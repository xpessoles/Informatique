\documentclass[10pt,fleqn]{article} % Default font size and left-justified equations
\usepackage[%
    pdftitle={Informatique : Transfert thermique},
    pdfauthor={Xavier Pessoles}]{hyperref}

%%%%%%%%%%%%%%%%%%%%%%%%%%%%%%%%%%%%%%%%%
% Original author:
% Mathias Legrand (legrand.mathias@gmail.com) with modifications by:
% Vel (vel@latextemplates.com)
% License:
% CC BY-NC-SA 3.0 (http://creativecommons.org/licenses/by-nc-sa/3.0/)
%%%%%%%%%%%%%%%%%%%%%%%%%%%%%%%%%%%%%%%%%



%----------------------------------------------------------------------------------------
%	MAIN TABLE OF CONTENTS
%----------------------------------------------------------------------------------------


% Part text styling
\titlecontents{part}[0cm]
{\addvspace{20pt}\centering\large\bfseries}
{}
{}
{}

% Chapter text styling
\titlecontents{chapter}[1.25cm] % Indentation
{\addvspace{12pt}\large\sffamily\bfseries} % Spacing and font options for chapters
{\color{bleuxp!60}\contentslabel[\Large\thecontentslabel]{1.25cm}\color{bleuxp}} % Chapter number
{\color{bleuxp}}  
{\color{bleuxp!60}\normalsize\;\titlerule*[.5pc]{.}\;\thecontentspage} % Page number

% Section text styling
\titlecontents{section}[1.25cm] % Indentation
{\addvspace{3pt}\sffamily\bfseries} % Spacing and font options for sections
{\color{bleuxp!60}\contentslabel[\thecontentslabel]{1.25cm} \color{bleuxp}} % Section number
{\color{bleuxp}}
{\hfill\color{bleuxp!60}\thecontentspage} % Page number
[]

% Subsection text styling
\titlecontents{subsection}[1.25cm] % Indentation
{\addvspace{1pt}\sffamily\small} % Spacing and font options for subsections
{\contentslabel[\thecontentslabel]{1.25cm}} % Subsection number
{}
{\ \titlerule*[.5pc]{.}\;\thecontentspage} % Page number
[]


% Subsection text styling
\titlecontents{subsubsection}[1.25cm] % Indentation
{\addvspace{1pt}\sffamily\small} % Spacing and font options for subsections
{\contentslabel[\thecontentslabel]{1.25cm}} % Subsection number
{}
{\ \titlerule*[.5pc]{.}\;\thecontentspage} % Page number
[]

% List of figures
\titlecontents{figure}[0em]
{\addvspace{-5pt}\sffamily}
{\thecontentslabel\hspace*{1em}}
{}
{\ \titlerule*[.5pc]{.}\;\thecontentspage}
[]

% List of tables
\titlecontents{table}[0em]
{\addvspace{-5pt}\sffamily}
{\thecontentslabel\hspace*{1em}}
{}
{\ \titlerule*[.5pc]{.}\;\thecontentspage}
[]

%----------------------------------------------------------------------------------------
%	MINI TABLE OF CONTENTS IN PART HEADS
%----------------------------------------------------------------------------------------

% Chapter text styling
\titlecontents{lchapter}[0em] % Indenting
{\addvspace{15pt}\large\sffamily\bfseries} % Spacing and font options for chapters
{\color{bleuxp}\contentslabel[\Large\thecontentslabel]{1.25cm}\color{bleuxp}} % Chapter number
{}  
{\color{bleuxp}\normalsize\sffamily\bfseries\;\titlerule*[.5pc]{.}\;\thecontentspage} % Page number

% Section text styling
\titlecontents{lsection}[0em] % Indenting
{\sffamily\small} % Spacing and font options for sections
{\contentslabel[\thecontentslabel]{1.25cm}} % Section number
{}
{}

% Subsection text styling
\titlecontents{lsubsection}[.5em] % Indentation
{\normalfont\footnotesize\sffamily} % Font settings
{}
{}
{}

%----------------------------------------------------------------------------------------
%	PAGE HEADERS
%----------------------------------------------------------------------------------------




\pagestyle{fancy}
 \renewcommand{\headrulewidth}{0pt}
 \fancyhead{}
 
 % ENTETES de page
 \fancyhead[L]{%
 \begin{tikzpicture}[overlay]
\node(logo) at (1,0)
    {\includegraphics[width=2cm]{logo_lycee.png}};
\end{tikzpicture}
 %\noindent\begin{minipage}[c]{2.6cm}%
 %\includegraphics[width=2cm]{logo_lycee.png}%
 %\end{minipage}
}

\fancyhead[C]{\rule{8cm}{.5pt}}

 \fancyhead[R]{%
 \noindent\begin{minipage}[c]{3cm}
 \begin{flushright}
 \footnotesize{\textit{\textsf{\xxtete}}}%
 \end{flushright}
 \end{minipage}
}

 \fancyfoot{}
 % PIEDS de page
\fancyfoot[C]{\rule{12cm}{.5pt}}
\renewcommand{\footrulewidth}{0.2pt}
\fancyfoot[C]{\footnotesize{\bfseries \thepage}}
\fancyfoot[L]{ 
\begin{minipage}[c]{.4\linewidth}
\noindent\footnotesize{{\xxauteur}}
\end{minipage}}

\fancyfoot[R]{\footnotesize{\xxpied}
\ifthenelse{\isodd{\value{page}}}{
\begin{tikzpicture}[overlay]
\node[shape=rectangle, 
      rounded corners = .25 cm,
	  draw= bleuxp,
	  line width=2pt, 
	  fill = bleuxp!10,
	  minimum width  = 2.5cm,
	  minimum height = 3cm,] at (\xxposongletx,\xxposonglety) {};
\node at (\xxposonglettext,\xxposonglety) {\rotatebox{90}{\textbf{\large\color{bleuxp}{\xxonglet}}}};
%{};
\end{tikzpicture}}{}
}



%
%
%
% Removes the header from odd empty pages at the end of chapters
\makeatletter
%\renewcommand{\cleardoublepage}{
%\clearpage\ifodd\c@page\else
%\hbox{}
%\vspace*{\fill}
%\thispagestyle{empty}
%\newpage
%\fi}

%\fancypagestyle{plain}{%
%\fancyhf{} % vide l’en-tête et le pied~de~page.
%%\fancyfoot[C]{\bfseries \thepage} % numéro de la page en cours en gras
%% et centré en pied~de~page.
%\fancyfoot[R]{\footnotesize{\xxpied}}
%\fancyfoot[C]{\rule{12cm}{.5pt}}
%\renewcommand{\footrulewidth}{0.2pt}
%\fancyfoot[C]{\footnotesize{\bfseries \thepage}}
%\fancyfoot[L]{ 
%\begin{minipage}[c]{.4\linewidth}
%\noindent\footnotesize{{\xxauteur}}
%\end{minipage}}}

\fancypagestyle{plain}{%
\fancyhf{} % vide l’en-tête et le pied~de~page.
\fancyfoot[C]{\rule{12cm}{.5pt}}
\renewcommand{\footrulewidth}{0.2pt}
\fancyfoot[C]{\footnotesize{\bfseries \thepage}}
\fancyfoot[L]{ 
\begin{minipage}[c]{.4\linewidth}
\noindent\footnotesize{{\xxauteur}}
\end{minipage}}
\fancyfoot[R]{\footnotesize{\xxpied}}
}







%----------------------------------------------------------------------------------------
%	SECTION NUMBERING IN THE MARGIN
%----------------------------------------------------------------------------------------
\setcounter{secnumdepth}{3}
\setcounter{tocdepth}{2}



\makeatletter
\renewcommand{\@seccntformat}[1]{\llap{\textcolor{bleuxp}{\csname the#1\endcsname}\hspace{1em}}}                    
\renewcommand{\section}{\@startsection{section}{1}{\z@}
{-4ex \@plus -1ex \@minus -.4ex}
{1ex \@plus.2ex }
{\normalfont\large\sffamily\bfseries}}
\renewcommand{\subsection}{\@startsection {subsection}{2}{\z@}
{-3ex \@plus -0.1ex \@minus -.4ex}
{0.5ex \@plus.2ex }
{\normalfont\sffamily\bfseries}}
\renewcommand{\subsubsection}{\@startsection {subsubsection}{3}{\z@}
{-2ex \@plus -0.1ex \@minus -.2ex}
{.2ex \@plus.2ex }
{\normalfont\small\sffamily\bfseries}}                        
\renewcommand\paragraph{\@startsection{paragraph}{4}{\z@}
{-2ex \@plus-.2ex \@minus .2ex}
{.1ex}
{\normalfont\small\sffamily\bfseries}}

%----------------------------------------------------------------------------------------
%	PART HEADINGS
%----------------------------------------------------------------------------------------


%----------------------------------------------------------------------------------------
%	CHAPTER HEADINGS
%----------------------------------------------------------------------------------------

% \newcommand{\thechapterimage}{}%
% \newcommand{\chapterimage}[1]{\renewcommand{\thechapterimage}{#1}}%
% \def\@makechapterhead#1{%
% {\parindent \z@ \raggedright \normalfont
% \ifnum \c@secnumdepth >\m@ne
% \if@mainmatter
% \begin{tikzpicture}[remember picture,overlay]
% \node at (current page.north west)
% {\begin{tikzpicture}[remember picture,overlay]
% \node[anchor=north west,inner sep=0pt] at (0,0) {\includegraphics[width=\paperwidth]{\thechapterimage}};
% \draw[anchor=west] (\Gm@lmargin,-9cm) node [line width=2pt,rounded corners=15pt,draw=bleuxp,fill=white,fill opacity=0.5,inner sep=15pt]{\strut\makebox[22cm]{}};
% \draw[anchor=west] (\Gm@lmargin+.3cm,-9cm) node {\huge\sffamily\bfseries\color{black}\thechapter. #1\strut};
% \end{tikzpicture}};
% \end{tikzpicture}
% \else
% \begin{tikzpicture}[remember picture,overlay]
% \node at (current page.north west)
% {\begin{tikzpicture}[remember picture,overlay]
% \node[anchor=north west,inner sep=0pt] at (0,0) {\includegraphics[width=\paperwidth]{\thechapterimage}};
% \draw[anchor=west] (\Gm@lmargin,-9cm) node [line width=2pt,rounded corners=15pt,draw=bleuxp,fill=white,fill opacity=0.5,inner sep=15pt]{\strut\makebox[22cm]{}};
% \draw[anchor=west] (\Gm@lmargin+.3cm,-9cm) node {\huge\sffamily\bfseries\color{black}#1\strut};
% \end{tikzpicture}};
% \end{tikzpicture}
% \fi\fi\par\vspace*{270\p@}}}

%-------------------------------------------

\def\@makeschapterhead#1{%
\begin{tikzpicture}[remember picture,overlay]
\node at (current page.north west)
{\begin{tikzpicture}[remember picture,overlay]
\node[anchor=north west,inner sep=0pt] at (0,0) {\includegraphics[width=\paperwidth]{\thechapterimage}};
\draw[anchor=west] (\Gm@lmargin,-9cm) node [line width=2pt,rounded corners=15pt,draw=bleuxp,fill=white,fill opacity=0.5,inner sep=15pt]{\strut\makebox[22cm]{}};
\draw[anchor=west] (\Gm@lmargin+.3cm,-9cm) node {\huge\sffamily\bfseries\color{black}#1\strut};
\end{tikzpicture}};
\end{tikzpicture}
\par\vspace*{270\p@}}
\makeatother



%----------------------------------------------------------------------------------------
%	
%----------------------------------------------------------------------------------------

\newcommand{\thechapterimage}{}%
\newcommand{\chapterimage}[1]{\renewcommand{\thechapterimage}{#1}}%
\def\@makechapterhead#1{%
{\parindent \z@ \raggedright \normalfont
\begin{tikzpicture}[remember picture,overlay]
\node at (current page.north west)
{\begin{tikzpicture}[remember picture,overlay]
\node[anchor=north west,inner sep=0pt] at (0,0) {\includegraphics[width=\paperwidth]{\thechapterimage}};
%\draw[anchor=west] (\Gm@lmargin,-9cm) node [line width=2pt,rounded corners=15pt,draw=bleuxp,fill=white,fill opacity=0.5,inner sep=15pt]{\strut\makebox[22cm]{}};
%\draw[anchor=west] (\Gm@lmargin+.3cm,-9cm) node {\huge\sffamily\bfseries\color{black}\thechapter. #1\strut};
\end{tikzpicture}};
\end{tikzpicture}
\par\vspace*{270\p@}
}}


%% Questions et exercices
\newcounter{numques}%Création d'un compteur qui s'appelle numques
\setcounter{numques}{0}%initialisation du compteur
\newcommand{\question}[1]{%Création d'une macro ayant un paramètre
\addtocounter{numques}{1}%chaque fois que cette macro est appelée, elle ajoute 1 au compteur numexos
\textbf{Question\, \textcolor{bleuxp}{\thenumques}\,}\,\textit{#1}}

\newcounter{numexo}%Création d'un compteur qui s'appelle numques
\setcounter{numexo}{0}%initialisation du compteur
\newcommand{\exer}[1]{%Création d'une macro ayant un paramètre
\refstepcounter{numexo} % incrément compteur et label
%\addtocounter{numexo}{1}%chaque fois que cette macro est appelée, elle ajoute 1 au compteur numexo
\noindent\textsf{\textbf{Exercice\, \textcolor{bleuxp}{\thenumexo}\, -- \, #1}}}



% \makeatletter             
% \renewcommand{\subparagraph}{\@startsection{exo}{5}{\z@}%
                                    % {-2ex \@plus-.2ex \@minus .2ex}%
                                    % {0ex}%               
% {\normalfont\bfseries Question \hspace{.7cm} }}
% \makeatother
% \renewcommand{\thesubparagraph}{\arabic{subparagraph}} 
% \makeatletter


%%%%%%%%%%%%
% Définition des vecteurs 
%%%%%%%%%%%%
\newcommand{\vect}[1]{\overrightarrow{#1}}
\newcommand{\axe}[2]{\left(#1,\vect{#2}\right)}
\newcommand{\couple}[2]{\left(#1,\vect{#2}\right)}
\newcommand{\angl}[2]{\left(\vect{#1},\vect{#2}\right)}

\newcommand{\rep}[1]{\mathcal{R}_{#1}}
\newcommand{\quadruplet}[4]{\left(#1;#2,#3,#4 \right)}
\newcommand{\repere}[4]{\left(#1;\vect{#2},\vect{#3},\vect{#4} \right)}
\newcommand{\base}[3]{\left(\vect{#1},\vect{#2},\vect{#3} \right)}


\newcommand{\vx}[1]{\vect{x_{#1}}}
\newcommand{\vy}[1]{\vect{y_{#1}}}
\newcommand{\vz}[1]{\vect{z_{#1}}}

\newcommand{\norm}[1]{\ensuremath{\left\Vert {#1}\right\Vert}}
\newcommand{\Ker}{\mathop{\mathrm{Ker}}\nolimits}

% d droit pour le calcul différentiel
\newcommand{\dd}{\text{d}}

\newcommand{\inertie}[2]{I_{#1}\left( #2\right)}
\newcommand{\matinertie}[7]{
\begin{pmatrix}
#1 & #6 & #5 \\
#6 & #2 & #4 \\
#5 & #4 & #3 \\
\end{pmatrix}_{#7}}
%%%%%%%%%%%%
% Définition des torseurs 
%%%%%%%%%%%%

\newcommand{\ec}[2]{%
\mathcal{E}_c\left(#1/#2\right)}

\newcommand{\pext}[3]{%
\mathcal{P}\left(#1\rightarrow#2/#3\right)}

\newcommand{\pint}[3]{%
\mathcal{P}\left(#1 \stackrel{\text{#3}}{\leftrightarrow} #2\right)}


 \newcommand{\torseur}[1]{%
\left\{{#1}\right\}
}

\newcommand{\torseurcin}[3]{%
\left\{\mathcal{#1} \left(#2/#3 \right) \right\}
}

\newcommand{\torseurci}[2]{%
\left\{\sigma \left(#1/#2 \right) \right\}
}
\newcommand{\torseurdyn}[2]{%
\left\{\mathcal{D} \left(#1/#2 \right) \right\}
}


\newcommand{\torseurstat}[3]{%
\left\{\mathcal{#1} \left(#2\rightarrow #3 \right) \right\}
}


 \newcommand{\torseurc}[8]{%
%\left\{#1 \right\}=
\left\{
{#1}
\right\}
 = 
\left\{%
\begin{array}{cc}%
{#2} & {#5}\\%
{#3} & {#6}\\%
{#4} & {#7}\\%
\end{array}%
\right\}_{#8}%
}

 \newcommand{\torseurcol}[7]{
\left\{%
\begin{array}{cc}%
{#1} & {#4}\\%
{#2} & {#5}\\%
{#3} & {#6}\\%
\end{array}%
\right\}_{#7}%
}

 \newcommand{\torseurl}[3]{%
%\left\{\mathcal{#1}\right\}_{#2}=%
\left\{%
\begin{array}{l}%
{#1} \\%
{#2} %
\end{array}%
\right\}_{#3}%
}

% Vecteur vitesse
 \newcommand{\vectv}[3]{%
\vect{V\left( {#1} \in {#2}/{#3}\right)}
}

% Vecteur force
\newcommand{\vectf}[2]{%
\vect{R\left( {#1} \rightarrow {#2}\right)}
}

% Vecteur moment stat
\newcommand{\vectm}[3]{%
\vect{\mathcal{M}\left( {#1}, {#2} \rightarrow {#3}\right)}
}




% Vecteur résultante cin
\newcommand{\vectrc}[2]{%
\vect{R_c \left( {#1}/ {#2}\right)}
}
% Vecteur moment cin
\newcommand{\vectmc}[3]{%
\vect{\sigma \left( {#1}, {#2} /{#3}\right)}
}


% Vecteur résultante dyn
\newcommand{\vectrd}[2]{%
\vect{R_d \left( {#1}/ {#2}\right)}
}
% Vecteur moment dyn
\newcommand{\vectmd}[3]{%
\vect{\delta \left( {#1}, {#2} /{#3}\right)}
}

% Vecteur accélération
 \newcommand{\vectg}[3]{%
\vect{\Gamma \left( {#1} \in {#2}/{#3}\right)}
}

% Vecteur omega
 \newcommand{\vecto}[2]{%
\vect{\Omega\left( {#1}/{#2}\right)}
}
% }$$\left\{\mathcal{#1} \right\}_{#2} =%
% \left\{%
% \begin{array}{c}%
%  #3 \\%
%  #4 %
% \end{array}%
% \right\}_{#5}}

\newcommand{\N}{\mathbb{N}}
\newcommand{\Z}{\mathbb{Z}}
\newcommand{\R}{\mathbb{R}}
\newcommand{\C}{\mathbb{C}}
\newcommand{\K}{\mathbb{K}}

\newcommand{\cA}{\mathscr{A}}
\newcommand{\cM}{\mathscr{M}}
\newcommand{\cL}{\mathscr{L}}
\newcommand{\cS}{\mathscr{S}}

\newcommand{\python}{\texttt{Python}}

\newcommand{\z}[1]{\Z_{#1}}
\newcommand{\ztimes}[1]{\Z_{#1}^{\times}}
\newcommand{\ii}[1]{[\![#1[\![}
\newcommand{\iif}[1]{[\![#1]\!]}
\newcommand{\llbr}{\ensuremath{\llbracket}}
\newcommand{\rrbr}{\ensuremath{\rrbracket}}
%\newcommand{\p}[1]{\left(#1\right)}
\newcommand{\ens}[1]{\left\{ #1 \right\}}
\newcommand{\croch}[1]{\left[ #1 \right]}
%\newcommand{\of}[1]{\lstinline{#1}}
% \newcommand{\py}[2]{%
%   \begin{tabular}{|l}
%     \lstinline+>>>+\textbf{\of{#1}}\\
%     \of{#2}
%   \end{tabular}\par{}
% }
\newcommand{\floor}[1]{\left\lfloor#1\right\rfloor}
\newcommand{\ceil}[1]{\left\lceil#1\right\rceil}
\newcommand{\abs}[1]{\left|#1\right|}


% Binaire, octal, hexa
\newcommand{\hex}[1]{\underline{\text{\texttt{#1}}}_{16}}
\newcommand{\oct}[1]{\underline{\text{\texttt{#1}}}_{8}}
\newcommand{\bin}[1]{\underline{\text{\texttt{#1}}}_{2}}
\DeclareMathOperator{\mmod}{\texttt{\%}}


% Fonctions et systèmes
\newcommand{\fct}[5][t]{%
  \begin{array}[#1]{rcl}
    #2 & \rightarrow & #3\\
    #4 & \mapsto     & #5\\
  \end{array}
}
\newcommand{\fonction}[5]{#1 : \left\{\begin{array}{rcl} #2& \longrightarrow &#3 \\ #4 &\longmapsto & #5\end{array}\right.}
\newenvironment{systeme}{\left\{ \begin{array}{rcl}}{\end{array}\right.}

% Matrices
\newcommand{\mat}[1]{
  \begin{pmatrix}
    #1
  \end{pmatrix}
}
\newcommand{\inv}{\ensuremath{^{-1}}}
\newcommand{\bpm}{\begin{pmatrix}}
\newcommand{\epm}{\end{pmatrix}}


% bases de données
\newcommand{\relat}[1]{\textsc{#1}}
\newcommand{\attr}[1]{\emph{#1}}
\newcommand{\prim}[1]{\uline{#1}}
\newcommand{\foreign}[1]{\#\textsl{#1}}


% Bases de données

\newcommand{\att}{\ensuremath{\mathbf{att}}}
\newcommand{\dom}{\ensuremath{\mathbf{dom}}}
\newcommand{\sort}{\ensuremath{\mathbf{sort}}}
\newcommand{\relname}{\ensuremath{\mathbf{relname}}}
\newcommand{\var}{\ensuremath{\mathbf{var}}}
\newcommand{\FILM}{\ensuremath{\mathtt{FILM}}}
\newcommand{\JOUE}{\ensuremath{\mathtt{JOUE}}}
\newcommand{\PERSONNE}{\ensuremath{\mathtt{PERSONNE}}}
\newcommand{\PERSONNAGE}{\ensuremath{\mathtt{PERSONNAGE}}}

\newcommand{\ttid}{\ensuremath{\mathtt{id}}}
\newcommand{\tttitre}{\ensuremath{\mathtt{titre}}}
\newcommand{\ttdate}{\ensuremath{\mathtt{date}}}
\newcommand{\ttidr}{\ensuremath{\mathtt{idrealisateur}}}
\newcommand{\ttdatenais}{\ensuremath{\mathtt{datenaissance}}}
\newcommand{\ttnom}{\ensuremath{\mathtt{nom}}}
\newcommand{\ttprenom}{\ensuremath{\mathtt{prenom}}}
\newcommand{\ttidacteur}{\ensuremath{\mathtt{idacteur}}}
\newcommand{\ttidfilm}{\ensuremath{\mathtt{idfilm}}}
\newcommand{\ttidpersonnage}{\ensuremath{\mathtt{idpersonnage}}}

\newcommand{\fv}{\mathrm{libre}}
\newcommand{\sem}[1]{[\![ #1 ]\!]}


\fichetrue
%\fichefalse

\proftrue
%\proffalse

%\tdtrue
\tdfalse

%\courstrue
\coursfalse

% -------------------------------------
% Déclaration des titres
% -------------------------------------

\def\discipline{Informatique \ifprof \\ Corrigé \else \fi}
\def\xxtete{Informatique}

\def\classe{PT -- PT $\star$}
\def\xxnumpartie{DM 01}
\def\xxpartie{Devoir maison 01}

\def\xxnumchapitre{Étude d'un tensiomètre électronique$\;$ }
\def\xxchapitre{\textit{$\;$ \\  D'après Banque PT 2016.}}

\def\xxtitreexo{Étude d'un tensiomètre électronique}
\def\xxsourceexo{\hspace{.2cm} D'après Banque PT 2016.}

\def\xxposongletx{2}
\def\xxposonglettext{1.45}
\def\xxposonglety{20}
\def\xxonglet{\textsf{DM 01}}

\def\xxactivite{}
\def\xxauteur{\textsl{La Martinière Monplaisir}}

\def\xxcompetences{%
\texttt{%
\textbf{Savoirs et compétences :}\\
\noindent \textbf{Résoudre :} à partir des modèles retenus :
\begin{itemize}[label=\ding{112},font=\color{ocre}] 
\item choisir une méthode de résolution analytique, graphique, numérique;
\item mettre en \oe{}uvre une méthode de résolution.
\end{itemize}
\begin{itemize}[label=\ding{112},font=\color{ocre}] 
\item \textit{Rés -- C1.1 :} Loi entrée sortie géométrique et cinématique -- Fermeture géométrique.
\end{itemize}
%
%\noindent \textit{Mod2 -- C4.1 :} Représentation par schéma bloc.
}}

\def\xxfigures{
%\includegraphics[width=.8\textwidth]{images/prot_01}
}%figues de la page de garde

\def\xxpied{%
Concours Blanc 2016}


\setcounter{secnumdepth}{5}
%---------------------------------------------------------------------------


\begin{document}

%\chapterimage{png/Fond_Cin}
\pagestyle{empty}


%%%%%%%% PAGE DE GARDE COURS
\ifcours
% ==== BANDEAU DES TITRES ==== 
\begin{tikzpicture}[remember picture,overlay]
\node at (current page.north west)
{\begin{tikzpicture}[remember picture,overlay]
\node[anchor=north west,inner sep=0pt] at (0,0) {\includegraphics[width=\paperwidth]{\thechapterimage}};
\draw[anchor=west] (-2cm,-8cm) node [line width=2pt,rounded corners=15pt,draw=ocre,fill=white,fill opacity=0.6,inner sep=40pt]{\strut\makebox[22cm]{}};
\draw[anchor=west] (1cm,-8cm) node {\huge\sffamily\bfseries\color{black} %
\begin{minipage}{1cm}
\rotatebox{90}{\LARGE\sffamily\textsc{\color{ocre}\textbf{\xxnumpartie}}}
\end{minipage} \hfill
\begin{minipage}[c]{14cm}
\begin{titrepartie}
\begin{flushright}
\renewcommand{\baselinestretch}{1.1} 
\Large\sffamily\textsc{\textbf{\xxpartie}}
\renewcommand{\baselinestretch}{1} 
\end{flushright}
\end{titrepartie}
\end{minipage} \hfill
\begin{minipage}[c]{3.5cm}
{\large\sffamily\textsc{\textbf{\color{ocre} \discipline}}}
\end{minipage} 
 };
\end{tikzpicture}};
\end{tikzpicture}
% ==== FIN BANDEAU DES TITRES ==== 


% ==== ONGLET 
\begin{tikzpicture}[overlay]
\node[shape=rectangle, 
      rounded corners = .25 cm,
	  draw= ocre,
	  line width=2pt, 
	  fill = ocre!10,
	  minimum width  = 2.5cm,
	  minimum height = 3cm,] at (18.3cm,0) {};
\node at (17.7cm,0) {\rotatebox{90}{\textbf{\Large\color{ocre}{\classe}}}};
%{};
\end{tikzpicture}
% ==== FIN ONGLET 


\vspace{3.5cm}

\begin{tikzpicture}[remember picture,overlay]
\draw[anchor=west] (-2cm,-6cm) node {\huge\sffamily\bfseries\color{black} %
\begin{minipage}{2cm}
\begin{center}
\LARGE\sffamily\textsc{\color{ocre}\textbf{\xxactivite}}
\end{center}
\end{minipage} \hfill
\begin{minipage}[c]{15cm}
\begin{titrechapitre}
\renewcommand{\baselinestretch}{1.1} 
\Large\sffamily\textsc{\textbf{\xxnumchapitre}}

\Large\sffamily\textsc{\textbf{\xxchapitre}}
\vspace{.5cm}

\renewcommand{\baselinestretch}{1} 
\normalsize\normalfont
\xxcompetences
\end{titrechapitre}
\end{minipage}  };
\end{tikzpicture}
\vfill

\begin{flushright}
\begin{minipage}[c]{.3\linewidth}
\begin{center}
\xxfigures
\end{center}
\end{minipage}\hfill
\begin{minipage}[c]{.6\linewidth}
\startcontents
%\printcontents{}{1}{}
\printcontents{}{1}{}
\end{minipage}
\end{flushright}

\begin{tikzpicture}[remember picture,overlay]
\draw[anchor=west] (4.5cm,-.7cm) node {
\begin{minipage}[c]{.2\linewidth}
\begin{flushright}
\includegraphics[width=2cm]{logoCC}
\end{flushright}
\end{minipage}
\begin{minipage}[c]{.2\linewidth}
\textsl{\xxauteur} \\
\textsl{\classe}
\end{minipage}
 };
\end{tikzpicture}

\newpage
\pagestyle{fancy}

%\newpage
%\pagestyle{fancy}

\else
\fi
%% FIN PAGE DE GARDE DES COURS

%%%%%%%% PAGE DE GARDE TD
\iftd
%\begin{tikzpicture}[remember picture,overlay]
%\node at (current page.north west)
%{\begin{tikzpicture}[remember picture,overlay]
%\draw[anchor=west] (-2cm,-3.25cm) node [line width=2pt,rounded corners=15pt,draw=ocre,fill=white,fill opacity=0.6,inner sep=40pt]{\strut\makebox[22cm]{}};
%\draw[anchor=west] (1cm,-3.25cm) node {\huge\sffamily\bfseries\color{black} %
%\begin{minipage}{1cm}
%\rotatebox{90}{\LARGE\sffamily\textsc{\color{ocre}\textbf{\xxnumpartie}}}
%\end{minipage} \hfill
%\begin{minipage}[c]{13.5cm}
%\begin{titrepartie}
%\begin{flushright}
%\renewcommand{\baselinestretch}{1.1} 
%\Large\sffamily\textsc{\textbf{\xxpartie}}
%\renewcommand{\baselinestretch}{1} 
%\end{flushright}
%\end{titrepartie}
%\end{minipage} \hfill
%\begin{minipage}[c]{3.5cm}
%{\large\sffamily\textsc{\textbf{\color{ocre} \discipline}}}
%\end{minipage} 
% };
%\end{tikzpicture}};
%\end{tikzpicture}

%%%%%%%%%% PAGE DE GARDE TD %%%%%%%%%%%%%%%
%\begin{tikzpicture}[overlay]
%\node[shape=rectangle, 
%      rounded corners = .25 cm,
%	  draw= ocre,
%	  line width=2pt, 
%	  fill = ocre!10,
%	  minimum width  = 2.5cm,
%	  minimum height = 2.5cm,] at (18.5cm,0) {};
%\node at (17.7cm,0) {\rotatebox{90}{\textbf{\Large\color{ocre}{\classe}}}};
%%{};
%\end{tikzpicture}

% PARTIE ET CHAPITRE
%\begin{tikzpicture}[remember picture,overlay]
%\draw[anchor=west] (-1cm,-2.1cm) node {\large\sffamily\bfseries\color{black} %
%\begin{minipage}[c]{15cm}
%\begin{flushleft}
%\xxnumchapitre \\
%\xxchapitre
%\end{flushleft}
%\end{minipage}  };
%\end{tikzpicture}

% BANDEAU EXO
\iflivret % SI LIVRET
\begin{tikzpicture}[remember picture,overlay]
\draw[anchor=west] (-2cm,-3.3cm) node {\huge\sffamily\bfseries\color{black} %
\begin{minipage}{5cm}
\begin{center}
\LARGE\sffamily\color{ocre}\textbf{\textsc{\xxactivite}}

\begin{center}
\xxfigures
\end{center}

\end{center}
\end{minipage} \hfill
\begin{minipage}[c]{12cm}
\begin{titrechapitre}
\renewcommand{\baselinestretch}{1.1} 
\large\sffamily\textbf{\textsc{\xxtitreexo}}

\small\sffamily{\textbf{\textit{\color{black!70}\xxsourceexo}}}
\vspace{.5cm}

\renewcommand{\baselinestretch}{1} 
\normalsize\normalfont
\xxcompetences
\end{titrechapitre}
\end{minipage}};
\end{tikzpicture}
\else % ELSE NOT LIVRET
\begin{tikzpicture}[remember picture,overlay]
\draw[anchor=west] (-2cm,-4.5cm) node {\huge\sffamily\bfseries\color{black} %
\begin{minipage}{5cm}
\begin{center}
\LARGE\sffamily\color{ocre}\textbf{\textsc{\xxactivite}}

\begin{center}
\xxfigures
\end{center}

\end{center}
\end{minipage} \hfill
\begin{minipage}[c]{12cm}
\begin{titrechapitre}
\renewcommand{\baselinestretch}{1.1} 
\large\sffamily\textbf{\textsc{\xxtitreexo}}

\small\sffamily{\textbf{\textit{\color{black!70}\xxsourceexo}}}
\vspace{.5cm}

\renewcommand{\baselinestretch}{1} 
\normalsize\normalfont
\xxcompetences
\end{titrechapitre}
\end{minipage}};
\end{tikzpicture}

\fi

\else   % FIN IF TD
\fi


%%%%%%%% PAGE DE GARDE FICHE
\iffiche
\begin{tikzpicture}[remember picture,overlay]
\node at (current page.north west)
{\begin{tikzpicture}[remember picture,overlay]
\draw[anchor=west] (-2cm,-2.25cm) node [line width=2pt,rounded corners=15pt,draw=ocre,fill=white,fill opacity=0.6,inner sep=40pt]{\strut\makebox[22cm]{}};
\draw[anchor=west] (1cm,-2.25cm) node {\huge\sffamily\bfseries\color{black} %
\begin{minipage}{1cm}
\rotatebox{90}{\LARGE\sffamily\textsc{\color{ocre}\textbf{\xxnumpartie}}}
\end{minipage} \hfill
\begin{minipage}[c]{14cm}
\begin{titrepartie}
\begin{flushright}
\renewcommand{\baselinestretch}{1.1} 
\large\sffamily\textsc{\textbf{\xxpartie} \\} 

\vspace{.2cm}

\normalsize\sffamily\textsc{\textbf{\xxnumchapitre -- \xxchapitre}}
\renewcommand{\baselinestretch}{1} 
\end{flushright}
\end{titrepartie}
\end{minipage} \hfill
\begin{minipage}[c]{3.5cm}
{\large\sffamily\textsc{\textbf{\color{ocre} \discipline}}}
\end{minipage} 
 };
\end{tikzpicture}};
\end{tikzpicture}

\iflivret
\begin{tikzpicture}[overlay]
\node[shape=rectangle, 
      rounded corners = .25 cm,
	  draw= ocre,
	  line width=2pt, 
	  fill = ocre!10,
	  minimum width  = 2.5cm,
	  minimum height = 2.5cm,] at (18.5cm,1.1cm) {};
\node at (17.9cm,1.1cm) {\rotatebox{90}{\textsf{\textbf{\large\color{ocre}{\classe}}}}};
%{};
\end{tikzpicture}
\else
\begin{tikzpicture}[overlay]
\node[shape=rectangle, 
      rounded corners = .25 cm,
	  draw= ocre,
	  line width=2pt, 
	  fill = ocre!10,
	  minimum width  = 2.5cm,
%	  minimum height = 2.5cm,] at (18.5cm,1.1cm) {};
	  minimum height = 2.5cm,] at (18.6cm,1cm) {};
\node at (18cm,1cm) {\rotatebox{90}{\textsf{\textbf{\large\color{ocre}{\classe}}}}};
%{};
\end{tikzpicture}

\fi

\else
\fi



\vspace{1cm}
\pagestyle{fancy}
\thispagestyle{plain}

\section{Présentation}
Dans le domaine médical, la pression artérielle est l'un des paramètres
les plus surveillés dans le cadre des maladies cardio-vasculaires. En effet,
quand la pression artérielle est trop élevée, pendant des durées très longues,
le muscle cardiaque finit par se fatiguer jusqu'à atteindre des insuffisances
cardiaques sévères ; on parle d'hypertension.

À l'inverse, une tension trop basse conduit à sous-alimenter les organes
et peut conduire à des étourdissements ou évanouissements. L'hypotension
n'est généralement pas considérée comme une maladie.
Communément, la mesure de la pression artérielle se fait par la détermination
de deux valeurs classiques : la pression systolique (pression la plus
élevée suite à la contraction du cœur) et la pression diastolique (pression
la moins élevée après le relâchement du cœur quand celui-ci se remplit de
sang).

Pour mesurer ces pressions, le médecin utilise classiquement la méthode
du brassard associé à son stéthoscope. Le médecin gonfle le brassard jusqu'à
couper la circulation sanguine dans le bras. Le médecin pose ensuite
le stéthoscope sur l'artère et dégonfle progressivement le brassard. Dès qu'il
écoute le sang passer à nouveau dans l'artère, il mesure la pression systolique
sur le manomètre associé au brassard. Puis il continue à dégonfler le
brassard jusqu'à ne plus écouter aucun bruit, là il lit la pression diastolique.
On peut également mesurer ces pressions avec un tensiomètre électronique (voir figure 1).
Il existe des modèles simples pour les particuliers et des modèles beaucoup plus complets dans
le cadre des milieux médicaux.

Les applications numériques seront données avec 1 chiffre significatif, sauf contre-ordre.

\section{Étude du capteur de pression}
\subsection{Étude du capteur : modélisation de la déformation de la membrane}
Le principe du capteur de pression repose sur l'effet piézorésistif (voir figure 2). La cellule
sensible du capteur est constituée d'une membrane se déformant sous l'effet de la force appliquée
par un micro-vérin. Le micro-vérin applique une force F proportionnelle à la pression du
brassard. Sur la membrane sont fixées 4 jauges de déformation. Les résistances électriques de ces
jauges varient sous l'effet d'une déformation. Puis cette variation de résistance est transformée
en tension électrique Vp au moyen d'un circuit de conditionnement.

L'objectif de cette partie est de relier la pression du brassard à la tension mesurée dans le
circuit de conditionnement en fonction des différents paramètres caractéristiques du capteur. On cherchera également à montrer que le capteur permet d'avoir une précision de moins d'un
mmHg.

En première approximation, la membrane peut être modélisée par une poutre de longueur
$2L$ encastrée à chacune de ses extrémités soumise à un effort d'intensité $2F$ en son milieu (figure
3(a)).
Les caractéristiques de la poutre sont :
\begin{itemize}
\item $E$ : le module d'Young;
\item $S$ : la surface de la section droite de la poutre;
\item $I$ : le moment d'inertie de la section autour de l'axe $(G,\vect{z})$;
\item $e$ : l'épaisseur de la membrane.
\end{itemize}

$G$ est le centre d'inertie de la section droite de la poutre.
On note $\vect{AG}=x\vect{x}$. On suppose que le déplacement de la section droite est $\vect{u}\left( G\right) = u_y(x) \vect{y}$.
La force appliquée par le vérin est notée $\vect{F}=-F\vect{y}$. 
Les effets de la pesanteur sont négligées.
Le modèle est considéré comme plan.


Pour simplifier le calcul de la déformée de la membrane, on adopte le modèle donné sur la figure 3(b). On supposera que la liaison en $B$ bloque la translation suivant $\vect{x}$ et la rotation suivant $(B,\vect{z})$.

\subparagraph{}
\textit{Justifier pourquoi le modèle de la figure 3(b) est équivalent à celui de la figure 3(a). Justifier
l'intérêt de cette simplification en déterminant le degré d'hyperstatisme des deux modèles.}

On notera les torseurs des actions mécaniques au niveau des liaisons sous la forme :

$\torseurstat{T}{i}{j}=\torseurcol{X_M}{Y_M}{-}{-}{-}{M_M}{M,R}$, le torseur de cohésion est noté $\left\{\mathcal{T}(coh) \right\}=\torseurcol{T_X}{T_Y}{-}{-}{-}{M_Z}{G,R}$ (action de la partie droite sur la partie gauche).


\subparagraph{}
\textit{Isoler le tronçon de poutre compris entre l'abscisse $[0, x]$. Écrire l'équation issue du principe
fondamental de la statique permettant d'obtenir la relation entre le moment fléchissant $M_Z$,
l'effort tranchant $T_Y$ et une (ou des) inconnue(s) de la liaison en $A$.
Effectuer un second isolement et écrire l'équation nécessaire permettant d'exprimer l'effort
tranchant $T_Y$ en fonction de $F$.
En déduire la relation entre le moment fléchissant $M_Z$, l'effort $F$ et une (ou des) inconnue(s)
de la liaison en $A$.}


On rappelle que la relation de comportement reliant le moment de flexion au déplacement $u_y$
est $M_z = EI \dfrac{d^2 u_y}{d x^2}$. On se place dans les hypothèses de Navier-Bernoulli en petits déplacements,
on a ainsi $\theta = \dfrac{du_y}{dx}$, $\theta$ la rotation de la section droite autour de l'axe $(G,\vect{z})$. 



\subparagraph{}
\textit{Après intégration, montrer que le déplacement se met sous la forme $u_y(x) = A_1x^3 + A_2x^2$
où vous préciserez les expressions des constantes $A_1$ et $A_2$ en fonction des paramètres du modèle
en éliminant les inconnues de la liaison en $A$.}

On rappelle que le problème étant hyperstatique, le moment de flexion est exprimé en fonction
d'inconnues de liaisons qui seront déterminées en appliquant les conditions limites.

On rappelle que la déformation dans une section droite de la poutre est donnée par 
$\varepsilon(x) = -y \dfrac{d\theta}{ d x}=-y\dfrac{d^2 u_y} {dx^2}$.

\subparagraph{}
\textit{Montrer que la déformation sur la peau supérieure de la membrane ($y = e/2$) se met sous
forme $\varepsilon = A_3 +A_4x$, où vous préciserez les expressions des constantes $A_3$ et $A_4$ en fonction des
paramètres du modèle.}

Quatre jauges de déformation sont disposées sur la peau extérieure de la membrane (voir
figure 4).


\subparagraph{}

\textit{Quelle serait la position idéale des jauges de déformation ? En pratique elles sont placées
en $x = \dfrac{L}{4}$ et en $x = \dfrac{3L}{4}$, donner la valeur des déformations de ces jauges et montrer que
$\varepsilon_1 = -\varepsilon_2 = -\varepsilon_3 = \varepsilon_4 = kF$ 
où vous donnerez l'expression de $k$ 
en fonction des paramètres du modèle.}

\subsection{Choix des jauges de déformations}


Les jauges de déformation disposées sur la membrane (voir figure 4) subissent une déformation
proportionnelle à l'intensité de la force F appliquée par le piston :
$\varepsilon_1 = -\varepsilon_2 = -\varepsilon_3 = \varepsilon_4 = kF$.

On cherche maintenant à déterminer la relation entre la variation de la résistance de la jauge
de déformation et l'effort presseur $F$.
Considérons une jauge de déformation de forme parallélépipédique constituée d'un matériau
homogène de résistivité $\rho$ (s'exprimant en $\Omega\cdot m$). On note $\ell$ sa longueur, $a$ et $b$ ses dimensions
transversales, $s$ sa section droite (voir figure 5).


La résistance électrique $R$ de ce corps d'épreuve s'exprime au moyen de la relation $R = \rho \dfrac{\ell}{s}$.


\subparagraph{}
\textit{Lorsque la jauge d'épreuve est soumise à une force $\vect{F}$ appliquée parallèlement à $\ell$, elle se
déforme : $\ell$, $s$ et $\rho$ subissent de petites variations valant respectivement $\delta \ell$, $\delta s$ et $\delta \rho$. Exprimer
la variation relative de résistance électrique $\dfrac{\delta R}{R}$.}

Les variations de $\ell$, de $a$ et $b$ ne sont pas indépendantes : $\dfrac{\delta a}{a}=\dfrac{\delta b}{b}=-\nu \dfrac{\delta \ell}{\ell}$ avec $\nu$ une constante dépendant du matériau, appelée coefficient de Poisson.

\subparagraph{}
\textit{Exprimer $\dfrac{\delta R}{R}$ en fonction de $\dfrac{\delta \ell}{\ell}$, $\nu$, $\dfrac{\delta \rho}{\rho}$.}



Dans le cas d'un métal, $\delta \rho$ est lié à la variation $\delta V$ du volume $V$ selon la loi $\dfrac{\delta \rho}{\rho}=c \dfrac{\delta V}{V}$ avec $c$ une constante appelée constante de Bridgman.

\subparagraph{}
\textit{Montrer que $\dfrac{\delta R}{R}$  peut s'écrire sous la forme : $\dfrac{\delta R}{R}=K\dfrac{\delta \ell }{\ell}$
avec $K$ une constante appelée facteur de jauge.}

Dans le cas d'un semi-conducteur, $\delta \rho$ est lié à la contrainte $\sigma = \dfrac{||\vect{F}||}{s}$
selon la loi $\dfrac{\delta \rho}{\rho}=\Pi \sigma$, avec $\Pi$ une constante appelée constante de piézorésistivité; de plus, le module de
 Young $E$ du matériau lie la contrainte $\sigma$ et la déformation $\dfrac{\delta \ell}{\ell}$ : $\dfrac{\delta \ell}{\ell} = \dfrac{\sigma}{E}$.


\subparagraph{}
\textit{Exprimer le facteur de jauge dans le cas d'un semi-conducteur.}
On donne : 
\begin{itemize}
\item pour le cuivre : $\nu =0,3$, $c = 1$;
\item pour le silicium : $\nu =0,4$, $E = 1011 Pa$, $\Pi= 10^{-9}Pa^{-1}$.
\end{itemize}

\subparagraph{}
\textit{Évaluer le facteur de jauge dans le cas du cuivre et dans le cas du silicium. Commenter.}
\subparagraph{}
\textit{Quel phénomène peut induire une erreur sur la mesure de la contrainte ?}

\subparagraph{Circuit de conditionnement : pont de Wheatstone}

\end{document}

Les 4 jauges semi-conductrices identiques, de résistance
R0 à l'équilibre, sont fixées sur la membrane
(voir figure 4). La déformation 
 de la jauge placée
en x vaut donc (x).
La déformation des jauges conduit à une variation
Ri (i = a, b, c ou d) de leur résistance. Les 4 jauges
sont branchées en pont de Wheatstone, alimenté sous
une tension Valim = 16 V. Les bornes A et B sont
branchées à un amplificateur de résistance d'entrée
très grande.


Q12. Déterminer l'expression de la tension Vp en fonction des paramètres du pont.


A l'équilibre le pont est réglé de manière à avoir Vp = 0.
Q13. Déterminer la variation de tension Vp = Vp lorsque les résistances subissent des variations
Ri en fonction des variations Ri, Valim, R0.
Q14. La jauge 1 correspond à la résistance Rb. Déterminer la position qu'il convient de donner
aux autres jauges dans le pont de Wheatstone (autrement dit faire le lien entre les jauges 2, 3
et 4 avec les résistances Ra, Rc, et Rd du pont).
Q15. Exprimer la tension Vp en fonction de Valim, 1 et du facteur de jauge K.
On considère un capteur de pression constitué d'une membrane en silicium ayant les caractéristiques
suivantes :
– surface Sm = 4 mm2
– longueur 2L = 4 mm;
– épaisseur e = 0,1 mm;
– module d'Young E = 1011 Pa ;
– moment d'inertie de la section I = 2.10−4 mm4.
Q16. Etablir la relation entre la tension Vp et la pression P = F
Sm
, puis définir et exprimer la
sensibilité de ce capteur de pression en fonction de K, e, L, E, I, Sm et Valim.
Q17. Quelle est la tension Vp obtenue pour une pression de 50 kPa ?

III Traitement numérique des informations



Le capteur fournit une tension u(t) qui va subir un traitement informatique par une carte de
commande. L'objectif de cette partie est de mettre en place les différents algorithmes permettant
de déterminer les pressions diastolique et systolique ainsi que le pouls du patient après avoir
filtré numériquement le signal. L'utilisation en milieu hospitalier nécessite la sauvegarde des
différentes données dans une base de données afin de réaliser un meilleur suivi du patient.
Les algorithmes demandés au candidat peuvent être réalisés dans le langage Python ou
dans le langage Scilab ; toutes les questions seront traitées dans le même langage.
Veillez à apporter les commentaires suffisants à la compréhension de votre programme et à
utiliser des noms de variables explicites.

III.1 Numérisation du signal
Le signal Vp(t) issu du conditionnement est amplifié pour donner la tension u(t) fournie par
le capteur ; cette tension u(t) est numérisé par un convertisseur analogique-numérique.
La tension u(t) mesurée appartient à [0, 5V] pour une pression comprise entre [0, 1350HPa].
La pression affichée est en mmHg, on rappelle que 1013HPa = 750 mmHg.
La résolution de l'affichage de la pression est de 1 mmHg, cependant la numérisation du
signal nécessite une précision plus importante de l'ordre de 0,02mmHg maximum.
Le constructeur a, à sa disposition, des convertisseurs analogique-numérique (CAN) dotés
d'une résolution de 10 bits, 12 bits et 16 bits.
Q18. Déterminer en justifiant votre réponse la résolution du (ou des) CAN permettant de
respecter le critère de précision.
On cherche à déterminer la quantité de mémoire nécessaire pour stocker une mesure.
Une mesure peut durer jusqu'à 60 s. La fréquence d'échantillonnage est de 1000 Hz.
L'ensemble des valeurs de la mesure va être stocké dans un tableau.



Q19. Choisir un type de stockage pour les valeurs parmi les types suivants : entier relatif codé
sur 8, 16, 32 ou 64 bits, entier non signé codé sur 8, 16, 32 ou 64 bits, réel codé sur 32 ou 64
bits. En déduire la quantité de mémoire nécessaire pour stocker le tableau, exprimée en octets
(1 octet = 8 bits).

III.2 Filtrage numérique du signal
La fréquence d'échantillonnage est notée fe, la période d'échantillonnage est notée Te. Le
signal numérisé est stocké dans une liste de taille N, notée Ue, contenant l'ensemble des valeurs
Ue(nTe) accessible par la commande Ue[n] avec n  0,N −1 en Python et par la commande
Ue(n) avec n  1,N en Scilab.
Pour simplifier les notations, toute fonction du temps Ug(t) évaluée au temps iTe sera notée
Ug,i au lieu de Ug(iTe).
Le signal numérisé est filtré pour supprimer les bruits de mesure et permet d'obtenir le
signal filtré numérique Uf sous forme d'une nouvelle liste Uf.
Le filtre numérique retenu est un filtre du 2nd ordre modélisé par l'équation différentielle
1
2
¨U
f (t) + 2z

˙U
f (t) + Uf (t) = Ue(t)
On choisit  = 20 rad/s et z = 0,7.
Q20. Justifier succinctement le choix de ces valeurs sachant que la fréquence cardiaque est au
maximum de 190 pulsations par minute.
Pour résoudre cette équation différentielle, on utilise le schéma numérique d'intégration de
Newmark :
˙U
f,i+1 = ˙Uf,i + (1 − )Te¨Uf,i + Te¨Uf,i+1
Uf,i+1 = Uf,i + Te˙Uf,i + T2
e (1/2 − )¨Uf,i + T2
e ¨Uf,i+1
où  et  sont des paramètres de la méthode.


Q21. En introduisant les relations du schéma de Newmark dans l'équation différentielle exprimée
au pas de temps (i+1)Te, montrer que la relation de récurrence peut se mettre sous la forme
suivante : ¨Uf,i+1 = B1

Ue,i+1 + B2¨Uf,i + B3˙Uf,i + B4Uf,i

avec B1 = 1
1/2 + 2z

Te + T2
e 
.
Vous exprimerez B2, B3 et B4 en fonction de z, , , .
Afin d'analyser les propriétés de la méthode de Newmark, on résout l'équation différentielle
pour un second membre simple avec différents pas de temps, pour un réglage des paramètres 
et  fixé. Les résultats sont synthétisés sur la figure 7.
Q22. Qualifier la stabilité du schéma d'intégration en fonction du pas de temps.
On définit l'erreur du schéma d'intégration par  = max
i0,N |Uf,i − Uf,exacte,i|. Le tableau
suivant donne l'évolution de l'erreur en fonction du pas de temps.


Te 0,01 0,001 0,0001
 0,05 0,005 0,0005
Q23. Etant donné l'évolution de l'erreur, proposer un ordre de convergence associé à la méthode
de Newmark dans la configuration donnée.


Dans le cadre de l'acquisition numérique, la période d'échantillonnage est fixée, donc le pas
de temps d'intégration aussi. On rappelle que la fréquence d'échantillonnage est de 1000 Hz.
Q24. Conclure quant à la qualité du calcul du signal filtré.
Q25. Ecrire une fonction newmark(gamma,beta,omega,z,e,Te) qui prend en argument les paramètres
 et  de la méthode, les paramètres  et z du filtre, le signal à filtrer e et la période
d'échantillonnage Te. Cette fonction retourne le signal filtré sous la forme d'une liste de même
taille que la variable e. On supposera les conditions initiales nulles. On pourra utiliser directement
les constantes B1, B2, B3 et B4 en les supposant définies comme variables globales.
Q26. Proposer une amélioration de la fonction newmark afin de minimiser la taille des variables
intermédiaires, donner la nouvelle fonction. Si vous pensez avoir déjà un algorithme optimal,
précisez-le sur votre copie en justifiant.

III.3 Détermination des pressions systolique, diastolique


Les courbes de la pression mesurée brute et filtrée sont données sur la figure 8. L'objectif
de cette partie est d'analyser la courbe afin d'extraire les données recherchées. On dispose de
deux tableaux : data contenant la pression mesurée filtrée et t contenant le temps associé à
chaque mesure (les données sont rangées par temps croissant).
Pour déterminer la pression systolique, il faut parcourir la courbe ; celle-ci présente des
successions de bosses et de creux qui seront respectivement nommés maximum local et minimun
local. La pression systolique correspond à la valeur du maximum local où le critère
maximum_local − minimum_local
minimum_local > 4 × 10−5 est vérifié pour la première fois en parcourant
les données par temps croissant.
Q27. Ecrire une fonction pression_systolique(data) qui prend en argument les données
mesurées filtrées et qui retourne la pression systolique. On supposera qu'il est toujours possible
de trouver la pression systolique dans le tableau data.

La pression diastolique correspond à la valeur du maximum local où le critère
maximum_local − minimum_local
minimum_local > 4 × 10−3 est vérifié pour la dernière fois en parcourant
les données par temps croissant et que le maximum local reste supérieur à 40 mmHg pour éviter
de détecter un bruit numérique.
Q28. Ecrire une fonction pression_diastolique(data) qui prend en argument les données
mesurées filtrées et qui retourne la pression diastolique. On supposera qu'il est toujours possible
de trouver la pression diastolique parmi dans le tableau data.

IV Stockage et analyse des résultats

Le tensiomètre est utilisé dans le domaine hospitalier lors des visites régulières des infirmières
dans les chambres des patients. Le tensiomètre doit permettre d'enregistrer les différentes
constantes pour chaque patient. Après utilisation, le tensiomètre est connecté au réseau
de l'hôpital et synchronise ses mesures avec la base de données des patients.
La base de données est composée de plusieurs tables dont :
– patients avec les attributs :
• id : identifiant associé à un patient, clé primaire
• numero_secu : le numéro de sécurité sociale du patient
• nom : le nom du patient
• prenom : le prénom (ou les prénoms) du patient
• adresse1 : adresse du patient
• adresse2 : complément d'adresse du patient
• code_postal : code postal de la ville du patient
• ville : ville du patient
• date_de_naissance : date de naissance du patient
• telephone : numéro de téléphone
• ...
– mesures avec les attributs :
• id : identifiant associé à une mesure, clé primaire

• type : le type de mesure stockée : résultat d'analyse sanguine, pression artérielle, radio...
• datetime : la date et l'heure de l'examen sous la forme '2016-01-01 00 :00 :00'
• pid : identifiant du patient associé à la mesure
• pdias : pression diastolique en mmHg.
• psyst : pression systolique en mmHg.
• pouls : pouls en pulsations par minute.
• ...
Remarque : Les tables contiennent bien d'autres attributs, la base de données d'autres
tables, seuls les éléments nécessaires à la suite ont été mis en place.
Q29. Rappeler en quelques lignes l'utilité d'une clé primaire dans chacune des tables et dire si
un autre attribut des tables aurait pu servir de clé primaire.


Pour le besoin du suivi médical des patients, les médecins doivent pouvoir avoir accès à
l'évolution des constantes en fonction du temps lors d'une hospitalisation.
A l'aide d'une interface graphique non étudiée ici, le médecin peut sélectionner les données
à extraire entre deux dates qui seront stockées dans les variables time1 et time2 stockées dans
le format de datetime. Il souhaite tracer les évolutions des pressions systolique, diastolique et
le pouls entre ces deux dates.
Q30. Donner la requête SQL à envoyer au serveur de gestion de la base de données permettant
de récupérer la valeur des champs datetime, pdias, psyst et pouls entre les deux dates time1
et time2 (stockées dans le format de datetime).
La requête précédente renvoie toutes les mesures sans distinction de type de mesures et sans
que ce soit lié à un unique patient.
Le médecin a sélectionné un patient avec son identifiant unique id, cette valeur est stockée
dans la variable id_patient.
Q31. Donner la requête SQL à envoyer au serveur de gestion de la base de données permettant
de récupérer la valeur des champs précédents entre les deux mêmes dates, pour le patient
id_patient et pour un type de mesures 'tension'.

La requête est envoyée et traitée à l'aide d'une fonction traitement_requete(requete)
qui retourne un tableau resultat_requete dont la première colonne est le champ datetime,
la deuxième le champ psyst, la troisième le champ pdias et la dernière pouls.
On supposera pour simplifier que la valeur de la colonne datetime est comprise comme un
entier par le langage de programmation retenu et qu'il peut être utilisé directement comme
abscisse des graphiques à produire dans la question suivante.
Q32. Donner la séquence d'instructions permettant de tracer un graphique comportant l'évolution
des pressions diastolique et systolique ainsi que le pouls en fonction du temps. Vous
veillerez à mettre des titres et légendes sur le graphique. Vous utiliserez des styles de lignes
différents pour chaque ligne dont vous expliciterez en commentaire le résultat visuel (couleur,
type de trait : continu, pointillé...). Des éléments de documentation sont donnés en annexe page
11.


A coté du graphique, le médecin souhaite avoir les valeurs minimale, maximale, moyenne et
médiane de chacune des constantes.
Q33. Ecrire une fonction analyse(valeurs) qui prend en argument une liste de valeurs et qui
retourne le minimum, le maximum et la moyenne.
Q34. Déterminer la complexité de votre fonction analyse(valeurs) en fonction de la taille de
la liste de valeurs notée n dans le meilleur et le pire des cas. Vous calculerez pour cela le nombre
de comparaisons et d'opérations (addition, soustraction, division, multiplication) réalisées.



Q35.A l'aide d'une méthode de tri de votre choix que vous expliciterez, écrire une fonction
mediane(valeurs) qui prend en argument une liste de valeurs et qui renvoie la valeur de la
médiane.
Q36. Déterminer la complexité de votre fonction dans le meilleur et dans le pire des cas en
comptant le nombre de comparaisons effectuées. Votre méthode est-elle efficace ? Justifier.
Le médecin souhaite obtenir les nom, prénom et le numéro de téléphone des patients qui
ont eu une pression systolique supérieure strictement à 160 mmHg, une pression diastolique
supérieure strictement à 110 mmHg, avec un pouls compris strictement entre 100 et 150.
Q37. Ecrire la requête SQL à envoyer au serveur de gestion de la base de données permettant
de répondre à la demande du médecin.





\end{document}
