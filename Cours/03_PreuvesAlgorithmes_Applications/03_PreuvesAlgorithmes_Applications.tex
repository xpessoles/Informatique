%\setchapterimage{bandeau}
\chapter*{TD \arabic{cptTD} \\ 
Correction des algorithmes -- 
\ifprof Corrigé \else Sujet \fi}
\addcontentsline{toc}{section}{TD \arabic{cptTD} :
Correction des algorithmes -- 
\ifprof Corrigé \else Sujet \fi}

\iflivret \stepcounter{cptTD} \else
\ifprof  \stepcounter{cptTD} \else \fi
\fi

\setcounter{question}{0}
%\marginnote{Sources}
%\marginnote[1cm]{
%\UPSTIcompetence[2]{C1-02}
%\UPSTIcompetence[2]{C2-04}}


%\subsection*{Exercice}
%
%\begin{marginfigure}
%\begin{lstlisting}
%def rechercheMaxi(L) :
%    n = len(L)
%    max = L[0]
%    for i in range(1,n) :
%        if L[i] > max :
%            max = L[i]
%    return max
%\end{lstlisting}
%\end{marginfigure}
%
%Soit la fonction de signature \lstinline{rechercheMaxi(L:[int]) -> int}.
%
%\question{Proposer un variant pour cet algorithme. Montrer que la boucle termine.}
%\ifprof
%\begin{corrige}
%$n-i$ est un variant.
%\end{corrige}
%\else
%\fi
%
%\question{Proposer un invariant pour cet algorithme. Montrer que cet algorithme est correct.}
%\ifprof
%\begin{corrige}
%Au début de la \ieme itération, \texttt{maxi} est la plus grande valeur de \texttt{L[:i]}. 
%\end{corrige}
%\else
%\fi

\subsection*{Exercice}

\begin{marginfigure}
\begin{lstlisting}
def rechercheMaxi(L) :
    n = len(L)
    max = L[0]
    i = 1
    while i < n :
        if L[i] > max :
            max = L[i]
        i = i+1
    return max
\end{lstlisting}
\end{marginfigure}

Soit la fonction de signature \lstinline{rechercheMaxi(L:[int]) -> int}.

\question{Proposer un variant pour cet algorithme. Montrer que la boucle termine.}
\ifprof
\begin{corrige}
$n-i$ est un variant.
\end{corrige}
\else
\fi

\question{Proposer un invariant pour cet algorithme. Montrer que cet algorithme est correct.}
\ifprof
\begin{corrige}
Au début de la \ieme itération, \texttt{maxi} est la plus grande valeur de \texttt{L[:i]}. 
\end{corrige}
\else
\fi

\subsection*{Exercice}

\begin{marginfigure}
\begin{lstlisting}
def x_in_L(x,L) :
    n = len(L)
    i = 0
    while i < n :
        if L[i] == x :
            return True
        i = i+1
    return False
\end{lstlisting}
\end{marginfigure}

Soit la fonction de signature \lstinline{x_in_L(x:int,L:[int]) -> bool}.

\question{Proposer un variant pour cet algorithme. Montrer que la boucle termine.}
\ifprof
\begin{corrige}
$n-i$ est un variant.
\end{corrige}
\else
\fi

\question{Proposer un invariant pour cet algorithme. Montrer que cet algorithme est correct.}
\ifprof
\begin{corrige}
Proposition d'invariant : << Au début de la \ieme itération, \texttt{x} n'appartient pas à \texttt{L[:i]}. >>
\end{corrige}
\else
\fi



\subsection*{Exercice}

On considère la fonction \texttt{mystere} suivante, qui étant donnés deux entiers \texttt{x} et \texttt{y}, renvoie un autre
entier.
\begin{lstlisting}
def mystere(x :int, y :int) -> int :
    a=0
    while y>0 :
        if y%2==1 :
            a=a+x
        x=x+x
        y=y//2
    return(a)
\end{lstlisting}

\question{Recopier et compléter les tableaux suivants donnant l’évolution des variables \texttt{x}, \texttt{y} et \texttt{a} ainsi
que de la quantité \texttt{a+x*y} lors des appels \texttt{f(7,20)} et \texttt{f(3,85)}.}
\ifprof
\begin{corrige}
\begin{center}
\begin{tabular}{l c c c c }
\hline
Pour \texttt{mystere(7,20)} & x  & y & a & $a+x*y$ \\
\hline
\hline
Fin du premier tour de boucle     &14 & 10 & 0 & 140 \\ \hline
Fin du deuxième tour de boucle  & 28 & 5 & 0 & 140\\ \hline
Fin du troisième tour de boucle   & 56 & 2 & 28 & 140\\ \hline
Fin du quatrième tour de boucle &  112 & 1 & 28 & 140\\ \hline
Fin du cinquième tour de boucle & 224 & 0 & 140 & 140\\ \hline
\end{tabular}
\end{center}
\end{corrige}
\else
\fi

\question{Montrer que la fonction \texttt{mystere} termine.}
\ifprof
\begin{corrige}
\begin{tabular}{l c c c c }
\hline
Pour \texttt{mystere(3,85)} & x  & y & a & $a+x*y$ \\
\hline
\hline
Fin du premier tour de boucle     & 6    & 42 & 3     & 255\\ \hline
Fin du deuxième tour de boucle & 12   & 21 &3      & 255 \\ \hline
Fin du troisième tour de boucle  & 24   & 10 & 15   & 255 \\ \hline
Fin du quatrième tour de boucle & 48   & 5 & 15     & 255 \\ \hline
Fin du cinquième tour de boucle & 96   & 2  & 63    & 255 \\ \hline
Fin du sixième tour de boucle     & 192  & 1 & 63    & 255 \\ \hline
Fin du septième tour de boucle   & 384 & 0 &  255 & 255 \\ \hline
\end{tabular}
\end{corrige}
\else
\fi

\question{Conjecturer la valeur de \texttt{mystere(x,y)} et démontrer cette conjecture à l’aide d’un invariant
de boucle bien choisi (on pourra appeler $x_i$ , $y_i$ et $a_i$ les valeurs respectives de $x$, $y$ et $a$
à l’entrée du \ieme  $i\geq 0$ tour de boucle).}
\ifprof
\begin{corrige}
On conjecture que \texttt{mystere(x,y)} renvoie $xy$.
Considérons l'invariant : << à l’entrée du i\ieme tour de boucle, $a_i+x_i y_i = x *y $>>.
\begin{itemize}
\item Pour$ i=0$, on a $a_0 = 0$, $x_0 = x$, $y_0 = y$ donc $a_0 + x_0 y_0 = xy$.
\item Soit  $i\geq 0$. On suppose qu’au rang $i$ , on a $a_i x_i y_i =xy$.
\begin{itemize}
\item Soit $y_i$  pair alors $x_{i+1}=x_i+x_i = 2x_i$, $y_{i+1}=\dfrac{y_i}{2}$ et $a_{i+1}=a_i +x_i$.
\begin{itemize}
\item Donc $a_{i+1}+x_{i+1}y_{i+1}=a_i+2x_i\dfrac{y_i}{2} = a_i +x_iy_i = xy$.
\end{itemize}
\item Soit $y_i$ impair alors $x_{i+1}=x_i + x_i = 2 x_i$, $y_{i+1}=\dfrac{y_i -1}{2}$ et $a_{i+1}=a_i x_i$.
\begin{itemize}
\item Donc $a_{i+1}+x_{i+1}y_{i+1}=a_i+x_i+2 x_i \dfrac{y_i}{2} = a_i +x_i+x_i\left( y_i -1 \right) = xy$.
\end{itemize}
\end{itemize}
\end{itemize}

Cela prouve l’invariant de boucle.
À la dernière étape, on a $a_{i-1}+x_{i-1}y_{i-1}=xy$ avec $y{i-1}=1$ donc $y_i = 0$, $a_i=a_{i-1}+x_{i-1}y_{i-1} = xy$ donc \texttt{mystere(x,y)} renvoie bien $x.y$ ce qui achève la correction de programme.

\end{corrige}
\else
\fi

\subsection*{Exercice}


Soit la fonction suivante où \texttt{L} est une fonction triée.
\begin{lstlisting}
def NombreDistinctsTri(L) :
    n=len(L)
    nombre=1
    for i in range(n-1) :
        if L[i]<L[i+1] :
            nombre=nombre+1
    return(nombre)
\end{lstlisting}

\question{Donner la signature de cette fonction.}

\question{Donner au moins une assertion permettant de valider que les entrées sont conformes aux attendus du concepteur.}

\question{Donner au moins un test permettant de valider la fonction.}

\question{Montrer que cette fonction termine.}

\question{Définir un invariant de la boucle constituant la fonction \texttt{NombreDistinctsTri(L)} et justifier la correction de cette fonction.}
\ifprof
\begin{corrige}
Un invariant de boucle est :  << à l’entrée du i\ieme tour, nombre contient le nombre d’éléments distincts dans \texttt{L[: i +1]} >>.
\begin{itemize}
\item Pour i = 0, nombre =1 = nombre d’éléments distincts dans L[: 1] qui est un singleton.
\item Supposons cet invariant à l’entrée du i\ieme tour. En sortie de ce tour :
\begin{itemize}
\item soit \texttt{L[i]=L[i+1]} alors nombre ne change pas et contient le nombre d’éléments distincts de \texttt{L[ :i+2]}.
\item soit \texttt{LL[i]=L[i+1]} alors nombre est incrémenté de 1 et contient le nombre d’éléments distincts de \texttt{L[ :i+2]}.
\end{itemize}
\item Donc en sortie du tout i, et donc à l’entrée du tour $i +1$, nombre contient le nombre d’éléments distincts de \texttt{L[ :i+2]}.
\item Au dernier tour, $i = n - 2$ . En entrée de boucle, nombre contient le nombre d’éléments distincts de \texttt{L[ :n-1]} et
nombre contient le nombre d’éléments distincts de \texttt{L[ :n]=L} ie le résultat cherché.
\end{itemize}
\end{corrige}
\else
\fi

\subsection*{Exercice}

Soit la fonction suivante réalisant un tri dit par insertion d'une liste. 
\begin{lstlisting}
def tri_par_selection(T):
    """trie le tableau T dans l'ordre croissant"""
    for i in range(len(T)):
        ind_min = i
        for j in range(i+1, len(T)):
            if T[j] < T[ind_min]:
                ind_min = j
        T[i],T[ind_min] = T[ind_min],T[i]
\end{lstlisting}

\question{Montrer que la propriété suivante est un invariant de boucle : au début de l'itération i, \texttt{T[0:i-1]} est trié et chacun des éléments de \texttt{T[0:i-1]} est inférieur ou égal aux éléments de \texttt{T[i:]}. }
\ifprof
\begin{corrige}

Voici l’invariant de boucle (du \lstinline{for i}) que l’on va utiliser pour prouver la correction :
\lstinline{T[0..i-1]} est trié et \lstinline{T[0..i-1]}$\leq$\lstinline{T[i..n-1]}.

Autrement dit : « la partie de gauche est triée et tous les éléments de la partie de
gauche déjà triés sont inférieurs à tous ceux de la partie de droite par encore triés ».

\paragraph*{Initialisation}
 Montrons que l’invariant est vrai avant l’entrée dans la boucle (\lstinline{for i}), donc qu’il est vrai lorsque \lstinline{i = 0}. Dans ce cas la partie de gauche est vide et est donc
triée (\lstinline{T[0..0-1]} est trié) et tous les éléments de la partie de droite sont supérieurs à ceux
de la partie de gauche puisque cette dernière est vide (\lstinline{T[0..0-1]}$\leq$\lstinline{T[0..n-1]}).

\paragraph*{Conservation}
 Montrons que l’invariant est conservé au cours d’une itération.
Supposons donc qu’au début de l’itération \lstinline{i}, on a \lstinline{T[0..i-1]} est trié et \lstinline{T[0..i-1]}$\leq$\lstinline{T[i..n-1]}.
Au cours de l’itération, \lstinline{T[i]} va être remplacé par l’élément minimum de \lstinline{T[i..n-1]}. Donc
\lstinline{T[i]} sera supérieur à T\lstinline{[0..i-1]} (d’après l’hypothèse) donc \lstinline{T[0..i]} est trié. De plus, \lstinline{T[i]} $\leq$
\lstinline{T[i+1..n-1]} donc on a \lstinline{T[0..i-1]}$\leq$T[i]$\leq$\lstinline{T[i+1..n-1]}et donc \lstinline{T[0..i]}$\leq$\lstinline{T[i+1..n-1]}.
L’invariant reste donc vrai après l’itération.

\paragraph*{Correction}
L’invariant reste en particulier vrai après la dernière itération, lorsque \lstinline{i= n-1}. On a donc en sortie de la boucle : \lstinline{T[0..n-1]} est trié (et \lstinline{T[0..n-1]}$\leq$\lstinline{T[n..n-1]} qui
est vraie puisque la partie de droite est vide). CQFD.

\paragraph*{Terminaison}
L’algorithme termine puisqu’il est composé de deux répétitives pour qui terminent
nécessairement.
\end{corrige}

\else
\fi


\subsection*{Exercice}

\question{Monter la terminaison et la correction de l'algorithme suviant.}

\begin{lstlisting}
def puiss(x, n):
    if n == 0:
        return 1
    else :
        return x*puiss(x,n-1)
\end{lstlisting}


