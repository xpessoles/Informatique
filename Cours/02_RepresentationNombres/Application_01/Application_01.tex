%\setchapterimage{fig_00.jpg}
\chapter*{Application \arabic{cptApplication} \\ 
\ifprof Corrigé \else Sujet \fi}
\addcontentsline{toc}{section}{Application \arabic{cptApplication} -- \ifprof Corrigé \else Sujet \fi}

\iflivret \stepcounter{cptApplication} \else
\ifprof  \stepcounter{cptApplication} \else \fi
\fi

\setcounter{question}{0}
%%\marginnote{Concours Centrale -- MP 2019}
%\marginnote[1cm]{
%\UPSTIcompetence[2]{C1-01}
%\UPSTIcompetence[2]{C2-03}
%}




\subsection*{Exercice 1}
\question{Réalisez la conversion des nombres suivants dans les autres systèmes de numération :
$(10050)_{(10)}$, $(1001 0001)_{(2)}$, $(A3F)_{16}$.}

\subsection*{Exercice 2}
\setcounter{question}{0}
On désire utiliser 12 bits pour comptabiliser des objets.

\question{Quel est le nombre maximum d'objets qu'il est possible de compter ?}

\question{Indiquer le numéro du premier et du dernier (dans les systèmes de numération décimale, binaire et hexadécimale).}

\subsection*{Exercice 3}
\setcounter{question}{0}
On désire compter 65000 objets. 

\question{Sur combien de bit peut-on réaliser cette opération ?}

\question{Quel est le premier et le dernier nombre (dans les systèmes de numération binaire et hexadécimale) ?}


\subsection*{Exercice 4}
\setcounter{question}{0}
Soit une machine où les nombres entiers sont codés sur 8 bits.

\question{Donner le plus grand et le plus petit nombre représentable selon que le codage utilisé est non signé ou signé.}

\question{Écrire dans les différents formats signés les nombres décimaux 1, $-1$, 111 et 55.}

\question{Quelles sont inversement les valeurs décimales codées par $4C$ et $B4$ suivant les différents codages signés et celui non signé.}
%\item Un périphérique de la machine lui délivre des données sur 8 bits dans le format signe et valeur absolue. Il transmet successivement $9A$ puis $3C$, quelles sont les valeurs signifiées ? 

\subsection*{Exercice 5}
\setcounter{question}{0}

\question{Effectuez les opérations arithmétiques suivantes dans les systèmes de numération binaire (codé sur 8 bits) : $71 + 35 =$, $15 - 25 =$, $121 - 75 =$, $-51 - 77 =$.}


\subsection*{Exercice 6}
\setcounter{question}{0}

\question{Écrire dans le format flottant simple précision (IEEE 754) les nombres 
1,0 ; 	$-1,0$; 	15,25  et $-3,26$. Les résultats seront donnés en hexadécimal.}


