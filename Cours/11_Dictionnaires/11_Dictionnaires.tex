% !TEX spellcheck = fr_FR
%\setchapterimage{.png}
\setchapterpreamble[u]{\margintoc}

\chapter{Dictionnaires}



\section{Introduction}

Les dictionnaires sont composés d'un nombre fini d'éléments auxquels on peut accéder par une clé qui fait partie de l'élément.
Chaque élément est donc une paire : une clé et une valeur (\textit{key} et \textit{value} en anglais).\\
Les dictionnaires ne sont pas ordonnés, on ne peut pas rechercher un élément à partir de sa position (indice) dans le dictionnaire mais seulement à partir de la clé.\\
Comme les listes, ce sont des objets \textit{itérables} car on peut parcourir leurs éléments à l'aide d'une boucle \lstinline{for}.


\section{Syntaxe}

\begin{itemize}
\item Un dictionnaire python est une succession de paires d'objets séparées par une virgule, délimité par des accolades \{ et \}.
\item Une paire est composée d'une clé et d'une valeur.% qui sont des objets de type quelconque (\lstinline{integer}, \lstinline{float}, \lstinline{string}, \lstinline{list}, \lstinline{tuple}, \lstinline{boolean}).
\item Un dictionnaire est dit \textbf{mutable} c'est-à-dire que l'on peut en modifier (voire en supprimer) un ou plusieurs éléments.
\item Les clés du dictionnaire doivent être non mutable (les types \lstinline{list} et \lstinline{dict} sont interdits).
\item Les valeurs du dictionnaire peuvent être des objets de type quelconque (\lstinline{integer}, \lstinline{float}, \lstinline{string}, \lstinline{list}, \lstinline{tuple}, \lstinline{boolean}).
\end{itemize}

\begin{center}
\texttt{dico=\{cle1 : valeur1, cle2 : valeur2, ... , clen : valeurn\}}
\end{center}

\begin{exemple}
\begin{itemize}
\item \texttt{velo={'guidon': 1, 'roue': 2, 'derailleur': 21, 'frein': 2}}
\item \texttt{pikachu={'pokemon':'souris', 'taille': 0.4, 'poids': 6, 'type':'electrik', 'talent':['statik','paratonnerre']}}
%\item \lstinline{altitudeRhone=\{(600,700):\verb![!'Joncin'\verb!]!,(700,800):\verb![!'Luere','Pin-Bouchain','Pilon'\verb!]!,(800,900):\verb![!'Brosses','Croix-Casard'\verb!]!\}}
\end{itemize}
\end{exemple}

\section{Manipulation des dictionnaires}
Le dictionnaire vide est désigné par \texttt{\{\}}.
\subsection*{Création}
La création d'un dictionnaire se fait en choisissant un nom et par les signes d'accolades
\texttt{mon\_dico=\{\}}.

\subsection*{Taille}
La taille du dictionnaire est donnée par la fonction prédéfinie \lstinline{length} : 
\lstinline{len(mon_dico)}.


\subsection*{Ajout d'un élément ou modification d'une valeur}
L'ajout d'un élément \lstinline{clé:valeur} ou la modification d'une valeur si la clé correspondante existe ont la même syntaxe : \lstinline{mon_dico['classe']='PTSI'}. 
%\textit{Remarque} : La clé n'est pas modifiable (non mutable) alors que la valeur l'est.

\subsection*{Suppression d'un élément}
La suppression d'un élément du dictionnaire est réalisée par la fonction \lstinline{del} en précisant la clé de l'élément à supprimer. La paire \lstinline{clé:valeur} est alors supprimée.
\lstinline{del(mon_dico['classe'])}


\subsection*{Lecture d'une valeur}

\begin{lstlisting}
pikachu['taille'] # renvoie la valeur associée à la clé 'taille' du dictionnaire pikachu.
\end{lstlisting}
%Cette instruction peut être associée à \lstinline{print} pour afficher la valeur ou affectée à une variable.

\subsection*{Parcours du dictionnaire}
On peut parcourir un dictionnaire par ses clés, ses valeurs ou ses éléments \lstinline{clé:valeur}. Pour parcourir la totalité du dictionnaire, on utilise une boucle bornée \lstinline{for}.

\marginnote{\texttt{for cle in velo} fonctionne aussi.}
\begin{center}
\begin{tabular}{lp{7cm}}
\hline \textbf{Instruction} & \textbf{Effet} \\
\hline
\texttt{for cle in velo.keys():} & Parcours des clés du dictionnaire \\
\texttt {for valeur in velo.values():} & Parcours des valeurs du dictionnaire\\
\texttt {for cle,valeur in velo.items():} & Parcours des éléments du dictionnaire\\
\hline
\end{tabular}
\end{center}


\subsection*{Liste des clés ou liste des valeurs}
On peut récupérer les différentes clés ou les différentes valeurs du dictionnaire sous forme de liste.

\begin{center}
\begin{tabular}{lp{7cm}}
\hline \textbf{Instruction} & \textbf{Effet} \\
\hline
\texttt{LesCles=list(velo.keys())} & \texttt{['guidon','roue','derailleur','frein']} est affecté à \texttt{LesCles}\\

\texttt{LesValeurs=list(velo.values())} & \texttt{[1, 2, 21, 2]} est affecté à \texttt{LesValeurs}\\
\hline
\end{tabular}
\end{center}


\subsection*{Vérification d'une clé}
Pour vérifier qu'une clé existe ou non dans un dictionnaire, on utilise le terme d'appartenance \lstinline{in}.
\sidenote{\texttt{'guidon' in velo} fonctionne aussi.}
\begin{center}
\begin{tabular}{lp{7cm}}
\hline \textbf{Instruction} & \textbf{Effet} \\
\hline
\texttt{'guidon' in velo.keys()} & renvoie le booléen \texttt{True} si la clé 'guidon' est dans \texttt{velo}, \texttt{False} sinon\\

\texttt{'guidon' not in velo.keys()} & renvoie le booléen \texttt{False} si la clé 'guidon' est dans \texttt{velo}, \texttt{True} sinon\\
\hline
\end{tabular}
\end{center}
