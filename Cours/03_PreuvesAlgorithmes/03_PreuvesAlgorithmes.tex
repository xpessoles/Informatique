% !TEX spellcheck = fr_FR
%\setchapterimage{.png}
\setchapterpreamble[u]{\margintoc}

\chapter{Anlayse des algorithmes}

\marginnote[8cm]{
\begin{itemize}
\item Terminaison.
\item Correction partielle. 
\item Correction totale.
\item Variant. Invariant.
\item Complexité.
\end{itemize}}




\section{Anlayse des algorithmes}

\subsection{Définitions}

\begin{defi}[Terminaison d'un algorithme]
Prouver la terminaison d'un algorithme signifie montrer que cet algorithme se terminera en un temps fini. On utilise pour cela un \textbf{variant de boucle}.
\end{defi}

\begin{defi}[Correction d'un algorithme]
Un algorithme est dit (partiellement) correct s'il est correct dès qu'il termine. 

\end{defi}
Prouver la correction d'un algorithme signifie montrer que cet algorithme fournit bien la solution au problème qu'il est sensé résoudre. On utilise pour cela un \textbf{invariant de boucle}.

\begin{defi}[Invariant de boucle]
Un invariant de boucle est une propriété dépendant des variables de l’algorithme, qui est vérifiée à
chaque passage dans la boucle.
\end{defi}
%\begin{defi}\textbf{Analyser} \\
%Prouver la correction d'un algorithme signifie montrer que cet algorithme fournit bien la solution au problème qu'il est sensé résoudre. On utilise pour cela un \textbf{invariant de boucle}.
%\end{defi}


\begin{defi}[Complexité]
  La complexité algorithmique est l'étude des ressources requises pour exécuter un algorithme, en fonction d'un paramètre (souvent, la taille des données d'entrée). Les deux ressources en général étudiées sont :
\begin{itemize}
\item le temps nécessaire à l'exécution de l'algorithme;
\item la mémoire nécessaire à l'exécution de l'algorithme (en plus des données d'entrée).
\end{itemize}
\end{defi}

\subsection{Un exemple ...}
\begin{obj}

L'objectif est ici de montrer la nécessité d'utiliser un invariant de boucle.

Pour cela, on propose la fonction suivante sensée déterminer le plus petit entier \texttt{n} strictement positif tel que $1 + 2 + . . . + n$ dépasse strictement la valeur entière strictement positive \texttt{v}. Cette fonction renvoie-t-elle le bon résultat ? Desfois ? Toujours ?
\end{obj}
\begin{lstlisting}
def foo(v:int) -> int:
    r = 0
    n = 0
    while r < v : 
        n = n+1
        r = r+n
    return n
\end{lstlisting}

\textit{Montrer intuitivement que \texttt{foo(v)} se termine pour $v\in\mathbb{N}^*$.}

L'algorithme se terminera si on sort de la boucle \texttt{while}. Il faut pour cela que la condition \texttt{r<v} devienne fausse (cette condition est vraie initialement). Pour cela, il faut que \texttt{r} devienne supérieure ou égale à \texttt{v} dont la valeur ne change jamais. 

\texttt{n} étant incrémenté de 1 à chaque itération, la valeur de \texttt{r} augmente donc à chaque itération. Il y aura donc un rang \texttt{n} au-delà duquel \texttt{r} sera supérieur à \texttt{v}. L'algorithme se termine donc. 

\bigskip

\textit{Que renvoie \texttt{foo(9)} ? Cela répond-il au besoin ?}

\begin{center}
\begin{tabular}{llll}
\hline
Début de la i\ieme itération & \texttt{r} & \texttt{n} & \texttt{r < v} \\ \hline 
Itération 1 & \texttt{0} & \texttt{0} & \texttt{0 < 9  } $\Rightarrow$ \texttt{  True} \\    %\hline
Itération 2& \texttt{1} & \texttt{1} & \texttt{1 < 9  } $\Rightarrow$ \texttt{  True} \\     %\hline
Itération 3 & \texttt{3} & \texttt{2} & \texttt{3 < 9  } $\Rightarrow$ \texttt{  True} \\    %\hline
Itération 4 & \texttt{6} & \texttt{3} & \texttt{6 < 9  } $\Rightarrow$ \texttt{  True} \\    %\hline
Itération 5 & \texttt{10} & \texttt{4} & \texttt{10< 9  } $\Rightarrow$ \texttt{  False} \\ \hline
\end{tabular}
\end{center}

La fonction renvoie 4. On a $1+2+3+4 = 10$. On dépasse strictement la valeur 9. La fonction répond au besoin dans ce cas. 

\bigskip

\textit{Que renvoie \texttt{foo(10)} ? Cela répond-il au besoin ?}

\begin{center}
\begin{tabular}{llll}
\hline
Début de la i\ieme itération & \texttt{r} & \texttt{n} & \texttt{r  < v} \\ \hline %\hline
Itération 1 & \texttt{0} & \texttt{0} & \texttt{0  < 10  } $\Rightarrow$ \texttt{  True} \\    %\hline
Itération 2& \texttt{1} & \texttt{1} & \texttt{1  < 10  } $\Rightarrow$ \texttt{  True} \\     %\hline
Itération 3 & \texttt{3} & \texttt{2} & \texttt{3  < 10  } $\Rightarrow$ \texttt{  True} \\    %\hline
Itération 4 & \texttt{6} & \texttt{3} & \texttt{6  < 10  } $\Rightarrow$ \texttt{  True} \\    %\hline
Itération 5 & \texttt{10} & \texttt{4} & \texttt{10 < 10  } $\Rightarrow$ \texttt{  False} \\ \hline
\end{tabular}
\end{center}
La fonction renvoie 4. On a $1+2+3+4 = 10$. On ne dépasse pas strictement la valeur 10. La fonction ne répond pas au besoin dans ce cas. 

\bigskip

\begin{resultat}
La fonction proposée ne remplit pas le cahier des charges. Aurait-on pu le prouver formellement ?
\end{resultat}

\section{Terminaison d'un algorithme}

\subsection{Variant de boucle}
\begin{defi}[Variant de boucle]
Un variant de boucle permet de prouver la terminaison d'une boucle conditionnelle.  Un variant de boucle est une \textbf{quantité entière positive} à l’entrée de chaque
itération de la boucle et qui \textbf{diminue strictement à chaque itération}.
\end{defi}

\begin{theoreme}
Si une boucle admet un variant de boucle, elle termine.
\end{theoreme}

\begin{prop}
Un algorithme qui n’utilise ni boucles inconditionnelles (boucle \texttt{for}) ni récursivité termine toujours. Ainsi,
la question de la terminaison n’est à considérer que dans ces deux cas.
\end{prop}

Reprenons l'exemple précédent. 
\begin{marginfigure}
\begin{lstlisting}
def foo(v:int) -> int:
    r = 0
    n = 0
    while r < v : 
        n = n+1
        r = r+n
    return n
\end{lstlisting}
\end{marginfigure}


Dans cet exemple montrons que la quantité $u_n = v-r$ est un variant de boucle : 
\begin{itemize}
\item initialement, $r=0$ et $v>0$; donc  $u_0 > 0$;
\item à la fin de l'itération $n$, on suppose que $u_n = v-r >0$ et que $u_n < u_{n-1}$;
\item à l'itération $n+1$ : 
\begin{itemize}
\item cas 1 : $r\geq v$. Dans ce cas, $n$ et $r$ n'évoluent pas l'hypothèse de récurrence reste vraie. On sort de la boucle \texttt{while}. L'algorithme termine,
\item cas 2 : $r < v$. Dans ce cas, à la fin de l'itération $n+1$,  montrons que  $u_{n+1} < u_{n}$ : $u_{n+1} = v - (r + n + 1) =  u_n -n - 1$ soit $u_{n+1} =u_n -n - 1$ et donc $u_{n+1} < u_{n}$. L'hypothèse de récurrence est donc vraie au rang $n+1$. 
\end{itemize}
\end{itemize}

Au final, $u_n = v-r$ est donc un variant de boucle et la boucle se termine.


\subsection{Un second exemple ressemblant...\sidenote{\url{https://marcdefalco.github.io/pdf/complet_python.pdf}}} 




Considérons l’algorithme suivant qui, étant donné un entier naturel $n$ strictement positif (inférieur à $2^{30}$), détermine le plus petit entier $k$ tel que $n \leq 2^k$.
\begin{lstlisting}
def plus_grande_puissance2(n):
    k = 0
    p = 1
    while p < n:
        k = k+1
        p = p*2
    return k
\end{lstlisting}
%\begin{demo}%
\textbf{[1]} $\quad$
Dans l’exemple précédent, la quantité $n - p$ est un variant de boucle :
\begin{itemize}
\item au départ, $n > 0$ et $p = 1$ donc $n - p \geq 0$;
\item comme il s’agit d’une différence de deux entiers, c’est un entier. Et tant que la condition
de boucle est vérifiée $p < n$ donc $n - p > 0$.
\item lorsqu’on passe d’une itération à la suivante, la quantité passe de $n-p$ à $n-2p$ or $2p-p > 0$
car $p \geq 1$. Il y a bien une stricte diminution.
\end{itemize}
%\end{demo}

%\begin{demo}
\textbf{[2]} $\quad$
Montrons que, la quantité $u_j = n - p$ est un variant de boucle :
\begin{itemize}
\item intialement, $n > 0$ et $p = 1$ donc $n - p \geq 0$;
\item à la fin de l'itération $j$, on suppose que $u_j = n - p >0 $ et$u_j < u_{j-1}$;
\item à la fin de l'itération suivante, $u_{j+1}=n-2p = u_j - p$. $p$ est positif donc $u_{j+1}$ est un entier et $u_{j+1}<u_j$. Par suite, ou bien $u_{j+1}<0$ c'est à dire que $n-p<0$ soit $p>n$. On sort donc de la boucle. Ou bien, $u_{j+1}>0$, et la boucle continue.
\end{itemize}
$n - p$ est donc un variant de boucle.
%\end{demo}

\section{Correction d'un algorithme}
\subsection{Invariant de boucle}

\begin{methode}~\\
Pour montrer qu'une propriété est un invariant de boucle dans une boucle \texttt{while} :
\begin{itemize}
\item le propriété doit être vérifiée avant d'entrer dans la boucle;
\item la propriété doit être vraie en entrée de boucle;
\item la propriété doit être vraie en fin de boucle.
\end{itemize}
\end{methode}
%
%\begin{defi}{Invariant de boucle}
%Soit une boucle. Une propriété est appelée un invariant de boucle lorsque :
%\begin{itemize}
%\item cette propriété est vérifiée avant d'entrer dans la boucle;
%\item si cette propriété est vérifiée en entrée d'itération, alors elle est vérifiée en sortie de l'itération.
%\end{itemize}
%\end{defi}


Reprenons un des exemples précédents. Reconsidérons l’algorithme suivant qui, étant donné un entier naturel $n$ strictement positif (inférieur à $2^{30}$), détermine le plus petit entier $k$ tel que $n \leq 2^k$.

\begin{marginfigure}
\begin{lstlisting}
def plus_grande_puissance2(n):
    k = 0
    p = 1
    while p < n:
        k = k+1
        p = p*2
    return k
\end{lstlisting}
\end{marginfigure}


%\begin{demo}
Montrons que la propriété suivante est un invariant de boucle : $p=2^k$ et $2^{k-1}<n$.
\begin{itemize}
\item \textbf{Initialisation : }à l'entrée dans la boucle $k=0$ et $p=1$, $n\in\mathbb{N}^*$
\begin{itemize}
\item d'une part  on a bien $1=2^0$;
\item d'autre part $2^{-1}<n$.
\end{itemize}
\item On considère que la propriété est vraie au $n$\ieme tour de boucle c'est à dire $p=2^k$ et $2^{k-1}<n$.
\item Au tour de boucle suivant : 
\begin{itemize}
\item \textbf{ou bien} $p>=n$. Dans ce cas, on sort de la boucle et on a toujours $p=2^k$ et $2^{k-1}<n$ (propriété d'invariance). La propriété est donc vraie au tour $n+1$.
\item \textbf{ou bien} $p<n$. Dans ce cas, il faut montrer que  $p=2^{k+1}$ et $2^{k}<n$. Etant entrés dans la boucle, $p<n \Rightarrow 2^k<n$. De plus, en fin de boucle, $p\rightarrow p *2$ et $k\rightarrow k+1$. On a donc $p\leftarrow 2^k *2=2^{k+1}$. 
\end{itemize}
\end{itemize}
La propriété citée est donc un invariant de boucle. 
%\end{demo}

\subsection{Un << contre exemple >>}


\begin{marginfigure}
\begin{lstlisting}
def foo(v:int) -> int:
    assert v>0
    r = 0
    n = 0
    while r <= v : 
        n = n+1
        r = r+n
    return n
\end{lstlisting}
\end{marginfigure}

Reprenons le tout premier exemple où on cherche le plus petit entier \texttt{n} strictement positif tel que $1 + 2 + . . . + n$ dépasse strictement la valeur entière strictement positive \texttt{v}.



La propriété suivante est-elle un invariant de boucle : 
$r=\sum\limits_{i=0}^n i$ et 
$\sum\limits_{i=0}^{n-1} i \leq v$, 
$n\in\mathbb{N}^*$ ?

La réponse est directement NON, car la phase d'initialisation n'est pas vérifiée car $n=0$ et $n\notin\mathbb{N}^*$.
Cela signifie donc que l'algorithme proposé ne répond pas au cahier des charges. 

Modifions alors l'algorithme ainsi.
\begin{marginfigure}
\begin{lstlisting}
def foo2(v:int) -> int:
    assert v>0
    r = 1
    n = 1
    while r <= v : 
        n = n+1
        r = r+n
    return n
\end{lstlisting}
\end{marginfigure}

Montrons que la propriété suivante est un invariant de boucle : $r=\sum\limits_{i=0}^n i$ et 
$\sum\limits_{i=0}^{n-1} i \leq v$, 
$n\in\mathbb{N}^*$.

\begin{itemize}
\item \textbf{Initialisation : }à l'entrée dans la boucle $r=1$ et $n=1$, $n\in\mathbb{N}^*$
\begin{itemize}
\item d'une part  on a bien $r=\sum\limits_{i=0}^1 i = 1$;
\item d'autre part$  \sum\limits_{i=0}^{0}i =0 < v$ et $v>0$ (spécification de la fonction).
\end{itemize}
\item On considère que la propriété est vraie au début du $n$\ieme  tour de boucle c'est-à-dire $r=\sum\limits_{i=0}^n i$ et $\sum\limits_{i=0}^{n-1} i \leq v$.
\item \'A la fin du $n$\ieme  tour de boucle, $n_{n+1} = n_n+1$ et 
$r_{n+1} = r_n + n_{n+1}=r_n+n_n+1 = \sum\limits_{i=0}^n \left(i\right)+n_n+1  = \sum\limits_{i=0}^{n+1} i$ ( car $n_n = n$). 
On a alors, 
\begin{itemize}
\item ou bien $r_{n+1}>v$ et on sort de la boucle; on peut renvoyer $n$.
\item ou bien $r_{n+1}\leq v$ et donc $\sum\limits_{i=0}^{n} i \leq v$. 
\end{itemize}
%
%\item Au tour de boucle suivant : 
%\begin{itemize}
%\item \textbf{ou bien} $r\geq v$ et la propriété vraie au rang $n$ reste vraie;
%\item \textbf{ou bien} $r < v$. À la fin de l'itération, on a donc $r\leftarrow r+n+1 $ et $n\leftarrow n+1 +1$.
%Or $r=\sum\limits_{i=0}^n i$; donc $r\leftarrow \sum\limits_{i=0}^{n+1} i$. Par ailleurs, étant entrés dans la boucle, 
%$r=\sum\limits_{i=0}^n i < v$. Les deux assertions sont donc vraies au rang $n+1$. 
%%$p<n$. Dans ce cas, il faut montrer que  $p=2^{k+1}$ et $2^{k}<n$. Etant entrés dans la boucle, $p<n \Rightarrow 2^k<n$. De plus, en fin de boucle, $p\leftarrow p *2$ et $k\leftarrow k+1$. On a donc $p\leftarrow 2^k *2=2^{k+1}$. 
%\end{itemize}
\end{itemize}
La propriété citée est donc un invariant de boucle. 
%
%\subsection{Correction partielle -- Correction totale}
%\begin{defi}{Correction partielle -- Correction totale}
%La correction est partielle quand le résultat est correct lorsque l'algorithme s'arrête, la correction est totale si elle est partielle et si l'algorithme termine.
%\end{defi}
