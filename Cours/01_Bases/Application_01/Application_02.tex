%\setchapterimage{fig_00.jpg}
\chapter*{Application \arabic{cptApplication} \\ 
\ifprof Corrigé \else Sujet \fi}
\addcontentsline{toc}{section}{Application \arabic{cptApplication} -- \ifprof Corrigé \else Sujet \fi}

\iflivret \stepcounter{cptApplication} \else
\ifprof  \stepcounter{cptApplication} \else \fi
\fi

\setcounter{question}{0}
%%\marginnote{Concours Centrale -- MP 2019}
%\marginnote[1cm]{
%\UPSTIcompetence[2]{C1-01}
%\UPSTIcompetence[2]{C2-03}
%}





\subsection*{Exercice -- Structures conditionnelles}

\question{Implémenter une fonction \lstinline{est_plus_grand(a:int,b:int)-> bool} renvoyant \texttt{True} si $a>b$, \texttt{False} sinon. }
\ifprof
\begin{corrige}
\begin{lstlisting}
def est_plus_grand(a:int,b:int)-> bool : 
    res = True
    if a>b : 
        res = True
    else :
        res = False
    return res

def est_plus_grand(a:int,b:int)-> bool : 
    if a>b : 
        return True
    else :
        return False

def est_plus_grand(a:int,b:int)-> bool : 
    return a>b
\end{lstlisting}
\end{corrige}
\else
\fi
\question{\'Ecrire une fonction \lstinline{neg(b)} qui renvoie la négation du booléen \texttt{b} sans utiliser \texttt{not}.}
\ifprof
\begin{corrige}
\begin{lstlisting}
def neg(b) :
    res = True
    if b == True :
        res = False
    else : 
        res = True
    return res
\end{lstlisting}
\end{corrige}
\else
\fi

\question{\'Ecrire une fonction \lstinline{ou(a,b)} qui renvoie le ou logique des booléen \texttt{a} et \texttt{b} sans utiliser \texttt{not}, \texttt{or} ni \texttt{and}.}
\ifprof
\begin{corrige}
\begin{lstlisting}
def ou(a,b):
    if a == True :
        return True
    if b == True :
        return True
    return False
\end{lstlisting}
\end{corrige}
\else
\fi

\question{\'Ecrire une fonction \lstinline{et(a,b)} qui renvoie le et logique des booléen \texttt{a} et \texttt{b} sans utiliser \texttt{not}, \texttt{or} ni \texttt{and}.}
\ifprof
\begin{corrige}
\begin{lstlisting}
def et(a,b):
    if a == True :
        if b == True :
            return True
    return False
\end{lstlisting}
\end{corrige}
\else
\fi

%\subsection*{Implémentations de fonctions prenant des listes en argument}

\subsection*{Exercice -- Un tout petit peu d'arithmétique}

\question{Implémenter la fonction \lstinline{unite(n:int)->int} renvoyant le chiffre des unités de l'entier \texttt{n}.}

\begin{lstlisting}
>>> unite(123)
	3
\end{lstlisting}
\ifprof
\begin{corrige}
\begin{lstlisting}
def unite(n:int)->int :
    return n%10
\end{lstlisting}
\end{corrige}
\else
\fi

\question{Implémenter la fonction \lstinline{dizaine(n:int)->int} renvoyant le chiffre des dizaines de l'entier \texttt{n}.}

\begin{lstlisting}
>>> dizaine(123)
	2
\end{lstlisting}
\ifprof
\begin{corrige}
\begin{lstlisting}
def dizaine(n:int)->int :
    # On récupere le nombre de dizaine grace à une division entiere
    nb = n//10
    return nb%10
\end{lstlisting}
\end{corrige}
\else
\fi

\question{Implémenter la fonction  \lstinline{unites_base8(n:int)->int} renvoyant le chiffre des unités de l'entier \texttt{n} en base 8.}
\ifprof
\begin{corrige}
\begin{lstlisting}
def unites_base8(n:int)->int :
    return n%8
\end{lstlisting}
\end{corrige}
\else
\fi

%\subsection*{Exercice -- Liste de zéros}
%
%\question{Implémenter la fonction \texttt{zeros\_01(n:int)->list} permettant de générer une liste de \texttt{n 0}. On utilisera une boucle \texttt{while}.}
%
%\begin{lstlisting}
%>>> zeros_01(4)
%	[0,0,0,0]
%\end{lstlisting}
%
%\question{Implémenter la fonction \texttt{zeros\_02(n:int)->list} permettant de générer une liste de \texttt{n 0}. On utilisera une boucle \texttt{for}.}
%
%\question{Implémenter la fonction \texttt{zeros\_03(n:int)->list} permettant de générer une liste de \texttt{n 0}. On utilisera une boucle \texttt{while}. On génèrera cette liste <<~en compréhension~>> (sans boucle \texttt{for} ou \texttt{while} explicite).}



\subsection*{Exercice -- Structures itératives}

\question{\'Ecrire la fonction \lstinline{somme_inverse(n:int)->float} calculant la somme des inverses des \texttt{n} premiers entiers non nuls (\texttt{n} exclus).}
\ifprof
\begin{corrige}
\begin{lstlisting}
def somme_inverse(n:int)->float :
    res = 0
    for i in range(1,n) : 
        res = res+1/i
    return res
\end{lstlisting}
\end{corrige}
\else
\fi


%\subsection*{Exercice -- Un petit peu de géométrie}
%
%\question{Implémenter la fonction \texttt{norme(A:list,B:list)->float} permettant calculer la norme du vecteur $\vect{AB}$ dans $\mathbb{R}^3$. Chacun des points sera constitué de la liste de ses coordonnées (par exemple \texttt{A=[xA,yA,zA]}).}
%
%\question{Implémenter la fonction \texttt{prod\_vect(u:list, v:list)->float} permettant calculer le produit vectoriel $\vect{u}\wedge \vect{v}$ dans $\mathbb{R}^3$.}



\subsection*{Exercice -- Suites d'entiers}

\question{Implémenter la fonction \lstinline{impairs(n:int)} permettant d'afficher la liste des \texttt{n} premiers entiers naturels impairs (\texttt{n} exclus).}
\ifprof
\begin{corrige}
\begin{lstlisting}
def impairs(n:int)->float :
    res = 0
    for i in range(1,n) : 
        if i%2 == 1 : 
            res = res + i
    return res

def impairs(n:int)->float :
    res = 0
    for i in range(1, n, 2) : 
        res = res + i
    return res
\end{lstlisting}
\end{corrige}
\else
\fi

\question{Implémenter la fonction \lstinline{multiples_5(d:int, f:int)} permettant d'afficher la liste de tous les multiples de 5 compris entre \texttt{d} et \texttt{f} (bornes incluses si ce sont des multiples de 5).}
\ifprof
\begin{corrige}
\begin{lstlisting}
def multiples_5(d:int, f:int) :
    for i in range(d,f+1):
        if i%5 == 0 : 
            print(i)
\end{lstlisting}
\end{corrige}
\else
\fi

\question{Implémenter la fonction \lstinline{cube(f:int)} permettant d'écire la liste de tous les cubes d'entiers naturels inférieurs ou égaux à \texttt{f} (inclus si \texttt{f} est un cube).}
\ifprof
\begin{corrige}
\begin{lstlisting}
def cube(f:int) :
    i = 0
    while i*i*i <=f :
        print(i*i*i)
\end{lstlisting}
\end{corrige}
\else
\fi

%
%\subsection*{Exercice -- Un petit peu de statistiques}
%
%\question{\'Ecrire une fonction \texttt{moy\_extr(L:list)->float} qui prend en argument une liste \texttt{L} et renvoie en sortie la moyenne du premier et du dernier élément de \texttt{L}.}
%
%\question{\'Ecrire une fonction \texttt{moyenne(L:list)->float} qui prend en argument une liste \texttt{L} et renvoie la moyenne des éléments de \texttt{L}.}
%
%\question{\'Ecrire une fonction \texttt{ecart\_type(L:list)->float} qui prend en argument une liste \texttt{L} et renvoie l'éacrt type des données : en notant $\overline{x}$ la moyenne des échantillons, on a  $\sigma = \sqrt{\dfrac{1}{n}\sum\limits_{i=1}^n \left(x_i^2  - \overline{x}\right)}$.}

