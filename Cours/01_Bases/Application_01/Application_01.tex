%\setchapterimage{fig_00.jpg}
\chapter*{Application \arabic{cptApplication} \\ 
\ifprof Corrigé \else Sujet \fi}
\addcontentsline{toc}{section}{Application \arabic{cptApplication} -- \ifprof Corrigé \else Sujet \fi}

\iflivret \stepcounter{cptApplication} \else
\ifprof  \stepcounter{cptApplication} \else \fi
\fi

\setcounter{question}{0}
%%\marginnote{Concours Centrale -- MP 2019}
%\marginnote[1cm]{
%\UPSTIcompetence[2]{C1-01}
%\UPSTIcompetence[2]{C2-03}
%}



\section*{Variables, types, affectations}
\question{Dans chaque cas, indiquer le type que vous utiliseriez pour modéliser les grandeurs suivantes :
\begin{enumerate}
\item le nombre de coté d'un polygones;
\item le nombre $\pi$;
\item le résultat d'un test d'égalité;
\item le nombre de pays dans le monde;
\item la vitesse moyenne d'un véhicule en \si{m.s^{-1}};
\item le résultat d'un test d'une inégalité;
\item le nombre de secondes dans une journée.
\end{enumerate}}
\ifprof
\begin{corrige}
\begin{enumerate}
\item le nombre de coté d'un polygones : \lstinline{int};
\item le nombre $\pi$ \lstinline{float};
\item le résultat d'un test d'égalité \lstinline{bool};
\item le nombre de pays dans le monde \lstinline{int};
\item la vitesse moyenne d'un véhicule en \si{m.s^{-1}} \lstinline{float};
\item le résultat d'un test d'une inégalité \lstinline{bool};
\item le nombre de secondes dans une journée \lstinline{int}.
\end{enumerate}
\end{corrige}
\else\fi

\section*{Fonctions}
\question{Ecrire une fonction \lstinline{perimetre_rectangle(a:float, b:float) -> float} qui calcule le périmètre d'un rectangle.}
\ifprof
\begin{corrige}
\begin{lstlisting}
def perimetre_rectangle(a:float, b:float) -> float : 
    return 2*(a+b)
\end{lstlisting}
\end{corrige}
\else
\fi

\question{Ecrire une fonction \lstinline{aire_rectangle(a:float,b:float) -> float} qui calcule l'aire d'un rectangle.}
\ifprof
\begin{corrige}
\begin{lstlisting}
def aire_rectangle(a:float,b:float) -> float :
    return a*b
\end{lstlisting}
\end{corrige}
\else
\fi

\question{Ecrire une fonction \lstinline{perimetre_cercle(R:float) -> float} qui calcule le périmètre d'un cercle.}
\ifprof
\begin{corrige}
\begin{lstlisting}
import math as m
def perimetre_cercle(R:float) -> float :
    return 2*m.pi*R
\end{lstlisting}
\end{corrige}
\else
\fi

\question{Ecrire une fonction \lstinline{aire_disque(R:float) -> float} qui calcule l'aire d'un disque.}
\ifprof
\begin{corrige}
\begin{lstlisting}
def aire_disque(R:float) -> float :
    return m.pi*R**2
\end{lstlisting}
\end{corrige}
\else
\fi

\question{Ecrire une fonction qui calcule l'aire d'un trapèze de hauteur $h$, de petite base $b$ et de grande base $B$.}
\ifprof
\begin{corrige}
\begin{lstlisting}
def aire_trapeze(h, b, B) -> float :
    return 0.5 * (b+B)*h
\end{lstlisting}
\end{corrige}
\else
\fi

\question{Ecrire une suite d'instructions qui permet de permuter les valeurs de deux variables \lstinline{a} et \lstinline{b}.}
\ifprof
\begin{corrige}
\begin{lstlisting}
a = 1
b = 2
c = a
a = b 
b = c
\end{lstlisting}
\end{corrige}
\else
\fi
