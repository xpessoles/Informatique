%\setchapterimage{fig_00.jpg}
\chapter*{Application \arabic{cptApplication} \\ 
\ifprof Corrigé \else Sujet \fi}
\addcontentsline{toc}{section}{Application \arabic{cptApplication} -- \ifprof Corrigé \else Sujet \fi}

\iflivret \stepcounter{cptApplication} \else
\ifprof  \stepcounter{cptApplication} \else \fi
\fi

\setcounter{question}{0}
%%\marginnote{Concours Centrale -- MP 2019}
%\marginnote[1cm]{
%\UPSTIcompetence[2]{C1-01}
%\UPSTIcompetence[2]{C2-03}
%}



\section*{Initiation à l'implémentations de fonctions}


\subsection*{Exercice -- Structures conditionnelles}

\question{Implémenter une fonction \texttt{est\_plus\_grand(a:int,b:int)-> bool} renvoyant \texttt{True} si $a>b$, \texttt{False} sinon. }



\question{\'Ecrire une fonction \texttt{neg(b)} qui renvoie la négation du booléen \texttt{b} sans utiliser \texttt{not}.}

\question{\'Ecrire une fonction \texttt{ou(a,b)} qui renvoie le ou logique des booléen \texttt{a} et \texttt{b} sans utiliser \texttt{not}, \texttt{or} ni \texttt{and}.}

\question{\'Ecrire une fonction \texttt{et(a,b)} qui renvoie le et logique des booléen \texttt{a} et \texttt{b} sans utiliser \texttt{not}, \texttt{or} ni \texttt{and}.}

%\subsection*{Implémentations de fonctions prenant des listes en argument}

\subsection*{Exercice -- Un tout petit peu d'arithmétique}

\question{Implémenter la fonction \texttt{unite(n:int)->int} renvoyant le chiffre des unités de l'entier \texttt{n}.}


\begin{lstlisting}
>>> unite(123)
	3
\end{lstlisting}

\question{Implémenter la fonction \texttt{dizaine(n:int)->int} renvoyant le chiffre des dizaines de l'entier \texttt{n}.}

\begin{lstlisting}
>>> dizaine(123)
	2
\end{lstlisting}

\question{Implémenter la fonction  \texttt{unites\_base8(n:int)->int} renvoyant le chiffre des unités de l'entier \texttt{n} en base 8.}

\subsection*{Exercice -- Liste de zéros}

\question{Implémenter la fonction \texttt{zeros\_01(n:int)->list} permettant de générer une liste de \texttt{n 0}. On utilisera une boucle \texttt{while}.}

\begin{lstlisting}
>>> zeros_01(4)
	[0,0,0,0]
\end{lstlisting}

\question{Implémenter la fonction \texttt{zeros\_02(n:int)->list} permettant de générer une liste de \texttt{n 0}. On utilisera une boucle \texttt{for}.}

\question{Implémenter la fonction \texttt{zeros\_03(n:int)->list} permettant de générer une liste de \texttt{n 0}. On utilisera une boucle \texttt{while}. On génèrera cette liste <<~en compréhension~>> (sans boucle \texttt{for} ou \texttt{while} explicite).}

\subsection*{Exercice -- Liste d'entiers}

\question{Implémenter la fonction \texttt{entiers\_01(n:int)->list} permettant de générer la liste des entiers compris entre 0 inclus et \texttt{n} exclus. On utilisera une boucle \texttt{while}.}

\begin{lstlisting}
>>> entiers_01(4)
	[0,1,2,3]
\end{lstlisting}

\question{Implémenter la fonction \texttt{entiers\_02(n:int)->list} permettant de générer la liste des entiers compris entre 0 inclus et \texttt{n} exclus. On utilisera une boucle \texttt{for}.}


\question{Implémenter la fonction \texttt{entiers\_03(n:int)->list} permettant de générer la liste des entiers compris entre 0 inclus et \texttt{n} exclus. On génèrera cette liste << en compréhension >> (sans boucle \texttt{for} ou \texttt{while} explicite).}

\question{\'Ecrire une fonction \texttt{carres(n:int)->list} qui prend en argument un entier naturel \texttt{n} et qui renvoie la liste des \texttt{n} premiers carrés d'entiers, en commençant par $0$.}

\question{\'Ecrire une fonction \texttt{somme\_racine(n:int)->float} permettant de calculer la somme des racines carrées des n premiers entiers naturels non nuls.}

\subsection*{Exercice -- Notion d'effet de bords}

On cherche à écrire une fonction prenant en argument une liste d'entiers (non vide) et incrémentant de $1$ le premier élément de cette liste.

\question{\'Ecrire une telle fonction \texttt{incr\_sans\_effet\_de\_bord(L:list)->list}, qui ne modifie pas la liste initiale et renvoie en sortie une nouvelle liste.}

\begin{lstlisting}
>>> L = [1,2,3]
>>> LL = incr_sans_effet_de_bord(L)
>>> print(L)
    [1,2,3]
>>> print(LL)
    [2,2,3]
\end{lstlisting}


\question{\'Ecrire une telle fonction \texttt{incr\_avec\_effet\_de\_bord(L:list)-> None}, qui modifie la liste initiale et ne renvoie rien en sortie (ponctuer par un \texttt{return None}).}

\begin{lstlisting}
>>> L = [1,2,3]
>>> LL = incr_sans_effet_de_bord(L)
>>> print(L)
    [2,2,3]
>>> print(LL)
    None
\end{lstlisting}


\subsection*{Exercice -- Structures itératives}

\question{\'Ecrire la fonction \texttt{somme\_inverse(n:int)->float} calculant la somme des inverses des \texttt{n} premiers entiers non nuls.}


\subsection*{Exercice -- Un petit peu de géométrie}

\question{Implémenter la fonction \texttt{norme(A:list,B:list)->float} permettant calculer la norme du vecteur $\vect{AB}$ dans $\mathbb{R}^3$. Chacun des points sera constitué de la liste de ses coordonnées (par exemple \texttt{A=[xA,yA,zA]}).}

\question{Implémenter la fonction \texttt{prod\_vect(u:list, v:list)->float} permettant calculer le produit vectoriel $\vect{u}\wedge \vect{v}$ dans $\mathbb{R}^3$.}



\subsection*{Exercice -- Suites d'entiers}

\question{Implémenter la fonction \texttt{impairs(n:int)->list} permettant de générer la liste des \texttt{n} premiers entiers naturels impairs.}

\question{Implémenter la fonction \texttt{multiples\_5(d:int, f:int)->list} permettant de générer la liste de tous les multuples de 5 compris entre \texttt{d} et \texttt{f} (bornes incluses si se sont des multiples de 5).}

\question{Implémenter la fonction \texttt{cube(f:int)->list} permettant de générer la liste de tous les cubes d'entiers naturels inférieurs ou égaux à \texttt{f} (inclus si \texttt{f} est un cube).}


\subsection*{Exercice -- Un petit peu de statistiques}

\question{\'Ecrire une fonction \texttt{moy\_extr(L:list)->float} qui prend en argument une liste \texttt{L} et renvoie en sortie la moyenne du premier et du dernier élément de \texttt{L}.}

\question{\'Ecrire une fonction \texttt{moyenne(L:list)->float} qui prend en argument une liste \texttt{L} et renvoie la moyenne des éléments de \texttt{L}.}

\question{\'Ecrire une fonction \texttt{ecart\_type(L:list)->float} qui prend en argument une liste \texttt{L} et renvoie l'éacrt type des données : en notant $\overline{x}$ la moyenne des échantillons, on a  $\sigma = \sqrt{\dfrac{1}{n}\sum\limits_{i=1}^n \left(x_i^2  - \overline{x}\right)}$.}



\subsection*{Exercice -- Chaînes de caractères}

\question{\'Ecrire une fonction \texttt{lettre(i:int)->str} qui prend en argument un entier \texttt{i} et renvoie la \texttt{i}\ieme\ lettre de l'alphabet.}


