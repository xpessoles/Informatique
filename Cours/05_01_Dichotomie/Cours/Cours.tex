%\setchapterimage{Fond_SLCI.png}
\setchapterpreamble[u]{\margintoc}
\chapter{Algorithmes dichotomiques}

%\marginnote[5cm]{
%\UPSTIcompetence[2]{C1-02}
%\UPSTIcompetence[2]{C2-04}
%}

%\marginnote[4cm]{\textbf{Qui}, \textit{Quoi}, Où.}


\section{Introduction}
Les méthodes de résolutions par un algorithme dichotomique font partie des algorithmes basés sur le principe de << diviser pour régner >>.
Elles utilisent la définition du terme \textbf{dichotomie} qui signifie diviser un tout en deux parties << opposées >>.
Certains algorithmes de tris sont basés sur ce principe de diviser pour régner.



Ce cours vous présente deux algorithmes dichotomiques :
\begin{itemize}
\item la recherche d'un élément dans une liste triée ;
\item la détermination de la racine d'une fonction quand elle existe.
\end{itemize}

\section{Recherche dichotomique dans une liste {triée}: Principe}

Lorsque vous cherchez le mot <<~hippocampe~>> dans le dictionnaire, vous ne vous amusez pas à parcourir chaque page depuis la lettre a jusqu'à tomber sur le mot <<~hippocampe~>>...\\
Dans une liste triée, il y a plus efficace ! Par exemple dans le dictionnaire, vous ouvrez à peu près au milieu, et suivant si le mot trouvé est \og inférieur \fg \ ou \og supérieur \fg \ à <<~hippocampe~>> (pour l'ordre alphabétique), vous poursuivez votre recherche dans l'une ou l'autre moitié du dictionnaire.


\begin{prop}
On se donne une liste \lstinline{L} de nombres de longueur \lstinline{n},     {triée dans l'ordre croissant}, et un nombre \lstinline{x0}. 

Pour chercher \lstinline{x0}, on va couper la liste en deux moitiés et chercher dans la moitié intéressante et ainsi de suite.

On appelle \lstinline{g} {l'indice} de l'élément du début de la sous-liste dans laquelle on travaille et \lstinline{d}     {l'indice} de l'élément de fin.

Au début, \lstinline{g = 0} et \lstinline{d = n-1} 

On souhaite construire un algorithme admettant l'invariant suivant:
\bigskip

\centerline{{si \lstinline{x0} est dans \lstinline{L} alors \lstinline{x0} est dans la sous-liste \lstinline{L[g:d]} (\lstinline{g} inclus et \lstinline{d} exclu).}}
\end{prop}



On va utiliser la méthode suivante.
\begin{itemize}
\item On compare \lstinline{x0} à <<~l'élément du milieu~>> : c'est \lstinline{L[m]} où \lstinline{m = (g+d)//2}
son indice est \lstinline{m} =\lstinline{ n//2} (division euclidienne)
%$$\begin{array}{|c|c|c|c|c|c|c|c|c|} 
%\hline \hspace*{3mm} & \hspace*{3mm} & \hspace*{3mm} &  \hspace*{3mm} & \hspace*{3mm} & \hspace*{3mm} & \hspace*{3mm} & \hspace*{3mm} & \hspace*{3mm}\\ \hline
%\end{array}$$
%\medskip
%$$\begin{array}{|c|c|c|c|} 
%\hline \hspace*{3mm} & \hspace*{3mm} & \hspace*{3mm} &  \hspace*{3mm} \\ \hline
%\end{array}$$

\item Si \lstinline{x0 = L[m]}, on a trouvé \lstinline{x0}, on peut alors s'arrêter.
\item Si \lstinline{x0} $<$ \lstinline{L[m]}, c'est qu'il faut chercher dans la première moitié de la liste, entre \lstinline{L[g]} et  \lstinline{L[m-1]} (\lstinline{L[m]} exclu).
%dans  {la première moitié de la liste \t{L[g:m]}}\\
\item Si \lstinline{x0} $>$ \lstinline{L[m]}, c'est qu'il faut chercher dans la seconde moitié de la liste, entre \lstinline{L[m+1]} et \lstinline{L[d]} (\lstinline{L[m]} exclu).
\end{itemize}

On poursuit jusqu'à ce qu'on a trouvé \lstinline{x0} ou lorsque l'on a épuisé la liste \lstinline{L}.





\section{Exemples d'application}
%\begin{enumerate}
%\item %En notant $g$ et $d$  les indices de gauche et de droite du morceau de la liste $l$ où l'on est en train de faire la %recherche, 

\marginnote{Cas 1 :
$\begin{cases}
g=0\\d=8\\m=4,L[m]>x0
\end{cases}$
$\begin{cases}
g=0\\d=3\\m=1,L[m]=x0
\end{cases}$.\\
C'est fini, on a bien trouvé $x_0$ dans la liste.}

\marginnote{Cas 2 :
$\begin{cases}
g=0\\d=8\\m=4,L[m]<x0
\end{cases}$, $\begin{cases}
g=5\\d=8\\m=6,L[m]>x0
\end{cases}$ $\begin{cases}
g=5\\d=5\\m=5,L[m]<x0
\end{cases}$ $\begin{cases}
g=6\\d=5
\end{cases}$.
C'est fini, on a épuisé la liste \lstinline{L} et on n'a pas trouvé $x0$.}

Indiquer pour les deux exemples suivants les valeurs successives de \lstinline{g} et \lstinline{d} :
\begin{enumerate}
\item \lstinline{x0 = 5} et \lstinline{L} $= \begin{array}{|c|c|c|c|c|c|c|c|c|} 
\hline -3 & 5 & 7 & 10 & 11 & 14 & 17 & 21 & 30 \\ \hline
\end{array}$



\item \lstinline{x0 = 11} et \lstinline{L} $= \begin{array}{|c|c|c|c|c|c|c|c|c|} 
\hline -2 & 1 & 2 & 7 & 8 & 10 & 13 & 16 & 17  \\ \hline
\end{array}$

\end{enumerate}


%\textbf{Remarque :} On en déduit que de manière générale, \lstinline{m = (g + d) // 2} (division euclidienne)\\
 %- si \lstinline{x0} $<$ \lstinline{L[m]}, \lstinline{g} est inchangé et \lstinline{d} prend la valeur de \lstinline{m}\\
 %- si \lstinline{L[m]} $\leq$ \lstinline{x0}, \lstinline{d} est inchangé et \lstinline{g} prend la valeur de \lstinline{m}
 

\section{Implémentation en Python}


La fonction \lstinline{recherche dichotomie} d'arguments une liste \lstinline{L} et un élément \lstinline{x0} renvoyant un booléen disant si \lstinline{x0} est dans la liste \lstinline{L} est proposée :



\begin{lstlisting}
def recherche dichotomie(L:list, x0:int)-> bool:
     n = len(L)
     g_ind = 0 # c'est l'indice de gauche
     d_ind = n - 1 # c'est l'indice de droite
     rep = False
     while g ind <= d_ind and rep == False:
         # si x0 est dans L alors L[g ind] <= x0 <= L[d ind]     {invariant}
         m = (d ind + g ind) // 2 
         if x0 == L[m]:
             rep = True
         elif x0 < L[m]:
             d_ind = m - 1
          else:
             g_ind = m + 1
           # si x0 est dans L alors L[g_ind] <= x0 <= L[d_ind]     {invariant}
     return(rep)
\end{lstlisting} 



%\begin{python}
%def dichotomie(L, x0):\\
%     n = len(L)\\
%     g = 0\\
%     d = n\\
%     while  {d - g > 1:}\\
%         m =  {(d + g) // 2} \\
%         if  {x0 < L[m]}\\
%              {d = m}\\
%          else:\\
%              {g = m}\\
%     return  {x0 == L[g]}
%\end{python}
\textbf{Remarque :} La terminaison de l'algorithme est obtenue avec $d-g$ qui est un entier positif qui décro\^{i}t strictement à chaque passage dans la boucle \lstinline{while} et joue le rôle de variant.
%\end{enumerate}


\section{Détermination de la racine d'une fonction par dichotomie}

\subsection{Principe théorique de la méthode par dichotomie}
On considère une fonction $f$ vérifiant : 
\begin{center} $f$ continue sur $\verb![!a,b\verb!]!$ ;  $f(a)$ et $f(b)$ de signes opposés.
\end{center} Le théorème des valeurs intermédiaires nous assure que $f$ possède au moins un zéro $\ell$ entre $a$ et $b$. La preuve, vue en cours de mathématiques, repose sur la méthode de dichotomie. Prenons le cas $f(a)<0$ et $f(b)>0$ et posons $g_0=a$, $d_0=b$. 
%\begin{center}
%\begin{tikzpicture}%[scale=2,xmin=-2.5,xmax=2.5,ymin=-1.5,ymax=1.5]
%\shorthandoff{:};
%%\draw[->] (\xmin,0)--(\xmax,0);
%\draw[->] (-2.5,0)--(2.5,0);
%%\draw[->] (-2.25,\ymin)--(-2.25,\ymax);
%\draw[->] (-2.25,-1.5)--(-2.25,1.5);
%\fenetre
%\draw[domain=-2:2, samples=200, very thick]  plot ({\x},{((\x)^5+3*(\x)-7)/34});
%\draw (-2,0)node{$\cdot$};
%\draw (-1,0)node{$\cdot$};
%\draw (1,0)node{$\cdot$};
%\draw (0,0)node{$\cdot$};
%\draw (2,0)node{$\cdot$};
%\draw (-2 , 0) node[below] {$a$};
%\draw (2 , 0) node[below] {$b$};
%\draw (1.26 , 0) node[below] {$\ell$};
% \end{tikzpicture}
% \end{center}
 On considère $m_0 = \dfrac{g_0+d_0}{2}$ et on évalue $f(m_0)$ : 
\begin{itemize}
 \item Si $f(m_0)\geq 0$, on va poursuivre la recherche d'un zéro dans l'intervalle  {$[g_0,m_0]$} \\On pose donc  : 
 $g_1 =  {g_0} \quad ;\quad  d_1 =  {m_0}$\vspace*{2mm}
  \item Sinon,  la recherche doit se poursuivre  dans l'intervalle  {$[m_0,d_0]$} \\ On pose donc  : 
 $g_1 =  {m_0} \quad ; \quad d_1 =  {d_0}$\vspace*{2mm}
 \item On recommence alors en considérant $m_1 = \dfrac{g_1+d_1}{2}$...\\
 \end{itemize}


 \begin{enumerate}
 \item Par quoi peut-on remplacer la condition "$f(m_k)\geq 0$" dans le cas général où $f(a)$ et $f(b)$ sont de signes contraires (pas forcément $f(a)<0$ et $f(b)>0$) ?

 \item \`A quelle condition sur $g_n$ et $d_n$ s'arrête-t-on, si l'on souhaite que $g_n$ et $d_n$ soient des solutions approchées de $\ell$ à une précision $\varepsilon$ ?

 \item  Au lieu de renvoyer $g_n$ et/ou $d_n$ comme valeurs approchées de $\ell$, que pourrait-on prendre ? 
 
 Que mettre comme condition d'arrêt pour avoir une précision $\varepsilon$  ? 

 \item Un étudiant propose de tester si $f(m_k)=0$.  Qu'en pensez-vous ?

\end{enumerate}


\subsection{Implémentation en Python et avec scipy}
\'Ecrivons une fonction \lstinline{zero dichotomie(f:function,a:float,b:float,epsilon:float)->float} d'arguments une fonction \lstinline{f}, des flottants \lstinline{a} et \lstinline{b} (tels que \lstinline{a < b}), et  la précision voulue \lstinline{epsilon} (flottant strictement positif). Cette fonction renverra une valeur approchée à \lstinline{epsilon} près d'un zéro de \lstinline{f}, compris entre \lstinline{a} et \lstinline{b}, obtenue par la méthode de dichotomie.

\begin{lstlisting}
def zero dichotomie(f:function, a:float, b:float, epsilon:float):
     g = a # c'est un flottant
     d = b # c'est un flottant
     while d - g > 2 * epsilon :
          m = (g + d) / 2
          if f(g) * f(m) <= 0:
               d = m
          else:
               g = m
     return ((g + d) / 2)
\end{lstlisting}

Effectuons un test avec la fonction $f : x \mapsto x^2-2$ sur l'intervalle $\verb![!1,2\verb!]!$, avec une précision de $10^{-6}$ :
 
\begin{lstlisting}
def f(x):
     return(x ** 2 - 2)
print (zero dichotomie(f, 1, 2, 10**(-6)))

# il s'affichera : 1.4142141342163086
\end{lstlisting}
%\begin{algorithm}
%\begin{algorithmic}
%
% {
%\STATE $g \leftarrow a$
%\STATE $d \leftarrow b$
%\WHILE{$d-g> 2\varepsilon$}
%\STATE $m \leftarrow (g+d)/2$
%\IF{$f(g)f(m)\leq 0$}
%\STATE $d \leftarrow m$
%\ELSE
%\STATE $g \leftarrow m$
%\ENDIF
%\ENDWHILE
%\STATE renvoyer $\dfrac{g+d}{2}$}
%~\newline
%~\newline
%~\newline
%\end{algorithmic}
%\end{algorithm}
% 

%\subsection{Fonction prédéfinie dans scipy}
Une telle fonction est déjà prédéfinie dans la bibliothèque \lstinline{scipy.optimize}, la fonction \lstinline{bisect} \linebreak (la méthode de dichotomie s'appelle aussi la méthode de la \textit{bisection}) : 
\begin{lstlisting}
import scipy.optimize as spo
print (spo.bisect(f, 1, 2)) 
# il s'affichera : 1.4142135623724243
\end{lstlisting}
La précision est un argument optionnel (à mettre après \lstinline{f}, \lstinline{a} et \lstinline{b}) et vaut $10^{-12}$ par défaut.

