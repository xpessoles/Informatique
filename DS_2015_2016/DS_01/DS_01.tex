\documentclass[10pt,fleqn]{article} % Default font size and left-justified equations
\usepackage[%
    pdftitle={CIN : Cinématique du solide},
    pdfauthor={Xavier Pessoles}]{hyperref}

%%%%%%%%%%%%%%%%%%%%%%%%%%%%%%%%%%%%%%%%%
% Original author:
% Mathias Legrand (legrand.mathias@gmail.com) with modifications by:
% Vel (vel@latextemplates.com)
% License:
% CC BY-NC-SA 3.0 (http://creativecommons.org/licenses/by-nc-sa/3.0/)
%%%%%%%%%%%%%%%%%%%%%%%%%%%%%%%%%%%%%%%%%



%----------------------------------------------------------------------------------------
%	MAIN TABLE OF CONTENTS
%----------------------------------------------------------------------------------------


% Part text styling
\titlecontents{part}[0cm]
{\addvspace{20pt}\centering\large\bfseries}
{}
{}
{}

% Chapter text styling
\titlecontents{chapter}[1.25cm] % Indentation
{\addvspace{12pt}\large\sffamily\bfseries} % Spacing and font options for chapters
{\color{bleuxp!60}\contentslabel[\Large\thecontentslabel]{1.25cm}\color{bleuxp}} % Chapter number
{\color{bleuxp}}  
{\color{bleuxp!60}\normalsize\;\titlerule*[.5pc]{.}\;\thecontentspage} % Page number

% Section text styling
\titlecontents{section}[1.25cm] % Indentation
{\addvspace{3pt}\sffamily\bfseries} % Spacing and font options for sections
{\color{bleuxp!60}\contentslabel[\thecontentslabel]{1.25cm} \color{bleuxp}} % Section number
{\color{bleuxp}}
{\hfill\color{bleuxp!60}\thecontentspage} % Page number
[]

% Subsection text styling
\titlecontents{subsection}[1.25cm] % Indentation
{\addvspace{1pt}\sffamily\small} % Spacing and font options for subsections
{\contentslabel[\thecontentslabel]{1.25cm}} % Subsection number
{}
{\ \titlerule*[.5pc]{.}\;\thecontentspage} % Page number
[]


% Subsection text styling
\titlecontents{subsubsection}[1.25cm] % Indentation
{\addvspace{1pt}\sffamily\small} % Spacing and font options for subsections
{\contentslabel[\thecontentslabel]{1.25cm}} % Subsection number
{}
{\ \titlerule*[.5pc]{.}\;\thecontentspage} % Page number
[]

% List of figures
\titlecontents{figure}[0em]
{\addvspace{-5pt}\sffamily}
{\thecontentslabel\hspace*{1em}}
{}
{\ \titlerule*[.5pc]{.}\;\thecontentspage}
[]

% List of tables
\titlecontents{table}[0em]
{\addvspace{-5pt}\sffamily}
{\thecontentslabel\hspace*{1em}}
{}
{\ \titlerule*[.5pc]{.}\;\thecontentspage}
[]

%----------------------------------------------------------------------------------------
%	MINI TABLE OF CONTENTS IN PART HEADS
%----------------------------------------------------------------------------------------

% Chapter text styling
\titlecontents{lchapter}[0em] % Indenting
{\addvspace{15pt}\large\sffamily\bfseries} % Spacing and font options for chapters
{\color{bleuxp}\contentslabel[\Large\thecontentslabel]{1.25cm}\color{bleuxp}} % Chapter number
{}  
{\color{bleuxp}\normalsize\sffamily\bfseries\;\titlerule*[.5pc]{.}\;\thecontentspage} % Page number

% Section text styling
\titlecontents{lsection}[0em] % Indenting
{\sffamily\small} % Spacing and font options for sections
{\contentslabel[\thecontentslabel]{1.25cm}} % Section number
{}
{}

% Subsection text styling
\titlecontents{lsubsection}[.5em] % Indentation
{\normalfont\footnotesize\sffamily} % Font settings
{}
{}
{}

%----------------------------------------------------------------------------------------
%	PAGE HEADERS
%----------------------------------------------------------------------------------------




\pagestyle{fancy}
 \renewcommand{\headrulewidth}{0pt}
 \fancyhead{}
 
 % ENTETES de page
 \fancyhead[L]{%
 \begin{tikzpicture}[overlay]
\node(logo) at (1,0)
    {\includegraphics[width=2cm]{logo_lycee.png}};
\end{tikzpicture}
 %\noindent\begin{minipage}[c]{2.6cm}%
 %\includegraphics[width=2cm]{logo_lycee.png}%
 %\end{minipage}
}

\fancyhead[C]{\rule{8cm}{.5pt}}

 \fancyhead[R]{%
 \noindent\begin{minipage}[c]{3cm}
 \begin{flushright}
 \footnotesize{\textit{\textsf{\xxtete}}}%
 \end{flushright}
 \end{minipage}
}

 \fancyfoot{}
 % PIEDS de page
\fancyfoot[C]{\rule{12cm}{.5pt}}
\renewcommand{\footrulewidth}{0.2pt}
\fancyfoot[C]{\footnotesize{\bfseries \thepage}}
\fancyfoot[L]{ 
\begin{minipage}[c]{.4\linewidth}
\noindent\footnotesize{{\xxauteur}}
\end{minipage}}

\fancyfoot[R]{\footnotesize{\xxpied}
\ifthenelse{\isodd{\value{page}}}{
\begin{tikzpicture}[overlay]
\node[shape=rectangle, 
      rounded corners = .25 cm,
	  draw= bleuxp,
	  line width=2pt, 
	  fill = bleuxp!10,
	  minimum width  = 2.5cm,
	  minimum height = 3cm,] at (\xxposongletx,\xxposonglety) {};
\node at (\xxposonglettext,\xxposonglety) {\rotatebox{90}{\textbf{\large\color{bleuxp}{\xxonglet}}}};
%{};
\end{tikzpicture}}{}
}



%
%
%
% Removes the header from odd empty pages at the end of chapters
\makeatletter
%\renewcommand{\cleardoublepage}{
%\clearpage\ifodd\c@page\else
%\hbox{}
%\vspace*{\fill}
%\thispagestyle{empty}
%\newpage
%\fi}

%\fancypagestyle{plain}{%
%\fancyhf{} % vide l’en-tête et le pied~de~page.
%%\fancyfoot[C]{\bfseries \thepage} % numéro de la page en cours en gras
%% et centré en pied~de~page.
%\fancyfoot[R]{\footnotesize{\xxpied}}
%\fancyfoot[C]{\rule{12cm}{.5pt}}
%\renewcommand{\footrulewidth}{0.2pt}
%\fancyfoot[C]{\footnotesize{\bfseries \thepage}}
%\fancyfoot[L]{ 
%\begin{minipage}[c]{.4\linewidth}
%\noindent\footnotesize{{\xxauteur}}
%\end{minipage}}}

\fancypagestyle{plain}{%
\fancyhf{} % vide l’en-tête et le pied~de~page.
\fancyfoot[C]{\rule{12cm}{.5pt}}
\renewcommand{\footrulewidth}{0.2pt}
\fancyfoot[C]{\footnotesize{\bfseries \thepage}}
\fancyfoot[L]{ 
\begin{minipage}[c]{.4\linewidth}
\noindent\footnotesize{{\xxauteur}}
\end{minipage}}
\fancyfoot[R]{\footnotesize{\xxpied}}
}







%----------------------------------------------------------------------------------------
%	SECTION NUMBERING IN THE MARGIN
%----------------------------------------------------------------------------------------
\setcounter{secnumdepth}{3}
\setcounter{tocdepth}{2}



\makeatletter
\renewcommand{\@seccntformat}[1]{\llap{\textcolor{bleuxp}{\csname the#1\endcsname}\hspace{1em}}}                    
\renewcommand{\section}{\@startsection{section}{1}{\z@}
{-4ex \@plus -1ex \@minus -.4ex}
{1ex \@plus.2ex }
{\normalfont\large\sffamily\bfseries}}
\renewcommand{\subsection}{\@startsection {subsection}{2}{\z@}
{-3ex \@plus -0.1ex \@minus -.4ex}
{0.5ex \@plus.2ex }
{\normalfont\sffamily\bfseries}}
\renewcommand{\subsubsection}{\@startsection {subsubsection}{3}{\z@}
{-2ex \@plus -0.1ex \@minus -.2ex}
{.2ex \@plus.2ex }
{\normalfont\small\sffamily\bfseries}}                        
\renewcommand\paragraph{\@startsection{paragraph}{4}{\z@}
{-2ex \@plus-.2ex \@minus .2ex}
{.1ex}
{\normalfont\small\sffamily\bfseries}}

%----------------------------------------------------------------------------------------
%	PART HEADINGS
%----------------------------------------------------------------------------------------


%----------------------------------------------------------------------------------------
%	CHAPTER HEADINGS
%----------------------------------------------------------------------------------------

% \newcommand{\thechapterimage}{}%
% \newcommand{\chapterimage}[1]{\renewcommand{\thechapterimage}{#1}}%
% \def\@makechapterhead#1{%
% {\parindent \z@ \raggedright \normalfont
% \ifnum \c@secnumdepth >\m@ne
% \if@mainmatter
% \begin{tikzpicture}[remember picture,overlay]
% \node at (current page.north west)
% {\begin{tikzpicture}[remember picture,overlay]
% \node[anchor=north west,inner sep=0pt] at (0,0) {\includegraphics[width=\paperwidth]{\thechapterimage}};
% \draw[anchor=west] (\Gm@lmargin,-9cm) node [line width=2pt,rounded corners=15pt,draw=bleuxp,fill=white,fill opacity=0.5,inner sep=15pt]{\strut\makebox[22cm]{}};
% \draw[anchor=west] (\Gm@lmargin+.3cm,-9cm) node {\huge\sffamily\bfseries\color{black}\thechapter. #1\strut};
% \end{tikzpicture}};
% \end{tikzpicture}
% \else
% \begin{tikzpicture}[remember picture,overlay]
% \node at (current page.north west)
% {\begin{tikzpicture}[remember picture,overlay]
% \node[anchor=north west,inner sep=0pt] at (0,0) {\includegraphics[width=\paperwidth]{\thechapterimage}};
% \draw[anchor=west] (\Gm@lmargin,-9cm) node [line width=2pt,rounded corners=15pt,draw=bleuxp,fill=white,fill opacity=0.5,inner sep=15pt]{\strut\makebox[22cm]{}};
% \draw[anchor=west] (\Gm@lmargin+.3cm,-9cm) node {\huge\sffamily\bfseries\color{black}#1\strut};
% \end{tikzpicture}};
% \end{tikzpicture}
% \fi\fi\par\vspace*{270\p@}}}

%-------------------------------------------

\def\@makeschapterhead#1{%
\begin{tikzpicture}[remember picture,overlay]
\node at (current page.north west)
{\begin{tikzpicture}[remember picture,overlay]
\node[anchor=north west,inner sep=0pt] at (0,0) {\includegraphics[width=\paperwidth]{\thechapterimage}};
\draw[anchor=west] (\Gm@lmargin,-9cm) node [line width=2pt,rounded corners=15pt,draw=bleuxp,fill=white,fill opacity=0.5,inner sep=15pt]{\strut\makebox[22cm]{}};
\draw[anchor=west] (\Gm@lmargin+.3cm,-9cm) node {\huge\sffamily\bfseries\color{black}#1\strut};
\end{tikzpicture}};
\end{tikzpicture}
\par\vspace*{270\p@}}
\makeatother



%----------------------------------------------------------------------------------------
%	
%----------------------------------------------------------------------------------------

\newcommand{\thechapterimage}{}%
\newcommand{\chapterimage}[1]{\renewcommand{\thechapterimage}{#1}}%
\def\@makechapterhead#1{%
{\parindent \z@ \raggedright \normalfont
\begin{tikzpicture}[remember picture,overlay]
\node at (current page.north west)
{\begin{tikzpicture}[remember picture,overlay]
\node[anchor=north west,inner sep=0pt] at (0,0) {\includegraphics[width=\paperwidth]{\thechapterimage}};
%\draw[anchor=west] (\Gm@lmargin,-9cm) node [line width=2pt,rounded corners=15pt,draw=bleuxp,fill=white,fill opacity=0.5,inner sep=15pt]{\strut\makebox[22cm]{}};
%\draw[anchor=west] (\Gm@lmargin+.3cm,-9cm) node {\huge\sffamily\bfseries\color{black}\thechapter. #1\strut};
\end{tikzpicture}};
\end{tikzpicture}
\par\vspace*{270\p@}
}}


%% Questions et exercices
\newcounter{numques}%Création d'un compteur qui s'appelle numques
\setcounter{numques}{0}%initialisation du compteur
\newcommand{\question}[1]{%Création d'une macro ayant un paramètre
\addtocounter{numques}{1}%chaque fois que cette macro est appelée, elle ajoute 1 au compteur numexos
\textbf{Question\, \textcolor{bleuxp}{\thenumques}\,}\,\textit{#1}}

\newcounter{numexo}%Création d'un compteur qui s'appelle numques
\setcounter{numexo}{0}%initialisation du compteur
\newcommand{\exer}[1]{%Création d'une macro ayant un paramètre
\refstepcounter{numexo} % incrément compteur et label
%\addtocounter{numexo}{1}%chaque fois que cette macro est appelée, elle ajoute 1 au compteur numexo
\noindent\textsf{\textbf{Exercice\, \textcolor{bleuxp}{\thenumexo}\, -- \, #1}}}



% \makeatletter             
% \renewcommand{\subparagraph}{\@startsection{exo}{5}{\z@}%
                                    % {-2ex \@plus-.2ex \@minus .2ex}%
                                    % {0ex}%               
% {\normalfont\bfseries Question \hspace{.7cm} }}
% \makeatother
% \renewcommand{\thesubparagraph}{\arabic{subparagraph}} 
% \makeatletter


%%%%%%%%%%%%
% Définition des vecteurs 
%%%%%%%%%%%%
\newcommand{\vect}[1]{\overrightarrow{#1}}
\newcommand{\axe}[2]{\left(#1,\vect{#2}\right)}
\newcommand{\couple}[2]{\left(#1,\vect{#2}\right)}
\newcommand{\angl}[2]{\left(\vect{#1},\vect{#2}\right)}

\newcommand{\rep}[1]{\mathcal{R}_{#1}}
\newcommand{\quadruplet}[4]{\left(#1;#2,#3,#4 \right)}
\newcommand{\repere}[4]{\left(#1;\vect{#2},\vect{#3},\vect{#4} \right)}
\newcommand{\base}[3]{\left(\vect{#1},\vect{#2},\vect{#3} \right)}


\newcommand{\vx}[1]{\vect{x_{#1}}}
\newcommand{\vy}[1]{\vect{y_{#1}}}
\newcommand{\vz}[1]{\vect{z_{#1}}}

\newcommand{\norm}[1]{\ensuremath{\left\Vert {#1}\right\Vert}}
\newcommand{\Ker}{\mathop{\mathrm{Ker}}\nolimits}

% d droit pour le calcul différentiel
\newcommand{\dd}{\text{d}}

\newcommand{\inertie}[2]{I_{#1}\left( #2\right)}
\newcommand{\matinertie}[7]{
\begin{pmatrix}
#1 & #6 & #5 \\
#6 & #2 & #4 \\
#5 & #4 & #3 \\
\end{pmatrix}_{#7}}
%%%%%%%%%%%%
% Définition des torseurs 
%%%%%%%%%%%%

\newcommand{\ec}[2]{%
\mathcal{E}_c\left(#1/#2\right)}

\newcommand{\pext}[3]{%
\mathcal{P}\left(#1\rightarrow#2/#3\right)}

\newcommand{\pint}[3]{%
\mathcal{P}\left(#1 \stackrel{\text{#3}}{\leftrightarrow} #2\right)}


 \newcommand{\torseur}[1]{%
\left\{{#1}\right\}
}

\newcommand{\torseurcin}[3]{%
\left\{\mathcal{#1} \left(#2/#3 \right) \right\}
}

\newcommand{\torseurci}[2]{%
\left\{\sigma \left(#1/#2 \right) \right\}
}
\newcommand{\torseurdyn}[2]{%
\left\{\mathcal{D} \left(#1/#2 \right) \right\}
}


\newcommand{\torseurstat}[3]{%
\left\{\mathcal{#1} \left(#2\rightarrow #3 \right) \right\}
}


 \newcommand{\torseurc}[8]{%
%\left\{#1 \right\}=
\left\{
{#1}
\right\}
 = 
\left\{%
\begin{array}{cc}%
{#2} & {#5}\\%
{#3} & {#6}\\%
{#4} & {#7}\\%
\end{array}%
\right\}_{#8}%
}

 \newcommand{\torseurcol}[7]{
\left\{%
\begin{array}{cc}%
{#1} & {#4}\\%
{#2} & {#5}\\%
{#3} & {#6}\\%
\end{array}%
\right\}_{#7}%
}

 \newcommand{\torseurl}[3]{%
%\left\{\mathcal{#1}\right\}_{#2}=%
\left\{%
\begin{array}{l}%
{#1} \\%
{#2} %
\end{array}%
\right\}_{#3}%
}

% Vecteur vitesse
 \newcommand{\vectv}[3]{%
\vect{V\left( {#1} \in {#2}/{#3}\right)}
}

% Vecteur force
\newcommand{\vectf}[2]{%
\vect{R\left( {#1} \rightarrow {#2}\right)}
}

% Vecteur moment stat
\newcommand{\vectm}[3]{%
\vect{\mathcal{M}\left( {#1}, {#2} \rightarrow {#3}\right)}
}




% Vecteur résultante cin
\newcommand{\vectrc}[2]{%
\vect{R_c \left( {#1}/ {#2}\right)}
}
% Vecteur moment cin
\newcommand{\vectmc}[3]{%
\vect{\sigma \left( {#1}, {#2} /{#3}\right)}
}


% Vecteur résultante dyn
\newcommand{\vectrd}[2]{%
\vect{R_d \left( {#1}/ {#2}\right)}
}
% Vecteur moment dyn
\newcommand{\vectmd}[3]{%
\vect{\delta \left( {#1}, {#2} /{#3}\right)}
}

% Vecteur accélération
 \newcommand{\vectg}[3]{%
\vect{\Gamma \left( {#1} \in {#2}/{#3}\right)}
}

% Vecteur omega
 \newcommand{\vecto}[2]{%
\vect{\Omega\left( {#1}/{#2}\right)}
}
% }$$\left\{\mathcal{#1} \right\}_{#2} =%
% \left\{%
% \begin{array}{c}%
%  #3 \\%
%  #4 %
% \end{array}%
% \right\}_{#5}}

\newcommand{\N}{\mathbb{N}}
\newcommand{\Z}{\mathbb{Z}}
\newcommand{\R}{\mathbb{R}}
\newcommand{\C}{\mathbb{C}}
\newcommand{\K}{\mathbb{K}}

\newcommand{\cA}{\mathscr{A}}
\newcommand{\cM}{\mathscr{M}}
\newcommand{\cL}{\mathscr{L}}
\newcommand{\cS}{\mathscr{S}}

\newcommand{\python}{\texttt{Python}}

\newcommand{\z}[1]{\Z_{#1}}
\newcommand{\ztimes}[1]{\Z_{#1}^{\times}}
\newcommand{\ii}[1]{[\![#1[\![}
\newcommand{\iif}[1]{[\![#1]\!]}
\newcommand{\llbr}{\ensuremath{\llbracket}}
\newcommand{\rrbr}{\ensuremath{\rrbracket}}
%\newcommand{\p}[1]{\left(#1\right)}
\newcommand{\ens}[1]{\left\{ #1 \right\}}
\newcommand{\croch}[1]{\left[ #1 \right]}
%\newcommand{\of}[1]{\lstinline{#1}}
% \newcommand{\py}[2]{%
%   \begin{tabular}{|l}
%     \lstinline+>>>+\textbf{\of{#1}}\\
%     \of{#2}
%   \end{tabular}\par{}
% }
\newcommand{\floor}[1]{\left\lfloor#1\right\rfloor}
\newcommand{\ceil}[1]{\left\lceil#1\right\rceil}
\newcommand{\abs}[1]{\left|#1\right|}


% Binaire, octal, hexa
\newcommand{\hex}[1]{\underline{\text{\texttt{#1}}}_{16}}
\newcommand{\oct}[1]{\underline{\text{\texttt{#1}}}_{8}}
\newcommand{\bin}[1]{\underline{\text{\texttt{#1}}}_{2}}
\DeclareMathOperator{\mmod}{\texttt{\%}}


% Fonctions et systèmes
\newcommand{\fct}[5][t]{%
  \begin{array}[#1]{rcl}
    #2 & \rightarrow & #3\\
    #4 & \mapsto     & #5\\
  \end{array}
}
\newcommand{\fonction}[5]{#1 : \left\{\begin{array}{rcl} #2& \longrightarrow &#3 \\ #4 &\longmapsto & #5\end{array}\right.}
\newenvironment{systeme}{\left\{ \begin{array}{rcl}}{\end{array}\right.}

% Matrices
\newcommand{\mat}[1]{
  \begin{pmatrix}
    #1
  \end{pmatrix}
}
\newcommand{\inv}{\ensuremath{^{-1}}}
\newcommand{\bpm}{\begin{pmatrix}}
\newcommand{\epm}{\end{pmatrix}}


% bases de données
\newcommand{\relat}[1]{\textsc{#1}}
\newcommand{\attr}[1]{\emph{#1}}
\newcommand{\prim}[1]{\uline{#1}}
\newcommand{\foreign}[1]{\#\textsl{#1}}


% Bases de données

\newcommand{\att}{\ensuremath{\mathbf{att}}}
\newcommand{\dom}{\ensuremath{\mathbf{dom}}}
\newcommand{\sort}{\ensuremath{\mathbf{sort}}}
\newcommand{\relname}{\ensuremath{\mathbf{relname}}}
\newcommand{\var}{\ensuremath{\mathbf{var}}}
\newcommand{\FILM}{\ensuremath{\mathtt{FILM}}}
\newcommand{\JOUE}{\ensuremath{\mathtt{JOUE}}}
\newcommand{\PERSONNE}{\ensuremath{\mathtt{PERSONNE}}}
\newcommand{\PERSONNAGE}{\ensuremath{\mathtt{PERSONNAGE}}}

\newcommand{\ttid}{\ensuremath{\mathtt{id}}}
\newcommand{\tttitre}{\ensuremath{\mathtt{titre}}}
\newcommand{\ttdate}{\ensuremath{\mathtt{date}}}
\newcommand{\ttidr}{\ensuremath{\mathtt{idrealisateur}}}
\newcommand{\ttdatenais}{\ensuremath{\mathtt{datenaissance}}}
\newcommand{\ttnom}{\ensuremath{\mathtt{nom}}}
\newcommand{\ttprenom}{\ensuremath{\mathtt{prenom}}}
\newcommand{\ttidacteur}{\ensuremath{\mathtt{idacteur}}}
\newcommand{\ttidfilm}{\ensuremath{\mathtt{idfilm}}}
\newcommand{\ttidpersonnage}{\ensuremath{\mathtt{idpersonnage}}}

\newcommand{\fv}{\mathrm{libre}}
\newcommand{\sem}[1]{[\![ #1 ]\!]}


\fichetrue
%\fichefalse

\proftrue
%\proffalse

%\tdtrue
\tdfalse

%\courstrue
\coursfalse

% -------------------------------------
% Déclaration des titres
% -------------------------------------

\def\discipline{Informatique}
\def\xxtete{Informatique}

\def\classe{PT -- PT $\star$}
\def\xxnumpartie{DS 1}
\def\xxpartie{Devoir Surveillé 1}

\def\xxnumchapitre{}
\def\xxchapitre{Construction d'une référence spécifiée}

\def\xxtitreexo{Prothèse Active Transtibiale}
\def\xxsourceexo{\hspace{.2cm} D'après concours Mines-Ponts -- 2013.}

\def\xxposongletx{2}
\def\xxposonglettext{1.45}
\def\xxposonglety{20}
\def\xxonglet{\textsf{DS 1}}

\def\xxactivite{TD 1}
\def\xxauteur{\textsl{Xavier Pessoles -- D'après ressouces de P. Beynet.}}

\def\xxcompetences{%
\texttt{%
\textbf{Savoirs et compétences :}\\
\noindent \textbf{Résoudre :} à partir des modèles retenus :
\begin{itemize}[label=\ding{112},font=\color{ocre}] 
\item choisir une méthode de résolution analytique, graphique, numérique;
\item mettre en \oe{}uvre une méthode de résolution.
\end{itemize}
\begin{itemize}[label=\ding{112},font=\color{ocre}] 
\item \textit{Rés -- C1.1 :} Loi entrée sortie géométrique et cinématique -- Fermeture géométrique.
\end{itemize}
%
%\noindent \textit{Mod2 -- C4.1 :} Représentation par schéma bloc.
}}

\def\xxfigures{
%\includegraphics[width=.8\textwidth]{images/prot_01}
}%figues de la page de garde

\def\xxpied{%
DS 1 : Construction d'une référence spécifiée
}


\setcounter{secnumdepth}{5}
%---------------------------------------------------------------------------


\begin{document}
%\chapterimage{png/Fond_Cin}
\pagestyle{empty}


%%%%%%%% PAGE DE GARDE COURS
\ifcours
% ==== BANDEAU DES TITRES ==== 
\begin{tikzpicture}[remember picture,overlay]
\node at (current page.north west)
{\begin{tikzpicture}[remember picture,overlay]
\node[anchor=north west,inner sep=0pt] at (0,0) {\includegraphics[width=\paperwidth]{\thechapterimage}};
\draw[anchor=west] (-2cm,-8cm) node [line width=2pt,rounded corners=15pt,draw=ocre,fill=white,fill opacity=0.6,inner sep=40pt]{\strut\makebox[22cm]{}};
\draw[anchor=west] (1cm,-8cm) node {\huge\sffamily\bfseries\color{black} %
\begin{minipage}{1cm}
\rotatebox{90}{\LARGE\sffamily\textsc{\color{ocre}\textbf{\xxnumpartie}}}
\end{minipage} \hfill
\begin{minipage}[c]{14cm}
\begin{titrepartie}
\begin{flushright}
\renewcommand{\baselinestretch}{1.1} 
\Large\sffamily\textsc{\textbf{\xxpartie}}
\renewcommand{\baselinestretch}{1} 
\end{flushright}
\end{titrepartie}
\end{minipage} \hfill
\begin{minipage}[c]{3.5cm}
{\large\sffamily\textsc{\textbf{\color{ocre} \discipline}}}
\end{minipage} 
 };
\end{tikzpicture}};
\end{tikzpicture}
% ==== FIN BANDEAU DES TITRES ==== 


% ==== ONGLET 
\begin{tikzpicture}[overlay]
\node[shape=rectangle, 
      rounded corners = .25 cm,
	  draw= ocre,
	  line width=2pt, 
	  fill = ocre!10,
	  minimum width  = 2.5cm,
	  minimum height = 3cm,] at (18.3cm,0) {};
\node at (17.7cm,0) {\rotatebox{90}{\textbf{\Large\color{ocre}{\classe}}}};
%{};
\end{tikzpicture}
% ==== FIN ONGLET 


\vspace{3.5cm}

\begin{tikzpicture}[remember picture,overlay]
\draw[anchor=west] (-2cm,-6cm) node {\huge\sffamily\bfseries\color{black} %
\begin{minipage}{2cm}
\begin{center}
\LARGE\sffamily\textsc{\color{ocre}\textbf{\xxactivite}}
\end{center}
\end{minipage} \hfill
\begin{minipage}[c]{15cm}
\begin{titrechapitre}
\renewcommand{\baselinestretch}{1.1} 
\Large\sffamily\textsc{\textbf{\xxnumchapitre}}

\Large\sffamily\textsc{\textbf{\xxchapitre}}
\vspace{.5cm}

\renewcommand{\baselinestretch}{1} 
\normalsize\normalfont
\xxcompetences
\end{titrechapitre}
\end{minipage}  };
\end{tikzpicture}
\vfill

\begin{flushright}
\begin{minipage}[c]{.3\linewidth}
\begin{center}
\xxfigures
\end{center}
\end{minipage}\hfill
\begin{minipage}[c]{.6\linewidth}
\startcontents
%\printcontents{}{1}{}
\printcontents{}{1}{}
\end{minipage}
\end{flushright}

\begin{tikzpicture}[remember picture,overlay]
\draw[anchor=west] (4.5cm,-.7cm) node {
\begin{minipage}[c]{.2\linewidth}
\begin{flushright}
\includegraphics[width=2cm]{logoCC}
\end{flushright}
\end{minipage}
\begin{minipage}[c]{.2\linewidth}
\textsl{\xxauteur} \\
\textsl{\classe}
\end{minipage}
 };
\end{tikzpicture}

\newpage
\pagestyle{fancy}

%\newpage
%\pagestyle{fancy}

\else
\fi
%% FIN PAGE DE GARDE DES COURS

%%%%%%%% PAGE DE GARDE TD
\iftd
%\begin{tikzpicture}[remember picture,overlay]
%\node at (current page.north west)
%{\begin{tikzpicture}[remember picture,overlay]
%\draw[anchor=west] (-2cm,-3.25cm) node [line width=2pt,rounded corners=15pt,draw=ocre,fill=white,fill opacity=0.6,inner sep=40pt]{\strut\makebox[22cm]{}};
%\draw[anchor=west] (1cm,-3.25cm) node {\huge\sffamily\bfseries\color{black} %
%\begin{minipage}{1cm}
%\rotatebox{90}{\LARGE\sffamily\textsc{\color{ocre}\textbf{\xxnumpartie}}}
%\end{minipage} \hfill
%\begin{minipage}[c]{13.5cm}
%\begin{titrepartie}
%\begin{flushright}
%\renewcommand{\baselinestretch}{1.1} 
%\Large\sffamily\textsc{\textbf{\xxpartie}}
%\renewcommand{\baselinestretch}{1} 
%\end{flushright}
%\end{titrepartie}
%\end{minipage} \hfill
%\begin{minipage}[c]{3.5cm}
%{\large\sffamily\textsc{\textbf{\color{ocre} \discipline}}}
%\end{minipage} 
% };
%\end{tikzpicture}};
%\end{tikzpicture}

%%%%%%%%%% PAGE DE GARDE TD %%%%%%%%%%%%%%%
%\begin{tikzpicture}[overlay]
%\node[shape=rectangle, 
%      rounded corners = .25 cm,
%	  draw= ocre,
%	  line width=2pt, 
%	  fill = ocre!10,
%	  minimum width  = 2.5cm,
%	  minimum height = 2.5cm,] at (18.5cm,0) {};
%\node at (17.7cm,0) {\rotatebox{90}{\textbf{\Large\color{ocre}{\classe}}}};
%%{};
%\end{tikzpicture}

% PARTIE ET CHAPITRE
%\begin{tikzpicture}[remember picture,overlay]
%\draw[anchor=west] (-1cm,-2.1cm) node {\large\sffamily\bfseries\color{black} %
%\begin{minipage}[c]{15cm}
%\begin{flushleft}
%\xxnumchapitre \\
%\xxchapitre
%\end{flushleft}
%\end{minipage}  };
%\end{tikzpicture}

% BANDEAU EXO
\iflivret % SI LIVRET
\begin{tikzpicture}[remember picture,overlay]
\draw[anchor=west] (-2cm,-3.3cm) node {\huge\sffamily\bfseries\color{black} %
\begin{minipage}{5cm}
\begin{center}
\LARGE\sffamily\color{ocre}\textbf{\textsc{\xxactivite}}

\begin{center}
\xxfigures
\end{center}

\end{center}
\end{minipage} \hfill
\begin{minipage}[c]{12cm}
\begin{titrechapitre}
\renewcommand{\baselinestretch}{1.1} 
\large\sffamily\textbf{\textsc{\xxtitreexo}}

\small\sffamily{\textbf{\textit{\color{black!70}\xxsourceexo}}}
\vspace{.5cm}

\renewcommand{\baselinestretch}{1} 
\normalsize\normalfont
\xxcompetences
\end{titrechapitre}
\end{minipage}};
\end{tikzpicture}
\else % ELSE NOT LIVRET
\begin{tikzpicture}[remember picture,overlay]
\draw[anchor=west] (-2cm,-4.5cm) node {\huge\sffamily\bfseries\color{black} %
\begin{minipage}{5cm}
\begin{center}
\LARGE\sffamily\color{ocre}\textbf{\textsc{\xxactivite}}

\begin{center}
\xxfigures
\end{center}

\end{center}
\end{minipage} \hfill
\begin{minipage}[c]{12cm}
\begin{titrechapitre}
\renewcommand{\baselinestretch}{1.1} 
\large\sffamily\textbf{\textsc{\xxtitreexo}}

\small\sffamily{\textbf{\textit{\color{black!70}\xxsourceexo}}}
\vspace{.5cm}

\renewcommand{\baselinestretch}{1} 
\normalsize\normalfont
\xxcompetences
\end{titrechapitre}
\end{minipage}};
\end{tikzpicture}

\fi

\else   % FIN IF TD
\fi


%%%%%%%% PAGE DE GARDE FICHE
\iffiche
\begin{tikzpicture}[remember picture,overlay]
\node at (current page.north west)
{\begin{tikzpicture}[remember picture,overlay]
\draw[anchor=west] (-2cm,-2.25cm) node [line width=2pt,rounded corners=15pt,draw=ocre,fill=white,fill opacity=0.6,inner sep=40pt]{\strut\makebox[22cm]{}};
\draw[anchor=west] (1cm,-2.25cm) node {\huge\sffamily\bfseries\color{black} %
\begin{minipage}{1cm}
\rotatebox{90}{\LARGE\sffamily\textsc{\color{ocre}\textbf{\xxnumpartie}}}
\end{minipage} \hfill
\begin{minipage}[c]{14cm}
\begin{titrepartie}
\begin{flushright}
\renewcommand{\baselinestretch}{1.1} 
\large\sffamily\textsc{\textbf{\xxpartie} \\} 

\vspace{.2cm}

\normalsize\sffamily\textsc{\textbf{\xxnumchapitre -- \xxchapitre}}
\renewcommand{\baselinestretch}{1} 
\end{flushright}
\end{titrepartie}
\end{minipage} \hfill
\begin{minipage}[c]{3.5cm}
{\large\sffamily\textsc{\textbf{\color{ocre} \discipline}}}
\end{minipage} 
 };
\end{tikzpicture}};
\end{tikzpicture}

\iflivret
\begin{tikzpicture}[overlay]
\node[shape=rectangle, 
      rounded corners = .25 cm,
	  draw= ocre,
	  line width=2pt, 
	  fill = ocre!10,
	  minimum width  = 2.5cm,
	  minimum height = 2.5cm,] at (18.5cm,1.1cm) {};
\node at (17.9cm,1.1cm) {\rotatebox{90}{\textsf{\textbf{\large\color{ocre}{\classe}}}}};
%{};
\end{tikzpicture}
\else
\begin{tikzpicture}[overlay]
\node[shape=rectangle, 
      rounded corners = .25 cm,
	  draw= ocre,
	  line width=2pt, 
	  fill = ocre!10,
	  minimum width  = 2.5cm,
%	  minimum height = 2.5cm,] at (18.5cm,1.1cm) {};
	  minimum height = 2.5cm,] at (18.6cm,1cm) {};
\node at (18cm,1cm) {\rotatebox{90}{\textsf{\textbf{\large\color{ocre}{\classe}}}}};
%{};
\end{tikzpicture}

\fi

\else
\fi



\vspace{2cm}
\pagestyle{fancy}
\thispagestyle{plain}

\section{Présentation}
\subsection{Mise en situation}
\ifprof
\else
En métrologie, il est nécessaire de construire une référence spécifiée plane à partir d'un nuage de points extraits d'une surface nominalement (réputée) plane. Ces points sont souvent mesurés à partir d'une machine à mesurer tridimensionnelle.

D'après la norme, la critère d'association choisi est de construire un <<plan tangent extérieur à la matière qui minimise l'écart maximum. Dans la pratique, il est assez difficile de déterminer le plan optimal. 

\begin{obj}
L'objectif de ce travail est de déterminer algorithmiquement, de manière approchée, le plan tangent extérieur matière qui minimise l'écart maximum. 
\end{obj}
\fi




\subsection{Mise en équation de détermination du plan des moindre carrés}
\ifprof
\else
\noindent\begin{minipage}[c]{.75\linewidth}
On suppose que la direction de mesure $\vect{z}$ est verticale ascendante pour la machine à mesurer tridimensionnelle. En conséquence, on traitera les plans qui ne contiennent pas le vecteur $\vect{z}$.

\begin{rappel}
$\quad$\\

Dans un repère orthonormé direct, l'équation d'un plan $\left( \mathcal{P}\right)$ est donnée par $z=ax+by+c$.

Le vecteur $\vect{n}(a,b,1)$ définit une normale au plan $\left( \mathcal{P}\right)$.

Le point de coordonnées $(0,0,c)$ appartient au plan $\left( \mathcal{P}\right)$.

\end{rappel}
\end{minipage} \hfill
\begin{minipage}[c]{.2\linewidth}
\begin{center}
\includegraphics[width=.95\textwidth]{images/MMT}
\end{center}
\end{minipage}
L'écart $e_i$ d'un point  $M_i$ suivant $\vect{z}$ de coordonnées $\left(x_i,y_i,z_i\right)$ au plan $\left( \mathcal{P}\right)$ est donné par 
$e_i = z_i -ax_i -by_i -c$.


\begin{defi}\textbf{Écarts}
La fonction écart $E$ est définie de la manière suivante : 
$$
E:(a,b,c)\rightarrow \sum\limits_{i=1}^{n} e_i^2 = \sum\limits_{i=1}^{n} \left(z_i -ax_i -by_i -c\right)^2
$$
\end{defi}


\begin{defi}\textbf{Minimisation des écarts}

Pour minimiser la fonction $E$, il faut résoudre le système d'équations suivant :

\begin{tabular}{p{4cm}p{4cm}p{4cm}}
\begin{eqnarray}\label{eq1}
\dfrac{\partial E(a,b,c)}{\partial a} = 0 
\end{eqnarray}
&
\begin{eqnarray}\label{eq2}
\dfrac{\partial E(a,b,c)}{\partial b} = 0 
\end{eqnarray}
&
\begin{eqnarray}\label{eq3}
\dfrac{\partial E(a,b,c)}{\partial c} = 0 
\end{eqnarray}% \\
\end{tabular}

\end{defi}


\begin{rem}
On remarque que la méthode des moindres carrés ne minimise pas la somme des écarts $e_i$ mais celle de leur carré. On pourrait vérifier que minimiser la fonction $E':(a,b,c)\rightarrow \sum\limits_{i=1}^{n} e_i$ à l’aide des relations obtenues par dérivations partielles, ne permettrait pas de déterminer les valeurs de $a$, $b$ et $c$.
\end{rem}


\fi
\section{Détermination du défaut de planéité}

\begin{obj}
L'objectif de cette partie est de rechercher le plan des moindres carrés c'est à dire de trouver les valeurs $a$, $b$ et $c$ qui minimisent les écarts entre le plan $\mathcal{P}$ et un nuage de points. 

Le plan des moindre carrés permettra de déterminer le défaut de planéité.
\end{obj}

\subsection{Conditionnement du problème}

\subparagraph{}
\textit{Montrer que la méthode des moindres carrés permet d'aboutir aux 3 équations suivantes :}

$$
\sum\limits_{i=1}^n \left(ax_i^2 + bx_i y_i  +c x_i -x_i z_i  \right) = 0 \quad 
\sum\limits_{i=1}^n  \left(ax_iy_i  + by_i^2 +cy_i -y_i z_i  \right) = 0 \quad 
\sum\limits_{i=1}^n  \left(ax_i +  by_i+c -z_i  \right) = 0
$$

\ifprof
\begin{corrige}

On a : 

$\dfrac{\partial E(a,b,c)}{\partial a} = 0 
\Leftrightarrow  \sum\limits_{i=1}^{n} \dfrac{\partial \left(z_i -ax_i -by_i -c\right)^2}{\partial a}  = 0
\Leftrightarrow   \sum\limits_{i=1}^{n} -2x_i  \left(z_i -ax_i -by_i -c\right)  = 0
$

Au final:  $\dfrac{\partial E(a,b,c)}{\partial a} = 0 
\Leftrightarrow   \sum\limits_{i=1}^{n}   \left(-x_iz_i +ax_i^2 +bx_i y_i +cx_i\right) = 0$

On a :

$\dfrac{\partial E(a,b,c)}{\partial b} = 0
\Leftrightarrow \sum\limits_{i=1}^{n} \dfrac{\partial \left(z_i -ax_i -by_i -c\right)^2}{\partial b}  =0 
\Leftrightarrow \sum\limits_{i=1}^{n} -2y_i  \left(z_i -ax_i -by_i -c\right) = 0
$

Au final : $\dfrac{\partial E(a,b,c)}{\partial b} = 0 
\Leftrightarrow   \sum\limits_{i=1}^{n}     \left(-y_iz_i + ax_i y_i  +by_i^2 +cy_i\right)  = 0$


On a :

$\dfrac{\partial E(a,b,c)}{\partial c} = 0
\Leftrightarrow \sum\limits_{i=1}^{n} \dfrac{\partial \left(z_i -ax_i -by_i -c\right)^2}{\partial c}  =0 
\Leftrightarrow \sum\limits_{i=1}^{n} -2  \left(z_i -ax_i -by_i -c\right) = 0
$

Au final : $\dfrac{\partial E(a,b,c)}{\partial b} = 0 
\Leftrightarrow   \sum\limits_{i=1}^{n}  \left(-z_i +ax_i +by_i +c\right) = 0$

\end{corrige}

\else
\fi

\vspace{.5cm}
On définit les grandeurs suivantes : 

$S_{xx} = \sum\limits_{i=1}^n x_i^2$, $S_{yy} = \sum\limits_{i=1}^n y_i^2$, $S_{xy} = \sum\limits_{i=1}^n x_i y_i$, $S_{xz} = \sum\limits_{i=1}^n x_i z_i$, $S_{yz} = \sum\limits_{i=1}^n y_i z_i$, 
$S_x  = \sum\limits_{i=1}^n  x_i$, $S_y  = \sum\limits_{i=1}^n  y_i$ et $S_z  = \sum\limits_{i=1}^n  z_i $.

%\begin{center}
%\begin{tabular}{ccc}
%$S_{xx} = \sum\limits_{i=1}^n x_i^2$ && $S_{yy} = \sum\limits_{i=1}^n y_i^2$  \\
%$S_{xy} = \sum\limits_{i=1}^n x_i y_i$ & $S_{xz} = \sum\limits_{i=1}^n x_i z_i$ & $S_{yz} = \sum\limits_{i=1}^n y_i z_i$  \\
%$S_x  = \sum\limits_{i=1}^n  x_i$ & $S_y  = \sum\limits_{i=1}^n  y_i$ & $S_y  = \sum\limits_{i=1}^n  x_i $\\
%\end{tabular}
%\end{center}

On note $X=(a,b,c)$ le vecteur solution du problème avec $X\in \mathcal{M}_{1,3} \left(\mathbb{R} \right)$.


\subparagraph{}
\textit{Montrer que le problème peut se mettre sous la forme du système linéaire suivant : $AX + B = 0$  où $A \in \mathcal{M}_{3,3} \left(\mathbb{R} \right)$ et  $B \in \mathcal{M}_{1,3} \left(\mathbb{R} \right)$. On donnera les expressions de $A$ et de $B$.}

\ifprof
\begin{corrige}
On a : 
$$
\sum\limits_{i=1}^{n}   \left(-x_iz_i +ax_i^2 +bx_i y_i +cx_i\right) = 0 
\Leftrightarrow    - \sum\limits_{i=1}^{n} x_iz_i +\sum\limits_{i=1}^{n}ax_i^2 +\sum\limits_{i=1}^{n}bx_i y_i +\sum\limits_{i=1}^{n}cx_i  = 0 
$$
$$
\Leftrightarrow    a\sum\limits_{i=1}^{n}x_i^2 +b\sum\limits_{i=1}^{n}x_i y_i +c\sum\limits_{i=1}^{n}x_i  = \sum\limits_{i=1}^{n} x_iz_i
\Leftrightarrow    aS_{xx} +bS_{xy} +cS_x  =S_{xz}
$$
De même, on détermine que : 
$$
\sum\limits_{i=1}^{n}     \left(-y_iz_i + ax_i y_i  +by_i^2 +cy_i\right)  = 0
\Leftrightarrow aS_{xy} + bS_{yy}+cS_y = S_{yz}
$$

$$
\sum\limits_{i=1}^{n}  \left(-z_i +ax_i +by_i +c\right) = 0
\Leftrightarrow aS_{x} + bS_{y}+cn = S_{z}
$$
On a donc : 
$$
\left\{
\begin{array}{l}
aS_{xx} +bS_{xy} +cS_x  =S_{xz} \\
aS_{xy} + bS_{yy}+cS_y = S_{yz} \\
aS_{x} + bS_{y}+cn = S_{z}
\end{array}
\right.
\Leftrightarrow
\left[ 
\begin{array}{ccc}
S_{xx} & S_{xy} & S_x \\
S_{xy} & S_{yy} & S_y \\
S_{x} & S_{y} & n \\
\end{array}
\right]
\cdot 
\left[ 
\begin{array}{c}
a \\b \\ c\end{array}
\right]
=
\left[ 
\begin{array}{c}
S_{xz} \\
S_{yz} \\ 
S_{z}\end{array}
\right]
\Leftrightarrow
AX = -B
$$

Le signe négatif provient de la formulation du problème dans la question. 

\end{corrige}

\else
\fi


\subsection{Résolution du problème}
\ifprof
\else
\begin{methode}
Au final, pour déterminer le plan des moindres carrés, la démarche est la suivante :
\begin{itemize}
\item mesurer les points$(x_i,y_i,z_i)$ grâce à la MMT et exporter les données dans un fichier (ici un fichier texte);
\item importer le fichier de points avec Python;
\item résoudre le système linéaire $AX+B=0$ pour déterminer $a$, $b$ et $c$ permettant d'obtenir l'équation du plan. 
\end{itemize}
\end{methode}
\fi

\subparagraph{}
\textit{Donner une méthode (ou le nom d'un algorithme) permettant de résoudre le système linéaire ci-dessus. Préciser les différentes étapes de cet algorithme ainsi que sa complexité.}


\ifprof
\begin{corrige}
Pour résoudre ce problème, il est possible d'utiliser l'algorithme du pivot de Gauss. Cet algorithme se déroule en deux phases majeures : 
\begin{enumerate}
\item la phase de la triangularisation de la matrice $A$ (en phase avec les transformations nécessaires sur la matrice $B$);
\item la phase de remontée permettant de déterminer $X$. 
\end{enumerate}

On peut montrer que la complexité algorithmique du pivot de Gauss est en $\mathcal{O}\left(n^3\right)$.
\end{corrige}
\else
\fi
\ifprof
\else
\vspace{.5cm}

Les points mesurés par une machine à mesurer tridimensionnelle sont stockés dans un fichier texte. Dans ce fichier sont inscrits successivement les coordonnés des points puis les coordonnées de la normale de contact entre le palpeur et la surface mesurée. Le fichier est de la forme suivante : 

\begin{tabular}{l}
\texttt{-101.88340,-155.21568,-50.30434,0,0,1}\\
\texttt{-99.21040,-145.54768,-50.304844,0,0,1}\\
\texttt{-82.43090,-129.69318,-50.292844,0,0,1}\\
\texttt{-59.72540,-134.12818,-50.301844,0,0,1}\\
\texttt{...} \\
\end{tabular}

\begin{rappel}
La fonction \texttt{split} permet de séparer les éléments d'une chaîne de caractère suivant un motif et de les stocker dans une liste :
\begin{python}
>>> a = "-101.88340,-155.21568,-50.30434,0,0,1"
>>> a = a.split(",")
           ['-59.72540', '-134.12818', '-50.301844', '0', '0', '1']
\end{python}
\end{rappel}
\fi

\subparagraph{}
\textit{Donner l'implémentation de la fonction \texttt{read\_file} permettant de lire un fichier de mesures formaté comme indiqué ci-dessus et permettant de retourner la liste des points mesurés.}
\vspace{.5cm}

\ifprof
\begin{corrige}
~\\
\begin{python}
def read_file(file):
    fid = open(file,'r')
    pts=[]
    for ligne in fid:
        ligne = ligne.split(",")
        print(ligne)
        pts.append([float(ligne[0]),float(ligne[1]),float(ligne[2])])
    return pts
\end{python}
\end{corrige}
\else
\fi

\ifprof
\else

On appelle \texttt{plan\_moindres\_carres} la fonction permettant de trouver la solution du système linéaire à partir d'une liste de points. Ces spécifications sont les suivantes :
\begin{py}
\begin{python}
def plan_moindres_carres(liste_pts):
    """
    Permet de déterminer les caractéristiques du plan des moindres carrés.
    Entrée : 
        * liste_pts (list) : liste de points de la forme [[x1,y1,z1],[x2,y2,z2],...]
    Sortie : 
        * pl(list) : solution du problème d'optimisation : retourne [a,b,c]
    """
\end{python}
\end{py}
\fi

\subparagraph{}
\textit{Donner l'implémentation du programme principal (main) permettant de lire le fichier de mesure appelé \texttt{mesures.txt} et de déterminer la liste des paramètres \texttt{[a,b,c]} correspondant aux paramètres du plan des moindres carrés. Vous utiliserez les fonctions précédemment définies.}
\ifprof
\begin{corrige}~\\
\begin{python}
liste_pts=read_file("mesures.txt")
plan = plan_moindres_carres(liste_pts)
\end{python}
\end{corrige}
\else
\fi


\subsection{Application -- Détermination du défaut de planéité}

\ifprof
\else

Pour déterminer le défaut de planéité d'une forme, il est nécessaire de déterminer la distance maximale entre les points les plus éloignés de part et d'autre du plan des moindres carrés.

On définit la distance algébrique $d$ d'un point $M$ de coordonnées $\left(x,y,z\right)$ au plan $\left( \mathcal{P}\right)$ par l'expression suivante :
$d = \dfrac{z - a x -b y -c}{\sqrt{a^2+b^2+1}}$.

On notera \texttt{pt} la liste contenant les coordonnées d'un point et \texttt{pl} la liste contenant le triplet \texttt{(a,b,c)}.

\fi 

\subparagraph{}
\textit{Donner l'implémentation de la fonction  \texttt{dist\_pt\_plan} en Python permettant de retourner la distance algébrique entre un point et un plan. Les spécifications de la fonction sont les suivantes :}
\begin{py}
\begin{python}
def dist_pt_plan(pt,plan):
    """
    Permet de calculer une distance point - plan
    Entrées : 
        * pt(list) : point de coordonnées [x,y,z]
        * plan(list) : caractéristiques du plan [a,b,c]
    Sortie : 
        * d(flt) : distance
    """
\end{python}
\end{py}
\ifprof
\begin{corrige}~\\
\begin{python}
def dist_pt_plan(pt,pl):
    return (pt[2]-pl[0]*pt[0]-pl[1]*pt[1]-pl[2])/(sqrt(pl[0]**2+pl[1]**2+1))
\end{python}
\end{corrige}
\else
\fi

\subparagraph{}
\textit{Donner l'implémentation de la fonction \texttt{defaut\_planeite} en Python permettant de retourner le défaut de planéité. On donne les spécifications de la fonction : }
\begin{py}
\begin{python} 
def defaut_planeite(pl,liste_pt):
    """
    Permet de calculer une distance point - plan
    Entrées : 
        * pl(list) : caractéristiques du plan [a,b,c]
        * liste_pts (list) : liste de points de la forme [[x1,y1,z1],[x2,y2,z2],...]
        
    Sortie : 
        * d(flt) : distance
    """
\end{python}
\end{py}
\ifprof
\begin{corrige}~\\
\begin{python}
def defaut_planeite(pl,liste_pt):
    dist_p=0
    dist_n=0
    for i in range(len(liste_pt)):
        dist=dist_pt_plan(liste_pt[i],pl)
        if dist_p<dist and dist>0 :
            dist_p=dist
        elif dist_n>dist and dist<0 :
            dist_n=dist
    return dist_p-dist_n
\end{python}
\end{corrige}
\else
\fi

\section{Construction d'une référence spécifiée associée à une surface nominalement plane}

\ifprof
\else

Pour associer un plan idéal à un nuage de points, la norme prévoit de lui associé un plan tangent extérieur matière minimisant les écarts. Mathématiquement, ce problème n'est pas simple à résoudre. Afin d'approcher la solution, on se propose d'utiliser, dans le cadre ce travail,  la méthode suivante : 

\begin{methode}
~\\
\begin{enumerate}
\item Détermination du plan par la méthode des moindres carrés.
\item Translation du plan au point le plus éloigné côté libre de la matière.
\item Balancement du plan afin de minimiser les défauts : on fait varier son orientation.
\item Vérification que tous les points sont situés sous le plan.
\item Changement au point le plus éloigné situé côté libre de la matière s'il y en a.
\end{enumerate}
Les étapes 4 et 5 peuvent être itératives.
\begin{warn}
\textbf{\textsf{Attention :}} les points 3, 4 et 5 ne sont qu'une piste envisagée dans le cadre de ce travail et ne constituent pas un algorithme utilisé dans les logiciels de mesure.
\end{warn}
\end{methode}

\newpage 

On donne la fonction suivante :

\begin{py}
\begin{python}
def mystere(pl,liste_pt):
    ind=0
    d=dist_pt_plan(liste_pt[ind],pl)
    for i in range(1,len(liste_pt)):
        temp=dist_pt_plan(liste_pt[i],pl)
        print(temp)
        if temp>d :
            ind=i
            d=dist_pt_plan(liste_pt[ind],pl)
    return ind
\end{python}
\end{py}

\fi 

\subparagraph{}
\textit{Quel est l'objectif de cette fonction ? En déduire le triplet correspondant à l'équation du plan répondant au point \textbf{2} de la méthode. }
\ifprof
\begin{corrige}
Cette fonction a pour but de retrouver l'indice du point étant le plus éloigné du plan des moindres carrés. Ce point est aussi du coté libre de la matière.

Le nouveau plan a donc pour caractéristiques : 

\texttt{[a,b,liste\_pt[ind][2]-X[0]*liste\_pt[ind][0]-X[1]*liste\_pt[ind][1]}.
\end{corrige}
\else
\fi

\ifprof
\else
Le balançage du plan par rapport à un point est réalisé grâce à la fonction suivante. 

\begin{py}
\begin{python}
def balancage(pl, pt):
    npt=25
    pas=0.000003
    dist=100000
    pl2=[None,None,None]
    for k in range(-npt,npt):
        pl2[0]=pl[0]+k*pas
        for j in range(-npt,npt):
            pl2[1]=pl[1]+j*pas
            pl2[2]=pt[2]-pl2[0]*pt[0]-pl2[1]*pt[1]
            temp=defaut_planeite(pl2,liste_pt)	
            if temp<dist :
                dist=temp
                f=pl2[0]
                g=pl2[1]
                h=pl2[2]
    pl2=[f,g,h]    
    return [pl2,dist]	
\end{python}
\end{py}

\fi

\subparagraph{}
\textit{Quel est l'objectif de cette fonction ? Que retourne-t-elle ? Donner un commentaire pour chacune des instructions.}
\ifprof
\begin{corrige}
~\\
\begin{minipage}[c]{.57\linewidth}
\begin{python}
 1. def balancage(pl, pt):
 2.    npt=25                            #
 3.    pas=0.000003                      #
 4.    dist=100000                       #
 5.    pl2=[None,None,None]              #
 6.    for k in range(-npt,npt):         #
 7.        pl2[0]=pl[0]+k*pas            #
 8.        for j in range(-npt,npt):     #
 9.            pl2[1]=pl[1]+j*pas        #
10.            pl2[2]=pt[2]-pl2[0]*pt[0]-pl2[1]*pt[1]  #
11.            temp=defaut_planeite(pl2,liste_pt)	   #
12.            if temp<dist :         #
13.                dist=temp          #
14.                f=pl2[0]           #
15.                g=pl2[1]           #
16.                h=pl2[2]           #
17.    pl2=[f,g,h]                    #    
18.    return [pl2,dist]	      #
\end{python}
\end{minipage}\hfill
\begin{minipage}[c]{.43\linewidth}
L'objectif de cette fonction est de déterminer une référence spécifiée en faisant varier l'orientation du plan des moindres carrés. Elle retourne le plan optimal ainsi que le défaut de planéité lié au nuage de point.

Les boucles \textsl{for} lignes 6 et 8 permettent de réaliser des petite variation autour d'un point \textsl{pt} prédéterminé. 
Ligne 10 on dispose alors des caractéristiques d'un plan ayant subi deux petites rotations. 
On détermine alors le défaut de planéité lié à ce nouveau plan. 

On cherche alors le nouveau plan << minimisant le défaut de planéité >> (en fait on ne minimise pas le défaut mais l'écart...).

\end{minipage}
\end{corrige}
\else
\fi

\subparagraph{}
\textit{Quelle est la complexité de cet algorithme si on cherche à améliorer le choix du plan optimal par rapport à un seul des points du nuage de points ? Comment évolue la complexité de ce programme si on cherche à réaliser le balançage en utilisant chacun des points mesurés ?}
\ifprof
\begin{corrige}
Pour avoir une <<meilleure>> solution du problème, on peut augmenter le nombre de pas de calcul. Les deux boucles imbriquées dépendant du nombre de points, on a une complexité en $\mathcal{O}(npt^2)$.

On note $n$ le nombre points palpés. Si on réalise un balançage sur chacun des points du nuage, la complexité sera en $\mathcal{O}(n\cdot npt^2)$
\end{corrige}
\else
\fi


\end{document}