\documentclass[10pt,fleqn]{article} % Default font size and left-justified equations
\usepackage[%
    pdftitle={Informatique : Enveloppe convexe},
    pdfauthor={Xavier Pessoles}]{hyperref}

%%%%%%%%%%%%%%%%%%%%%%%%%%%%%%%%%%%%%%%%%
% Original author:
% Mathias Legrand (legrand.mathias@gmail.com) with modifications by:
% Vel (vel@latextemplates.com)
% License:
% CC BY-NC-SA 3.0 (http://creativecommons.org/licenses/by-nc-sa/3.0/)
%%%%%%%%%%%%%%%%%%%%%%%%%%%%%%%%%%%%%%%%%



%----------------------------------------------------------------------------------------
%	MAIN TABLE OF CONTENTS
%----------------------------------------------------------------------------------------


% Part text styling
\titlecontents{part}[0cm]
{\addvspace{20pt}\centering\large\bfseries}
{}
{}
{}

% Chapter text styling
\titlecontents{chapter}[1.25cm] % Indentation
{\addvspace{12pt}\large\sffamily\bfseries} % Spacing and font options for chapters
{\color{bleuxp!60}\contentslabel[\Large\thecontentslabel]{1.25cm}\color{bleuxp}} % Chapter number
{\color{bleuxp}}  
{\color{bleuxp!60}\normalsize\;\titlerule*[.5pc]{.}\;\thecontentspage} % Page number

% Section text styling
\titlecontents{section}[1.25cm] % Indentation
{\addvspace{3pt}\sffamily\bfseries} % Spacing and font options for sections
{\color{bleuxp!60}\contentslabel[\thecontentslabel]{1.25cm} \color{bleuxp}} % Section number
{\color{bleuxp}}
{\hfill\color{bleuxp!60}\thecontentspage} % Page number
[]

% Subsection text styling
\titlecontents{subsection}[1.25cm] % Indentation
{\addvspace{1pt}\sffamily\small} % Spacing and font options for subsections
{\contentslabel[\thecontentslabel]{1.25cm}} % Subsection number
{}
{\ \titlerule*[.5pc]{.}\;\thecontentspage} % Page number
[]


% Subsection text styling
\titlecontents{subsubsection}[1.25cm] % Indentation
{\addvspace{1pt}\sffamily\small} % Spacing and font options for subsections
{\contentslabel[\thecontentslabel]{1.25cm}} % Subsection number
{}
{\ \titlerule*[.5pc]{.}\;\thecontentspage} % Page number
[]

% List of figures
\titlecontents{figure}[0em]
{\addvspace{-5pt}\sffamily}
{\thecontentslabel\hspace*{1em}}
{}
{\ \titlerule*[.5pc]{.}\;\thecontentspage}
[]

% List of tables
\titlecontents{table}[0em]
{\addvspace{-5pt}\sffamily}
{\thecontentslabel\hspace*{1em}}
{}
{\ \titlerule*[.5pc]{.}\;\thecontentspage}
[]

%----------------------------------------------------------------------------------------
%	MINI TABLE OF CONTENTS IN PART HEADS
%----------------------------------------------------------------------------------------

% Chapter text styling
\titlecontents{lchapter}[0em] % Indenting
{\addvspace{15pt}\large\sffamily\bfseries} % Spacing and font options for chapters
{\color{bleuxp}\contentslabel[\Large\thecontentslabel]{1.25cm}\color{bleuxp}} % Chapter number
{}  
{\color{bleuxp}\normalsize\sffamily\bfseries\;\titlerule*[.5pc]{.}\;\thecontentspage} % Page number

% Section text styling
\titlecontents{lsection}[0em] % Indenting
{\sffamily\small} % Spacing and font options for sections
{\contentslabel[\thecontentslabel]{1.25cm}} % Section number
{}
{}

% Subsection text styling
\titlecontents{lsubsection}[.5em] % Indentation
{\normalfont\footnotesize\sffamily} % Font settings
{}
{}
{}

%----------------------------------------------------------------------------------------
%	PAGE HEADERS
%----------------------------------------------------------------------------------------




\pagestyle{fancy}
 \renewcommand{\headrulewidth}{0pt}
 \fancyhead{}
 
 % ENTETES de page
 \fancyhead[L]{%
 \begin{tikzpicture}[overlay]
\node(logo) at (1,0)
    {\includegraphics[width=2cm]{logo_lycee.png}};
\end{tikzpicture}
 %\noindent\begin{minipage}[c]{2.6cm}%
 %\includegraphics[width=2cm]{logo_lycee.png}%
 %\end{minipage}
}

\fancyhead[C]{\rule{8cm}{.5pt}}

 \fancyhead[R]{%
 \noindent\begin{minipage}[c]{3cm}
 \begin{flushright}
 \footnotesize{\textit{\textsf{\xxtete}}}%
 \end{flushright}
 \end{minipage}
}

 \fancyfoot{}
 % PIEDS de page
\fancyfoot[C]{\rule{12cm}{.5pt}}
\renewcommand{\footrulewidth}{0.2pt}
\fancyfoot[C]{\footnotesize{\bfseries \thepage}}
\fancyfoot[L]{ 
\begin{minipage}[c]{.4\linewidth}
\noindent\footnotesize{{\xxauteur}}
\end{minipage}}

\fancyfoot[R]{\footnotesize{\xxpied}
\ifthenelse{\isodd{\value{page}}}{
\begin{tikzpicture}[overlay]
\node[shape=rectangle, 
      rounded corners = .25 cm,
	  draw= bleuxp,
	  line width=2pt, 
	  fill = bleuxp!10,
	  minimum width  = 2.5cm,
	  minimum height = 3cm,] at (\xxposongletx,\xxposonglety) {};
\node at (\xxposonglettext,\xxposonglety) {\rotatebox{90}{\textbf{\large\color{bleuxp}{\xxonglet}}}};
%{};
\end{tikzpicture}}{}
}



%
%
%
% Removes the header from odd empty pages at the end of chapters
\makeatletter
%\renewcommand{\cleardoublepage}{
%\clearpage\ifodd\c@page\else
%\hbox{}
%\vspace*{\fill}
%\thispagestyle{empty}
%\newpage
%\fi}

%\fancypagestyle{plain}{%
%\fancyhf{} % vide l’en-tête et le pied~de~page.
%%\fancyfoot[C]{\bfseries \thepage} % numéro de la page en cours en gras
%% et centré en pied~de~page.
%\fancyfoot[R]{\footnotesize{\xxpied}}
%\fancyfoot[C]{\rule{12cm}{.5pt}}
%\renewcommand{\footrulewidth}{0.2pt}
%\fancyfoot[C]{\footnotesize{\bfseries \thepage}}
%\fancyfoot[L]{ 
%\begin{minipage}[c]{.4\linewidth}
%\noindent\footnotesize{{\xxauteur}}
%\end{minipage}}}

\fancypagestyle{plain}{%
\fancyhf{} % vide l’en-tête et le pied~de~page.
\fancyfoot[C]{\rule{12cm}{.5pt}}
\renewcommand{\footrulewidth}{0.2pt}
\fancyfoot[C]{\footnotesize{\bfseries \thepage}}
\fancyfoot[L]{ 
\begin{minipage}[c]{.4\linewidth}
\noindent\footnotesize{{\xxauteur}}
\end{minipage}}
\fancyfoot[R]{\footnotesize{\xxpied}}
}







%----------------------------------------------------------------------------------------
%	SECTION NUMBERING IN THE MARGIN
%----------------------------------------------------------------------------------------
\setcounter{secnumdepth}{3}
\setcounter{tocdepth}{2}



\makeatletter
\renewcommand{\@seccntformat}[1]{\llap{\textcolor{bleuxp}{\csname the#1\endcsname}\hspace{1em}}}                    
\renewcommand{\section}{\@startsection{section}{1}{\z@}
{-4ex \@plus -1ex \@minus -.4ex}
{1ex \@plus.2ex }
{\normalfont\large\sffamily\bfseries}}
\renewcommand{\subsection}{\@startsection {subsection}{2}{\z@}
{-3ex \@plus -0.1ex \@minus -.4ex}
{0.5ex \@plus.2ex }
{\normalfont\sffamily\bfseries}}
\renewcommand{\subsubsection}{\@startsection {subsubsection}{3}{\z@}
{-2ex \@plus -0.1ex \@minus -.2ex}
{.2ex \@plus.2ex }
{\normalfont\small\sffamily\bfseries}}                        
\renewcommand\paragraph{\@startsection{paragraph}{4}{\z@}
{-2ex \@plus-.2ex \@minus .2ex}
{.1ex}
{\normalfont\small\sffamily\bfseries}}

%----------------------------------------------------------------------------------------
%	PART HEADINGS
%----------------------------------------------------------------------------------------


%----------------------------------------------------------------------------------------
%	CHAPTER HEADINGS
%----------------------------------------------------------------------------------------

% \newcommand{\thechapterimage}{}%
% \newcommand{\chapterimage}[1]{\renewcommand{\thechapterimage}{#1}}%
% \def\@makechapterhead#1{%
% {\parindent \z@ \raggedright \normalfont
% \ifnum \c@secnumdepth >\m@ne
% \if@mainmatter
% \begin{tikzpicture}[remember picture,overlay]
% \node at (current page.north west)
% {\begin{tikzpicture}[remember picture,overlay]
% \node[anchor=north west,inner sep=0pt] at (0,0) {\includegraphics[width=\paperwidth]{\thechapterimage}};
% \draw[anchor=west] (\Gm@lmargin,-9cm) node [line width=2pt,rounded corners=15pt,draw=bleuxp,fill=white,fill opacity=0.5,inner sep=15pt]{\strut\makebox[22cm]{}};
% \draw[anchor=west] (\Gm@lmargin+.3cm,-9cm) node {\huge\sffamily\bfseries\color{black}\thechapter. #1\strut};
% \end{tikzpicture}};
% \end{tikzpicture}
% \else
% \begin{tikzpicture}[remember picture,overlay]
% \node at (current page.north west)
% {\begin{tikzpicture}[remember picture,overlay]
% \node[anchor=north west,inner sep=0pt] at (0,0) {\includegraphics[width=\paperwidth]{\thechapterimage}};
% \draw[anchor=west] (\Gm@lmargin,-9cm) node [line width=2pt,rounded corners=15pt,draw=bleuxp,fill=white,fill opacity=0.5,inner sep=15pt]{\strut\makebox[22cm]{}};
% \draw[anchor=west] (\Gm@lmargin+.3cm,-9cm) node {\huge\sffamily\bfseries\color{black}#1\strut};
% \end{tikzpicture}};
% \end{tikzpicture}
% \fi\fi\par\vspace*{270\p@}}}

%-------------------------------------------

\def\@makeschapterhead#1{%
\begin{tikzpicture}[remember picture,overlay]
\node at (current page.north west)
{\begin{tikzpicture}[remember picture,overlay]
\node[anchor=north west,inner sep=0pt] at (0,0) {\includegraphics[width=\paperwidth]{\thechapterimage}};
\draw[anchor=west] (\Gm@lmargin,-9cm) node [line width=2pt,rounded corners=15pt,draw=bleuxp,fill=white,fill opacity=0.5,inner sep=15pt]{\strut\makebox[22cm]{}};
\draw[anchor=west] (\Gm@lmargin+.3cm,-9cm) node {\huge\sffamily\bfseries\color{black}#1\strut};
\end{tikzpicture}};
\end{tikzpicture}
\par\vspace*{270\p@}}
\makeatother



%----------------------------------------------------------------------------------------
%	
%----------------------------------------------------------------------------------------

\newcommand{\thechapterimage}{}%
\newcommand{\chapterimage}[1]{\renewcommand{\thechapterimage}{#1}}%
\def\@makechapterhead#1{%
{\parindent \z@ \raggedright \normalfont
\begin{tikzpicture}[remember picture,overlay]
\node at (current page.north west)
{\begin{tikzpicture}[remember picture,overlay]
\node[anchor=north west,inner sep=0pt] at (0,0) {\includegraphics[width=\paperwidth]{\thechapterimage}};
%\draw[anchor=west] (\Gm@lmargin,-9cm) node [line width=2pt,rounded corners=15pt,draw=bleuxp,fill=white,fill opacity=0.5,inner sep=15pt]{\strut\makebox[22cm]{}};
%\draw[anchor=west] (\Gm@lmargin+.3cm,-9cm) node {\huge\sffamily\bfseries\color{black}\thechapter. #1\strut};
\end{tikzpicture}};
\end{tikzpicture}
\par\vspace*{270\p@}
}}


%% Questions et exercices
\newcounter{numques}%Création d'un compteur qui s'appelle numques
\setcounter{numques}{0}%initialisation du compteur
\newcommand{\question}[1]{%Création d'une macro ayant un paramètre
\addtocounter{numques}{1}%chaque fois que cette macro est appelée, elle ajoute 1 au compteur numexos
\textbf{Question\, \textcolor{bleuxp}{\thenumques}\,}\,\textit{#1}}

\newcounter{numexo}%Création d'un compteur qui s'appelle numques
\setcounter{numexo}{0}%initialisation du compteur
\newcommand{\exer}[1]{%Création d'une macro ayant un paramètre
\refstepcounter{numexo} % incrément compteur et label
%\addtocounter{numexo}{1}%chaque fois que cette macro est appelée, elle ajoute 1 au compteur numexo
\noindent\textsf{\textbf{Exercice\, \textcolor{bleuxp}{\thenumexo}\, -- \, #1}}}



% \makeatletter             
% \renewcommand{\subparagraph}{\@startsection{exo}{5}{\z@}%
                                    % {-2ex \@plus-.2ex \@minus .2ex}%
                                    % {0ex}%               
% {\normalfont\bfseries Question \hspace{.7cm} }}
% \makeatother
% \renewcommand{\thesubparagraph}{\arabic{subparagraph}} 
% \makeatletter


%%%%%%%%%%%%
% Définition des vecteurs 
%%%%%%%%%%%%
\newcommand{\vect}[1]{\overrightarrow{#1}}
\newcommand{\axe}[2]{\left(#1,\vect{#2}\right)}
\newcommand{\couple}[2]{\left(#1,\vect{#2}\right)}
\newcommand{\angl}[2]{\left(\vect{#1},\vect{#2}\right)}

\newcommand{\rep}[1]{\mathcal{R}_{#1}}
\newcommand{\quadruplet}[4]{\left(#1;#2,#3,#4 \right)}
\newcommand{\repere}[4]{\left(#1;\vect{#2},\vect{#3},\vect{#4} \right)}
\newcommand{\base}[3]{\left(\vect{#1},\vect{#2},\vect{#3} \right)}


\newcommand{\vx}[1]{\vect{x_{#1}}}
\newcommand{\vy}[1]{\vect{y_{#1}}}
\newcommand{\vz}[1]{\vect{z_{#1}}}

\newcommand{\norm}[1]{\ensuremath{\left\Vert {#1}\right\Vert}}
\newcommand{\Ker}{\mathop{\mathrm{Ker}}\nolimits}

% d droit pour le calcul différentiel
\newcommand{\dd}{\text{d}}

\newcommand{\inertie}[2]{I_{#1}\left( #2\right)}
\newcommand{\matinertie}[7]{
\begin{pmatrix}
#1 & #6 & #5 \\
#6 & #2 & #4 \\
#5 & #4 & #3 \\
\end{pmatrix}_{#7}}
%%%%%%%%%%%%
% Définition des torseurs 
%%%%%%%%%%%%

\newcommand{\ec}[2]{%
\mathcal{E}_c\left(#1/#2\right)}

\newcommand{\pext}[3]{%
\mathcal{P}\left(#1\rightarrow#2/#3\right)}

\newcommand{\pint}[3]{%
\mathcal{P}\left(#1 \stackrel{\text{#3}}{\leftrightarrow} #2\right)}


 \newcommand{\torseur}[1]{%
\left\{{#1}\right\}
}

\newcommand{\torseurcin}[3]{%
\left\{\mathcal{#1} \left(#2/#3 \right) \right\}
}

\newcommand{\torseurci}[2]{%
\left\{\sigma \left(#1/#2 \right) \right\}
}
\newcommand{\torseurdyn}[2]{%
\left\{\mathcal{D} \left(#1/#2 \right) \right\}
}


\newcommand{\torseurstat}[3]{%
\left\{\mathcal{#1} \left(#2\rightarrow #3 \right) \right\}
}


 \newcommand{\torseurc}[8]{%
%\left\{#1 \right\}=
\left\{
{#1}
\right\}
 = 
\left\{%
\begin{array}{cc}%
{#2} & {#5}\\%
{#3} & {#6}\\%
{#4} & {#7}\\%
\end{array}%
\right\}_{#8}%
}

 \newcommand{\torseurcol}[7]{
\left\{%
\begin{array}{cc}%
{#1} & {#4}\\%
{#2} & {#5}\\%
{#3} & {#6}\\%
\end{array}%
\right\}_{#7}%
}

 \newcommand{\torseurl}[3]{%
%\left\{\mathcal{#1}\right\}_{#2}=%
\left\{%
\begin{array}{l}%
{#1} \\%
{#2} %
\end{array}%
\right\}_{#3}%
}

% Vecteur vitesse
 \newcommand{\vectv}[3]{%
\vect{V\left( {#1} \in {#2}/{#3}\right)}
}

% Vecteur force
\newcommand{\vectf}[2]{%
\vect{R\left( {#1} \rightarrow {#2}\right)}
}

% Vecteur moment stat
\newcommand{\vectm}[3]{%
\vect{\mathcal{M}\left( {#1}, {#2} \rightarrow {#3}\right)}
}




% Vecteur résultante cin
\newcommand{\vectrc}[2]{%
\vect{R_c \left( {#1}/ {#2}\right)}
}
% Vecteur moment cin
\newcommand{\vectmc}[3]{%
\vect{\sigma \left( {#1}, {#2} /{#3}\right)}
}


% Vecteur résultante dyn
\newcommand{\vectrd}[2]{%
\vect{R_d \left( {#1}/ {#2}\right)}
}
% Vecteur moment dyn
\newcommand{\vectmd}[3]{%
\vect{\delta \left( {#1}, {#2} /{#3}\right)}
}

% Vecteur accélération
 \newcommand{\vectg}[3]{%
\vect{\Gamma \left( {#1} \in {#2}/{#3}\right)}
}

% Vecteur omega
 \newcommand{\vecto}[2]{%
\vect{\Omega\left( {#1}/{#2}\right)}
}
% }$$\left\{\mathcal{#1} \right\}_{#2} =%
% \left\{%
% \begin{array}{c}%
%  #3 \\%
%  #4 %
% \end{array}%
% \right\}_{#5}}

\newcommand{\N}{\mathbb{N}}
\newcommand{\Z}{\mathbb{Z}}
\newcommand{\R}{\mathbb{R}}
\newcommand{\C}{\mathbb{C}}
\newcommand{\K}{\mathbb{K}}

\newcommand{\cA}{\mathscr{A}}
\newcommand{\cM}{\mathscr{M}}
\newcommand{\cL}{\mathscr{L}}
\newcommand{\cS}{\mathscr{S}}

\newcommand{\python}{\texttt{Python}}

\newcommand{\z}[1]{\Z_{#1}}
\newcommand{\ztimes}[1]{\Z_{#1}^{\times}}
\newcommand{\ii}[1]{[\![#1[\![}
\newcommand{\iif}[1]{[\![#1]\!]}
\newcommand{\llbr}{\ensuremath{\llbracket}}
\newcommand{\rrbr}{\ensuremath{\rrbracket}}
%\newcommand{\p}[1]{\left(#1\right)}
\newcommand{\ens}[1]{\left\{ #1 \right\}}
\newcommand{\croch}[1]{\left[ #1 \right]}
%\newcommand{\of}[1]{\lstinline{#1}}
% \newcommand{\py}[2]{%
%   \begin{tabular}{|l}
%     \lstinline+>>>+\textbf{\of{#1}}\\
%     \of{#2}
%   \end{tabular}\par{}
% }
\newcommand{\floor}[1]{\left\lfloor#1\right\rfloor}
\newcommand{\ceil}[1]{\left\lceil#1\right\rceil}
\newcommand{\abs}[1]{\left|#1\right|}


% Binaire, octal, hexa
\newcommand{\hex}[1]{\underline{\text{\texttt{#1}}}_{16}}
\newcommand{\oct}[1]{\underline{\text{\texttt{#1}}}_{8}}
\newcommand{\bin}[1]{\underline{\text{\texttt{#1}}}_{2}}
\DeclareMathOperator{\mmod}{\texttt{\%}}


% Fonctions et systèmes
\newcommand{\fct}[5][t]{%
  \begin{array}[#1]{rcl}
    #2 & \rightarrow & #3\\
    #4 & \mapsto     & #5\\
  \end{array}
}
\newcommand{\fonction}[5]{#1 : \left\{\begin{array}{rcl} #2& \longrightarrow &#3 \\ #4 &\longmapsto & #5\end{array}\right.}
\newenvironment{systeme}{\left\{ \begin{array}{rcl}}{\end{array}\right.}

% Matrices
\newcommand{\mat}[1]{
  \begin{pmatrix}
    #1
  \end{pmatrix}
}
\newcommand{\inv}{\ensuremath{^{-1}}}
\newcommand{\bpm}{\begin{pmatrix}}
\newcommand{\epm}{\end{pmatrix}}


% bases de données
\newcommand{\relat}[1]{\textsc{#1}}
\newcommand{\attr}[1]{\emph{#1}}
\newcommand{\prim}[1]{\uline{#1}}
\newcommand{\foreign}[1]{\#\textsl{#1}}


% Bases de données

\newcommand{\att}{\ensuremath{\mathbf{att}}}
\newcommand{\dom}{\ensuremath{\mathbf{dom}}}
\newcommand{\sort}{\ensuremath{\mathbf{sort}}}
\newcommand{\relname}{\ensuremath{\mathbf{relname}}}
\newcommand{\var}{\ensuremath{\mathbf{var}}}
\newcommand{\FILM}{\ensuremath{\mathtt{FILM}}}
\newcommand{\JOUE}{\ensuremath{\mathtt{JOUE}}}
\newcommand{\PERSONNE}{\ensuremath{\mathtt{PERSONNE}}}
\newcommand{\PERSONNAGE}{\ensuremath{\mathtt{PERSONNAGE}}}

\newcommand{\ttid}{\ensuremath{\mathtt{id}}}
\newcommand{\tttitre}{\ensuremath{\mathtt{titre}}}
\newcommand{\ttdate}{\ensuremath{\mathtt{date}}}
\newcommand{\ttidr}{\ensuremath{\mathtt{idrealisateur}}}
\newcommand{\ttdatenais}{\ensuremath{\mathtt{datenaissance}}}
\newcommand{\ttnom}{\ensuremath{\mathtt{nom}}}
\newcommand{\ttprenom}{\ensuremath{\mathtt{prenom}}}
\newcommand{\ttidacteur}{\ensuremath{\mathtt{idacteur}}}
\newcommand{\ttidfilm}{\ensuremath{\mathtt{idfilm}}}
\newcommand{\ttidpersonnage}{\ensuremath{\mathtt{idpersonnage}}}

\newcommand{\fv}{\mathrm{libre}}
\newcommand{\sem}[1]{[\![ #1 ]\!]}


\fichetrue
%\fichefalse

\proftrue
%\proffalse

%\tdtrue
\tdfalse

%\courstrue
\coursfalse

% -------------------------------------
% Déclaration des titres
% -------------------------------------

\def\discipline{Informatique \ifprof \\ Corrigé \else \fi}
\def\xxtete{Informatique}

\def\classe{PT -- PT $\star$}
\def\xxnumpartie{DS 3}
\def\xxpartie{Devoir Surveillé 3}

\def\xxnumchapitre{Enveloppe convexe$\;$ }
\def\xxchapitre{\textit{D'après Concours X -- ENS MP/PC}}

\def\xxtitreexo{Prothèse Active Transtibiale}
\def\xxsourceexo{\hspace{.2cm} D'après X 2015 -- MP/PC.}

\def\xxposongletx{2}
\def\xxposonglettext{1.45}
\def\xxposonglety{20}
\def\xxonglet{\textsf{DS 3}}

\def\xxactivite{}
\def\xxauteur{\textsl{Xavier Pessoles -- Patricia Bessonnat}}

\def\xxcompetences{%
\texttt{%
\textbf{Savoirs et compétences :}\\
\noindent \textbf{Résoudre :} à partir des modèles retenus :
\begin{itemize}[label=\ding{112},font=\color{ocre}] 
\item choisir une méthode de résolution analytique, graphique, numérique;
\item mettre en \oe{}uvre une méthode de résolution.
\end{itemize}
\begin{itemize}[label=\ding{112},font=\color{ocre}] 
\item \textit{Rés -- C1.1 :} Loi entrée sortie géométrique et cinématique -- Fermeture géométrique.
\end{itemize}
%
%\noindent \textit{Mod2 -- C4.1 :} Représentation par schéma bloc.
}}

\def\xxfigures{
%\includegraphics[width=.8\textwidth]{images/prot_01}
}%figues de la page de garde

\def\xxpied{%
DS 3 : Enveloppe convexe}


\setcounter{secnumdepth}{5}
%---------------------------------------------------------------------------


\begin{document}
%\chapterimage{png/Fond_Cin}
\pagestyle{empty}


%%%%%%%% PAGE DE GARDE COURS
\ifcours
% ==== BANDEAU DES TITRES ==== 
\begin{tikzpicture}[remember picture,overlay]
\node at (current page.north west)
{\begin{tikzpicture}[remember picture,overlay]
\node[anchor=north west,inner sep=0pt] at (0,0) {\includegraphics[width=\paperwidth]{\thechapterimage}};
\draw[anchor=west] (-2cm,-8cm) node [line width=2pt,rounded corners=15pt,draw=ocre,fill=white,fill opacity=0.6,inner sep=40pt]{\strut\makebox[22cm]{}};
\draw[anchor=west] (1cm,-8cm) node {\huge\sffamily\bfseries\color{black} %
\begin{minipage}{1cm}
\rotatebox{90}{\LARGE\sffamily\textsc{\color{ocre}\textbf{\xxnumpartie}}}
\end{minipage} \hfill
\begin{minipage}[c]{14cm}
\begin{titrepartie}
\begin{flushright}
\renewcommand{\baselinestretch}{1.1} 
\Large\sffamily\textsc{\textbf{\xxpartie}}
\renewcommand{\baselinestretch}{1} 
\end{flushright}
\end{titrepartie}
\end{minipage} \hfill
\begin{minipage}[c]{3.5cm}
{\large\sffamily\textsc{\textbf{\color{ocre} \discipline}}}
\end{minipage} 
 };
\end{tikzpicture}};
\end{tikzpicture}
% ==== FIN BANDEAU DES TITRES ==== 


% ==== ONGLET 
\begin{tikzpicture}[overlay]
\node[shape=rectangle, 
      rounded corners = .25 cm,
	  draw= ocre,
	  line width=2pt, 
	  fill = ocre!10,
	  minimum width  = 2.5cm,
	  minimum height = 3cm,] at (18.3cm,0) {};
\node at (17.7cm,0) {\rotatebox{90}{\textbf{\Large\color{ocre}{\classe}}}};
%{};
\end{tikzpicture}
% ==== FIN ONGLET 


\vspace{3.5cm}

\begin{tikzpicture}[remember picture,overlay]
\draw[anchor=west] (-2cm,-6cm) node {\huge\sffamily\bfseries\color{black} %
\begin{minipage}{2cm}
\begin{center}
\LARGE\sffamily\textsc{\color{ocre}\textbf{\xxactivite}}
\end{center}
\end{minipage} \hfill
\begin{minipage}[c]{15cm}
\begin{titrechapitre}
\renewcommand{\baselinestretch}{1.1} 
\Large\sffamily\textsc{\textbf{\xxnumchapitre}}

\Large\sffamily\textsc{\textbf{\xxchapitre}}
\vspace{.5cm}

\renewcommand{\baselinestretch}{1} 
\normalsize\normalfont
\xxcompetences
\end{titrechapitre}
\end{minipage}  };
\end{tikzpicture}
\vfill

\begin{flushright}
\begin{minipage}[c]{.3\linewidth}
\begin{center}
\xxfigures
\end{center}
\end{minipage}\hfill
\begin{minipage}[c]{.6\linewidth}
\startcontents
%\printcontents{}{1}{}
\printcontents{}{1}{}
\end{minipage}
\end{flushright}

\begin{tikzpicture}[remember picture,overlay]
\draw[anchor=west] (4.5cm,-.7cm) node {
\begin{minipage}[c]{.2\linewidth}
\begin{flushright}
\includegraphics[width=2cm]{logoCC}
\end{flushright}
\end{minipage}
\begin{minipage}[c]{.2\linewidth}
\textsl{\xxauteur} \\
\textsl{\classe}
\end{minipage}
 };
\end{tikzpicture}

\newpage
\pagestyle{fancy}

%\newpage
%\pagestyle{fancy}

\else
\fi
%% FIN PAGE DE GARDE DES COURS

%%%%%%%% PAGE DE GARDE TD
\iftd
%\begin{tikzpicture}[remember picture,overlay]
%\node at (current page.north west)
%{\begin{tikzpicture}[remember picture,overlay]
%\draw[anchor=west] (-2cm,-3.25cm) node [line width=2pt,rounded corners=15pt,draw=ocre,fill=white,fill opacity=0.6,inner sep=40pt]{\strut\makebox[22cm]{}};
%\draw[anchor=west] (1cm,-3.25cm) node {\huge\sffamily\bfseries\color{black} %
%\begin{minipage}{1cm}
%\rotatebox{90}{\LARGE\sffamily\textsc{\color{ocre}\textbf{\xxnumpartie}}}
%\end{minipage} \hfill
%\begin{minipage}[c]{13.5cm}
%\begin{titrepartie}
%\begin{flushright}
%\renewcommand{\baselinestretch}{1.1} 
%\Large\sffamily\textsc{\textbf{\xxpartie}}
%\renewcommand{\baselinestretch}{1} 
%\end{flushright}
%\end{titrepartie}
%\end{minipage} \hfill
%\begin{minipage}[c]{3.5cm}
%{\large\sffamily\textsc{\textbf{\color{ocre} \discipline}}}
%\end{minipage} 
% };
%\end{tikzpicture}};
%\end{tikzpicture}

%%%%%%%%%% PAGE DE GARDE TD %%%%%%%%%%%%%%%
%\begin{tikzpicture}[overlay]
%\node[shape=rectangle, 
%      rounded corners = .25 cm,
%	  draw= ocre,
%	  line width=2pt, 
%	  fill = ocre!10,
%	  minimum width  = 2.5cm,
%	  minimum height = 2.5cm,] at (18.5cm,0) {};
%\node at (17.7cm,0) {\rotatebox{90}{\textbf{\Large\color{ocre}{\classe}}}};
%%{};
%\end{tikzpicture}

% PARTIE ET CHAPITRE
%\begin{tikzpicture}[remember picture,overlay]
%\draw[anchor=west] (-1cm,-2.1cm) node {\large\sffamily\bfseries\color{black} %
%\begin{minipage}[c]{15cm}
%\begin{flushleft}
%\xxnumchapitre \\
%\xxchapitre
%\end{flushleft}
%\end{minipage}  };
%\end{tikzpicture}

% BANDEAU EXO
\iflivret % SI LIVRET
\begin{tikzpicture}[remember picture,overlay]
\draw[anchor=west] (-2cm,-3.3cm) node {\huge\sffamily\bfseries\color{black} %
\begin{minipage}{5cm}
\begin{center}
\LARGE\sffamily\color{ocre}\textbf{\textsc{\xxactivite}}

\begin{center}
\xxfigures
\end{center}

\end{center}
\end{minipage} \hfill
\begin{minipage}[c]{12cm}
\begin{titrechapitre}
\renewcommand{\baselinestretch}{1.1} 
\large\sffamily\textbf{\textsc{\xxtitreexo}}

\small\sffamily{\textbf{\textit{\color{black!70}\xxsourceexo}}}
\vspace{.5cm}

\renewcommand{\baselinestretch}{1} 
\normalsize\normalfont
\xxcompetences
\end{titrechapitre}
\end{minipage}};
\end{tikzpicture}
\else % ELSE NOT LIVRET
\begin{tikzpicture}[remember picture,overlay]
\draw[anchor=west] (-2cm,-4.5cm) node {\huge\sffamily\bfseries\color{black} %
\begin{minipage}{5cm}
\begin{center}
\LARGE\sffamily\color{ocre}\textbf{\textsc{\xxactivite}}

\begin{center}
\xxfigures
\end{center}

\end{center}
\end{minipage} \hfill
\begin{minipage}[c]{12cm}
\begin{titrechapitre}
\renewcommand{\baselinestretch}{1.1} 
\large\sffamily\textbf{\textsc{\xxtitreexo}}

\small\sffamily{\textbf{\textit{\color{black!70}\xxsourceexo}}}
\vspace{.5cm}

\renewcommand{\baselinestretch}{1} 
\normalsize\normalfont
\xxcompetences
\end{titrechapitre}
\end{minipage}};
\end{tikzpicture}

\fi

\else   % FIN IF TD
\fi


%%%%%%%% PAGE DE GARDE FICHE
\iffiche
\begin{tikzpicture}[remember picture,overlay]
\node at (current page.north west)
{\begin{tikzpicture}[remember picture,overlay]
\draw[anchor=west] (-2cm,-2.25cm) node [line width=2pt,rounded corners=15pt,draw=ocre,fill=white,fill opacity=0.6,inner sep=40pt]{\strut\makebox[22cm]{}};
\draw[anchor=west] (1cm,-2.25cm) node {\huge\sffamily\bfseries\color{black} %
\begin{minipage}{1cm}
\rotatebox{90}{\LARGE\sffamily\textsc{\color{ocre}\textbf{\xxnumpartie}}}
\end{minipage} \hfill
\begin{minipage}[c]{14cm}
\begin{titrepartie}
\begin{flushright}
\renewcommand{\baselinestretch}{1.1} 
\large\sffamily\textsc{\textbf{\xxpartie} \\} 

\vspace{.2cm}

\normalsize\sffamily\textsc{\textbf{\xxnumchapitre -- \xxchapitre}}
\renewcommand{\baselinestretch}{1} 
\end{flushright}
\end{titrepartie}
\end{minipage} \hfill
\begin{minipage}[c]{3.5cm}
{\large\sffamily\textsc{\textbf{\color{ocre} \discipline}}}
\end{minipage} 
 };
\end{tikzpicture}};
\end{tikzpicture}

\iflivret
\begin{tikzpicture}[overlay]
\node[shape=rectangle, 
      rounded corners = .25 cm,
	  draw= ocre,
	  line width=2pt, 
	  fill = ocre!10,
	  minimum width  = 2.5cm,
	  minimum height = 2.5cm,] at (18.5cm,1.1cm) {};
\node at (17.9cm,1.1cm) {\rotatebox{90}{\textsf{\textbf{\large\color{ocre}{\classe}}}}};
%{};
\end{tikzpicture}
\else
\begin{tikzpicture}[overlay]
\node[shape=rectangle, 
      rounded corners = .25 cm,
	  draw= ocre,
	  line width=2pt, 
	  fill = ocre!10,
	  minimum width  = 2.5cm,
%	  minimum height = 2.5cm,] at (18.5cm,1.1cm) {};
	  minimum height = 2.5cm,] at (18.6cm,1cm) {};
\node at (18cm,1cm) {\rotatebox{90}{\textsf{\textbf{\large\color{ocre}{\classe}}}}};
%{};
\end{tikzpicture}

\fi

\else
\fi



\vspace{1cm}
\pagestyle{fancy}
\thispagestyle{plain}

\section{Présentation}

\ifprof
\else
\noindent
\begin{minipage}[c]{.7\linewidth}
La recherche de l'enveloppe convexe d'un nuage de points est un problème fréquemment rencontré par les industriels ou les chercheurs. En robotique par exemple, cette recherche permet d'éviter les collisions entre l'effecteur et des obstacles, en traitement des images, elle permet de détecter des contours de formes et reconnaitre des caractères ...

Dans le cadre de ce sujet, on cherche à trouver l'enveloppe d'un nuage de points dans le plan par une méthode de balayage. 
\begin{defi}
On appelle $\mathcal{E}$ l'enveloppe convexe d'un ensemble $\mathcal{P}$ de points $P_i\in \mathbb{R}^2$ et $C$ l'ensemble de points constituant cette enveloppe. 
$\mathcal{E}$ est convexe si pour tous couples de points  $(P_i,P_j) \in {C}^2$, le segment $[P_i,P_j]$ est inclus dans $\mathcal{E}$.
\end{defi}

\textit{Remarque :} $\mathcal{P}=\left\{P_0,P_1,P_2, ..., P_{10} \right\}$, $\mathcal{E}=\left\{P_0, P_2, P_4, P_5, P_9, P_{10} \right\}$.

\end{minipage} \hfill
\begin{minipage}[c]{.28\linewidth}
\begin{center}
\includegraphics[width=.95\textwidth]{images/enveloppe}

\textit{Nuage de points et enveloppe convexe}
\end{center}
\end{minipage} 

Afin de déterminer l'enveloppe convexe à partir d'un nuage de points quelconques, on commence par trier les points par ordre croissant des abscisses. Quand deux (ou plus) points ont la même abscisse, leur ordre est choisi arbitrairement.

Pour déterminer l'enveloppe, on procède par balayage de gauche à droite : 
\begin{enumerate}
\item chaque point $P_i$ appartenant à $\mathcal{P}$ est <<visité>> une fois par ordre croissant;
\item pour chaque $P_i$ on met à jour l'ensemble $\mathcal{E}$ contenant l'enveloppe convexe des points situés à gauche de $P_i$. Pour cela : 
\begin{itemize}
\item on crée deux sous parties de $\mathcal{P}$, $\mathcal{E}_s$ (contenant les points au dessus de la droite $(P_0;P_i)$) et $\mathcal{E}_i$ (contenant les points au dessous de la droite $(P_0;P_i)$). Pour la quatrième figure, on obtiendrait donc $\mathcal{E}_s = \{P_0, P_2, P_4, P_6 \}$, $\mathcal{E}_s = \{P_0, P_5, P_6 \}$. L'enveloppe est donc déterminée par l'union de 
$\mathcal{E}_s$ et $\mathcal{E}_i$ après suppression des doublons $P_0$ et $P_6$.
\item on donnera ultérieurement la méthode permettant d'obtenir $\mathcal{E}_s$ et $\mathcal{E}_i$.
\end{itemize}

\end{enumerate}

%\begin{minipage}[c]{.22\linewidth}
%\begin{center}
%\includegraphics[width=.95\textwidth]{images/balayage_1}
%\end{center}
%\end{minipage} \hfill
%\begin{minipage}[c]{.22\linewidth}
%\begin{center}
%\includegraphics[width=.95\textwidth]{images/balayage_2}
%\end{center}
%\end{minipage} \hfill
%\begin{minipage}[c]{.22\linewidth}
%\begin{center}
%\includegraphics[width=.95\textwidth]{images/balayage_3}
%\end{center}
%\end{minipage} \hfill
%\begin{minipage}[c]{.22\linewidth}
%\begin{center}
%\includegraphics[width=.95\textwidth]{images/balayage_4}
%\end{center}
%\end{minipage} 

\begin{minipage}[c]{.22\linewidth}
\begin{center}
\includegraphics[width=.95\textwidth]{images/orientation_1}
\end{center}
\end{minipage} \hfill
\begin{minipage}[c]{.22\linewidth}
\begin{center}
\includegraphics[width=.95\textwidth]{images/orientation_2}
\end{center}
\end{minipage} \hfill
\begin{minipage}[c]{.22\linewidth}
\begin{center}
\includegraphics[width=.95\textwidth]{images/orientation_3}
\end{center}
\end{minipage} \hfill
\begin{minipage}[c]{.22\linewidth}
\begin{center}
\includegraphics[width=.95\textwidth]{images/orientation_4}
\end{center}
\end{minipage} 

\begin{center}
\begin{tabular}{|c||c|c|c|c|c|c|c|c|c|c|c|c|}
\hline 
Indice & 0 & 1 & 2 & 3 & 4 & 5 & 6 & 7 & 8 & 9 & 10  \\
\hline\hline
$x$ & 0 & 2 & 2 & 4 & 5 & 7 & 7 & 9 & 10 & 11 & 12 \\
\hline 
$y$ & 2 & 3 & 8 & 6 & 10 & 0 & 3 & 7 & 4 & 2 & 8 \\
\hline
\end{tabular}
\end{center}

Les points seront gérés sous formes de piles d'entiers. On considère donc que les fonctions ci-dessous ont déjà été définies : 
\begin{itemize}
\item \texttt{creer\_pile()} : fonction retournant une pile vide;
\item \texttt{est\_vide(p)} : fonction prenant comme argument une pile \texttt{p} et retournant \texttt{True} si la pile est vide \texttt{False} sinon;
\item \texttt{push(p,i)} : fonction prenant comme argument une pile \texttt{p} et un entier \texttt{i},  ne retournant rien mais ajoutant \texttt{i} au sommet de la pile;
\item \texttt{top(p)} :  fonction prenant comme argument une pile \texttt{p} (supposée non vide) et retournant la valeur de l'entier au sommet de la pile;
\item \texttt{pop(p)} :  fonction prenant comme argument une pile \texttt{p} (supposée non vide) et retournant et supprimant la valeur de l'entier au sommet de la pile.
\end{itemize}

\fi


\section{Réalisation de l'algorithme}
\subsection{Tri des points}
\ifprof
\else
Les points mesurés et non triés sont stockés dans une liste \texttt{P} contenant la liste \texttt{[abscisse\_i,ordonnée\_i]} de chacun des points. Ainsi, \texttt{P} est de la forme \texttt{[[x\_0,y\_0],[x\_1,y\_1],[x\_2,y\_2],...,[x\_n,y\_n])}.
\fi
\subparagraph{}
\textit{Citer 3 algorithmes de tris et donner leur complexité dans le meilleur des cas, le pire des cas et le cas moyen.} ~\\
\ifprof
\begin{corrige} ~\\
\begin{center}
\begin{tabular}{|c|c|c|c|}
\hline
& Tri par insertion & Tri rapide & Tri fusion \\
\hline 

{Pire des cas}&  $ C(n)=\mathcal{O}\left(n^2 \right)$ & $ C(n)=\mathcal{O}\left(n^2 \right)$ & $ C(n)=\mathcal{O}\left(n^2\right)$ \\ \hline
Cas moyen & $ C(n)=\mathcal{O}\left(n^2 \right)$ &$ C(n)=\mathcal{O}\left(n \log n\right)$  & $ C(n)=\mathcal{O}\left(n \log n\right)$ \\ \hline
Meilleur des cas   & $ C(n)=\mathcal{O}\left(n \right)$ & $ C(n)=\mathcal{O}\left(n \log n \right)$ & $ C(n)=\mathcal{O}\left(n \log n\right)$ \\ \hline
\end{tabular}
\end{center}
\end{corrige}
\else
\fi

\subparagraph{}
\textit{On donne l'algorithme d'un tri dans le document réponse (à la question suivante). Donner son nom et remplacer les fonctions <<mystere>> par un nom plus adéquat.}
\subparagraph{}
\textit{Modifier l'algorithme de tri donné pour qu'il prenne en argument la liste des points \texttt{P}. Pour cela vous barrerez et réécrirez les lignes à modifier. }

\ifprof
\begin{corrige} ~\\
Il s'agit du tri rapide.

\begin{py}
\begin{python}
def segmente(tab,i,j):
    g =i+1
    d=j
    p=tab[i][0]
    while g<=d :
        while d>=0 and tab[d][0]>p:
            d=d-1
        while g<=j and tab[g][0]<=p:
            g=g+1
        if g<d :
            tab[g],tab[d]=tab[d],tab[g]
            d=d-1
            g=g+1
    k=d
    tab[i],tab[d]=tab[d],tab[i]
    return k
    
def tri_quicksort(tab,i,j):
    if i<j :
        k = segmente(tab,i,j)
        tri_quicksort(tab,i,k-1)
        tri_quicksort(tab,k+1,j)

\end{python}
\end{py}
\end{corrige}
\else
\fi

\subsection{Fonction orientation}
\ifprof
\else
\begin{defi}
Étant donnés trois points $p_i,p_j,p_k$ du nuage $P$, distincts ou non, le test d'orientation renvoie +1 si la séquence $\left(p_i,p_j,p_k\right)$ est orientée positivement, -1 si elle est orientée négativement et 0 si les trois points sont alignés (c'est-à-dire deux au moins sont égaux). 
\end{defi}


Pour déterminer l'orientation de $\left(p_i,p_j,p_k\right)$, il suffit de calculer l'aire signée du triangle. Cette aire est la moitié du déterminant de la matrice $2\times 2$ formée par les vecteurs $\vect{p_i p_j}$ et $\vect{p_i p_k}$.

\begin{center}
\includegraphics[width=.7\textwidth]{images/orientation}
\end{center}

\fi

\subparagraph{}
\textit{En utilisant les points dans le tableau de la page précédente, donner le résultat du test d'orientation pour les points d'indices suivants : 
\begin{itemize}
\item $i=0$, $j=1$, $k=2$;
\item $i=6$, $j=7$, $k=8$.
\end{itemize}}
\ifprof
\begin{corrige} ~\\
Les points $P_0$, $P_3$ et $P_4$ les points sont orientés positivement. On peut le voir graphiquement. Sinon, on a : $\vect{P_0 P_1} = (2,1) $ et $\vect{P_0 P_2} = (2,6)$. $\begin{vmatrix} 2 & 2 \\ 1 & 6\end{vmatrix}=12-2 = 10$.

Mes points $P_6$, $P_7$ et $P_8$ les points sont orientés négativement. On peut le voir graphiquement. Sinon, on a : $\vect{P_6 P_7} = (2,4) $ et $\vect{P_6 P_8} = (3,1)$. $\begin{vmatrix} 2 & 3 \\ 4 & 1\end{vmatrix}=2-12 = -10$.
\end{corrige}
\else
\fi

\subparagraph{}
\textit{Écrire une fonction $\texttt{orientation(tab,i,j,k)}$ prenant comme paramètres la liste des points ainsi que l'indice des points à tester. Cette fonction renverra $-1$, $0$ ou $1$ suivant le résultat du test d'orientation des points d'indices $i$, $j$ et $k$. }
\ifprof
\begin{corrige}~\\
\begin{py}
\begin{python}
def orientation(tab,i,j,k):
    p1, p2, p3 = tab[i], tab[j], tab[k]
    p12 = [p2[0]-p1[0],p2[1]-p1[1]]
    p13 = [p3[0]-p1[0],p3[1]-p1[1]]
    det = p12[0]*p13[1]-p12[1]*p13[0]
    if det <0 :
        return -1
    elif det >0 :
        return 1
    else :
        return 0
\end{python}
\end{py}
\end{corrige}
\else
\fi

\subsection{Création de l'enveloppe supérieure}
\ifprof
\else
La liste des points triés sont contenus dans la liste \texttt{[tab]}. Les \textbf{indices} des points constituants l'enveloppe supérieure sont stockés dans la  \textbf{pile} \texttt{Es}. Pour créer l'enveloppe à partir du point $P_i$ et de l'enveloppe supérieure $\mathcal{E}_s$ à l'étape $i-1$, on procède ainsi : 
\begin{itemize}
\item on considère le point $P_i$ ainsi que les deux points au sommet de la pile \texttt{Es};
\item si l'orientation des trois points précédents est négative, on dépile l'élément au sommet de \texttt{Es} (élimination du sommet de \texttt{Es});
\item on continue ce processus d'élimination jusqu'à ce que l'orientation devienne positive ou jusqu'à ce qu'on ait atteint le dernier élément de la pile $\mathcal{E}_s$;
\item l'indice du point visité est alors ajouté au sommet de \texttt{Es}.
\end{itemize}

On remarque qu'à la fin du traitement de $P_i$, les points $P_i$ et $P_0$ appartiennent aux bords de l'enveloppe.

\begin{obj}
L'objectif est d'écrire une fonction $\texttt{enveloppe\_Sup(tab,Es,i)}$ qui prend en paramètre la liste des points, la pile \texttt{Es} contenant les indices des points de l'enveloppe supérieure, et le l'indice \texttt{i} considéré.
\end{obj}
\fi

\subparagraph{}
\textit{À l'instant $0$, quelle instruction permet de tester si \texttt{Es} est vide ? Que faut-il alors faire pour initialiser \texttt{Es} ? À l'instant $i$, que faire si \texttt{Es} ne contient qu'un seul élément ?}
\ifprof
\begin{corrige}
À l'instant $0$, \texttt{Es} est vide. On peut le tester avec la fonction \texttt{est\_vide}. On empile donc l'élément $0$. 

À l'instant $i$, si \texttt{Es} est vide, on empile l'élément $i$. 

À l'instant $i$, si \texttt{Es} ne contient qu'un seul élément, on empile l'élément $i$. 


\end{corrige}
\else
\fi

\subparagraph{}
\textit{Écrire la fonction $\texttt{enveloppe\_Sup(tab,Es,i)}$.}
\ifprof
\begin{corrige}~\\

\begin{py}
\begin{python}
def enveloppe_Sup_rec(tab,Es,i):
    if est_vide(Es):
        push(Es,i)
    else :
        p1 = i
        p2 = pop(Es)
        if est_vide(Es):
            push(Es,p2)
            push(Es,p1)
        else :
            p3 = top(Es)
            if orientation(tab,p1,p2,p3)>0:
                push(Es,p2)
                push(Es,p1)
            else :
                enveloppe_Sup_rec(tab,Es,i)
\end{python}
\end{py}
\end{corrige}             
\else
\fi

\ifprof
\else

\vspace{.5cm}
\textsl{Il resterait à réaliser l'algorithme permettant de déterminer l'enveloppe inférieure (même procédure que pour l'enveloppe supérieure mais en raisonnant de manière symétrique).}
\fi

\subsection{Analyse de l'algorithme final}

\subparagraph{}
\textit{On donne dans le document réponse l'algorithme permettant de retourner la pile contenant les indices des sommets de l'enveloppe convexe à partir d'un nuage de points. Commenter les blocs d'instructions de ce programme. Donner la complexité de l'algorithme dans le pire des cas. }
\ifprof
\begin{corrige}~\\
\begin{py}
\begin{python}
def enveloppe_convexe(tab):
    tri(tab)                         # Tri de la liste des points
    es = creer_pile()                # Initialisation de l'enveloppe supérieure
    ei = creer_pile()                # Initialisation de l'enveloppe inférieure
    for i in range (len(tab)):       # On balaye les points et on crée les enveloppes inf et sup
        enveloppe_Inf(tab,ei,i)
        enveloppe_Sup(tab,es,i)
    pop(es)                          # Tous les points vont être transférés dans l'enveloppe inf pour 
                                     # créer l'enveloppe finale. Le dernier point de la liste initiale 
                                     # est dans les deux enveloppes. On supprime donc le doublon.

    while (not(est_vide(es))):       # On dépile l'enveloppe sup dans l'enveloppe inf
        push(ei,pop(es))             # 
    pop(ei)                          # Le point 0 étant un doublons on le supprime.
    return (ei)                      # 
\end{python}
\end{py}
\end{corrige}
\else
\fi



\end{document}