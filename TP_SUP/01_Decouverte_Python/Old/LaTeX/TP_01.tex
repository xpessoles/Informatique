\documentclass[11pt,oneside]{article}
\input{coursHeadings}
\usepackage{algorithm}
\usepackage{algorithmic}


% Python sources
\usepackage{listings}
\usepackage{textcomp}
\usepackage{setspace}
%\usepackage{palatino}

%\usepackage{color}
\definecolor{Bleu}{rgb}{0.1,0.1,1.0}
\definecolor{Noir}{rgb}{0,0,0}
\definecolor{Grau}{rgb}{0.5,0.5,0.5}
\definecolor{DunkelGrau}{rgb}{0.15,0.15,0.15}
\definecolor{Hellbraun}{rgb}{0.5,0.25,0.0}
\definecolor{Magenta}{rgb}{1.0,0.0,1.0}
\definecolor{Gris}{gray}{0.5}
\definecolor{Vert}{rgb}{0,0.5,0}
\definecolor{SourceHintergrund}{rgb}{1,1.0,0.95}

%
\renewcommand{\lstlistlistingname}{Listings}
\renewcommand{\lstlistingname}{Listing}

\lstnewenvironment{python}[1][]{
\lstset{
language=python,
basicstyle=\ttfamily\footnotesize\setstretch{1}, 	
stringstyle=\color{red}, 
showstringspaces=false, 
alsoletter={1234567890},
otherkeywords={\ , \}, \{},
keywordstyle=\color{blue},
emph={access,and,break,class,continue,def,del,elif ,else,
except,exec,finally,for,from,global,if,import,in,i s,
lambda,not,or,pass,print,raise,return,try,while},
emphstyle=\color{black}\bfseries,
emph={[2]True, False, None, self},
emphstyle=[2]\color{green},
emph={[3]from, import, as},
emphstyle=[3]\color{blue},
upquote=true,
morecomment=[s]{"""}{"""},
commentstyle=\color{Hellbraun}\slshape, 
%emph={[4]1, 2, 3, 4, 5, 6, 7, 8, 9, 0},
emphstyle=[4]\color{blue},
literate=*{:}{{\textcolor{blue}:}}{1}
{=}{{\textcolor{blue}=}}{1}
{-}{{\textcolor{blue}-}}{1}
{+}{{\textcolor{blue}+}}{1}
{*}{{\textcolor{blue}*}}{1}
{!}{{\textcolor{blue}!}}{1}
{(}{{\textcolor{blue}(}}{1}
{)}{{\textcolor{blue})}}{1}
{[}{{\textcolor{blue}[}}{1}
{]}{{\textcolor{blue}]}}{1}
{<}{{\textcolor{blue}<}}{1}
{>}{{\textcolor{blue}>}}{1},
%framexleftmargin=1mm, framextopmargin=1mm, frame=shadowbox, rulesepcolor=\color{blue},#1
backgroundcolor=\color{SourceHintergrund}, 
framexleftmargin=1mm, framexrightmargin=1mm, framextopmargin=1mm, frame=single, framerule=1pt, rulecolor=\color{black},#1
}}{}
\usepackage[%
    pdftitle={Complexité des algorithmes},
    pdfauthor={Xavier Pessoles},
    colorlinks=true,
    linkcolor=blue,
    citecolor=magenta]{hyperref}

\usepackage{pifont}


% \makeatletter \let\ps@plain\ps@empty \makeatother
%% DEBUT DU DOCUMENT
%% =================
\sloppy
\hyphenpenalty 10000

\newcommand{\Pointilles}[1][3]{%
\multido{}{#1}{\makebox[\linewidth]{\dotfill}\\[\parskip]
}}


\colorlet{shadecolor}{orange!15}

\newtheorem{theorem}{Theorem}


\begin{document}


\newboolean{prof}
\setboolean{prof}{true}
%------------- En tetes et Pieds de Pages ------------
\pagestyle{fancy}
\renewcommand{\headrulewidth}{0pt}

\fancyhead{}
\fancyhead[L]{%
\noindent\noindent\begin{minipage}[c]{2.6cm}
%Lycée Rouvière PTSI
\includegraphics[width=2cm]{png/logo_ptsi.png}%
\end{minipage}
}

\fancyhead[C]{\rule{12cm}{.5pt}}

\fancyhead[R]{%
\noindent\begin{minipage}[c]{3cm}
\begin{flushright}
\footnotesize{\textit{\textsf{Informatique}}}%
\end{flushright}
\end{minipage}
}

\renewcommand{\footrulewidth}{0.2pt}

\fancyfoot[C]{\footnotesize{\bfseries \thepage}}
\fancyfoot[L]{\footnotesize{2012 -- 2013} \\ X. \textsc{Pessoles}}
\ifthenelse{\boolean{prof}}{%
\fancyfoot[R]{\footnotesize{TP 1 -- CI 2 : Algorithmique \& Programmation}}
}{%
\fancyfoot[R]{\footnotesize{TP 1 -- CI 2 : Algorithmique \& Programmation}}
}



\begin{center}
 \huge\textsc{CI 2 -- Algorithmique et Programmation}

% \large\textsc{Introduction à la programmation}
\end{center}

\begin{center}
 \LARGE\textsc{TP 1 -- Découverte de Python}
\end{center}

\vspace{.5cm}



\begin{minipage}[c]{.3\linewidth}
\begin{center}

\end{center}
\end{minipage} \hfill
\begin{minipage}[c]{.3\linewidth}
\begin{center}

\end{center}
\end{minipage} \hfill
\begin{minipage}[c]{.3\linewidth}
\begin{center}

\end{center}
\end{minipage}

\vspace{.5cm}


%\begin{center}
%\includegraphics[width=.9\textwidth]{png/cyclev.png}

%\textit{Cycle de conception d'un produit}
%\end{center}

%\begin{prob}
%\textsc{Problématique :}

%En phase d'avant conception d'un produit, quels sont les critères qui vont permettre de choisir les matériaux à utiliser ?
%\end{prob}



\begin{savoir}
\textsc{Objectifs :}
\begin{itemize}
\item découvrir l'environnement Python;
\item découvrir les limites de Python;
\item manipuler différents types.
\end{itemize}
\end{savoir}
 



\begin{rem}
Suivant les goûts du programmeur, il existe plusieurs interpréteurs ou plusieurs éditeurs de texte. Dans le cadre de ce TP, nous proposons d'utiliser celui qui est directement installé avec Python : IDLE.
\end{rem}

Pour ouvrir Python :
\begin{enumerate}
\item dans le menu démarrer;
\item sélectionner Programmes;
\item ...
\end{enumerate}

\vspace{.5cm}

\begin{minipage}[c]{.45\linewidth}
\begin{center}
\includegraphics[width=.9\textwidth]{png/shell}

\textit{Shell Python -- Interpréteur de commande}
\end{center}
\end{minipage} \hfill
\begin{minipage}[c]{.45\linewidth}
\begin{center}
\includegraphics[width=.9\textwidth]{png/editeur}

\textit{Éditeur de texte}
\end{center}
\end{minipage}


\setlength{\parskip}{0ex plus 0.2ex minus 0ex}
 \renewcommand{\contentsname}{}
 \renewcommand{\baselinestretch}{1}

\tableofcontents

 \renewcommand{\baselinestretch}{1.2}
\setlength{\parskip}{2ex plus 0.5ex minus 0.2ex}


% \vspace{1cm}
\textit{Ce document évolue. Merci de signaler toutes erreurs ou coquilles.}



\subsection*{Exercice 1 -- Premiers pas avec Python}
\begin{obj}
\begin{itemize}
\item Découvrir l'interpréteur (\textsf{shell})
\item Découvrir l'éditeur de texte
\end{itemize}
\end{obj}



Dans l'interpréteur python, saisir les instructions suivantes :
\begin{py}
\begin{python}
>>> 4+3
>>> 4*3
>>> 7/2
>>> 7/2.
>>> 7//2
>>> 7//2.
>>> 7**2
>>> 7\%2
\end{python}
\end{py}

\paragraph{}
\textit{Quel est le but de chacune de ces instructions ? Quelle est la différence entre \textsf{7//2} et \textsf{7//2.} ?}

Dans l'interpréteur python, saisir les instructions suivantes :
\begin{py}
\begin{python}
>>> "Abracadabra"
>>> Abracadabra
\end{python}
\end{py}

\paragraph{}
\textit{Quelle est la différence entre les 2 instructions ? Expliquez ?}


Dans l'interpréteur python, saisir l'instruction suivante.
\begin{py}
\begin{python}
>>> "Abrac"+"adabra"
\end{python}
\end{py}


\paragraph{}
\textit{Quelle est la différence entre les 2 instructions ? Expliquez ?}

De manière générale pour afficher du texte, on utiliser l'instruction \textsl{print}.

On va maintenant affecter des variables. Dans l'interpréteur python, saisir les instructions suivantes :

\begin{py}
\begin{python}
>>> a=1
>>> b=2
>>> a=b
>>> b=a
\end{python}
\end{py}

\paragraph{} 
\textit{Pour chacune des instructions précédentes, quelles sont les valeurs de a et de b ?}


Dans l'interpréteur python, saisir les instructions suivantes :
\begin{py}
\begin{python}
>>> a=1
>>> b=2
>>> a<b
>>> a>b
>>> a==b
>>> a=!b
\end{python}
\end{py}


\paragraph{} 
\textit{Quel est le résultat de ces opérations ? Quel est leur but ?}


On va maintenant utiliser l'éditeur de texte d'IDLE pour saisir des lignes de programmation. Chacune des lignes d'instruction sera exécutée une par une jusqu'à la fin du programme (ou jusqu'à ce qu'il y ait une erreur sur une ligne). Pour cela :
\begin{itemize}
\item aller dans le menu Fichier;
\item ouvrir une nouvelle fenêtre;
\item sauvegarder votre fichier : dans le noms de fichiers ne pas utiliser d'accent, d'espace, d'apostrophe. Si nécessaire utiliser le caractère \_.
\end{itemize}

Saisir les instructions suivantes :
\begin{py}
\begin{python}
print("Chuck Norris counted to infinty - twice")
\end{python}
\end{py}

Pour l'exécuter :
\begin{itemize}
\item sauvegarder votre programme;
\item dans le menu Run, Cliquer sur Run Module (ou sur F5).
\end{itemize}

\paragraph{} 
\textit{Quel est le résultat ?}



La commande \textsf{input()} permet à l'utilisateur de saisir une chaîne de caractère. La méthode \textsf{float(str)} permet de convertir une chaîne de caractère en nombre flottant.
La méthode \textsf{float(str)} permet de convertir un nombre en chaîne de caractère.

\paragraph{}
\textit{Écrire un programme qui permet à l'utilisateur de saisir un nombre, de calculer le carré de ce nombre et d'afficher la phrase suivante (si le nombre saisi est 8):}

\texttt{Le carré de 8 vaut 64}.

\subsection*{Exercice 2 -- Un peu d'algorithmique}

\begin{defi}
\textbf{Algorithme\footnote{\url{https://fr.wikipedia.org/wiki/Algorithme}}}

Le mot algorithme vient du nom latinisé du mathématicien perse Al-Khawarizmi, surnommé « le père de l'algèbre ». Un algorithme est une suite finie et non-ambiguë d’opérations ou d'instructions permettant de résoudre un problème.
\end{defi}

\begin{obj}
\begin{itemize}
\item Découvrir la boucle \textsf{for}
\item Identifier quelques limites de Python
\end{itemize}

\end{obj}


\setcounter{paragraph}{0}
\paragraph{}
\textit{Expliquer le rôle des instructions suivantes :}

\begin{py}
\begin{python}
>>> range(5) 
>>> range(0,6)
>>> range(0,10,2)
>>> range(10,2,-2)
\end{python}
\end{py}

On donne la structure d'une boucle \textsf{for} :
\begin{py}
\begin{python}
for i in range (0,10): # il est indispensable de finir l'instruction for par :
    print(i)           # il est indispensable que chaque instruction dans la boucle for soit precedee de 4 espaces

                       # une boucle for se termine par un retour a la ligne sans indentation
\end{python}
\end{py}

\paragraph{}
\textit{Tester la boucle précédente.}


\subsubsection*{La légende de l'échiquier \footnote{\url{http://www.ac-paris.fr/portail/jcms/d_5773/tableur-en-3eme-la-legende-de-l-echiquier}}}


\end{document}