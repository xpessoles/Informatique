\section{Introduction aux graphes}

\subsection{Vocabulaire des graphes}

\begin{defi}{Graphe}%\cite{ref_01}
Un graphe est un ensemble de \textbf{sommets} et  \textbf{relations} entre ces sommets.

Lorsque deux sommets sont en relation, on dit qu'il existe une \textbf{arête} entre ces sommets.
\end{defi}

\begin{defi}{Graphe non orienté -- Arêtes}
Un graphe non orienté $G$ est un couple $G=(S,A)$, où $S$ est un ensemble fini de sommets (appelés aussi n\oe uds)  et où $A$ est un ensemble fini de paires ordonnées de sommets, appelées arêtes.

On note $x - y$ l'arête $\{x,y\}$. $x$ et $y$ sont les deux extrémités de l'arête.
\end{defi}

\begin{defi}{Graphe orienté -- Arcs}\cite{ref_01}
Un graphe orienté $G$ est un couple $G=(S,A)$, où $S$ est un ensemble fini de sommets et où $A$ est un ensemble fini de paires ordonnées de sommets, appelées arcs.

On note $x\to y$ l'arc $(x,y)$. $x$ est l'extrémité initiale de l'arc, $y$ est son extrémité terminale. On dit que $y$ est successeur de $x$ et que $x$ est prédécesseur de $y$. 
\end{defi}

\begin{multicols}{2}
\begin{figure}[H]
\includegraphics[width=.8\linewidth]{fig_02}
\captionsetup{justification=centering}
\caption{Graphe non orienté \label{fig_02}}
\end{figure}

\begin{figure}[H]
\includegraphics[width=.8\linewidth]{fig_03}
\captionsetup{justification=centering}
\caption{Graphe orienté}
\end{figure}
\end{multicols}

\begin{rem}
On peut noter le graphe non orienté $G=\left(\llbracket 1,6\rrbracket,E\right)$ où $E=\left(
\left\{1,2\right\},\left\{2,3\right\},\left\{3,4\right\},\left\{1,4\right\},\left\{1,3\right\},\left\{1,5\right\},\left\{1,6\right\}\right)$ désigne les arêtes. 

On peut noter le graphe orienté $G=\left(\llbracket 1,6\rrbracket,E\right)$ où $E=\left(
\left(1,2\right),\left(2,3\right),\left(3,4\right),\left(1,4\right),\left(1,3\right),\left(1,5\right),\left(1,6\right)\right)$ désigne les arcs. 
\end{rem}

\begin{defi}{Adjacence}
Deux arcs (resp. arêtes) d'un graphe orienté (resp. non orienté) sont dits adjacents s'ils ont au moins une extrémité commune. 

Deux sommets d'un graphe non orienté sont dits adjacents s'il existe une arête les joignant. 

Dans un graphe orienté, le sommet $y$ est dit adjacent au sommet $x$ s'il existe un arc $x\to y$.
\end{defi}


\begin{defi}{Sommet (ou n\oe{}uds)}
\end{defi}

\begin{defi}{Arc, arête}
\end{defi}

\begin{defi}{Chemin d'un sommet à un autre}
\end{defi}

\begin{defi}{Cycle}
\end{defi}

\begin{defi}{Connexité dans les graphes non orientés}
\end{defi}

\subsection{Notations}
\begin{defi}{Degré d'un sommet}
On appelle degré d'un sommet $s$ et on note $d\left(s\right)$ le nombres d'arcs (ou d'arêtes) dont $s$ est une extrémité.
\end{defi}

\begin{defi}{Degré entrant et sortant}
On note $s$ le sommet d'un graphe orienté. On note : 
\begin{itemize}
\item $d_{+}\left(s\right)$ le demi-degré extérieur de $s$, c'est-à-dire le nombre d'arcs ayant leur extrémité initiale en $s$ (ces arcs sont dits incidents à $s$ vers l'extérieur);
\item $d_{-}\left(s\right)$ le demi-degré intérieur de $s$, c'est-à-dire le nombre d'arcs ayant leur extrémité finale en $s$ (ces arcs sont dits incidents à $s$ vers l'intérieur).
\end{itemize}

Dans ce cas, on a  $d\degres\left(s\right)=d_{-}\left(s\right)+d_{+}\left(s\right)$.
\end{defi}


\begin{exemple}~\\

\begin{multicols}{2}
\begin{figure}[H]
\centering
\includegraphics[width=.8\linewidth]{fig_04}
\captionsetup{justification=centering}
\caption{Graphe orienté \label{fig_04}}
\end{figure}

\begin{itemize}
\item $d_{-}\left(S_1\right)=3$.
\item $d_{+}\left(S_1\right)=4$.
\item $d\degres\left(S_1\right)=7$.
\end{itemize}
\end{multicols}

\end{exemple}

\subsection{Implémentation des graphes}

\subsubsection{Liste d'adjacence}
\begin{defi}{Liste d'adjacence}
Soit un graphe de $n$ sommets d'indices $i \in \llbracket 0, n-1\rrbracket$. Pour implémenter le graphe, on utilise une liste \texttt{G} de taille $n$ pour laquelle, \texttt{G[i]} est la liste des voisins de \texttt{i}.
\end{defi}
\begin{rem}
Cette implémentation est plutôt réservée au graphes << creux >> ayant peu d'arêtes.
\end{rem}

\begin{exemple} ~\\
\begin{minipage}[b]{.47\linewidth}
\begin{center}
\includegraphics[width=.7\linewidth]{fig_05}
\end{center}
\begin{lstlisting}
G = [[1,5,6],[0,2,6],[1,3],[2,4],[3,5],
      [4,0],[1,0]]
\end{lstlisting}
\end{minipage}\hfill
\begin{minipage}[b]{.47\linewidth}
\begin{center}
\includegraphics[width=.7\linewidth]{fig_06}
\end{center}
Dans ca cas, le graphe est orienté. La liste d'adjacence contient la liste des successeurs. 
\begin{lstlisting}
G = [[1,5,6],[],[1],[6],[3,5],[],[0,1]]
\end{lstlisting}
\end{minipage}
\end{exemple}

Dans la même idée, il est aussi possible d'utiliser des dictionnaires d'adjacence dans lequel les clés sont les sommets, et les valeurs sont des listes de voisins ou de sucesseurs. 
\begin{lstlisting}
# Graphe non orienté 
G = {"S0":["S1","S5","S6"],"S1":["S0","S2","S6"],"S2":["S1","S3"],"S3":["S2","S4"],"S4":["S3","S5"],"S5":["S4","S0"],"S6":["S1","S0"]}
# Graphe orienté 
G = {"S0":["S1","S5","S6"],"S1":[],"S2":["S1"],"S3":["S6"],"S4":["S3","S5"],"S5":[],"S6":["S1","S0"]}
\end{lstlisting}


\subsubsection{Matrice d'adjacence}
\begin{defi}{Matrice d'adjacence}
Soit un graphe de $n$ sommets d'indices $i \in \llbracket 0, n-1\rrbracket$ et $E$ l'ensemble des arêtes 
(on notera $G=\left( \llbracket 0, n-1\rrbracket,E\right)$. Pour implémenter le graphe, on utilise la matrice d'adjacence carrée de taille $n$, $\mathcal{M}_n$ \texttt{G} de taille $n$ pour laquelle,
$m_{i,j}=\left\{
\begin{array}{l}
\texttt{True } \text{ si } \{i,j\}\in E\\
\texttt{False } \text{ sinon } 
\end{array}
\right.$ avec $i,j\in \llbracket 0, n-1\rrbracket$. 


\end{defi}
\begin{rem}
Cette implémentation est plutôt réservée au graphes << denses >> ayant << beaucoup >> d'arêtes.
\end{rem}


\begin{exemple} ~\\
\begin{minipage}[b]{.47\linewidth}
\begin{center}
\includegraphics[width=.7\linewidth]{fig_05}
\end{center}

On a dans ce cas 

\footnotesize{$
M = $

$
\begin{pmatrix}
\texttt{False} & \texttt{True} & \texttt{False} & \texttt{False} & \texttt{False} & \texttt{True} & \texttt{True} \\
\texttt{True} & \texttt{False} & \texttt{True} & \texttt{False} & \texttt{False} & \texttt{False} & \texttt{True} \\ 
\texttt{False} & \texttt{True} & \texttt{False} & \texttt{True} & \texttt{False} & \texttt{False} & \texttt{False} \\
\texttt{False} & \texttt{False} & \texttt{True} & \texttt{False} & \texttt{True} & \texttt{False} & \texttt{False} \\
\texttt{False} & \texttt{False} & \texttt{False} & \texttt{True} & \texttt{False} & \texttt{True} & \texttt{False} \\
\texttt{True} & \texttt{False} & \texttt{False} & \texttt{False} & \texttt{True} & \texttt{False} & \texttt{False} \\
\texttt{True} & \texttt{True} & \texttt{False} & \texttt{False} & \texttt{False} & \texttt{False} & \texttt{False} \\
\end{pmatrix}$}

ou 

\footnotesize{$
M = 
\begin{pmatrix}
\texttt{0} & \texttt{1} & \texttt{0} & \texttt{0} & \texttt{0} & \texttt{1} & \texttt{1} \\
\texttt{1} & \texttt{0} & \texttt{1} & \texttt{0} & \texttt{0} & \texttt{0} & \texttt{1} \\ 
\texttt{0} & \texttt{1} & \texttt{0} & \texttt{1} & \texttt{0} & \texttt{0} & \texttt{0} \\
\texttt{0} & \texttt{0} & \texttt{1} & \texttt{0} & \texttt{1} & \texttt{0} & \texttt{0} \\
\texttt{0} & \texttt{0} & \texttt{0} & \texttt{1} & \texttt{0} & \texttt{1} & \texttt{0} \\
\texttt{1} & \texttt{0} & \texttt{0} & \texttt{0} & \texttt{1} & \texttt{0} & \texttt{0} \\
\texttt{1} & \texttt{1} & \texttt{0} & \texttt{0} & \texttt{0} & \texttt{0} & \texttt{0} \\
\end{pmatrix}$}


\end{minipage}\hfill
\begin{minipage}[b]{.47\linewidth}
\begin{center}
\includegraphics[width=.7\linewidth]{fig_06}
\end{center}
Dans ca cas, le graphe est orienté. On a 
On a dans ce cas 

\footnotesize{$
M = $

$
\begin{pmatrix}
\texttt{False} & \texttt{True} & \texttt{False} & \texttt{False} & \texttt{False} & \texttt{False} & \texttt{True} \\
\texttt{False} & \texttt{False} & \texttt{False} & \texttt{False} & \texttt{False} & \texttt{False} & \texttt{False} \\ 
\texttt{False} & \texttt{True} & \texttt{False} & \texttt{False} & \texttt{False} & \texttt{False} & \texttt{False} \\
\texttt{False} & \texttt{False} & \texttt{False} & \texttt{False} & \texttt{False} & \texttt{False} & \texttt{True} \\
\texttt{False} & \texttt{False} & \texttt{False} & \texttt{True} & \texttt{False} & \texttt{True} & \texttt{False} \\
\texttt{False} & \texttt{False} & \texttt{False} & \texttt{False} & \texttt{False} & \texttt{False} & \texttt{False} \\
\texttt{True} & \texttt{True} & \texttt{False} & \texttt{False} & \texttt{False} & \texttt{False} & \texttt{False} \\
\end{pmatrix}$}

ou 
\footnotesize{$
M =
\begin{pmatrix}
\texttt{0} & \texttt{1} & \texttt{0} & \texttt{0} & \texttt{0} & \texttt{0} & \texttt{1} \\
\texttt{0} & \texttt{0} & \texttt{0} & \texttt{0} & \texttt{0} & \texttt{0} & \texttt{0} \\ 
\texttt{0} & \texttt{1} & \texttt{0} & \texttt{0} & \texttt{0} & \texttt{0} & \texttt{0} \\
\texttt{0} & \texttt{0} & \texttt{0} & \texttt{0} & \texttt{0} & \texttt{0} & \texttt{1} \\
\texttt{0} & \texttt{0} & \texttt{0} & \texttt{1} & \texttt{0} & \texttt{1} & \texttt{0} \\
\texttt{0} & \texttt{0} & \texttt{0} & \texttt{0} & \texttt{0} & \texttt{0} & \texttt{0} \\
\texttt{1} & \texttt{1} & \texttt{0} & \texttt{0} & \texttt{0} & \texttt{0} & \texttt{0} \\
\end{pmatrix}$}

\end{minipage}
\end{exemple}


\begin{rem}
\begin{itemize}
\item Dans le cas d'un graphe non orienté, la matrice est symétrique. 
\item Si on avait un bouclage sur un sommet, il y aurait des valeurs non nulles sur la diagonale. 
\end{itemize}
\end{rem}

\section{Parcours d'un graphe}
\subsection{Piles et files}

\subsection{Parcours générique d'un graphe}

\subsection{Parcours en largeur}

\subsection{Parcours en profondeur}

\subsection{Détection de la présence des cycles}

\subsection{Connexité d'un graphe non orienté}

\section{Pondération d'un graphe}



\section{Recheche du plus court chemin}
\subsection{Algorithme de Dijkstra}

\subsection{Algorithme A$\star$}

\begin{defi}{}
\end{defi}

\begin{defi}{}
\end{defi}

\begin{defi}{}
\end{defi}

\begin{defi}{}
\end{defi}

\begin{defi}{}
\end{defi}
