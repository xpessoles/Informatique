%%%ENTETES%%%
%%%%%%%%%%%
\newcommand{\entetedebut}{
{\noindent \textbf{PTSI2 -- 2019/2020 -- Maths} \hfill Lycée La Martinière-Monplaisir -- Lyon \vspace{2mm}}
\hrule
\begin{center} 
}
\newcommand{\entetedebutinfo}{
{\noindent \textbf{PTSI -- 2021/2022 -- Info} \hfill Lycée La Martinière-Monplaisir -- Lyon \vspace{2mm} }
\hrule
\begin{center} 
}

\newcommand{\entetefin}{
\end{center}
\hrule
\vspace*{0.5cm}
}

\newcommand{\entetedebutsnow}{
\begin{tikzpicture}[decoration=Koch snowflake]
\draw (0,0)--(12,0)node[midway,above]
}

\newcommand{\entetefinsnow}{
\draw decorate{ decorate{ decorate{ decorate{ (12,0)--(17,0) }}}};
\end{tikzpicture}
\vspace*{1cm}
}
%%% Lorqu'on utilise les entetes koch snowflake, il faut juste mettre tout le titre entre { } et un point-virgule à la fin, car on est dans un tikzpicture.
\newcommand{\entetecours}{
\entetedebut
\textbf{\textsf{\Large Ch \numero. \titre.}}
\entetefin
}
\newcommand{\entetecoursinfo}{
\entetedebutinfo
\textbf{\textsf{\Large Ch \numero. \titre.}}
\entetefin
}


\newcommand{\entete}{
\entetedebut
\textbf{\textsf{\Large \titre.}}
\entetefin
}
\newcommand{\enteteinfo}{
\entetedebutinfo
\textbf{\textsf{\Large \titre.}}
\entetefin
}


\newcommand{\entetetd}{
\entetedebut
\textbf{\textsf{\Large TD \numero. \titre.}}
\entetefin
}

\newcommand{\entetetp}{
\entetedebut
\textbf{\textsf{\Large TP \numero. \titre.}}
\entetefin
}

\newcommand{\entetetpinfo}{
\entetedebutinfo
\textbf{\textsf{\Large TP \numero. \titre.}}
\entetefin
}



\newcommand{\enteteindic}{
\entetedebut
\textbf{\textsf{\Large Indications et solutions pour le TD \numero.}}
\entetefin
}


\newcommand{\entetecor}{
\entetedebut
\textbf{\textsf{\Large Corrigés pour le TD \numero.}}
\entetefin
}