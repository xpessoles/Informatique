\section{Contexte}

\section{Signature des fonctions -- Signature de type}
\begin{defi}
La \textbf{signature d'une fonction} définit les entrées et les sorties d'une fonction. 
\end{defi}


La signature peut comporter par exemple le type  des paramètres d'entrées ou de sorties, des conditions sur ces paramètres. 

La façon la plus simple de signer une fonction, est d'indiquer des commentaires juste après la déclaration de la fonction. 

\begin{exemple}
Commentaire simple
\end{exemple}


Pour aller plus loin, il est possible d'indiquer certains mots clés dans une fonction (docstring, etc...).

\begin{exemple}
Commentaires avec docstring
\end{exemple}

Une méthode un peu plus spécifique à Python est d'utiliser les annotations de type. 

\begin{exemple}
Exemple d'annotation de type
\end{exemple}


Une fois toutes ces précautions prises, aucun contrôle n'est effectué lors de l'exécution. >> mypy .

\section{Assertion}

\subsection{Vers la gestion d'exceptions -- hors programme}


\section{Génération de tests}


\subsection{Vers Pytest -- hors programme}


