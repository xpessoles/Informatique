\begin{enumerate}
\item  \textbf{Lisez attentivement  tout l'énoncé
    avant de commencer.}
\item Commencez la séance en créant un dossier au nom du TP dans le répertoire dédié à l'informatique de votre compte. 
\item Après la séance, vous devez rédiger un compte-rendu de TP et
l'envoyer au format électronique à votre enseignant.
\item Vous rendrez un compte-rendu sous forme d'un fichier d'extension \texttt{.py}, en respectant exactement les spécifications données plus bas. 
\item Ce TP est à faire en binôme, vous ne rendrez donc qu'un  compte-rendu pour deux.
\item Ayez toujours un crayon et un papier sous la main. Quand vous réfléchissez à une question, utilisez les !
\item Vous devez être autonome. Ainsi, avant de poser une question à l'enseignant, merci de commencer par :
\begin{itemize}
  \item relire l'énoncé du TP (beaucoup de réponses se trouvent dedans) ;
  \item relire les passages du cours\footnote{Dans le cas fort 
improbable où vous ne vous en souviendriez pas.} relatifs à votre problème ;
  \item effectuer une recherche dans l'aide disponible sur votre ordinateur (ou sur internet) concernant votre question.
\end{itemize}
  Il est alors raisonnable d'appeler votre enseignant pour lui demander des explications ou une confirmation !
\end{enumerate}

On s'intéresse dans ce TP à la manipulation d'images en Python. {\bf On commencera par se rendre sur le site de la classe et par enregistrer dans le répertoire du TP les fichiers suivants} : 
\begin{itemize}
  \item \texttt{degrade.pgm} ;
  \item \texttt{joconde.pgm}.
\end{itemize}

\section*{Instructions de rendu}

Attention : suivez précisément ces instructions. Vous enverrez à votre enseignant un fichier d'extension  \texttt{.py} (script \python) nommé
\begin{center}
  \texttt{tp08\_nom1\_nom2.py},
\end{center}
 où nom1 et nom2 sont à remplacer par ceux des membres du binôme. Le nom de ce 
fichier ne devra comporter ni espace, ni accent, ni apostrophe, ni majuscule.
Dans ce fichier, vous respecterez les consignes suivantes.
\begin{itemize}
  \item \'Ecrivez d'abord en commentaires (ligne débutant par \#), le titre du TP, les noms et prénoms des étudiants du groupe.
  \item Commencez chaque question par son numéro écrit en commentaires.
  \item Les questions demandant une réponse écrite seront rédigées en commentaires.
  \item Les questions demandant une réponse sous forme de fonction ou de script respecteront pointilleusement les noms de variables et de fonctions demandés.
\end{itemize}