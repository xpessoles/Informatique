Attention : suivez précisément ces instructions. 

Votre fichier portera un nom du type 
\begin{center}
  \texttt{tp05$\_$durif$\_$berne.py},
\end{center}
 où les noms de vos enseignants sont à remplacer par ceux des membres du binôme. Le nom de ce 
fichier ne devra comporter ni espace, ni accent, ni apostrophe, ni majuscule.
Dans ce fichier, vous respecterez les consignes suivantes.
\begin{itemize}
  \item \'Ecrivez d'abord en commentaires (ligne débutant par \#), le titre du TP, les noms et prénoms des étudiants du groupe.
  \item Commencez chaque question par son numéro écrit en commentaires.
  \item Les questions demandant une réponse écrite seront rédigées en commentaires.
  \item Les questions demandant une réponse sous forme de fonction ou de script respecteront pointilleusement les noms de variables et de fonctions demandés.
\end{itemize}
Les figures seront sauvegardées sous le nom \texttt{tp05$\_$Qnum$\_$nom1$\_$nom2.png}, où Qnum est le numéro de la question, et nom1, nom2 les noms des membres du binôme.



\subsection*{Première approche de la complexité}
	
Étant donné deux suites $(u_n)$ et $(v_n)$  de réels strictement positifs, on dit que $(u_n)$ est dominée par la suite $(v_n)$ lorsque $\left(\Frac{u_n}{v_n}\right)$ est une suite bornée. On note alors $u_n=\O{n\to +\infty}(v_n)$.

Par exemple:
\begin{itemize}
\item  si $(u_n)$ converge alors $u_n=\O{n\to +\infty}(1)$. Réciproque fausse;

\item   $3n=\O{n\to +\infty}(n^2)$, $5n^2=\O{n\to +\infty}(n^2)$, $\ln(n)=\O{n\to +\infty}(n\ln(n)^2)$, $an^2+bn+c\ln(n)=\O{n\to +\infty}(n^2)$ ...

\item  pour tout polynôme de degré $p$, $P=a_px^p+a_{p-1}x^{p-1}+...+a_1x+a_0$, on a $P(n)=\O{n\to +\infty}(n^p)$.
\end{itemize}

En programmation, on dira qu'un programme a:
\begin{itemize}
\item une complexité linéaire lorsque le nombre d'opérations effectuées est en $\O{n\to +\infty}(n)$;
\item une complexité logarithmique lorsque le nombre d'opérations effectuées est en $\O{n\to +\infty}(\log(n))$;
\item une complexité quadratique lorsque le nombre d'opérations effectuées est en $\O{n\to +\infty}(n^2)$.
\end{itemize}