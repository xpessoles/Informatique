%\section*{Fonctionnement du devoir}

Vos réponses dépendent d'un paramètre $\alpha$, unique pour chaque étudiant, qui vous est donné en haut de votre fiche réponse. 
On considère la suite $u$, définie comme suit. 
\begin{equation*}
  u_0 = \alpha ~\textrm{ et }~ \forall n\in\N,~ u_{n+1} = (15\,091 \times u_n) ~[64\,007]. 
\end{equation*}
Nous vous en proposons l'implémentation suivante. 
\begin{lstlisting}
def u(alpha,n):
    """u_n, u_0 = alpha"""
    x = alpha
    for i in range(n):
        x = (15091 * x) % 64007
    return x
\end{lstlisting}
Pour s'assurer que vous avez bien codé la suite $u$, en voici quelques valeurs. \\
\indent
\begin{lstlisting}
def u(alpha,n):
    """u_n, u_0 = alpha"""
    x = alpha
    for i in range(n):
        x = (15091 * x) % 64007
    return x

print('u(100,0) = ',u(100,0),'\n')
print('u(1515,987) = ',u(1515,987),'\n')
print('u(496,10**4) = ',u(496,10**4))
\end{lstlisting}


