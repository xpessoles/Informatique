%\section*{Fonctionnement du devoir}

\begin{enumerate}
\item \textbf{Lisez attentivement  tout l'énoncé avant de commencer.}
\item Ce devoir est à réaliser seul, en utilisant Python 3.
\item Nous vous conseillons de commencez par créer un dossier au nom du DS dans le répertoire dédié à l'informatique de votre compte. 
\item Nous vous conseillons d'utiliser pyzo comme environnement de développement intégré.
\item Nous vous rappelons qu'il est possible d'obtenir de l'aide dans l'interpréteur de votre environnement de développement intégré en tapant \\
\texttt{help(nom\_fonction)}.
%\item Vous inscrirez vos réponses sur la feuille réponse fournie. Attention : lisez attentivement le paragraphe suivant.
\item Vous inscrirez vos réponses sur google form dont le lien est donné sur le site de la classe sous forme numérique.
\item Lorsque la réponse demandée est un réel, on attend que l'écart entre la réponse que vous donnez et la valeur attendue soit strictement inférieur à $10^{-4}$. Donnez donc des valeurs avec 5 chiffres après la virgule. On pourra pour cela utiliser la fonction \texttt(round(,)). Par exemple : 
\begin{lstlisting}
round(1.8754398,5)
\end{lstlisting}

renvoie $1.87544$
\item On rappelle que si $x$ est un entier, on note $x \% n$ le reste de la division euclidienne de $x$ par $n$.
\item Vos réponses dépendent d'un paramètre $\alpha$, unique pour chaque étudiant, qui vous a été donné sur le site de la classe.
\item Vous pourrez télécharger le fichier DS04$\_$2021$\_$2022$\_$nom.py présent sur le site de la classe. 
\item Vous sauvegardez ce fichier sur votre espace personnel en remplaçant "nom" par votre nom.
\item Une fois le TP terminé vous enverrez ce script via le lien de dépôt sur le site de la classe.
\end{enumerate}



%On considère la suite $u$, définie comme suit qui est celle déjà saisie sur le script DS04$\_$2021$\_$2022$\_$nom.py. 
%\begin{equation*}
%  u_0 = \alpha ~\textrm{ et }~ \forall n\in\N,~ u_{n+1} = (15\,091 \times u_n) ~[64\,007]. 
%\end{equation*}
%Nous vous en proposons l'implémentation suivante. 
%\begin{lstlisting}
%def u(alpha,n):
%    """u_n, u_0 = alpha"""
%    x = alpha
%    for i in range(n):
%        x = (7 * x) % 20
%    return x+5
%\end{lstlisting}


