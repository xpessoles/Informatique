Suivez précisément ces instructions. Vous enverrez à votre enseignant un fichier d'extension .py (script Python) nommé
tp13$\_$durif$\_$rebout.py,
où les noms de vos enseignants sont à remplacer par ceux des membres du binôme. Le nom de ce fichier ne devra
comporter ni espace, ni accent, ni apostrophe, ni majuscule. Dans ce fichier, vous respecterez les consignes suivantes.

\begin{itemize}
\item Ecrivez d'abord en commentaires, le titre du TP, les noms et prénoms des étudiants du groupe.
\item Commencez chaque question par son numéro écrit en commentaires.
\item Les questions demandant une réponse écrite seront rédigées en commentaires.
\item Les questions demandant une réponse sous forme de fonction ou de script respecteront pointilleusement les
noms de variables et de fonctions demandés.
\end{itemize}

Commencer le TP par recopier les fonctions vues dans le cours pour utiliser l'algorithme du pivot de Gauss.
Les figures demandées porteront toutes un nom du types tp13$\_$durif$\_$rebout$\_$num.png, où les noms de
vos enseignants sont à remplacer par ceux des membres du binôme et où
\begin{itemize}
\item num vaut q03 pour la question 03;
\item num vaut q07 pour la question 07;
\end{itemize}
