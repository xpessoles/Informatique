\question{} Ecrire une fonction \texttt{matches(joueurs:list)->list} prenant en argument une liste de numéros de joueurs et renvoyant une listes de matchs sous la forme de couple de joueurs.

\begin{center}
\begin{lstlisting}
def matches(joueurs): 
    """ Fonction recursive qui renvoie une liste de matches
    a partir d'une liste de joueurs """ 
    #s'il n'y a qu'un seul joueur, on n'organise aucun match
    if len(joueurs)==1: 
        return [] 
    #on enleve le dernier joueur de la liste, et on demande les matchs sans lui
    else:
        dernier_joueur = joueurs.pop()
        vs=matches(joueurs) 
        print(vs)
    #on rajoute un match entre lui et tous les autres joueurs 
        for j in joueurs:
            vs.append([j,dernier_joueur]) 
    #on le remet dans la liste des joueurs, et on renvoie 
    #la liste des matchs 
        joueurs.append(dernier_joueur) 
        return vs
    
\end{lstlisting}
\end{center}


En affichant ce qu'il se passe après chaque appel de la fonction "\textit{matches}", on obtient :

\begin{lstlisting}
>>> joueurs=list(range(1,5))
>>> matches(joueurs)
[]
[[1, 2]]
[[1, 2], [1, 3], [2, 3]]
[[1, 2], [1, 3], [2, 3], [1, 4], [2, 4], [3, 4]]
\end{lstlisting}