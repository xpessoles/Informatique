\question{\'Ecrire une fonction \texttt{dichotomie$\_$rec(x0,L)} qui renvoie \texttt{True} ou \texttt{False} selon que \texttt{x0} figure ou non dans \texttt{L} par cette méthode. On utilisera une boucle \texttt{while} que l'on interrompra soit lorsque l'on a trouvé $x0$, soit lorsque l'on a fini de parcourir la liste.}

\begin{lstlisting}
def appartient_dicho_rec(e : int , t : list) -> bool:
    """Renvoie un booléen indiquant si e est dans t
    Préconditions : t est un tableau de nombres trié par ordre croissant e est un nombre"""
    g = 0 # Limite gauche de la tranche où l'on recherche e
    d = len(t)-1 # Limite droite de la tranche où l'on recherche e
    while g <= d: # La tranche où l'on cherche e n'est pas vide
        m = (g+d)//2 # Milieu de la tranche où l'on recherche e
        pivot = t[m]
        if e == pivot: # On a trouvé e
            return True
        elif e < pivot:
            d = m-1 # On recherche e dans la partie gauche de la tranche
            appartient_dicho_rec(e,t[g:d])
        else:
            g = m+1 # On recherche e dans la partie droite de la tranche
            appartient_dicho_rec(e,t[g:d])
    return False
\end{lstlisting}
