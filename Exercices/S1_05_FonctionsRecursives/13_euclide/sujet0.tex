% Options for packages loaded elsewhere
\PassOptionsToPackage{unicode}{hyperref}
\PassOptionsToPackage{hyphens}{url}
%
\documentclass[
]{article}
\usepackage{lmodern}
\usepackage{amssymb,amsmath}
\usepackage{ifxetex,ifluatex}
\ifnum 0\ifxetex 1\fi\ifluatex 1\fi=0 % if pdftex
  \usepackage[T1]{fontenc}
  \usepackage[utf8]{inputenc}
  \usepackage{textcomp} % provide euro and other symbols
\else % if luatex or xetex
  \usepackage{unicode-math}
  \defaultfontfeatures{Scale=MatchLowercase}
  \defaultfontfeatures[\rmfamily]{Ligatures=TeX,Scale=1}
\fi
% Use upquote if available, for straight quotes in verbatim environments
\IfFileExists{upquote.sty}{\usepackage{upquote}}{}
\IfFileExists{microtype.sty}{% use microtype if available
  \usepackage[]{microtype}
  \UseMicrotypeSet[protrusion]{basicmath} % disable protrusion for tt fonts
}{}
\makeatletter
\@ifundefined{KOMAClassName}{% if non-KOMA class
  \IfFileExists{parskip.sty}{%
    \usepackage{parskip}
  }{% else
    \setlength{\parindent}{0pt}
    \setlength{\parskip}{6pt plus 2pt minus 1pt}}
}{% if KOMA class
  \KOMAoptions{parskip=half}}
\makeatother
\usepackage{xcolor}
\IfFileExists{xurl.sty}{\usepackage{xurl}}{} % add URL line breaks if available
\IfFileExists{bookmark.sty}{\usepackage{bookmark}}{\usepackage{hyperref}}
\hypersetup{
  hidelinks,
  pdfcreator={LaTeX via pandoc}}
\urlstyle{same} % disable monospaced font for URLs
\setlength{\emergencystretch}{3em} % prevent overfull lines
\providecommand{\tightlist}{%
  \setlength{\itemsep}{0pt}\setlength{\parskip}{0pt}}
\setcounter{secnumdepth}{-\maxdimen} % remove section numbering
\ifluatex
  \usepackage{selnolig}  % disable illegal ligatures
\fi

\author{}
\date{}

\begin{document}

L'algorithme d'Euclide permet, étant donnés deux entiers \emph{a} et
\emph{b}, de calculer leur plus grand commun diviseur (pgcd) \emph{d}.
Cet algorithme se base sur la propriété suivante~:

\[\left\{ \begin{matrix}
\text{pgcd}\left( a,b \right) = a,\ si\ b = 0 \\
\text{pgcd}\left( a,b \right) = pgcd\left( b,a\ \%\ b \right),\ sinon \\
\end{matrix} \right.\ \]

où \emph{a} \% \emph{b} représente le reste de la division euclidienne
de \emph{a} par \emph{b}.

\begin{enumerate}
\def\labelenumi{\arabic{enumi}.}
\item
  Ecrire une fonction récursive pgcd(a,b) qui calcule le plus grand
  commun diviseur de deux entiers en utilisant l'algorithme d'Euclide.
\end{enumerate}

Le théorème de Bézout nous assure également l'existence de deux entiers
\emph{u} et \emph{v} tels que~:

\emph{a.u} + \emph{b.v} = \emph{d} (\emph{u} et \emph{v} sont des
coefficients de Bézout de \emph{a} et \emph{b}).

Une version étendue de l'algorithme d'Euclide permet de calculer, en
plus du pgcd \emph{d} des valeurs possibles pour les coefficients de
Bézout \emph{u} et \emph{v}.

Cet algorithme prend en entrée deux entiers \emph{a} et \emph{b}. Il
procède de la manière suivante :

\begin{itemize}
\item
  Si \emph{b} = 0 alors \emph{d} = \emph{a}, \emph{u} = 1 et \emph{v} =
  0.
\item
  Sinon, on applique récursivement l'algorithme sur les entiers \emph{b}
  et (\emph{a} \% \emph{b}).
\end{itemize}

On obtient ainsi \emph{d}', \emph{u}' et \emph{v}' tels que :

\emph{d}' = pgcd(\emph{b}, \emph{a} \% b) ; et : \emph{b.u}' + (\emph{a}
\% \emph{b}).\emph{v'} = \emph{d'}

On en déduit la solution pour \emph{a} et b grâce aux égalités :

\emph{d} = \emph{d'}, \emph{u} = \emph{v'} et \emph{v} = \emph{u}' --
(\emph{a}//\emph{b}).\emph{v'}

\begin{enumerate}
\def\labelenumi{\arabic{enumi}.}
\setcounter{enumi}{1}
\item
  En déduire une fonction bezout qui étant donnés deux entiers \emph{a}
  et \emph{b} calcule le triplet (\emph{d},\emph{u},\emph{v}) comme
  expliqué ci-dessus.
\end{enumerate}

\end{document}
