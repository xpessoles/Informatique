\question{Pour votre valeur de $\alpha$ donner \texttt{n=u(alpha,10).}}

\begin{lstlisting}
def u(alpha,n):
    """u_n, u_0 = alpha"""
    x = alpha
    for i in range(n):
        x = (7 * x) % 20
    return x+5

n=u(alpha,10)
\end{lstlisting}



\question{Écriture sous Python \texttt{fib$\_$it(n:int)->F(n)} calculant le terme de rang $n$ ($F_n$) de la suite de Fibonacci par une méthode itérative. Donner la valeur de $F(n)$.}



\begin{lstlisting}
def fib_it(n):
    u=0
    v=1
    if n<=1:
        v=n
    for k in range(2,n+1):
        u,v=v,u+v
    return v

\end{lstlisting}


\question{Écriture sous Python \texttt{fib$\_$rec(n:int)->F(n)} calculant le terme de rang $n$ ($F_n$) de la suite de Fibonacci par une méthode récursive naïve. Donner la valeur de $F(n)$.}

\begin{lstlisting}
def fib_rec(n):
    if n==0 or n==1:
        return n
    else:
        Fn1=fib_rec(n-1)
        Fn2=fib_rec(n-2)
        return Fn1+Fn2

\end{lstlisting}



\question{Exécuter cette suite d'instruction et donner la valeur de C dans votre cas.}

\begin{lstlisting}
C=0
def compte_appel():
    global C
    C+=1

alpha=100

n=u(alpha,10)

for k in range(n):
    compte_appel()

print(n,C)

\end{lstlisting}



\question{En vous inspirant de la question précédente et avec la forme récursive Donner le nombre d'appel à la fonction \texttt{fib$\_$rec(n:int)->F(n)} par une méthode récursive "naïve" avec \texttt{n=u(alpha,10)}..}

\begin{lstlisting}
S=0
def fib_rec(n):
    global S
    S+=1
    if n==0 or n==1:
        return n
    else:
        Fn1=fib_rec(n-1)
        Fn2=fib_rec(n-2)
        return Fn1+Fn2
        
fib_rec(n)
print(S)
\end{lstlisting}


\question{Proposer un nouvelle fonction de signature \texttt{fib$\_$rec2(n:int)->int} en modifiant l'algorithme
précédent en y ajoutant la déclaration d'une liste. On ne calcule le n\up{ème} terme que si celui-ci n'a jamais été calculé. On pourra
commencer par déclarer une liste contenant n + 1 fois la valeur 0. Donner le nombre d'appel à la fonction \texttt{fib$\_$rec2(n:int)->F(n)} par cette méthode avec \texttt{n=u(alpha,10)}.}


\begin{lstlisting}
Lrec=(n+1)*[0]

S=0
def fib_rec2(n):
    global S,Lrec
    S+=1
    if n==0 or n==1:
        return n
    else:
        if Lrec[n-1]==0:
            Fn1=fib_rec2(n-1)
            Lrec[n-1]=Fn1
        else:
            Fn1=Lrec[n-1]
        if Lrec[n-2]==0:
            Fn2=fib_rec2(n-2)
            Lrec[n-2]=Fn2
        else:
            Fn2=Lrec[n-2]
        if Lrec[n]==0:
            Lrec[n]=Fn1+Fn2
        return Lrec[n]

print(fib_rec2(n),n,S)

\end{lstlisting}