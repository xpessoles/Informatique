\subsection*{Définitions}
\begin{defi}%Fonctions récusives]
Une fonction récursive est une fonction qui s'appelle elle-même.

On appelle récursion l'appel de la fonction à elle-même.
\end{defi}

La programmation est un paradigme de programmation au même titre que la programmation itérative. Un programme écrit de manière récursive peut être traduit de manière itérative, même si dans certain cas, cela peut s'avérer délicat.


\begin{methode}
\begin{itemize}
\item Une fonction récursive doit posséder une condition d'arrêt (ou cas de base).
\item Une fonction récursive doit s'appeler elle-même (récursion).
\item L'argument de l'étape de récursion doit évoluer de manière à se ramener à la condition d'arrêt.
\end{itemize}
\end{methode}
 
 


\subsection*{Exemples}

\subsubsection*{Suites récurrentes}
Soit la suite $u_n$ définie par récurrence pour tout $n\in\mathbb{N}^*$ par 
$
\left\{
\begin{array}{ll} 
u_1 = 1 \\
u_{n+1} = \dfrac{u_n + 6}{u_n + 2} \\
\end{array}
\right.
$.

\noindent\begin{minipage}[c]{.45\linewidth}
\begin{lstlisting}
def un_it (n : int) -> float :
    if n == 1 :
        return 1
    else : 
        u = 1
        for i in range(2,n+1):
            u = (u+6)/(u+2)
        return u
\end{lstlisting}
\end{minipage} \hfill
\begin{minipage}[c]{.45\linewidth}
\begin{lstlisting}
def un_rec (n : int) -> float :
    if n == 1 :
        return 1
    else : 
        return (un_rec(n-1)+6)/(un_rec(n-1)+2)
\end{lstlisting}
\end{minipage} 

%\subsubsection*{Recheche dichotomique}
%\begin{lstlisting}
%def appartient_dicho(e : int , t : list) -> bool:
%    """Renvoie un booléen indiquant si e est dans t
%    Préconditions : t est un tableau de nombres trié par ordre croissant e est un nombre"""
%    g = 0 # Limite gauche de la tranche où l'on recherche e
%    d = len(t)-1 # Limite droite de la tranche où l'on recherche e
%    while g <= d: # La tranche où l'on cherche e n'est pas vide
%        m = (g+d)//2 # Milieu de la tranche où l'on recherche e
%        pivot = t[m]
%        if e == pivot: # On a trouvé e
%            return True
%        elif e < pivot:
%            d = m-1 # On recherche e dans la partie gauche de la tranche
%        else:
%            g = m+1 # On recherche e dans la partie droite de la tranche
%    return False
%\end{lstlisting}
%
%\begin{lstlisting}
%def appartient_dicho_rec(e : int , t : list) -> bool:
%    """Renvoie un booléen indiquant si e est dans t
%    Préconditions : t est un tableau de nombres trié par ordre croissant e est un nombre"""
%    g = 0 # Limite gauche de la tranche où l'on recherche e
%    d = len(t)-1 # Limite droite de la tranche où l'on recherche e
%    while g <= d: # La tranche où l'on cherche e n'est pas vide
%        m = (g+d)//2 # Milieu de la tranche où l'on recherche e
%        pivot = t[m]
%        if e == pivot: # On a trouvé e
%            return True
%        elif e < pivot:
%            d = m-1 # On recherche e dans la partie gauche de la tranche
%            appartient_dicho_rec(e,t[g:d])
%        else:
%            g = m+1 # On recherche e dans la partie droite de la tranche
%            appartient_dicho_rec(e,t[g:d])
%    return False
%\end{lstlisting}

%\subsection{Exponentiation rapide}
%L'exponentiation rapide est un algorithme permettant de calculer $x^n$ en utilisant l'algorithme récursif suivant : 
%$
%\text{puissance}(x,n) =
%\left\{
%\begin{array}{ll}
%x & \text{si }n=1 \\
%\text{puissance}(x^2,n/2) & \text{si } $n$ \text{ est pair} \\
%x\times \text{puissance}(x^2,(n-1)/2) & \text{si } $n$ \text{ est impar pair} \\
%\end{array}
%\right.
%.$
%
%\noindent\begin{minipage}[c]{.45\linewidth}
%\begin{lstlisting}
%def expo_it(x : float, n : int) -> float :
%    res=1
%    while n!=0:
%        if n%2==1:
%            res = res*x
%        x*=x
%        n//=2
%    return res
%\end{lstlisting}
%\end{minipage} \hfill
%\begin{minipage}[c]{.45\linewidth}
%\begin{lstlisting}
%def expo_rec(x : float, n : int) -> float :
%    if n==0:
%        return 1
%    elif n%2==0:
%        return expo_rec(x*x,n//2)
%    else:
%        return x*expo_rec(x*x,n//2)
%\end{lstlisting}
%\end{minipage} 
%
%
%
%
%\section{Acticité préparatoire}
%\textbf{Pour réaliser l'activité associée à ce cours, suivre le lien suivant : }
%\begin{itemize} 
%\item Sujet : \url{https://bit.ly/36OLBiW}
%\item Corrige : \url{https://bit.ly/3BjucwN}
%\end{itemize}
%
%\newpage
%\section{QCM}
%
%\question{Que retourne la commande suivante \texttt{mystere(4)} ?}
%\begin{lstlisting}
%def mystere(n):
%    if n>0 :
%        return mystere(n-2)
%    else :
%        return n==0
%\end{lstlisting}
%
%\begin{enumerate}
%\item 0.
%\item False.
%\item True. % +
%\item L'exécution génère une erreur.
%\end{enumerate}
%
%\question{Laquelle de ces fonctions retourne True lorsqu'on exécute \texttt{f(5)} ?}
%\begin{lstlisting}
%def f1(n):
%    if n==0 :
%        return True
%    else :
%        return f(n-2)
%        
%def f2(n):
%    if n<=0 :
%        return True
%    else :
%        f(n-2)
%        
%def f3(n):
%    if n<=0 :
%        return True
%    return f(n-2)
%
%def f4(n):
%    if n==0 :
%        return True
%    f(n-2)
%\end{lstlisting}
%
%
%\begin{enumerate}
%\item \texttt{f1}.
%\item \texttt{f2}.
%\item \texttt{f3}. %+
%\item \texttt{f4}.
%\end{enumerate}
%
%\question{ Quel affichage obtient-on en exécutant \texttt{affiche(3)} ?}
%\begin{lstlisting}
%def affiche(n):
%    print(n)
%    if n>=0:
%        affiche(n-1)
%\end{lstlisting}
%
%\begin{enumerate}
%\item 3, 2, 1, 0 (avec des retours à la ligne entre chaque valeurs).
%\item 0, 1, 2, 3 (avec des retours à la ligne entre chaque valeurs).
%\item 3, 2, 1, 0, -1 (avec des retours à la ligne entre chaque valeurs). %+
%\item 3.
%\end{enumerate}
%
%\question{Une seule des fonctions définies ci-dessous retourne \texttt{'ccccc'} à l'appel de \texttt{replique(5,'c')}. Déterminer laquelle.}
%\begin{lstlisting}
%def replique(a,b): # Fonction 1
%    if a==1:
%        return b
%    else :
%        return replique( a-1 , b+b)
%
%def replique(a,b): # Fonction 2
%    if a==1:
%        return b
%    elif a%2 == 0:
%        return replique( a-2 , b+b)
%    else :
%        return b + replique( a-2 , b+b)
%
%def replique(a,b): # Fonction 3
%    if a==1:
%        return b
%    elif a%2 == 0:
%        return replique( a//2 , b+b)
%    else :
%        return b + replique( a//2 , b+b)
%
%def replique(a,b): # Fonction 4
%    if a==1:
%        return b
%    else :
%        replique( a-1 , b+b)
%\end{lstlisting}
%
%\begin{enumerate}
%\item Fonction 1.
%\item Fonction 2.
%\item Fonction 3.
%\item Fonction 4.
%\end{enumerate}
%
%\question{Que retourne l'instruction \texttt{copy(3,'A')} ?}
%\begin{lstlisting}
%def copy(n,s):
%    if n==0:
%        return s
%    return copy(n-1, s+s)
%\end{lstlisting}
%
%\begin{enumerate}
%\item \texttt{'AAA'}.
%\item \texttt{'AAAAAA'}.
%\item \texttt{'AAAAAAAA'}.
%\item \texttt{'3A'}.
%\end{enumerate}
%
%\question{Que retourne l'instruction : \texttt{mystere(3,'\$')} ?}
%\begin{lstlisting}
%def mystere(n,s):
%    if n==0:
%        return s
%    return s + mystere(n-1, s)
%\end{lstlisting}
%
%\begin{enumerate}
%\item \texttt{'\$\$\$'}.
%\item \texttt{'\$2\$'}.
%\item \texttt{'\$\$\$\$'}. %+
%\item L'exécution déclenche une erreur.
%\end{enumerate}
%
%\question{Que retourne la commande \texttt{f(3,4)} ?}
%\begin{lstlisting}
%def f(a,b):
%    if a == 0 :
%        return b
%    return f(a-1, b+1)
%\end{lstlisting}
%
%\begin{enumerate}
%\item 4.
%\item 5.
%\item 6.
%\item 7. %+
%\end{enumerate}
%
%\question{Que retourne la commande \texttt{mystere(3)} ?}
%\begin{lstlisting}
%def mystere(n):
%    if n>0 :
%        return mystere(n-2)
%    else :
%        return n==0
%\end{lstlisting}
%
%\begin{enumerate}
%\item True.
%\item False. %+
%\item RecursionError.
%\item 0.
%\end{enumerate}
%
%\question{On propose de créer une fonction récursive permettant de calculer  $x^n$. Compléter la fonction proposée.}
%\begin{lstlisting}
%def puissance(x,n):
%    if n > 0 : 
%        return .......
%    return 1
%\end{lstlisting}
%
%\begin{enumerate}
%\item \texttt{puissance(x,n-1)}.
%\item \texttt{x*puissance(x,n-1)}. %+
%\item Quoi que l'on écrive, cette fonction ne donnera pas le résultat attendu.
%\item \texttt{x**(n-1)*puissance(x,n-1)}.
%\end{enumerate}
%
%\question{Que renvoi ce programme en console ?}
%\begin{lstlisting}
%def ed(L,M=[]):
%    if len(L) == 0 : return M
%    a=L.pop()
%    if a not in M : M.append(a)
%    return ed(L,M)
%L=[2, 3, 2, 6, 8, 9, 9, 10, 9, 3, 6, 7, 8, 8, 9]
%print(ed(L))
%\end{lstlisting}
%
%\begin{enumerate}
%\item \texttt{None}.
%\item \texttt{[9, 8, 7, 6, 3, 10, 2]}. %+
%\item \texttt{[9, 8, 8, 7, 6, 3, 9, 10, 9, 9, 8, 6, 2, 3, 2]}.
%\item \texttt{[2, 10, 3, 6, 7, 8, 9]}.
%\end{enumerate}
%
%\question{Que retourne le programme suivant ?}
%\begin{lstlisting}
%def A(x):
%    if x <= 1 : return x
%    return B(x+1)
%
%def B(x) :
%    return A(x-2)+4
%
%print(A(4))
%\end{lstlisting}
%
%\begin{enumerate}
%\item 13. %+
%\item 1.
%\item 12.
%\item Une erreur de type : "RecursionError: maximum recursion depth exceeded in comparison".
%\end{enumerate}
