On appelle palindrome un mot qui peut se lire indifféremment de gauche à droite ou de droite à gauche. Ainsi par exemple \textit{radar}, \textit{rotor} ou kayak sont trois palindromes.

\question{ Écrire une fonction récursive \texttt{palindrome(x:str) -> bool} prenant en paramètre une chaîne de caractère \texttt{x} et qui retourne \texttt{True} si il s'agit d'un palindrome et \texttt{False} sinon. (Les espaces seront gérés comme de simples caractères). Pour cela on remarquera que :}
\textit{
\begin{itemize}
\item tout mot de longueur $\leq 1$ est un palindrome ;
\item un mot est un palindrome si et seulement si ses premier et dernier caractères sont identiques et son sous-mot allant du 2\up{ème} caractère jusqu'à l'avant dernier est un palindrome.
\end{itemize}}


