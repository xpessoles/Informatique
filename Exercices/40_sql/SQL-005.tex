
\section*{Base de données des Pokemon}
\label{sec:bdd}

Nous allons utiliser la base de données issue du site \url{http://veekun.com/}. Un fichier nommé << veekun-pokedex.sqlite >> doit être présent sur le bureau de votre ordinateur. Ouvrir cette base de données avec DB Browser for SQLite.


%elle contient les colonnes :
%
%\begin{itemize}%[itemsep=0mm]
%	\item \texttt{id} (clé primaire) : identifiant du pokemon 
%	\item \texttt{identifier} : nom du pokemon
%	\item \texttt{species\_id} : identifiant d'espèce
%	\item \texttt{height} : hauteur
%	\item \texttt{weight} : poids
%	\item \texttt{base\_experience} : expérience
%	\item \texttt{order} : ordre
%	\item \texttt{is\_default} : défaut
%\end{itemize}

\subsection*{Structure de la table de données}

\question{}
En utilisant DB Browser for SQLite, donner le nombre de tables contenu dans la base de données. 


Dans un premier temps, nous allons utiliser uniquement la table \texttt{pokemon} qui répertorie les pokémons.

\question{}
Donner le schéma relationnel de cette table. On le donnera sous la forme nom\_table(attribut\_1 : type, attribut\_2 : type, ...).

% Q3
\question{}
Donner la définition d'une clé primaire.

\subsection*{Table des pokemons}


%\question{}
%Combien y a t-il de pokémons dans cette base ? 

% Q4
\question{}
Quelle est la taille de pikachu ?

% Q5
\question{}
Quelle est le poids de pikachu ?

% Q6
\question{}
En utilisant une des valeurs précédentes, quels pokemons sont plus grands (strictement) que pikachu ?

% Q7
\question{}
\textbf{Sans utiliser} une des valeurs précédentes, quels pokemons sont plus grands (strictement) que pikachu ?

%Q8
\question{}
Combien y a t-il de pokemons plus grands (strictement) que pikachu ?

%Q9
\question{}
Combien de pokemons ont la même taille que pikachu (lui y compris)? 

%Q10
\question{}
Parmi les pokemons ayant la même taille que pikachu, donner le nom et le poids du plus gros. 


% Q11
\question{}
Donner le nom et la taille et le poids du plus grand pokemon. 


% Q12
\question{}
Quel pokémon est le plus petit ? (Il peut y en avoir plusieurs ...)


% Q13
\question{}
Lister le nombre de pokemons par taille en les classant du plus grand au plus petit. 

% Q14
\question{}
Quel est le nombre maximal de pokemons ayant la même taille ? 
Donner la taille et le nombre.

% Q15
\question{}
Quels est le nom et la taille du second pokemon le plus grand ? 

% Q16
\question{}
Quelle est la taille moyenne des pokemons ? (deux décimales après la virgule).

%Les paramètres réglés par défaut (angle et vitesse) dans notre application de capture permettent d'atteindre un pokemon de hauteur 8. 

Notre niveau d'expérience permet d'attraper des pokemons de hauteur égale à 8 à 0.5 près (inclus).
Nous souhaiterions donc savoir combien de pokemons pourront être attrapés sans changer ces réglages.

%\begin{qexo}
% Q17
\question{}
 Combien de pokémons sont capturés avec le réglage par défaut ?
%\end{qexo}

\section*{Classement des pokemons}
Maintenant nous souhaitons placer les pokemons sur une carte selon leurs propriétés, pour cela nous allons utiliser les tables suivantes.


La table \texttt{pokemon\_species} contient les colonnes :

\begin{itemize}[itemsep=0mm]
	\item \texttt{id} (clé primaire) : identifiant du pokemon ;
	\item \texttt{identifier} : nom du pokemon;
	\item \texttt{generation\_id} : identifiant de génération qui correspond aussi au numéro de la génération ;
%	\item \texttt{evolves\_from\_species\_id} : origine d'évolution (\emph{non utilisé dans notre étude});
%	\item \texttt{evolution\_chain\_id} : chaine d'évolution (\emph{non utilisé dans notre étude});
%	\item \texttt{color\_id} : identifiant de couleur (\emph{non utilisé dans notre étude});
%	\item \texttt{shape\_id} : identifiant de forme (\emph{non utilisé dans notre étude});
	\item \texttt{habitat\_id} : identifiant d'habitat;
%	\item \texttt{gender\_rate} : taux de genre (\emph{non utilisé dans notre étude});
%	\item \texttt{capture\_rate} : taux de capture (\emph{non utilisé dans notre étude});
%	\item \texttt{base\_happiness} : (\emph{non utilisé dans notre étude});
%	\item \texttt{is\_baby} : (\emph{non utilisé dans notre étude});
%	\item \texttt{hatch\_counter} : (\emph{non utilisé dans notre étude});
%	\item \texttt{has\_gender\_differences} : (\emph{non utilisé dans notre étude});
%	\item \texttt{growth\_rate\_id} : (\emph{non utilisé dans notre étude});
%	\item \texttt{forms\_switchable} : (\emph{non utilisé dans notre étude});
%	\item \texttt{order} : (\emph{non utilisé dans notre étude});
%	\item \texttt{conquest\_order} : (\emph{non utilisé dans notre étude}).
\end{itemize}
D'autres attributs existent, mais ils ne seront pas utilisés dans notre étude. 

La table \texttt{pokemon\_habitats} contient les colonnes :

\begin{itemize}[itemsep=0mm]
	\item \texttt{id} (clé primaire) : identifiant d'habitat;
	\item \texttt{identifier} : nom de l'habitat;
\end{itemize}

 % Q18
\question{}
\'Ecrire la requête SQL permettant d'afficher le nom du pokemon et le nom de son habitat.

 % Q19
\question{}
Combien de pokemons vivent en forêt ('forest' en anglais) ?

 % Q20
\question{}
Combien de pokemons de la generation 3 vivent en forêt ('forest' en anglais) ?


%
%
%
%\begin{qexo}
%\end{qexo}
%\texttt{pokemon(id : integer, identifier : string, species\_id : integer, height : integer, weight : integer, base\_experience : integer, order : integer, is\_default : boolean )} 
%
%\begin{qexo}
%\end{qexo}
%La clé primaire d'une table est une contrainte d'unicité, composée d'une ou plusieurs colonnes, et qui permet d'identifier de manière unique chaque ligne de la table.
%
%
%\begin{qexo}
%\end{qexo}
%\texttt{select count(*) from pokemon;}  \boxed{811}
%
%\begin{qexo}
%\end{qexo}
%\texttt{select height from pokemon where identifier = 'pikachu';} \boxed{4} 
%
%\begin{qexo}
%\end{qexo}
%\texttt{select count(*) from pokemon where height > (select height from pokemon where identifier = 'pikachu');} \boxed{678} 
%
%\begin{qexo}
%\end{qexo}
%\texttt{select count(*) from pokemon where height = (select height from pokemon where identifier = 'pikachu');} \boxed{63} ;
%\texttt{select max(weight) from pokemon where height = (select height from pokemon where identifier = 'pikachu');} \boxed{600}
%
%\begin{qexo}
%\end{qexo}
%\texttt{select identifier from pokemon where height = (select max(height) from pokemon );} \boxed{\text{wailord}}
%\texttt{select identifier from pokemon where height = (select min(height) from pokemon );} \boxed{\text{joltik et flabebe}} 
%
%\begin{qexo}
%\end{qexo}
%\texttt{select avg(height) from pokemon;} \boxed{12.2503082614057}
%
%
%
%\begin{qexo}
%\end{qexo}
% \texttt{select count(*) from pokemon where height <= 8.5 and height >= 7.5;} \boxed{44} 
%
%
%
%\begin{qexo}
%\end{qexo}
%\texttt{select S.identifier,H.identifier from pokemon\_species as S join pokemon\_habitats as H on S.habitat\_id = H.id;}
%
%\begin{qexo}
%\end{qexo}
%\texttt{select count(*) from pokemon\_species as S join pokemon\_habitats as H on S.habitat\_id = H.id where H.identifier = 'forest';}  \boxed{71} 
%
%\begin{qexo}
%\end{qexo}
%\texttt{select count(*) from pokemon\_species as S join pokemon\_habitats as H on S.habitat\_id = H.id where H.identifier = 'forest' and S.generation\_id = 3;} \boxed{29}
%
%
%
%
%
%
%
%
%
%
%
%
