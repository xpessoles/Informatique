\lstset{language=SQL,morekeywords={REFERENCES}}

\question{}
Il y a 172 tables (valeur indiquée dans l'onglet Tables).

\question{}
\texttt{pokemon(id : integer, identifier : string, species\_id : integer, height : integer, weight : integer, bASe\_experience : integer, order : integer, is\_default : boolean )} 

% Q3
\question{}
La clé primaire d'une table est une contrainte d'unicité, composée d'une ou plusieurs colonnes et qui permet d'identifier de manière unique chaque ligne de la table.

%
%\question{}
%\texttt{SELECT count(*) FROM pokemon;}  \boxed{811}

% Q4
\question{}
\begin{lstlisting}
SELECT height 
	FROM pokemon 
	WHERE identifier = 'pikachu';
\end{lstlisting}

\boxed{4} 

% Q5
\question{}
\begin{lstlisting}
SELECT weight 
	FROM pokemon 
	WHERE identifier = "pikachu";
\end{lstlisting}
\boxed{60} 

% Q6
\question{}
\begin{lstlisting}
SELECT identifier 
	FROM pokemon
	WHERE height> 4;
\end{lstlisting}

% Q7
\question{}
\begin{lstlisting}
SELECT identifier 
	FROM pokemon 
	WHERE height > 
		(SELECT height 
		FROM pokemon 
		WHERE identifier = "pikachu");
\end{lstlisting}

%Q8
\question{}
\begin{lstlisting}
SELECT count(*) 
	FROM pokemon 
	WHERE height > 
		(SELECT height 
		FROM pokemon 
		WHERE identifier = 'pikachu');
\end{lstlisting}
\boxed{678} 

%Q9
\question{}
\begin{lstlisting}
SELECT count(*) 
	FROM pokemon 
	WHERE height = 
		(SELECT height 
		FROM pokemon 
		WHERE identifier = 'pikachu');
\end{lstlisting}
 \boxed{63} ;


%Q10
\question{}
\begin{lstlisting}
SELECT identifier,max(weight) 
	FROM pokemon 
	WHERE height = 
		(SELECT height 
		FROM pokemon 
		WHERE identifier = 'pikachu'); 
\end{lstlisting}
\boxed{aron, 600}




% Q11
\question{}
\begin{lstlisting}
SELECT identifier, height, weight
	FROM pokemon 
	WHERE height = 
		(SELECT max(height) FROM pokemon);
\end{lstlisting}
\boxed{\text{wailord, 145, 3980}}


% Q12
\question{}
\begin{lstlisting}
SELECT identifier
	FROM pokemon 
	WHERE height = 
		(SELECT min(height) FROM pokemon); 
\end{lstlisting}

\boxed{\text{joltik et flabebe}} 


% Q13
\question{}
\begin{lstlisting}
SELECT height,count(height) 
	FROM pokemon 
	GROUP BY height
	order BY height DESC
\end{lstlisting}

% Q14
\question{}
\begin{lstlisting}
SELECT height,MAX(ch) FROM 
	(SELECT height,count(height) AS ch
		FROM pokemon 
		GROUP BY height)
\end{lstlisting}
\boxed{\text{6,68}} 

% Q15
\question{}
\begin{lstlisting}
SELECT identifier,max(height) FROM 
	(SELECT identifier,height 
	FROM pokemon 
	WHERE height!=(SELECT max(height) from pokemon))
\end{lstlisting}
\boxed{\text{rayquaza-mega, 108}} 



\question{}% Q16
\begin{lstlisting}
SELECT avg(height) FROM pokemon;
\end{lstlisting}
\boxed{12.2503082614057}
 
 % Q17
\question{}
\begin{lstlisting}
SELECT count(*) 
	FROM pokemon 
	WHERE height <= 8.5 AND height >= 7.5;
\end{lstlisting}
\boxed{44} 

 % Q18
\question{}
\begin{lstlisting}
SELECT S.identifier,H.identifier 
	FROM pokemon\_species AS S 
	JOIN pokemon\_habitats AS H 
	ON S.habitat\_id = H.id;
\end{lstlisting}

 % Q19
\question{}
\begin{lstlisting}
SELECT count(*) 
	FROM pokemon\_species AS S 
	JOIN pokemon\_habitats AS H 
	ON 
	S.habitat\_id = H.id 
	WHERE H.identifier = 'forest';
\end{lstlisting}  
\boxed{71} 

 % Q20
\question{}
\begin{lstlisting}
SELECT count(*) 
	FROM pokemon\_species AS S 
	JOIN pokemon\_habitats AS H 
	ON S.habitat\_id = H.id 
	WHERE H.identifier = 'forest' AND S.generation\_id = 3;
\end{lstlisting} 
\boxed{29}


