\section*{Autour des listes et des dictionnaires}

Soit une liste \lstinline{L} de chaînes de caractères. 

\question{Implémenter une fonction \texttt{comptage\_lettres(L:[str]): -> \{str:int\} } renvoyant le dictionnaire de comptage dont les clefs sont les lettres de la liste \lstinline{L} et les valeurs le nombre d'occurences.}

\begin{exemple}
Si \lstinline{L=['a', 'b', 'a', 'a', 'c', 'e', 'z', 'z', 'a', 'a']}, \lstinline{comptage_lettres(L)} devra renvoyer \texttt{\{'a':5,'b':1,'c':1,'z':3\}}.
\end{exemple}


Soit une liste \lstinline{L} d'entiers. 

\question{Implémenter une fonction \texttt{comptage(L:[int]): -> \{int:int\} } renvoyant le dictionnaire de comptage dont les clefs sont les entiers de la liste \lstinline{L} et les valeurs le nombre d'occurences.}

\begin{exemple}
Si \lstinline{L=[0, 4, 6, 1, 4, 3, 3, 5, 3, 8]}, \lstinline{comptage(L)} devra renvoyer \texttt{\{0:1,4:2,6:1,1:1,3:3,5:1,8:1\}}.
\end{exemple}

\question{Implémenter une fonction \lstinline{suppression_doublons(L:[int]): -> []} renvoyant une liste où tous les doublons auront été supprimés et ce, en conservant l'ordre initial \sidenote{On pourra remarquer que la liste demandée correspond à la liste des clés du dictionnaire de comptage.}.}


\begin{exemple}
Si \lstinline{L=[0, 4, 6, 1, 4, 3, 3, 5, 3, 8]}, \lstinline{suppression_doublons(L)} devra renvoyer \lstinline{L=[0, 4, 6, 1, 3, 5, 8]}.
\end{exemple}

