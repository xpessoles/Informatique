On suppose que la liste à trier est constituée d’entiers de l’intervalle  $\llbracket 0; k \llbracket$. L’algorithme
fonctionne suivant le principe suivant. On parcourt une fois la liste et on comte  le nombre d’éléments de
la liste égaux à 0, 1, ..., $k-1$. Pour ce faire on utilise une liste de taille $k$. On peut alors facilement procéder à une
réécriture de la liste initiale, de sorte qu’en sortie elle soit constituée des mêmes éléments, mais triés dans l’ordre
croissant.


L’algorithme prend en entrée la liste $L$ à trier, ainsi qu’un entier $k$ tel que tous les éléments de la
liste soient des entiers de l’intervalle $\llbracket 0; k \llbracket$. On procède en deux étapes : d’abord compter les éléments de chaque type,
ensuite réécrire la liste $L$.

\begin{lstlisting}
def tri_comptage(L,k):
    C=[0]*k
    for i in range(len(L)):
        C[L[i]]=C[L[i]]+1
        p=0
    for i in range(k):
        for j in range(C[i]):
            L[p]=i
            p+=1
\end{lstlisting}