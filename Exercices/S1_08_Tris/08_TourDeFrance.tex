%\exer{Classement de l'étape 18 Embrun -- Valloire -- 208 km}
%\setcounter{numques}{0}


On donne la bibliothèque de tri \texttt{tris.py} dans laquelle différents tris ont été implémentés.
On dispose ainsi des fonctions : 
\begin{itemize}
\item \texttt{tri\_insertion};
\item \texttt{tri\_rapide};
\item \texttt{tri\_fusion}.
\end{itemize}

\vspace{0.5cm}

Pour augmenter la limite de récursivité de Python, on utilisera les instructions suivantes :\\
\texttt{import sys}\\
\texttt{sys.setrecursionlimit(100000)}.




Les coureurs du tour de France sont en train de terminer la dix huitième étape du Tour de France qui sépare Embrun et Valloire. 

Le fichier \texttt{classement\_general.txt} rassemble le classement général à l'issue de l'étape 17. Le fichier \texttt{etape\_18.txt} contient le classement de l'étape 18 uniquement. Dans le fichier texte, les champs sont séparés par des tabulations.

\begin{obj}
  L'objectif est de réaliser le classement général après la dix huitième étape.
\end{obj}


\subsection*{Lecture des fichiers de résultat}



\question{\'Ecrire la fonction \texttt{chargeClassement(fichier:str)->list} permettant de lire un fichier de classement 
et de renvoyer une liste de la forme \texttt{[[Nom\_1, Dossard\_1, Temps\_1], [Nom\_2, Dossard\_2, Temps\_2], ...]}.\\
\tiny{[['JULIAN ALAPHILIPPE ', '21', "69h 39' 16''"]]}
}
\ifprof
\begin{lstlisting}
def chargeClassement(fichier):
    f=open(fichier,'r')
    fichier=f.readlines()
    f.close() #ne pas oublier de fermer le fichier...
    L=[]
    for ligne in fichier:
        ligne=ligne.split('\t') # coupe aux tabulations
        L1=[]
        L1.append(ligne[1])
        L1.append(ligne[2])
        L1.append(ligne[4])
        L.append(L1)
    return L
\end{lstlisting}
\else\fi


\question{\'Ecrire la fonction \texttt{convertirTemps(temps:str)->int} qui évalue le temps exprimé en heure, minutes et secondes en une valeur en seconde.\\
\tiny{convertirTemps("69h 39' 16''")=250756s}}
\ifprof
\begin{lstlisting}
def convertirTemps(temps:str):
    '''temps est un str de la forme "06h 09' 39''" '''
    t_course=temps.split('h') #on peut aussi couper a h
    heure=int(t_course[0])
    t_course2=t_course[1].split("'")
    duree=int(t_course2[1])+60*int(t_course2[0])+3600*heure
    return duree
\end{lstlisting}
\else\fi



\question{\'Ecrire la fonction \texttt{classement(fichier:str)->list}  permettant de lire un fichier de classement 
et de renvoyer une liste de la forme \texttt{[[Nom\_1, Dossard\_1, Temps\_1], [Nom\_2, Dossard\_2, Temps\_2], ...]} les temps exprimés en seconde. Vous devez utiliser les deux fonctions précédentes.\\
\tiny{LG[:1] affiche [['JULIAN ALAPHILIPPE ', '21', 250756]] et L18[:] affiche [['NAIRO QUINTANA ', '61', 20055]]}}
\ifprof
\begin{lstlisting}
def classement(fichier):
    L=chargeClassement(fichier)
    for element in L:
        element[2]=convertirTemps(element[2])
    return L
\end{lstlisting}
\else\fi


\subsection*{Classement en fin d'étape}
Dans une première approche, on souhaite réaliser le classement général après la fin de l'étape 18.

\begin{rem}
Pour faire une copie complète d'une liste de listes, il faut utiliser la fonction \texttt{deepcopy} du module \texttt{copy}. C'est une fonction récursive (ou profonde) qui construit un nouvel objet composé, puis récursivement, insère dans l'objet composé des copies des objets trouvés dans l'objet original. Attention les objets récursifs (objets composés qui, directement ou indirectement, contiennent une référence à eux-mêmes) peuvent causer une boucle récursive infinie.
\end{rem}

\question{Réaliser la fonction \texttt{ajoutTemps(liste1:list,liste2:list)->list} qui à partir des deux listes du classement de l'étape 18 et du classement général renvoie la liste de la forme  \texttt{[[Nom\_1, Dossard\_1, Temps\_1], [Nom\_2, Dossard\_2, Temps\_2], ...]} dont les temps sont la somme des temps des deux listes pour chaque coureur.\\
\tiny{['NAIRO QUINTANA ', '61', 271381]}}
\ifprof
\begin{lstlisting}
def ajoutTemps(L1,L2):
    '''L1 est la classement de l'etape et L2 le classement general certains ont ete
    disqualifies ou ont abandonne'''
    assert len(L1)<=len(L2)
    Lnew=deepcopy(L1)
    n=len(Lnew)
    for i in range(n):
        d=Lnew[i][1]
        j=0
        while d!=L2[j][1]: #attention certains n'ont pas fait l'etape completement
        # et sont disqualifies
            j+=1
        Lnew[i][-1]=Lnew[i][-1]+L2[j][-1]
    return Lnew
\end{lstlisting}
\else\fi



\question{Quelle méthode de tri vous semble la mieux adaptée au tri du classement général ?}
\ifprof
\begin{lstlisting}
\end{lstlisting}
\else\fi





\question{Modifier les algorithmes de tris pour pouvoir trier la liste obtenue en sortie de la fonction \texttt{ajoutTemps(liste1:list,liste2:list)->list} suivant le temps de course d'un coureur. Le classement général a-t-il changé à l'issue de la dix huitième étape ?\\
\tiny{[['JULIAN ALAPHILIPPE ', '21', 271129], ['EGAN BERNAL ', '2', 271219], ['GERAINT THOMAS ', '1', 271224]]}}
\ifprof
\begin{lstlisting}
def tri_insertion_modifie(tab):
    '''Trie la liste t
    Entree :
        Une liste
    Sortie :
        La liste est modifiee mais n est pas renvoyee'''
    for i in range(1,len(t)) :
        element=t[i]
        x=t[i][-1]
        k=0
        while (k<i and x>t[k][-1]) :
            k=k+1
        for j in range(i,k,-1) :
            t[j]=t[j-1]
        t[k]=element
\end{lstlisting}
\else\fi



%\newline
\begin{rem}
Travaillant sur une liste de listes, la méthode \texttt{sort}  n'est plus adaptée. On peut donc utiliser la méthode \texttt{sorted} en utilisant une clef de tri (la clef correspondant à la colonne sur laquelle on souhaite trier la liste), tri de la liste \texttt{Liste} sur la colonne i :

\texttt{sorted(Liste, key=lambda colonnes: colonnes[i])}
\end{rem}



\subsection*{Classement en cours d'étape -- Implémentation d'une file}
On cherche à reconstituer le classement général au fur et à mesure que les coureurs arrivent. Dans cette partie le classement de l'étape (liste de listes) sera vu comme une \textbf{file} FIFO (First In First Out) où le premier élément est le premier coureur arrivé et le dernier élément est le dernier coureur à avoir passé la ligne d'arrivée.  

\question{Implémenter les fonctions élémentaires liées à la gestion des files : \texttt{enfiler}, \texttt{defiler}, \texttt{est\_vide}. À l'intérieur de ces fonctions, on s'autorise les méthodes liées aux listes (\texttt{append}, \texttt{pop}, ...).}
\ifprof
\begin{lstlisting}
def enfiler(file, element):
    return file.append(element)

def defiler(file):
    return file.pop(0)

def est_vide(file):
    return len(file)==0
\end{lstlisting}
\else\fi


\question{Implémenter la fonction \texttt{ajout} ayant pour but d'ajouter le temps de l'étape d'un coureur dans le classement général et de mettre à jour ce classement. La gestion du classement de l'étape devra être réalisé grâce à une liste.}
\ifprof
\begin{lstlisting}
def ajout(L_etape_triee,LG):
    ''' ajout au classement general le temps de la nouvelle etape et refait le classement'''
    new_classement=[]
    while est_vide(L_etape_triee)==False: #tant que la file n'est pas vide
        cycliste=defiler(L_etape_triee)# on prend le premier element et on l'enleve
        i=0
        new_classement.append(cycliste)# on place le nouvel arrivant a la queue de la liste
                                        # triee du classement general
        while cycliste[1]!=LG[i][1]: #on cherche le meme dossard de cycliste
            i=i+1
        new_classement[-1][-1]=new_classement[-1][-1]+LG[i][-1] #on additionne les temps du
                                                # classement general et du classement d'etape
        tri_insertion_modifie(new_classement) #on trie le nouveau classement
    return new_classement
\end{lstlisting}
\else\fi


\question{Quelle pourrait être l'utilité de la fonction \texttt{enfiler} dans un tel contexte ?}
\ifprof
\begin{lstlisting}
\end{lstlisting}
\else\fi

\vspace{1cm}


\textit{ANNEXE : opérations et fonctions Python disponibles}


Pour la copie de liste de listes, le module \texttt{copy} avec la fonction \texttt{deepcopy} sont efficaces.\\
Ci-dessous un exemple avec la fonction \texttt{copy} du module \texttt{copy} et avec la fonction \texttt{deepcopy} du module \texttt{copy}.

\begin{lstlisting}
from copy import copy, deepcopy

L = [['JULIAN ALAPHILIPPE ', '21', 250756], ['GERAINT THOMAS ', '1', 250851],['STEVEN KRUIJSWIJK ', '81', 250863]]

L_copy = copy(L)   # copie superficielle
L_deepcopy = deepcopy(L)  # copie profonde

L[1][0] = 5

copy:  [['JULIAN ALAPHILIPPE ', '21', 250756], [5, '1', 250851],
['STEVEN KRUIJSWIJK ', '81', 250863]]
deepcopy:  [['JULIAN ALAPHILIPPE ', '21', 250756], ['GERAINT THOMAS ', '1', 250851],
['STEVEN KRUIJSWIJK ', '81', 250863]]
\end{lstlisting}









