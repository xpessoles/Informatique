

\question{Ecrire la fonction on pourra (au choix) utiliser l'une des signatures suivante : \texttt{tri\_comptage(L:list,k:int) -> None}  ou \texttt{tri\_comptage(L:list,k:int) -> list} permettant de réaliser un tri par comptage (avec ou sans effet de bord).}

\begin{lstlisting}
def tri_comptage(L:list,k:int):
    C=k*[0]
    a=[]
    for x in L:
        for i in range(k):
            if x==i:
                C[i]+=1
    for i in range(k):
        a+=C[i]*[i]
    return a


def tri_comptage2(L,k):
    C=[0]*k
    for i in range(len(L)):
        C[L[i]]=C[L[i]]+1
    p=0
    for i in range(k):
        for j in range(C[i]):
            L[p]=i
            p+=1
\end{lstlisting}

\question{La fonction proposée agit-elle avec effet de bord ? Sans effet de bord ?}

Selon la version du code la méthode se fait avec ou sans effet de bord.
La première méthode proposée est sans effet de bord alors que la deuxième oui.

\question{La fonction proposée réalise-t-elle un tri stable ? un tri en place ?}

\ifprof
\begin{lstlisting}
def tri_comptage(L,k):
    C=[0]*k
    for i in range(len(L)):
        C[L[i]]=C[L[i]]+1
    p=0
    for i in range(k):
        for j in range(C[i]):
            L[p]=i
            p+=1
            
# Il s'agit ici d'un tri avec effet de bord et non stable. 
\end{lstlisting}
\else
\fi