\exer{Tri à bulles}
\setcounter{numques}{0}
Pour trier une liste selon la méthode du tri à bulles, on réalise des balayages successifs : à chaque balayage, on compare les éléments du tableau 2 à 2 et on les réordonne. 

\question{En utilisant la liste  \texttt{[10, 3, 7, 5, 9, 7, 8, 0, 8]}, écrire la séquence d'échanges permettant d'arriver à la liste triée.}
\ifprof
Première série de comparaison sur la liste faisant monter 10 : 
[10, 3, 7, 5, 9, 7, 8, 0, 8],
[3, 10, 7, 5, 9, 7, 8, 0, 8],
[3, 7, 10, 5, 9, 7, 8, 0, 8],
[3, 7, 5, 10, 9, 7, 8, 0, 8],
[3, 7, 5, 9, 10, 7, 8, 0, 8],
[3, 7, 5, 9, 7, 10, 8, 0, 8],
[3, 7, 5, 9, 7, 8, 10, 0, 8],
[3, 7, 5, 9, 7, 8, 0, 10, 8],
[3, 7, 5, 9, 7, 8, 0, 8, 10]\\
Deuxième série de comparaison à partir des deux premiers éléments :
[3, 7, 5, 9, 7, 8, 0, 8, 10],
[3, 5, 7, 9, 7, 8, 0, 8, 10],
[3, 5, 7, 7, 9, 8, 0, 8, 10],
[3, 5, 7, 7, 8, 9, 0, 8, 10],
[3, 5, 7, 7, 8, 0, 9, 8, 10],
[3, 5, 7, 7, 8, 0, 8, 9, 10]\\
Troisième série de comparaison à partir des deux premiers éléments :
[3, 5, 7, 7, 8, 0, 8, 9, 10],
[3, 5, 7, 7, 0, 8, 8, 9, 10]\\
Quatrième série de comparaison à partir des deux premiers éléments :
[3, 5, 7, 7, 0, 8, 8, 9, 10],
[3, 5, 7, 0, 7, 8, 8, 9, 10]\\
Cinquième série de comparaison à partir des deux premiers éléments :
[3, 5, 7, 0, 7, 8, 8, 9, 10],
[3, 5, 0, 7, 7, 8, 8, 9, 10]\\
Sixième série de comparaison à partir des deux premiers éléments :
[3, 5, 0, 7, 7, 8, 8, 9, 10],
[3, 0, 5, 7, 7, 8, 8, 9, 10]\\
Septième série de comparaison à partir des deux premiers éléments :
[3, 0, 5, 7, 7, 8, 8, 9, 10],
[0, 3, 5, 7, 7, 8, 8, 9, 10]\\
Huitième série de comparaison à partir des deux premiers éléments :
[0, 3, 5, 7, 7, 8, 8, 9, 10]
\else\fi

\question{Donner un algorithme naïf permettant de trier un algorithme selon la méthode du tri à bulles.}
\ifprof
\begin{lstlisting}
def tri_bulles_naif(l):
    for i in range(len(l)):
        for j in range(len(l)-1): #on ne peut pas comparer le dernier élément avec un suivant
            if l[j]>l[j+1]:
                l[j],l[j+1]=l[j+1],l[j]
\end{lstlisting}
\else\fi

\question{Dans quel cas est-on dans le meilleur des cas ? Quelle alors la complexité de l'algorithme ?}
\ifprof
Quand la liste est triée, il n'y a pas d'échange d'élément. On a qu'en même les deux boucles \texttt{for} imbriquées.
\else\fi

\question{Dans quel cas est-on dans le pire des cas ? Quelle alors la complexité de l'algorithme ?}
\ifprof
Quand la liste est triée dans le sens inverse, il y a échange de tous les éléments. On a qu'en même les deux boucles \texttt{for} imbriquées.
\else\fi

\question{En remarquant qu'à l'étape $i$, les $i$ derniers éléments sont triés, proposer un nouvel algorithme du tri à bulles.}
\ifprof
\begin{lstlisting}
def tri_bulles(l):
    for i in range(0,len(l)-1):
        for j in range(0,len(l)-i-1):
            if l[j]>l[j+1]:
                l[j],l[j+1]=l[j+1],l[j]
\end{lstlisting}
\else\fi

\question{En remarquant qu'à l'itération $i$, il est possible d'arrêter le tri là où la dernière inversion a eu lieu à l'étape $i-1$, proposer un nouvel algorithme du tri à bulles.}
\ifprof
\begin{lstlisting}
def tri_bulles_optimise(tableau):
    permutation = True
    passage = 0
    while permutation == True:
        permutation = False
        passage = passage + 1
        for en_cours in range(0, len(tableau) - passage):
            if tableau[en_cours] > tableau[en_cours + 1]:
                permutation = True
                tableau[en_cours], tableau[en_cours + 1] = tableau[en_cours + 1],tableau[en_cours]
\end{lstlisting}
\else\fi