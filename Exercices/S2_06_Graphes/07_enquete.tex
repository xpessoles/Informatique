\subsection{Graphe d'intervalles : Énigme de Claude BERGE}
A l'époque, le Duc de Densmore avait été tué par l'explosion d'une bombe artisanale, qui avait également détruit le château de Densmore
où il s'était retiré. Les journaux d'alors relataient que le testament, détruit lui aussi par l'explosion, avait tout pour
déplaire à l'une de ses huit ex-femmes. Or, avant sa mort, le Duc les avait toutes invitées à passer quelques jours dans sa retraite écossaise.

Les suspects sont donc ces huit femmes présentes sur l'île au moment des faits. Chacune n'y ayant effectué qu'un seul
séjour. L'une d'entre elles a dû se cacher à un moment propice pour aller préparer et poser la bombe.

Les faits étant assez anciens, les suspects se souviennent des personnes qu'elles ont croisées, mais pas dans quel ordre. Voici les témoignages :

\begin{itemize}
\item Ann a déclaré y avoir rencontré Betty, Cynthia, Emily, Felicia et Georgia ;
\item Betty a déclaré y avoir rencontré Ann, Cynthia et Helen ;
\item Cynthia a déclaré y avoir rencontré Ann, Betty, Diana, Émily et Helen;
\item Diana a déclaré y avoir rencontré Cynthia et Emily ;
\item Emily a déclaré y avoir rencontré Ann, Cynthia, Diana et Felicia ;
\item Felicia a déclaré y avoir rencontré Ann et Emily;
\item Georgia a déclaré y avoir rencontré Ann et Helen ;
\item Helen a déclaré y avoir rencontré Betty, Cynthia et Georgia.
\end{itemize}

\medskip 
Précision supplémentaire, personne d'autre n'est venu sur l'île. Par ailleurs si deux femmes ne se sont pas vues,
c'est soit qu'elles n'étaient pas l'île en même temps, soit que l'une d'entre elles était cachée dans le labyrinthe et préparait la bombe artisanale ... 

\medskip
L'inspecteur qui vous a précédé sur l'affaire a constaté que les témoignages concordaient et avaient conclue
à la culpabilité du majordome mort dans l'explosion avec le duc. Mais ce n'est pas le cas. Pouvez-vous trouver le coupable ?


\vfill
\textbf{Références :}\\
T. Kovaltchouk, \textit{Informatique Commune PCSI}, Reims\\
UPSTI, \textit{Informatique Commune}