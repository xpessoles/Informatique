\section*{Le problème du choix d'activité}
%\setcounter{cexo}{0}
\subsection*{Présentation du problème}

Le problème du choix d'activité (PCA) concerne le problème d'ordonnancement de plusieurs activités qui rivalisent pour l'utilisation exclusive d'une ressource commune.\\
Une application concrète est l'allocation d'une salle à différents conférenciers au cours d'une journée. Plusieurs personnes souhaitent effectuer une conférence, mais ces conférences ont lieu à des horaires définis imposés par la disponibilité de chaque conférencier.\\
 Certaines conférences démarrent alors que d'autres ne sont pas encore terminées. Il n'est donc pas possible d'attribuer la salle à chaque conférencier pour chacune de ses conférences, et il faut faire des choix.\\ 
La solution optimale est celle qui permettra au plus grand nombre de conférences de se tenir.


\begin{figure}[h]
	\centering
%		\includegraphics{image/conferences.png}
		\includegraphics[width=\linewidth]{conf.png}
		\caption{Conférences parmi lesquelles choisir pour établir le planning de la salle de conférences}
	\label{fig:conf}
\end{figure}


Les conférences sont définies par leur heure de début et leur heure de fin. La conférence $c_i$ aura pour heure de début $d(c_i)$ et pour heure de fin $f(c_i)$.

Nous supposerons que les conférences ont été triées dans l'ordre croissant de leurs dates de fin. Si nous travaillons avec un ensemble C de n conférences, on a :
\begin{center} 
$f(c_1)<f(c_2)<\ldots<f(c_{n-1})<f(c_n)$
\end{center} 

On a bien affaire ici à un problème d'optimisation :
\begin{itemize}
\item soit $S={\{s_1,s_2,\ldots,s_k\}}$ l'ensemble des conférences constituant la solution ;
\item l'objectif : maximiser le nombre de conférences : maximiser $|S|$ ;
\item la contrainte :  les conférences doivent être compatibles, pour tout $i,j\in {\{1,…,k\}}$, on doit avoir $f(s_i )\leqslant d(s_j)$ ou $f(s_j )\leqslant d(s_j)$.
\end{itemize}

\subsection*{La solution gloutonne}

Dans notre application, le premier choix glouton consiste à choisir la conférence qui termine le plus tôt possible, préservant ainsi la disponibilité de la salle pour les conférences ultérieures.

Lorsqu'un choix de conférence a été fait, on procède de la même manière pour le choix suivant, mais il y a la contrainte de compatibilité : il faut choisir une conférence compatible avec celles qui ont déjà été choisies : On ne peut choisir qu'une conférence qui débute après la fin de la dernière conférence choisie. Si on note $C$ l'ensemble des conférences, et $S$ l'ensemble des conférences qui constituent notre solution optimale, à l'itération $k$ on a :
$S={s_1,s_2,\ldots,s_k}$ et $S\subset C$.

Alors il faut choisir la conférence $c_i \in C-S$ telle que $d(c_i )\leq f(s_k)$. 
L’algorithme construit alors un ensemble $S$ de conférences mutuellement compatibles, c’est-à-dire tel que :
$d(s_1 )<f(s_1 )\leq d(s_2)<f(s_2)\leq \ldots \leq d(s_j )<f(s_j)$.

\subsection*{Algorithme et test à la main }

\textbf{Paramètres : } $C$ est la liste des conférences, représentées par leurs dates de début et de fin, et leur intitulé \lstinline{[d(ci),f(ci),'ci']}.

\textbf{Résultat : } La liste \lstinline{Planning} de taille maximale de conférences compatibles.

\textbf{Algorithme : }
\begin{itemize}
\item la liste \lstinline{planning} est initialisée avec la première conférence ;
\item la variable \lstinline{fin} prend la valeur de la fin de la première conférence ;
\item pour tous les éléments $c_i$ de la liste \lstinline{C}, on teste si $c_i[0]$>=fin, alors on insère $c_i$ dans le planning et on remplace la valeur de \lstinline{fin} par $c_i[1]$, sinon, on ne fait rien.
\end{itemize}

%\begin{lstlisting}
%Planning <- [C[0]]   #les conférences étant classées par ordre croissant de leur heure de fin, 
%Fin <-  C[0][1]        #on sait que la première valeur de C correspond au premier choix glouton 
%Pour tous les elements c_i de la liste C :
%	Si c_i[0] >= Fin :
%		Inserer c_i a la suite de Planning
%		Fin <- c_i[1]
%\end{lstlisting}



\begin{question}
Exécuter à la main cette méthode en utilisant l'ensemble de conférences illustré sur la figure \ref{fig:conf}.
\end{question}

\subsection*{Implémentation}
\label{sec:Implémentation}

On modélise l'ensemble \lstinline{C} des conférences par une liste dont chaque élément modélise une conférence. Chaque conférence est modélisée par une liste de la date de début de type \lstinline{float} avec par exemple 8.45 qui correspond à 8h45, la date de fin de type \lstinline{float}, et l'intitulé de la conférence de type \lstinline{str}. Cette liste est ordonnée par ordre croissant de la date de fin des conférences. Ainsi, l'exemple représenté sur la figure \ref{fig:conf} pourrait être modélisé par la liste :

 

\lstinline{L=[[8,9,"C1"],[8.45,10.3,"C2"],[8.1,11.3,"C3"],[11.15,11.45,"C4"],[12,12.45,"C5"],[11,13,"C6"], [12.3,14,"C7"],[16,17,"C8"],[15,18,"C9"],[16.2,18.15,"C10"]]}


 

 

\begin{question}
\'Ecrire la fonction \lstinline{choix_conferences(C:list)->list} qui prend en argument une liste de conférences \lstinline{C}, et qui applique la méthode décrite précédemment. 
Le résultat est une liste de conférences mutuellement compatibles, de longueur maximale.
\end{question}

 


\begin{question}
\'Ecrire les instructions permettant d'utiliser la fonction \lstinline{choix_conferences(C:list)->list} pour déterminer la liste des conférences retenues parmi les conférences listées dans \lstinline{L}. Comparer le résultat avec celui obtenu avec le test à la main.
\end{question}

 

On observe que le résultat obtenu est bien un optimal. Cependant, il existe d'autres solutions, toutes autant optimales, par exemple C2, C4, C7, C10 ou encore C1, C4, C5, C9.

\subsection*{Le choix glouton et le sous problème optimal}
\label{sec:LeChoixGloutonEtLeSousProblèmeOptimal}
Un algorithme glouton ne mène pas forcément à une solution optimale d'un problème d'optimisation.
Pour qu'un algorithme glouton soit adapté à la résolution d'un problème, il faut que le problème ait la propriété de sous-structure optimale, c'est-à-dire que la solution optimale du problème contient des solutions optimales des sous-problèmes.

Une résolution par la méthode gloutonne nécessite par ailleurs que les choix gloutons successifs engendrent une solution optimale globalement. C’est la propriété de choix glouton.

Pour notre application :
$S={s_1,s_2,…,s_k}$  est le résultat de notre algorithme. On montre par récurrence que pour tout $j\leq k$, le sous-ensemble $S_j={s_1,s_2,… ,s_j}$ de S est un sous ensemble d’une solution optimale :
\begin{itemize}
	\item Pour j=0 : l'ensemble $S_j$ est un ensemble vide, et est donc un sous-ensemble d'une solution optimale,
	\item Pour j>0 : Soit T une solution optimale telle que pour $j<k$,  $S_j\subset T$  (T contient les activités $s_1,s_2,… ,s_j$). On note b l’élément suivant $s_j$ dans T: 
	\begin{itemize}
		\item On a $f(s_{j+1}  ) \leq f(b)$ (en raison du choix glouton, on choisit l’activité que finit le plus tôt),
		\item les éléments de T sont tous compatibles, ainsi chaque élément succédant à b débute après la fin de b et de $s_{j+1}$.
	\end{itemize}
Ainsi $T- \left\{b\right\} \bigcup \left\{s_{j+1} \right\}$ est une solution optimale, de même cardinal  que $S$, et elle contient $s_1,s_2,… ,s_{j+1} $
\end{itemize}
	
$S$ est donc une solution optimale.
On a bien démontré que l’algorithme du choix d’activité est correct, présente la propriété de sous-structure optimale, et que le choix glouton permet d’arriver à une solution optimale globalement.


\begin{question}
\'Ecrire la fonction \lstinline{choix_conferences_recursif(C:list,k:int,n:int)->list} qui prend en argument une liste de conférences \lstinline{C}, un indice \lstinline{k} et \lstinline{n} la taille de C, et qui applique la méthode décrite précédemment de manière récursive. 
Le résultat est une liste de conférences mutuellement compatibles, de longueur maximale.
\end{question}

\vfill


Références :
\begin{itemize}
\item Algorithmique – Cormen Leiserson Rivest Stein - Dunod 
\item Matroïdes et algorithmes gloutons : UNE INTRODUCTION - Pierre Béjian 
\item Algorithmes gloutons - Ressource éduscol NSI 
\item Algorithmes gloutons - Emmanuel Beffara - lis-lab 
\end{itemize}

