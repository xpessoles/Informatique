%\exer{}
\setcounter{numques}{0}

\textit{D'après documents de Serge Bays.}

Nous considérons la variante « entière » du problème du sac à dos.
Nous sommes devant un ensemble de $n$ objets. Chaque objet noté $o_i$ a une valeur notée $v_i$ et un poids
noté $p_i$. Il s’agit d’emporter dans son sac à dos l’ensemble d’objets qui a la plus grande valeur sachant que
le sac supporte un poids maximum $P$. Comment résoudre ce problème, quels objets doit-on prendre ?

Pour appliquer une stratégie gloutonne, nous devons définir ce que nous entendons par le meilleur
choix à chaque étape.

Il y a trois manières ici de définir un meilleur choix. Parmi les objets qui n’ont pas encore été pris,
soit on choisit un objet qui a la valeur maximale, soit un objet qui a le poids minimal, soit un objet qui a le
rapport valeur/poids maximal.

Nous ne considérons que les objets ayant un poids $p_i \leq P$. L’algorithme glouton consiste, à chaque
étape, à choisir parmi ces objets celui qui représente le choix optimal. Nous le notons $O_1$, sa valeur $V_1$ et
son poids $P_1$. Ensuite, nous recommençons parmi les objets de poids $p_i \leq P - P_1$. Et ainsi de suite.

Cette variante est dite entière parce que chaque objet est pris ou pas. Il existe une version fractionnaire.
Le principe de base est le même, mais cette fois il est possible de prendre des fractions d’objets. Un
algorithme glouton donne une solution optimale à cette variante fractionnaire.
Prenons un exemple : le sac à dos peut contenir \SI{15}{kg}. Les poids des objets sont en \si{kg}, les valeurs
en euro.

\begin{center}
\begin{tabular}{|c|c|c|c|}
Objet & Valeur & Poids & Valeur/Poids \\ \hline \hline 
Objet 1 & 126  & 14   & 9 \\ \hline
Objet 2 & 32    & 2    & 16 \\ \hline
Objet 3 & 20    & 5    & 4 \\ \hline
Objet 4 & 5      & 1    & 5 \\ \hline
Objet 5 & 18    & 6    & 3 \\ \hline
Objet 6 & 80    & 8    & 10 \\ \hline
\end{tabular}
\end{center}


Un objet est représenté par une liste comme \texttt{['objet 1', 126, 14]}.

On a donc : 
\begin{lstlisting}	
objets = [[’objet 1’, 126, 14], [’objet 2’, 32, 2], [’objet 3’, 20, 5], [’objet 4’, 5, 1], [’objet 5’, 18, 6], [’objet 6’, 80, 8]]
\end{lstlisting}

\question{Définir la fonction \texttt{def valeur(objet:list) -> float} qui renvoie la valeur d'un objet.}

\question{Définir la fonction \texttt{def poids(objet:list) -> float} qui renvoie le poids d'un objet.}

\question{Définir la fonction \texttt{def rapport(objet:list) -> float} qui renvoie le rapport valeur/poids d'un objet.}

L'algorithme glouton du sac à dos est défini ainsi : 
\begin{itemize}
\item définir la fonction \texttt{glouton} qui prend en paramètres une liste d’objets, un poids maximal (celui que peut supporter le sac à dos) et le type de choix utilisé (par valeur, par poids, ou par valeur/poids);
\item trier la liste d'objets suivant le type de choix utilisé par ordre décroissant;
\item définir la variable \texttt{reponse:list} sert à stocker les objets choisis;
\item parcourir la liste triée et ajouter les noms des objets un par un tant que le poids ne dépasse pas le poids maximal. 
\item stocker valeur totale et le poids total sont stockés dans deux variables \texttt{valeur} et \texttt{poids}.
\end{itemize}

\begin{rem}
Tri d'une liste selon un critère : 
\begin{lstlisting}
copie = sorted(liste, key=critere, reverse=True)
\end{lstlisting}
\texttt{critere} peut être une des fonctions précédemment définies.
\end{rem}

\question{Implémenter la fonction \texttt{glouton} qui renvoie (dans l'ordre) \texttt{reponse} et \texttt{valeur}.}

\question{Exécuter la fonction \texttt{glouton} pour les différents types de choix. On observant la nature des différents objets, le choix optimal est-il parmi les choix proposés ?}

\question{Estimer le nombre de tours de boucle nécessaire pour exécuter l'algorithme \texttt{glouton} en fonction du nombre d'objets \texttt{n}.}

On cherche maintenant à tester l'ensemble des combinaisons possibles permettant de remplir le sac à dos. 

\question{Estimer (grossièrerement) le nombre de combinaisons possibles parmi la liste d'objets qui permettrait de remplir le sac à dos. Conclure sur l'intérêt d'un algorithme glouton.}

\question{Proposer une variante récursive de l'algorithme glouton.}
