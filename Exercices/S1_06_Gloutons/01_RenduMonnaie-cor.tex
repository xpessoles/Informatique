

\question{Écrire une fonction \texttt{rendre\_monnaie(caisse:list, cout:float, somme\_client:float)-> list} prenant en arguments deux flottants \texttt{cout} et \texttt{somme\_client} représentant le coût d'un produit et la somme donnée par le client en \euro{} ainsi que le contenu de la caisse. Cette fonction renvoie la liste des billets à rendre par le client.}

\begin{lstlisting}

def rendre_monnaie(caisse:list, cout:float, somme_client:float) -> list :
    """
    Algorithme glouton du rendu de monnaie

    Parameters
    ----------
    caisse : list
        liste de liste contenant les valeurs des billets et leur quantité.
        On utilise une liste (plutôt qu'un dictionnaire) car les valeurs
        doivent être triées
    cout : float
        coût à payer en euros
    somme_client : float
        somme donnée par le client (en euros)

    Returns
    -------
    list
        DESCRIPTION.

    """
    rendu = []
    somme = int(100*somme_client)
    cout = int(100*cout)
    delta = somme- cout
    while delta > 0 :
        for valeur in caisse :
            while valeur[0]<=delta :
                delta = delta - valeur[0]
                rendu.append(valeur[0])
    return rendu


\end{lstlisting}


\question{Ecrire une fonction \texttt{rendre\_monnaie\_v2(caisse:list, cout:float, somme\_client:float)-> list} ayant le même objectif que la précédente. Cette fonction devra de plus mettre à jour la caisse. Elle devra prendre en compte que la caisse peut manquer de billets. Elle renverra une liste vide s'il n'est pas possible de rendre la monnaie.}

\begin{lstlisting}

def rendre_monnaie_v2(caisse:list, cout:float, somme_client:float) -> list :
    """
    Algorithme glouton du rendu de monnaie avec MAJ de la caisse.
    """
    rendu = []
    somme = int(100*somme_client)
    cout = int(100*cout)
    delta = somme- cout
    while delta > 0 :
        for i in range(len(caisse)):
            while caisse[i][0]<=delta and caisse[i][1]>0:
                delta = delta - caisse[i][0]
                caisse[i][1]=caisse[i][1]-1
                rendu.append(caisse[i][0])
        print(delta)
        if delta>0 :
            return []
    return rendu

\end{lstlisting}


\question{Que retourne la fonction \texttt{rendre\_monnaie} ? Est-ce le rendu optimal ?}

La solution optimale serait [2000,500,100] et l'algorithme \texttt{rendre\_monnaie\_v2(caisse, 34, 50)}, l'algorithme renverra \texttt{[1000, 100, 100, 100, 100, 100, 100]}.

\question{Conclure << qualitativement >>.}

L'algorithme glouton renvoie un optimum local et non pas un optimum global comme le montre l'exemple précédent.