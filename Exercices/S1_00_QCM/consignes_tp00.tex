
\subsection*{Présentation des enseignants}


\begin{center}
\begin{tabular}{ccc}
MPSI 1 & MPSI 1 \& 2 & MPSI 2 \\
\url{xpessoles@lamartin.fr} & \url{jpberne@lamartin.fr} & \url{edurif@lamartin.fr} \\
Xavier Pessoles & Jean-Philippe Berne & Emilien Durif \\
Enseignant de SII en PSI$\star$ et MP  & Enseignant de Mathématiques en MP & Enseignant de SII en PSI et MPSI 2 \\  
et d'informatique en MPSI 1 &  et d'informatique en MPSI 1 \& 2  &  et d'informatique en MPSI 2 \\
\end{tabular}
\end{center}


\subsection*{Site de la classe}
\begin{center}
\url{https://mpsilamartin.github.io/info}
\end{center}

\subsection*{Travail à faire }

Afin d'évaluer vos connaissances Python, nous vos invitons à réaliser le QCM suivant \textbf{avant le 3 septembre 23h59}.
\url{https://forms.gle/H7tuEt2u68nXb14N7} ou encore 


\begin{center}

\qrcode{https://forms.gle/H7tuEt2u68nXb14N7}

\end{center}

\vspace{.5cm}


\subsection*{Utilisation des ordinateurs personnels}

L'utilisation des ordinateurs personnels est possible en TP. Ils requièrent l'installation de Python. Pour les utilisateurs de Windows, il est par exemple possible d'utiliser Anaconda \url{https://repo.anaconda.com/archive/Anaconda3-2021.05-Windows-x86_64.exe}. 

Anaconda est aussi disponible sous Mac et sous GNU/Linux.



\subsection*{Jeudi 2 septembre}

Si vous le souhaitez, vous pouvez venir en salle B306 entre 15h30 et 17h30 pour réaliser le QCM et/ou réaliser l'installation de Python sur votre ordinateur.

