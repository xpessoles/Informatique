%\exer{Fonction \texttt{mystere}}
\ifprof 
\else
\begin{lstlisting}
def mystere(n,p):
    if p < 0 or p > n :
        return 0
    else :
        f = 1
        for i in range(p) :
            f = f * (n + 1 - p + i) // (i + 1)
    return f
\end{lstlisting}
\fi

\question{Que renvoie la fonction \texttt{mystere(4,8)} ?}
\ifprof
\begin{corrige}
\begin{lstlisting}
0
\end{lstlisting}
\end{corrige}
\else
\fi

\vspace{-.5cm}

\question{Que renvoie la fonction \texttt{mystere(8,4)} ? Pour cela, on recopiera et on complètera le tableau suivant.}
\ifprof 
\else
\vspace{-1cm}
\begin{center}
\begin{tabular}{|c|c|}
\hline 
Valeur de \texttt{i} en entrée de boucle & Valeur de \texttt{f} en entrée de boucle \\
\hline 
0 & 1 \\
\hline 
... & ...\\
\hline 
\end{tabular}
\end{center}
\fi

\ifprof
\begin{corrige}
\begin{lstlisting}
0 1
1 5
2 15
3 35

70
\end{lstlisting}
\end{corrige}
\else
\fi
