\exer{}
\setcounter{numques}{0}

\question{} Donnons ces invariant et variant pour l'algorithme de vérification de la conjecture de syracuse (nous ne 
pouvons malheureusement pas le faire pour l'exemple \textbf{\ref{ex-syracuse}}, puisque comme son 
nom l'indique, la conjecture de Syracuse n'a jamais été demontrée). 

Conjecture de Syracuse : on note $f : \mathbb{N}^{*} \to \mathbb{N}^{*}$ l'application vérifiant, pour tout $n$ pair
$f(n)=n/2$ et tout $n$ impair et $f(n)=3n+1$.

On conjecture que pour tout entier $n$, il existe $k$ tel que
$f^{k}(n)=1$.

Voici un algorithme calculant, pour tout $n$ donné, le plus petit
entier $k$ vérifiant $f^{k}(n) = 1$ :

%\begin{lstlisting}
\begin{lstlisting}
def f(n):
    """Fonction de Syracuse.
    Précondition : n est un entier strictement positif"""
    if n % 2 == 0:
        return n // 2
    else:
        return 3 * n + 1
\end{lstlisting}

\begin{lstlisting}        
def syracuse(n):
    """Renvoie le premier entier k tel que  f\^k(n) = 0.
    Précondition : n est un entier strictement positif"""
    x = n
    k = 0
    while x != 1:
        x = f(x)
        k = k + 1
    return k
\end{lstlisting}
%\end{lstlisting}

