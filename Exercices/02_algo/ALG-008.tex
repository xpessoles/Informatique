\exer{}
\setcounter{numques}{0}

On s'intéresse au problème du codage d'une suite de bits (représentée sous la forme d'un tableau de 0 ou 1), de manière à pouvoir réparer une erreur de transmission. 
On fixe un entier naturel non nul \texttt{k}.
Un tableau de bits \texttt{b} sera codé avec un niveau de redondance \texttt{k} en répétant chaque bit $2k+1$ fois. 
Pour décoder un tableau avec un niveau de redondance \texttt{k}, on le découpe en blocs de $2k+1$ bits. Dans chaque bloc, on effectue un \og vote \fg\ et l'on considère la valeur majoritaire. 

Exemple : Avec $k = 2$ (et donc un niveau de redondance de $2$), le tableau 
\begin{equation*}
  b = [0,1,0]
\end{equation*}
sera codé en 
\begin{equation*}
  c = [\underbrace{0,0,0,0,0}_{5~\textrm{bits}},\underbrace{1,1,1,1,1}_{5~\textrm{bits}},\underbrace{0,0,0,0,0}_{5~\textrm{bits}}].
\end{equation*}
Imaginons qu'après transmission, le tableau reçu soit 
\begin{equation*}
  c' = [\underbrace{0,0,0,1,0}_{1~\textrm{erreur}},\underbrace{0,1,1,0,1}_{2~\textrm{erreurs}},\underbrace{0,1,0,1,1}_{3~\textrm{erreurs}}].
\end{equation*}
On le décode alors en 
\begin{equation*}
  b' = [0,1,1].
\end{equation*}

\question\ Écrire une fonction \texttt{code(b,k)} renvoyant le tableau codant un tableau \texttt{b}, avec un niveau de redondance~\texttt{k}. 

\question\ Écrire une fonction \texttt{decode(c,k)} renvoyant le tableau décodant un tableau \texttt{c}, avec un niveau de redondance~\texttt{k}. 



