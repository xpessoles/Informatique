\exer{}
\setcounter{numques}{0}

Commencez par recopier le code suivant dans votre script. 

\begin{lstlisting}
def binaire(k,n):
    """Renvoie le tableau de n bits écrivant k en binaire
    Précondition : 0 <= k <= 2**n -1 """
    L = [0]*n
    p = k
    for i in range(n):
        L[n-1-i] = p % 2
        p = p // 2
    return L
\end{lstlisting}



Soit \texttt{k} un entier écrit en binaire avec \texttt{n} chiffres : $\texttt{k} = \underline{a_{\texttt{n}-1}\ldots a_{1}a_{0}}_{2}$, c'est-à-dire que 
\begin{equation*}
  \texttt{k} = \sum_{j=0}^{\texttt{n}-1} a_j 2^j.
\end{equation*}



\question{} Écrire un invariant portant sur \texttt{L} et \texttt{p} dans la boucle \texttt{for} de la fonction \texttt{binaire(k,n)} et justifier que cette fonction renvoie bien le résultat demandé.