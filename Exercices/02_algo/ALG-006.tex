\exer{}
\setcounter{numques}{0}

On appelle \emph{nombre parfait} tout entier naturel non nul qui est égal à la somme de ses diviseurs stricts, c'est-à-dire de 
ses diviseurs autres que lui-même.

Par exemple, 26 n'est pas parfait, car ses diviseurs stricts sont 1, 2 et 13, et $1+2+13=16\neq 28$.
Mais 28 est parfait, car ses diviseurs stricts sont 1, 2, 4, 7 et 14, et $1+2+4+7+14=28$.

\question\ Écrire une fonction Python \texttt{parfait(n)} prenant en entrée un entier naturel non nul \texttt{n}, et renvoyant un booléen donnant la valeur de 
vérité de l'assertion \og \texttt{n} est parfait \fg. 

\question\ Écrire les éventuels variants et invariants permettant de montrer que cette fonction renvoie le bon résultat.

%\question\ Montrer que les variants et/ou invariants donnés à la question précédente sont bien des variants/invariants 
% de boucle et justifier que la fonction écrite donne le bon résultat. 