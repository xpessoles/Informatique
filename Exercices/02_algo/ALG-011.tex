\exer{}
\setcounter{numques}{0}

Soit $n \geq 2$ un entier. Un diviseur strict de $n$ est un entier $1 \leq d \leq n-1$ qui divise $n$ (c'est-à-dire que le reste de la division euclidienne de $n$ par $d$ est nul). 

Deux entiers $n_1$ et $n_2$ sont dits \emph{amicaux} si la somme des diviseurs stricts de $n_1$ vaut $n_2$ et si la somme des diviseurs stricts de $n_2$ vaut $n_1$. 

\bigskip{}

\question{} Écrire une fonction \texttt{amicaux(n,m}) qui prend en argument deux entiers naturels \texttt{n} et \texttt{m} et renvoie la valeur de vérité de \og \texttt{n} et \texttt{m} sont amicaux \fg{}.

On pourra écrire une fonction auxiliaire, au besoin. 