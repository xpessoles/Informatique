%\exer{Résolution d'une équation par dichotomie}

\ifprof
\else
Soit l'équation \emph{E} : \quad $x\ln(x)-1=0$ .

On admet que \emph{E} admet une unique solution $\alpha$ sur $\mathbb R$ et que
$\alpha\in[1;\text{e}]$. On prendra e$\approx2.8$. On peut aussi noter qu'en python \texttt{e} est un attribut de la bibliothèque \texttt{math}. On y a donc accès par \texttt{math.e}.
\fi

\question{Ecrire la fonction \texttt{f(x)} calculant $x\ln(x)-1$. On considèrera qu'on dispose de la fonction \texttt{ln}.}
\ifprof
\begin{corrige}~\\ \vspace{-.5cm}
\begin{lstlisting}
import math as m
def f(x):
    return x*m.log(x)-1
\end{lstlisting}
\end{corrige}
\else
\fi

\question{Dessiner un schéma illustrant la rechercher du 0 d'une fonction par une méthode dichotomique.}
\ifprof
\begin{corrige}~\\ \vspace{-.5cm}
\begin{lstlisting}
def dicho(f,a,b,epsilon):
    m  = (a+b)/2
    while (b-a)/2 > epsilon :
        if f(a)*f(m) > 0 :
            a=m
        else :
            b=m
        m = (a+b)/2
    return m 
\end{lstlisting}
\end{corrige}
\else
\fi

\question{Ecrire la fonction \texttt{dichotomie(f,a,b,epsilon)} renvoyant une valeur approchée à \texttt{epsilon} près de cette solution par dichotomie.}

\ifprof
\begin{corrige}~\\ \vspace{-.5cm}
\begin{lstlisting}
def dicho(f,a,b,epsilon):
    m  = (a+b)/2
    while (b-a)/2 > epsilon :
        if f(a)*f(m) > 0 :
            a=m
        else :
            b=m
        m = (a+b)/2
    return m 
\end{lstlisting}
\end{corrige}
\else
\fi
