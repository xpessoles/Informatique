\question{Écrire une fonction \texttt{chercheMax(L)} qui prend comme argument une liste \texttt{L} d'entiers \texttt{int} et qui renvoie le plus grand élément de la liste.}

%\begin{xxpy}~\\
%\vspace{-1cm}
\begin{lstlisting}
def chercheMax(L):
    '''prend comme argument une liste L d'entiers int et 
    qui renvoie le plus grand élément de la liste'''
    max=L[0]
    for i in range(1,len(L)):
        if L[i]>max:
            max=L[i]
    return max
\end{lstlisting}
%\end{xxpy}


\question{Écrire une fonction \texttt{chercheMaxIndice(L)} qui prend comme argument une liste \texttt{L} d'entiers \texttt{int} et qui renvoie l'indice du plus grand élément de la liste et s'il y en plusieurs renvoie le plus petit.}

%\begin{xxpy}~\\
%\vspace{-1cm}
\begin{lstlisting}
def chercheMaxIndice(L):
    '''prend comme argument une liste L d'entiers int et
     qui renvoie l'indice du plus grand élément de la liste'''
    max=L[0]
    indmax=0
    for i in range(1,len(L)):
        if L[i]>max:
            max=L[i]
            indmax=i
    return indmax
\end{lstlisting}
%\end{xxpy}
