\question{Écrire une fonction \texttt{ajouteUnFor(L)} qui prend comme argument une liste \texttt{L} de flottants et qui ajoute 1 à chaque élément de la liste. On utilisera une boucle \textbf{for}.}

%\begin{xxpy}~\\
%\vspace{-1cm}
\begin{lstlisting}
def ajouteUnFor(L):
    '''prend comme argument une liste L de flottants et ajoute 1
     à chaque élément de la liste avec un boucle for'''
    for i in range(len(L)):
        L[i]+=1
\end{lstlisting}
%\end{xxpy}

\question{Écrire une fonction \texttt{ajouteUnWhile(L)} qui prend comme argument une liste \texttt{L} de flottants et qui ajoute 1 à chaque élément de la liste. On utilisera une boucle \textbf{while}.}

%\begin{xxpy}~\\
%\vspace{-1cm}
\begin{lstlisting}
def ajouteUnWhile(L):
    '''prend comme argument une liste L de flottants et ajoute 1
     à chaque élément de la liste avec une boucle while'''
    i=0
    while i<len(L):
        L[i]+=1
        i+=1
 \end{lstlisting}
%\end{xxpy}

\question{Expliquer pourquoi il n'est pas indispensable que la fonction renvoie la liste modifiée.}

Un liste est un objet mutable ça veut dire qu'elle est modifiable.