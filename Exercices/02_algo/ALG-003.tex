\exer{}
\setcounter{numques}{0}

Rien de tel que de faire du camping pour profiter de la nature. Cependant sur Algoréa, les 
moustiques sont particulièrement pénibles et il faut faire attention à l'endroit où l'on s'installe, 
si l'on ne veut pas être sans cesse piqué.\\

Vous disposez d'une carte sur laquelle est indiquée, pour chaque parcelle de terrain, si le nombre 
de moustiques est supportable ou non. Votre objectif est de trouver le plus grand camping carré 
évitant les zones à moustiques qu'il est possible de construire.\\

\noindent ENTRÉE :  un tableau \texttt{t} de $n$ éléments correspondant à des lignes, chacune de 
ces lignes contenant $p$ élements, qui ne sont que des 0 et des 1. 0 signifie qu'il n'y a pas de 
moustiques et 1 qu'il y a des moustiques.\\
SORTIE : un entier : la taille maximale du côté d'un carré ne comportant que des 0 et dont les 
bords sont parallèles aux axes.\\

\noindent EXEMPLE 1 :\\
entrée : \texttt{t}= $\begin{matrix}[\  [1,\ &0,\ &0,\ &1,\ &0,\ &0,\ &1],\\
                          [0,\ &0,\ &0,\ &0,\ &0,\ &0,\ &0],\\
                          [1,\ &0,\ &0,\ &0,\ &0,\ &0,\ &0],\\
                          [0,\ &0,\ &0,\ &0,\ &0,\ &0,\ &0],\\
                          [0,\ &1,\ &0,\ &0,\ &0,\ &0,\ &1],\\
                          [1,\ &0,\ &0,\ &0,\ &1,\ &0,\ &1]\ ]
                         \end{matrix}$\\
sortie : 4.\\

\noindent EXEMPLE 2 :\\
entrée : \texttt{t}= $\begin{matrix}[\  [0,\ &0,\ &0,\ &1,\ &1,\ &1,\ &1],\\
                          [0,\ &0,\ &0,\ &1,\ &1,\ &1,\ &1],\\
                          [0,\ &0,\ &0,\ &1,\ &1,\ &1,\ &1],\\
                          [1,\ &1,\ &1,\ &1,\ &1,\ &0,\ &0],\\
                          [1,\ &1,\ &1,\ &1,\ &1,\ &0,\ &0]\ ]
                         \end{matrix}$\\
sortie : 3.