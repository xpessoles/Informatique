\exer{}
\setcounter{numques}{0}

Un marchand de légumes très maniaque souhaite ranger ses petits pois en les regroupant en boîtes de 
telle sorte que chaque boîte contienne un nombre factoriel de petits pois. On rappelle qu'un nombre 
est factoriel s'il est de la forme $1$, $1 \times 2$, $1 \times 2 \times 3$, $1 \times 2 \times 3 
\times 4\cdots$ et qu'on les note sous la forme suivante :\\
$$n!=1\times 2\times 3\times \cdots\times (n-1)\times n$$

Il souhaite également utiliser le plus petit nombre de boîtes possible.\\
Ainsi, s'il a 17 petits pois, il utilisera :\\
   \hspace*{2cm} 2 boîtes de 3! = 6 petits pois\\
   \hspace*{2cm} 2 boîtes de 2! = 2 petits pois\\
   \hspace*{2cm} 1 boîte de 1! = 1 petits pois.\\
ce qui donne bien $2 \times 3! + 2 \times 2! + 1 \times 1! = 12 + 4 + 1 = 17$.\\

D'une manière générale, s'il a \texttt{nbp} petits pois, il doit trouver une suite 
\texttt{a\_1,a\_2$\cdots$,a\_p} d'entiers positifs ou nuls avec \texttt{a\_p}$>0$ et telle que :
\texttt{nbp=a\_1 x  1!+a\_2 x  2!+$\cdots$+a\_p x  p!} avec 
\texttt{a\_1+$\cdots$+a\_p} minimal. \\
Remarque mathématique : si à chaque étape on cherche le plus grand entier \texttt{k} possible tel 
que \texttt{k!} soit inférieur au nombre de petits pois restant, on est sûrs d'obtenir la 
décomposition optimale : en termes informatiques, on dit que l'algorithme \textit{glouton} est 
optimal.\\

\noindent ENTRÉE : un entier, \texttt{nbp}, le nombre total de petits poids.\\
\noindent SORTIE : un couple constitué de l'entier \texttt{p} et du tableau 
\texttt{[ a\_1, a\_2, $\cdots$, a\_p ]}.\\

\noindent EXEMPLE :\\
entrée : \texttt{17}\\
sortie : \texttt{(3, [ 1, 2, 2 ] )}.