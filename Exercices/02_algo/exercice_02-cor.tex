%\exer{Une suite ...}
\ifprof 
\else
On considère la suite $u_n$ définie pour $n \in \mathbb{N}$ par $u_n = 15\times 0,9^n +3$.
\fi
\question{Ecrire la fonction \texttt{terme\_n(n:int)-> float} permettant de calculer le nième terme de la suite~$u_n$.}
\ifprof
\begin{corrige}~\\ \vspace{-.5cm}
\begin{lstlisting}
def terme_n(n) :
    return 15*0.9**n+3
\end{lstlisting}
\end{corrige}
\else
\fi

\question{Ecrire la fonction \texttt{liste\_for(n:int)-> list} renvoyant la liste des $n$ premiers termes de la suite~$u_n$. Cette fonction devra utiliser la fonction \texttt{terme\_n} et une boucle \texttt{for}.}
\ifprof
\begin{corrige}~\\ \vspace{-.5cm}
\begin{lstlisting}
def liste_for(n):
    L = []
    for i in range(n):
        L.append(terme_n(i))
    return L
\end{lstlisting}
\end{corrige}
\else
\fi


\question{Ecrire la fonction \texttt{liste\_c(n:int)-> list} renvoyant la liste des $n$ premiers termes de la suite $u_n$. Cette fonction devra utiliser la fonction \texttt{terme\_n}. La liste renvoyée sera une liste en comprehésion.}
\ifprof
\begin{corrige}~\\ \vspace{-.5cm}
\begin{lstlisting}
def liste_c(n):
    return [terme_n(i) for i in range(n)]
\end{lstlisting}
\end{corrige}
\else
\fi

