\exer{}
\setcounter{numques}{0}

On ne s'intéresse pas ici à la validité d'un nombre écrit en chiffre romains, mais à sa valeur. On rappelle
quelques principes de base. Les sept caractères de la numération romaine sont :\\

\begin{tabular}{|c|c|c|c|c|c|c|}
 \hline
 I&V&X&L&C&D&M\\ \hline
 1&5&10&50&100&500&1000\\ \hline
\end{tabular}\\

Certaines lettres sont dites d'\emph{unité}. Ainsi on dit que I est une unité pour V et X, X est une 
unité pour L et C, C est une unité pour D et M.\\

Pour trouver la valeur d'un nombre écrit en chiffres romains, on s'appuie sur les règles suivantes :
\begin{itemize}
\item toute lettre placée à la droite d'une lettre dont valeur est supérieure ou égale à la sienne s'ajoute
à celle-ci ;
\item toute lettre d'unité placée immédiatement à la gauche d'une lettre plus forte qu'elle indique que
le nombre qui lui correspond doit être retranché au nombre qui suit ;
\item les valeurs sont groupées en ordre décroissant, sauf pour les valeurs à retrancher selon la règle
précédente ;
1
\item la même lettre ne peut pas être employée quatre fois consécutivement sauf M.
\end{itemize}

Par exemple,  DXXXVI = 536, CIX = 109 et MCMXL = 1940.

\begin{enumerate}
\item Écrire une fonction \texttt{valeur (caractere)} qui retourne la valeur décimale d'un caractère romain. Cette
fonction doit renvoyer 0 si le caractère n'est pas l'un des 7 chiffres romains.
\item  Écrire la fonction principale \texttt{conversion (romain)} qui permet de convertir un nombre romain en nombre 
décimal. Cette fonction doit prendre en argument une chaîne de caractères \texttt{romain}. Si cette chaîne est 
écrite en majuscule et correspond à un nombre romain correctement écrit, la fonction doit renvoyer le nombre décimal 
égal au nombre romain passé en argument. Sinon, la fonction doit renvoyer -1.
% Voici un algorithme pour vous aider :
% \begin{itemize}
% \item initialiser une variable \texttt{decimal} à 0 ;
% \item  parcourir \texttt{romain} en lisant à chaque fois un caractère et le caractère suivant, notés \texttt{car} et 
% \texttt{carsuiv} ;
% \item  transformer \texttt{car} et \texttt{carsuiv} en leur valeur \texttt{val} et \texttt{valsuiv} selon la fonction 
% précédente ;
% \item  si \texttt{val < valsuiv} on ajoute -val\right à \texttt{decimal}. Sinon, on ajoute \texttt{val} à 
% \texttt{decimal} ;
% \item  renvoyer \texttt{decimal}.               
% \end{itemize}
\end{enumerate}
