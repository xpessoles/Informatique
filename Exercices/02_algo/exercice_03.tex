%\exer{Résolution d'une équation du second degré}

On considère l'équation $ax^2 + bx+c = 0$. 

\question{Ecrire la fonction \texttt{discriminant(a:float,b:float,c:float)-> float} permettant de calculer le discriminant de l'équation.}
\ifprof
\begin{corrige}~\\ \vspace{-.5cm}
\begin{lstlisting}
def discriminant(a,b,c):
    delta = b**2-4*a*c
    return delta
\end{lstlisting}
\end{corrige}
\else
\fi


\question{Ecrire la fonction \texttt{solve(a:float,b:float,c:float)->list} permettant de calculer la ou les solutions de l'équation. La fonction renverra la liste des (ou de la solution) dans $\mathbb{R}$. Elle renverra une liste vide s'il n'y a pas de solutions dans $\mathbb{R}$.}
\ifprof
\begin{corrige}~\\ \vspace{-.5cm}
\begin{lstlisting}
def solve(a,b,c):
    delta = discriminant(a, b, c)
    if delta >= 0 : 
        delta = delta**0.5
        p1,p2 = (-b-delta)/2/a,(-b+delta)/2/a
        return [p1,p2]
    else :
        return []
\end{lstlisting}
\end{corrige}
\else
\fi


