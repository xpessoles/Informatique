\exer{}
\setcounter{numques}{0}

\question{}
%On peut utiliser une structure de liste puisque cette structure en sera pas amenée à évoluer. On peut choisir de donner les valeurs de billets en centime afin de n'utiliser que des entiers et éviter des erreurs d'arrondis.
\begin{lstlisting}
valeurs=[2000, 1000, 500, 200, 100, 50, 20, 10, 5, 2, 1]
\end{lstlisting}

\question{}

\begin{lstlisting}
def rendre_monnaie(cout,somme_client,valeurs):
    """
    Retourne une liste nombre_billets donnant le nombre de billets ou pièce à rendre selon le type de billet ou pièce
    Keywords arguments :
    cout : somme à payer
    somme_client : argent donné par le client
    """
    nombre_billets=[0]*len(valeurs)
    montant_a_rendre = somme_client - cout
    k=0#Indice du billet dans la liste valeur
    while montant_a_rendre>0:
        nombre_billets[k]=montant_a_rendre//valeurs[k]
        montant_a_rendre-=nombre_billets[k]*valeurs[k]
        k+=1
    return nombre_billets
\end{lstlisting}

\question{}

\begin{lstlisting}
def afficher_rendu_monnaie(cout,somme_client,valeurs):
    cout=100*cout
    somme_client=100*somme_client
    nombre_billets=rendre_monnaie(cout,somme_client,valeurs)
    for k in range(len(nombre_billets)):
        if valeurs[k]>200:
            print(str(int(nombre_billets[k]))+' : billet de '+str(int(valeurs[k]/100))+' euros')
        elif valeurs[k]>=100:
            print(str(int(nombre_billets[k]))+' : pièce de '+str(int(valeurs[k]/100))+' euros')
        else:
            print(str(int(nombre_billets[k]))+' : pièce de '+str(int(valeurs[k]))+' centimes')
\end{lstlisting}