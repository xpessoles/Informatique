On donne la \texttt{fonctionMystere(n)} définie comme suit.

\vspace{-1cm}
\begin{lstlisting}
def fonctionMystere(n) :
    if n==0 or n==1:
         return 1
    else :
        res = 1
    for i in range (2,n+1) :
        res = res * i
    return res
\end{lstlisting}



\question{Si $n=5$ quelles sont les valeurs que va prendre la variable \texttt{i} ? }

\question{Si $n=4$ donner les valeurs successives que vont prendre les variables \texttt{i} et \texttt{res}  lorsqu'on exécute l'algorithme. }

\question{Quel est le nom mathématique usuel donné à la fonction \texttt{fonctionMystere} ?}