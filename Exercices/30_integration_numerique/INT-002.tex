Le but est d'obtenir un encadrement de
\quad $\displaystyle I=4\int_0^1\frac{\text{d}x}{1+x^2}$

\begin{minipage}[c]{.5\linewidth}
\question{}
Compléter cet algorithme et le coder en python afin d'obtenir une valeur approchée de $I$ par la méthode des rectangles à gauche en utilisant les champs suivants : \fbox{x+h}
\fbox{(b-a)/n}
\fbox{h}
\fbox{somme+f(x)}
\fbox{f(a)}.

\question{} A partir de la question précédente, écrire une fonction que l'on appellera \pyv{rect_gauche}, qui prendra en argument une fonction \pyv{f}, les variables \pyv{a} et \pyv{b} définissant le domaine d'intégration et le paramètre \pyv{n} définissant le nombre de subdivisions. Cette fonction renverra la valeur approché de \pyv{I}.
\end{minipage} \hfill
\begin{minipage}[r]{.35\linewidth}
\begin{lstlisting}
a=0
b=1
n=100
h=
x=a
somme=
for k in range(1,n) :
    x=
    somme=
print(somme*
\end{lstlisting}
\end{minipage}

\question{}
Modifier cet algorithme pour que la méthode soit celle des rectangles à droite. On appellera la fonction associée \pyv{rect_droit}, elle prendra les mêmes arguments et en renverra la valeur approché de \pyv{I}.

\question{} Définir la fonction \pyv{f(x)} permettant d'utiliser les deux fonctions précédentes avec la fonction à intégrer. 


\question{}
Tester ces fonctions en augmentant le nombre de subdivisions. et en traçant les résultats obtenus sur l'estimation de $I$ pour différentes valeurs de $n$. Vous sauvegarderez le graphe obtenu sous le nom \textbf{"tp10$\_$q05$\_$vos$\_$noms.png"} et vous l'enverrez à votre professeur (On veillera à choisir des valeurs de $n$ et une échelle de représentation graphique pertinentes).



\question{}
Justifier que la méthode des rectangles à droite donne un minorant de $I$ et que la méthode des rectangles à gauche donne un majorant. 

Pour tout $n\in\mathbb{N}$ soit l'intégrale : 
$$I_n=\int_0^1\frac{x}{1+x^n} \text{d}x$$

\question{} Écrire une fonction que l'on appellera \pyv{calcul_in} d'argument $n$ qui renvoie une valeur approchée de $I_n$ par une la méthodes des rectangles à gauche (avec une subdivision de 100 intervalles).

\question{} A l'aide d'un graphe choisi judicieusement (que vous sauvegarderez sous le nom \textbf{"tp10$\_$q08$\_$vos$\_$noms.png"} et vous l'enverrez à votre professeur), afficher quelques valeurs de cette suite afin de conjecturer sa monotonie et son comportement asymptotique.