\exer{[INT-003]}
\setcounter{numques}{0}~\\

\question{} 
\begin{lstlisting}

def trapeze(f,a,b,nb):
    res = 0
    pas = abs(b-a)/nb
    for i in range(nb):
        x0 = a+pas*i
        x1 = a+pas*(i+1)
        res = res + (x1-x0)*0.5*(f(x0)+f(x1))
    return res

\end{lstlisting}

\question{} 

\begin{lstlisting}

from math import pi,exp

def erf(x,nb):
    """
    x : borne supérieure de l'intégrale
    nb : nombre d'échantillons
    """
    if x>=0:
        return trapeze(fexp,0,x,nb)
    else :
        return None

def fexp(u):
    return (2/math.sqrt(math.pi))*exp(-u*u)

\end{lstlisting}

\question{}

\begin{lstlisting}
# Paramètres physiques du problème

D = 2.8E-7         # Diffusivité thermique du sol terrestre
To = 278           # Température de la surface du sol terrestre pour t < 0
T1 = 258           # Température de la surface du sol terrestre pour t >= 0
Tf = 273.15      # Température de fusion de l'eau sous la pression de 1,013 bar

def Temperature (z, t) :
    u = z / (2 * sqrt (D * t))
    T = T1 + (To - T1) * erf (u,500)
    return T
\end{lstlisting}

\question{} 

\begin{lstlisting}

ListeErreur = list ()

for i in range (0, 41, 1) :
    x = 0 + i * 0.05
    ListeErreur.append (erf (x,500))


\end{lstlisting}

\question{}

\begin{lstlisting}

z = 0
t = 864000         # 864 000 = 10 * 24 * 3600 secondes = 10 jours
T = Temperature (z, t)
while T < Tf :
    z = z + 0.01     # 0.01 m = 1 cm : précision de la recherche
    T = Temperature (z, t)
    
print ("La canalisation doit être enterrée à une profondeur minimale de", int (z * 100), "cm")

\end{lstlisting}