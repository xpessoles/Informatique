\exer{[INT-periodependule]}
\setcounter{numques}{0}~\\

On considère un pendule de masse $m=1+0,01\cdot \alpha$, de longueur $L=\alpha$ que l'on abandonne, sans vitesse initiale, à un angle de $\theta_0=\dfrac{\pi}{4\cdot \alpha}$ avec $g=\SI{9,81}{m.s^{-2}}$.

En appliquant la conservation de l'énergie, on trouve l'équation suivante :


\begin{minipage}{0.4\textwidth}

\begin{align*}
\dfrac{1}{2}mL^2\dot{\theta}(t)^2+mgl\left(1-\cos\theta(t)\right)=mgL\left(1-\cos\theta_0\right)
\end{align*}

On en déduit : 
$
\dfrac{\dd\theta(t)}{\dd t}=\sqrt{\dfrac{2g}{L}\left(\cos\theta(t)-\cos\theta_0\right)}
$.

Ce qui donne : 
$\dd t=\dfrac{\dd \theta(t)}{\sqrt{\dfrac{2g}{L}\left(\cos\theta(t)-\cos\theta_0\right)}}$.
\end{minipage}
\begin{minipage}{0.6\textwidth}
\begin{center}
\begin{tikzpicture}[scale=0.9] 
\draw (0,0)--++(-2,+5); 
\draw[dashed] ({5.4*sin(35)-2},{5-5.4*cos(35)})--(-2,+5); 
\node at (-0.7,2.4) {L}; 
\node at (-1.5,2.5) {$\theta(t)$}; 
\node at (-0.6,0.8) {$\theta_0$};
\draw [rounded corners=4pt,color=white,ball color=gray,smooth] (0,0) circle (0.2); 
\draw [rounded corners=4pt,color=white,ball color=gray,smooth] ({5.4*sin(35)-2},{5-5.4*cos(35)}) circle (0.2); 
\draw [dashed] (-2,-0.4)--++(0,5.4); 
\draw [dotted] (-2,-0.4) arc (-90:-55:5.4); 
\draw [dotted] (-2,-0.4) arc (-90:-110:5.4); 
\draw [thick] (-2,3) arc (-90:{atan(2/5)-90}:2); 
\draw [thick] (-2,1) arc (-90:{35-90}:4); 
\fill [pattern=north east lines,rotate=0] (-2.5,5) rectangle (-1.5,5.3); %bloc qui tient le pendule 
\draw[thick] (-2.5,5) --++ (1,0); %bloc qui tient le pendule 
\end{tikzpicture}
\end{center}
\end{minipage}




La période est donc : $
T=\displaystyle{\int_0^T\dd t=4\times \int_0^{\theta_0}}\dfrac{\dd \theta(t)}{\sqrt{\dfrac{2g}{L}\left(\cos\theta(t)-\cos\theta_0\right)}}$.

On rappelle que la période avec l'approximation des petites oscillations est donnée par : 
$
T_0=\sqrt{\dfrac{L}{g}}\times 2\pi
$.

\begin{rem}
$T=\displaystyle{\int_0^{\theta_0}f(\theta)\dd \theta}$ est une intégrale généralisée car $f(\theta)$ tend vers l'infini. Donc certaines méthodes numériques d'intégration donnent des résultats infinis.

\textbf{Pour les méthodes posant problèmes, il faut alors remplacer la borne supérieur $\theta_0$ par $\theta_0(1-\varepsilon)$ avec $\varepsilon=10^{-3}$.}
\end{rem}


\question{} Calculer $T_0$.

Pour les méthodes d'intégration numérique, on renverra les résultats avec 100 subdivisions.

\question{} Calculer $T$ (noté $T_{g}$) par la méthode des rectangles à gauche.

\question{} Calculer $T$ (noté $T_{d}$) par la méthode des rectangles à droite.

\question{} Calculer $T$ (noté $T_{t}$) par la méthode des trapèzes.

On note $\varepsilon_g=\vert T_g-T_0\vert$,  $\varepsilon_d=\vert T_d-T_0\vert$,  $\varepsilon_t=\vert T_t-T_0\vert$, les erreurs entre les différentes approximation et l'estimation de $T_0$.

\question{} Renvoyer $\varepsilon_g$,  $\varepsilon_d$,  $\varepsilon_t$.


