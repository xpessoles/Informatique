
	
	\question{Quel est le nom de la variable locale dont le contenu est retourné par la fonction ?}
	
La variable locale est \texttt{res}.
	
	\question{Faire tourner \og{}à la main\fg{} la fonction pour $x=2$ et $n=10$ en complétant le tableau suivant puis encadrer le nombre d'opérations dans exporapide en fonction de $ln(n)/ln(2)$.}
	
	\begin{center}
		\begin{tabular}{|l|p{1cm}|p{1cm}|p{1cm}|}
			\hline & \texttt{p} & \texttt{res} & \texttt{y}\\
			\hline &&&\\
			sortie du 1{\textrm{er}} tour de boucle & 5 & 1 & 4\\
sortie du 2{\textrm{er}} tour de boucle & 2 & 4 & 16\\
sortie du 3{\textrm{er}} tour de boucle & 1 & 4 & 256\\
sortie du 4{\textrm{er}} tour de boucle & 0 & 1024 & 65536\\
		\end{tabular}
		
	\end{center}
	
	\begin{lstlisting}
	# Le nombre d'opérations effectuées est exactement n (1 produit à chaque tour)
\end{lstlisting}
	

	\question{Ecrire une fonction \texttt{Pnaif(x,P)} d'arguments un réel \texttt{x} et \texttt{P} la liste des coefficients du polynôme, qui renvoie $P(x)$ à l'aide de la fonction \texttt{exponaif}. Compter le nombre d'opérations.}

\begin{lstlisting}

def Pnaif(x,n):
    S=0
    for i in range(n):
        S=S+i*exponaif(x,i)
    return(S)

# n+n(n+1)/2 ~ n**2/2 opérations. Quadratique

\end{lstlisting}

	\question{ Faire de même pour une fonction 
	\texttt{Prapide(x,n)} qui renvoie $P(x)$ à l'aide de la fonction \texttt{exporapide}. On admettra que la complexité est en O(n ln(n)).}
	
\begin{lstlisting}

def Prapide(x,n):
    S=0
    for i in range(n):
        S=S+i*exporapide(x,i)
    return(S)

# O(log(i)) pour chaque i*x**i. Il reste la somme des n termes.
# D'où n+somme des log(i)=O(n.ln(n)).

\end{lstlisting}

	\question{\'Ecrire une fonction \texttt{horner(x,L)} de paramètres un réel $x$ et une liste $L$ représentant un polynôme $P$, renvoie la valeur de $P(x)$ par la méthode de Hörner. Compter le nombre d'opérations.}
	
	
	\begin{lstlisting}
	def horner(x,L):
    """L est la liste des coefficients"""
    n=len(L)-1
    S=0
    for i in range(n+1):
        S=S*x+L[n-i]
    return(S)
    
    
# 2n opérations. Linéaire mais plus intéressante que ci-dessus.

\end{lstlisting}



	\question{Définir la liste $N$ des entiers naturels compris entre 0 et 100.}
	
	
\begin{lstlisting}
N=[i for i in range(101)]
\end{lstlisting}

	\question{Grâce à la fonction \texttt{perf$\_$counter} de la bibliothèque \texttt{time},  écrire une fonction \texttt{Temps\_calcul(x)} qui:}
\begin{itemize}
\item définit 3 listes \texttt{Tn}, \texttt{Tr} et \texttt{Th} contenant les temps de calcul de $P(x)$ pour $P=\displaystyle\sum_{k=0}^n k.x^k$ lorsque $n$ décrit $N$ avec respectivement la méthode naïve, la méthode rapide puis la méthode de Hörner.\\
\item trace les trois courbes  \texttt{Tn}, \texttt{Tr} et \texttt{Th} en fonction de $N$ (on prendra $x=2$). Interpréter le résultat (on pourrait démontrer que les temps d'exécution des trois programmes sont de l'ordre de $n**2$ pour la méthode naïve (on parle de complexité quadratique), de l'ordre de $n\ln(n)$ pour l'exporapide, et de l'ordre de $n$ pour la méthode de Hörner (complexité linéaire)). 
\end{itemize}


\begin{lstlisting}
import time as t

def Temps_calcul_P(x):
    # Le polynôme est donné par une liste des coefficients.
    Tn,Tr,Th=[],[],[]
    for n in N:
        L=[k for k in range(n+1)]
        tps=t.perf_counter()
        Pnaif(x,n)
        Tn.append(t.perf_counter()-tps)
        tps=t.perf_counter()
        Sr=0
        Prapide(x,n)
        Tr.append(t.perf_counter()-tps)
        tps=t.perf_counter()
        Phorner(x,L)
        Th.append(t.perf_counter()-tps)
    plt.plot(N,Th,label='méthode horner')
    plt.plot(N,Tr,label='méthode rapide')
    plt.plot(N,Tn,label='méthode naïve')
    plt.legend()
    plt.show()

\end{lstlisting}
	
