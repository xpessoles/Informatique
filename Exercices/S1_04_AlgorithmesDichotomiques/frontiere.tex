%Calcul de frontière par dichotomie
L'objectif de cette partie est d'adapter le principe de recherche dichotomique à un problème
voisin : la recherche d'un plus petit  élément plus grand qu'une borne donnée.

On utilisera la fonction \texttt{Lu} qui à partir de \texttt{alpha} et pour \texttt{n=100} vous génèrera une liste à traiter dans cette partie.


\begin{lstlisting}
def Lu(alpha,n):
    """u_n, u_0 = alpha"""
    x = alpha
    M=[]
    n=100+(-1)**alpha*2*alpha%97
    a=137
    c=187
    m=2**8
    for i in range(n):
        x = (a * x+c) % m
        xi=x-128
        if xi not in M:
            M.append(xi)
    M.sort()
    return M
    
L=Lu(alpha,100)
\end{lstlisting}


\question{Écrire une fonction \texttt{frontiere(L:list,x:int)->int} prenant en argument une liste L (générée par \texttt{Lu(alpha,100)}, supposée triée, et un élément x, et renvoyant
le plus petit indice i tel que x est inférieur ou égal à \texttt{L[i]}, ou \texttt{len(L)} si un tel indice n'existe pas. Renvoyer le résultat de  \texttt{frontiere(L=Lu(alpha,100),alpha)}. Il est conseillé de raisonner en itératif ici.}

Par exemple :
\begin{itemize}
\item \texttt{frontiere([1,3,4,6],4)} doit renvoyer 2 ;
\item \texttt{frontiere([1,3,4,6],7)} doit renvoyer 4.
\end{itemize}

La fonction devra avoir une structure dichotomique, de façon à être en coût logarithmique (en la longueur de L).

\question{Donner le nombre d'itérations pour la fonction \texttt{frontiere(Lu(alpha,100),alpha)}.}

%
%\question{Concevoir un jeu de test permettant de vérifier tous les cas limites pour votre fonction.}