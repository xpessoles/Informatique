

\question{Si l'on souhaite que $g_n$ et $d_n$ soient des solutions approchées de $\ell$ à une précision $\varepsilon$, quelle est la condition d'arrêt de l'algorithme ? Préciser alors la valeur approchée de $\ell$ qui sera renvoyée par la fonction.}


\begin{lstlisting}
# Pour une valeur à epsilon près, on s_arrete lorsque 0<d-g<2*epsilon et on renvoie (g+d)/2
\end{lstlisting}
	
\question{\'Ecrire une fonction \texttt{recherche\_zero(f,a,b,epsilon)} qui renvoie une valeur approchée du zéro de \texttt{f} sur \texttt{[a,b]} a epsilon près.}


\begin{lstlisting}
def recherche_zero(f,a,b,epsilon):
    g,d=a,b
    while d-g>2*epsilon:
        m=(g+d)/2
        if f(m)*f(g)<=0:
            d=m
        else:
            g=m
    return((g+d)/2)

\end{lstlisting}
	
\question{Tester la fonction avec $f:x\mapsto x^2-2$ sur $[0,2]$ et $\varepsilon=0,001$.}


\begin{lstlisting}
def f(x):
    return(x**2-2)
\end{lstlisting}
	
\question{Avec une erreur de $\varepsilon=\Frac 1{2^p}$, combien y a-t-il de comparaisons au final en fonction de $p$ ?}



\begin{lstlisting}
# Avec epsilon = 1/2**p, il faut compter combien il y a de tours de boucles. En sortie du kieme tour de boucle, d-g vaut (b-a)/2**k. Il y a donc k tours de boucles avec (b-a)/2**k<=1/2**(p-1) soit k>=p-1+log_2(b-a) soit une complexité logarithmique encore.
\end{lstlisting}