



\question{Écrire une fonction \texttt{frontiere(L:list,x:int)->int} prenant en argument une liste L (générée par \texttt{L=Lu(alpha,100)}, supposée triée, et un élément x, et renvoyant
le plus petit indice i tel que x est inférieur ou égal à \texttt{L[i]}, ou \texttt{len(L)} si un tel indice n'existe pas. Renvoyer le résultat de  \texttt{frontiere(L=Lu(alpha,100),alpha)}. Il est conseillé de raisonner en itératif ici.}

\begin{lstlisting}
def frontiere(Lf,x):
    a=0
    b=len(Lf)-1
    i=0
    if x>Lf[-1]:
        return b+1,i
    while b>=a and i<10:
        m=(b+a)//2
        #print(a,b,m,Lf[m])
        if Lf[m]==x:
            return m,i
        elif Lf[m]>x:
            b=m-1
        else:
            a=m+1
        i+=1
    return m
    
    
print(frontiere(Lu(alpha,100),alpha))
\end{lstlisting}


\question{Donner le nombre d'itérations pour la fonction \texttt{frontiere(L=Lu(alpha,100),alpha)}.}


\begin{lstlisting}
def frontiere(Lf,x):
    a=0
    b=len(Lf)-1
    i=0
    if x>Lf[-1]:
        return b+1,i
    while b>=a and i<10:
        m=(b+a)//2
        #print(a,b,m,Lf[m])
        if Lf[m]==x:
            return m,i
        elif Lf[m]>x:
            b=m-1
        else:
            a=m+1
        i+=1
    return m,i
    
it=frontiere(Lu(alpha,100),alpha)[1]
\end{lstlisting}


%
%\question{Concevoir un jeu de test permettant de vérifier tous les cas limites pour votre fonction.}