\subsection*{Recherche d'un zéro d'une fonction}
\ifprof
\else
Soit une fonction $f:[a,b]\to\mathbb{R}$ ($a<b$) vérifiant : 
$f$ continue sur $[a,b]$ et $f(a)\cdot f(b)\<0$ c'est-à-dire que $f(a)$ et $f(b)$ de signes opposés.

Le théorème des valeurs intermédiaires s'applique et assure que $f$ possède au moins un zéro $\ell$ entre $a$ et $b$. 

%\eject 

\begin{center}
	\begin{tikzpicture}[scale=2]
		\shorthandoff{:};
		\draw[->] (-2.5,0)--(2.5,0);
		\draw[->] (-2.25,-1.5)--(-2.25,1.5);
%		\fenetre
		\draw[domain=-2:2, samples=200, very thick]  plot ({\x},{((\x)^5+3*(\x)-7)/34});
		\draw (-2,0)node{$\cdot$};
		\draw (-1,0)node{$\cdot$};
		\draw (1,0)node{$\cdot$};
		\draw (0,0)node{$\cdot$};
		\draw (2,0)node{$\cdot$};
		\draw (-2 , 0) node[below] {$g_0=a$};
		\draw (2 , 0) node[below] {$d_0=b$};
		\draw (1 , 0.15) node[above] {$g_2=m_1$};
		\draw (2 , 0.15) node[above] {$d_2=b$};
		\draw (2 , -0.2) node[below] {$d_1=b$};
		\draw (0 , 0) node[below] {$g_1=m_0$};
		\draw (1.26 , 0) node[below] {$\ell$};
	\end{tikzpicture}
\end{center}

L'idée consiste à créer une suite d'intervalles $[g_n,d_n]$ tels que pour tout entier naturel $n$, 
$g_n\le\ell\le d_n$ et $0\le g_n-d_n=\dfrac{g_{n-1}-d_{n-1}}{2}$.

%\medskip 

On considère $m_0 = \dfrac{g_0+d_0}{2}$ et on évalue $f(m_0)$ : 

%\bigskip 

\begin{itemize}
	\item si $f(m_0)\times f(d_0)\ge 0$, on va poursuivre la recherche d'un zéro dans l'intervalle $[g_1,d_1]=[g_0,m_0]$;
	\item sinon,  on poursuit la recherche dans l'intervalle $[g_1,d_1]=[m_0,d_0]$;
	\item On recommence alors en considérant $m_1 = \dfrac{g_1+d_1}{2}$ ...
\end{itemize}

\fi
%\bigskip

\question{Si l'on souhaite que $g_n$ et $d_n$ soient des solutions approchées de $\ell$ à une précision $\varepsilon$, quelle est la condition d'arrêt de l'algorithme ? Préciser alors la valeur approchée de $\ell$ qui sera renvoyée par la fonction.}
\ifprof
\begin{corrige}
\begin{lstlisting}
# Pour une valeur à epsilon près, on s_arrete lorsque 0<d-g<2*epsilon et on renvoie (g+d)/2
\end{lstlisting}
\end{corrige}
\else
\fi

\question{\'Ecrire une fonction \texttt{recherche\_zero(f,a,b,epsilon)} qui renvoie une valeur approchée du zéro de \texttt{f} sur \texttt{[a,b]} a epsilon près.}
	\ifprof
\begin{corrige}
\begin{lstlisting}
def recherche_zero(f,a,b,epsilon):
    g,d=a,b
    while d-g>2*epsilon:
        m=(g+d)/2
        if f(m)*f(g)<=0:
            d=m
        else:
            g=m
    return((g+d)/2)

\end{lstlisting}
\end{corrige}
\else
\fi

\question{Tester la fonction avec $f:x\mapsto x^2-2$ sur $[0,2]$ et $\varepsilon=0,001$.}
\ifprof
\begin{corrige}
\begin{lstlisting}
def f(x):
    return(x**2-2)
\end{lstlisting}

\begin{lstlisting}
# Avec epsilon = 1/2**p, il faut compter combien il y a de tours de boucles. En sortie du kieme tour de boucle, d-g vaut (b-a)/2**k. Il y a donc k tours de boucles avec (b-a)/2**k<=1/2**(p-1) soit k>=p-1+log_2(b-a) soit une complexité logarithmique encore.
\end{lstlisting}
\end{corrige}
\else
\fi	

%\question{Avec une erreur de $\varepsilon=\Frac 1{2^p}$, combien y a-t-il de comparaisons au final en fonction de $p$ ?}