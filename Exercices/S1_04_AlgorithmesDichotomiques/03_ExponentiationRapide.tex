\subsection*{Valeur d'un polynôme par plusieurs méthodes}
	
\question{Écrire une fonction \lstinline{exponaif(x,n)} d'arguments un réel $x$ et un entier naturel $n$, qui renvoie la valeur de $x^n$ par la méthode naïve $x^n=x\times x \times ... \times x$ ($n$ termes). Compter le nombre d'opérations dans exponaïf.}
\ifprof
\begin{corrige}
\begin{lstlisting}
def exponaif(x,n):
    p=1
    for i in range(n):
        p=p*x
    return(p)
\end{lstlisting}
\end{corrige}
\else
\fi

%\medskip  Montrer que le nombre d'opérations effectuées a une complexité linéaire.
\ifprof
\else
Une autre méthode, celle de l'exponentiation rapide consiste à remarquer que 
$x^n=\left\{\begin{tabular}{ccc} ${(x^2)}^{n/2}$ &si &$n$ pair \\ ${x\times (x^2)}^{(n-1)/2}$& si& $n$ impair\end{tabular}\right.$

Le code itératif correspondant est le suivant:

\begin{lstlisting}
def expo_rapide(x,n) :
    p,res,y = n,1,x
    while p>0 :
        if p%2==1 :
            res=res*y
        p=p//2
        y=y*y
    return(res)
\end{lstlisting}
\fi
%\bigskip 
%\begin{minipage}{0.5\linewidth}
%def expo\_rapide(x,n):\\
%	\textcolor{white}{aaaa}  p,res,y = n,1,x\\
%	\textcolor{white}{aaaa} while p>0:\\
%	\textcolor{white}{aaaa}\textcolor{white}{aaaa} if p\%2==1:\\
%	\textcolor{white}{aaaa}\textcolor{white}{aaaa}\textcolor{white}{aaaa}res=res*y\\
%	\textcolor{white}{aaaa}\textcolor{white}{aaaa} p=p//2\\
%	\textcolor{white}{aaaa}\textcolor{white}{aaaa} y=y*y\\
%	\textcolor{white}{aaaa} return(res)\end{minipage}
	
	\question{Quel est le nom de la variable locale dont le contenu est retourné par la fonction ?}
\ifprof
\begin{corrige}
La variable locale est \texttt{res}.
\end{corrige}
\else
\fi	
	\question{Faire tourner \og{}à la main\fg{} la fonction pour $x=2$ et $n=10$ en complétant le tableau suivant puis encadrer le nombre d'opérations dans exporapide en fonction de $\ln(n)/\ln(2)$.}
\ifprof
\else	
	\begin{center}
		\begin{tabular}{lp{1cm}p{1cm}p{1cm}}
		\hline
			& \texttt{p} & \texttt{res} & \texttt{y}\\
			\hline
			Sortie du 1{\textrm{er}} tour de boucle && & \\[3mm]
			Sortie du 2\ieme\  tour de boucle && & \\[3mm]
			\ldots && & \\[3mm]
			\ldots&& & \\[3mm]
\hline
		\end{tabular}
	\end{center}
\fi

\ifprof
\begin{corrige}
\begin{center}
\begin{tabular}{|l|p{1cm}|p{1cm}|p{1cm}|}
\hline & \texttt{p} & \texttt{res} & \texttt{y}\\
\hline &&&\\
sortie du 1{\textrm{er}} tour de boucle & 5 & 1 & 4\\
sortie du 2{\textrm{er}} tour de boucle & 2 & 4 & 16\\
sortie du 3{\textrm{er}} tour de boucle & 1 & 4 & 256\\
sortie du 4{\textrm{er}} tour de boucle & 0 & 1024 & 65536\\
\end{tabular}
\end{center}
\begin{lstlisting}
# Le nombre d'opérations effectuées est exactement n (1 produit à chaque tour)
\end{lstlisting}
\end{corrige}
\else
\fi	
	
	%\q Montrer que le nombre d'opérations effectuées a une complexité logarithmique.


On considère un polynôme $P(x)=\sum_{k=0}^n a_k.x^k$ que l'on modélisera en Python par la liste $P=[a_0,a_1,...,a_n]$. Dans la suite, on prendra pour tout $k\in\mathbb{N}$, $a_k=k$.

	\question{Ecrire une fonction \lstinline{Pnaif(x,P)} d'arguments un réel \lstinline{x} et \lstinline{P} la liste des coefficients du polynôme, qui renvoie $P(x)$ à l'aide de la fonction \lstinline{exponaif}. Compter le nombre d'opérations.}
\ifprof
\begin{corrige}
\begin{lstlisting}

def Pnaif(x,n):
    S=0
    for i in range(n):
        S=S+i*exponaif(x,i)
    return(S)

# n+n(n+1)/2 ~ n**2/2 opérations. Quadratique

\end{lstlisting}
\end{corrige}
\else
\fi
	%\medskip Donner un équivalent du nombre d'opérations faites pour ce calcul et vérifier que la complexité est quadratique.

	\question{ Faire de même pour une fonction 
	\lstinline{Prapide(x,n)} qui renvoie $P(x)$ à l'aide de la fonction \lstinline{exporapide}. On admettra que la complexité est en $\mathcal{O}(n \ln(n))$.}
\ifprof
\begin{corrige}
\begin{lstlisting}
def Prapide(x,n):
    S=0
    for i in range(n):
        S=S+i*exporapide(x,i)
    return(S)

# O(log(i)) pour chaque i*x**i. Il reste la somme des n termes.
# D'où n+somme des log(i)=O(n.ln(n)).
\end{lstlisting}
\end{corrige}
\else
\fi
	
	%On peut montrer dans ce cas que la complexité est dominée par $n.\ln(n)$ (on admettra ce résultat).
\ifprof
\else
 Une dernière méthode consiste à utiliser le schéma de Hörner :
 
 $P(x)= ({(a_nx+a_{n-1})x+a_{n-2})x+...+a_1)x+a_0}$
\fi

	\question{\'Ecrire une fonction \lstinline{horner(x,L)} de paramètres un réel $x$ et une liste $L$ représentant un polynôme $P$, renvoie la valeur de $P(x)$ par la méthode de Hörner. Compter le nombre d'opérations.}
\ifprof
\begin{corrige}
\begin{lstlisting}
def horner(x,L):
    """L est la liste des coefficients"""
    n=len(L)-1
    S=0
    for i in range(n+1):
        S=S*x+L[n-i]
    return(S)
# 2n opérations. Linéaire mais plus intéressante que ci-dessus.
\end{lstlisting}
\end{corrige}
\else
\fi
	%\medskip Compter le nombre d'opérations au total pour calculer $P(x)$ et en donner que la complexité du programme est linéaire.


\ifprof
\else
	\bigskip On désire maintenant visualiser les temps d'exécution des trois fonctions précédentes pour des grandes valeurs de $n$.
\fi

	\question{Définir la liste $N$ des entiers naturels compris entre 0 et 100.}
\ifprof
\begin{corrige}
\begin{lstlisting}
N=[i for i in range(101)]
\end{lstlisting}
\end{corrige}
\else
\fi
	\question{Grâce à la fonction \lstinline{perf_counter} de la bibliothèque \lstinline{time},  écrire une fonction \lstinline{Temps_calcul(x)} qui:}
\begin{itemize}
\item définit 3 listes \lstinline{Tn}, \lstinline{Tr} et \lstinline{Th} contenant les temps de calcul de $P(x)$ pour $P=\sum_{k=0}^n k.x^k$ lorsque $n$ décrit $N$ avec respectivement la méthode naïve, la méthode rapide puis la méthode de Hörner.\\
\item trace les trois courbes  \lstinline{Tn}, \lstinline{Tr} et \lstinline{Th} en fonction de $N$ (on prendra $x=2$). Interpréter le résultat (on pourrait démontrer que les temps d'exécution des trois programmes sont de l'ordre de $n**2$ pour la méthode naïve (on parle de complexité quadratique), de l'ordre de $n\ln(n)$ pour l'exporapide, et de l'ordre de $n$ pour la méthode de Hörner (complexité linéaire)). 
\end{itemize}
\ifprof
\begin{corrige}
\begin{lstlisting}
import time as t

def Temps_calcul_P(x):
    # Le polynôme est donné par une liste des coefficients.
    Tn,Tr,Th=[],[],[]
    for n in N:
        L=[k for k in range(n+1)]
        tps=t.perf_counter()
        Pnaif(x,n)
        Tn.append(t.perf_counter()-tps)
        tps=t.perf_counter()
        Sr=0
        Prapide(x,n)
        Tr.append(t.perf_counter()-tps)
        tps=t.perf_counter()
        Phorner(x,L)
        Th.append(t.perf_counter()-tps)
    plt.plot(N,Th,label='méthode horner')
    plt.plot(N,Tr,label='méthode rapide')
    plt.plot(N,Tn,label='méthode naïve')
    plt.legend()
    plt.show()

\end{lstlisting}
\end{corrige}
\else
\fi	
