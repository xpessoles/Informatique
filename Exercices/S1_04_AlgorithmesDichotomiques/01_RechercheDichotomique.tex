%
%
%
%\section{Recherche dichotomique}

\subsection*{Recherche d'un élément dans une liste triée}

On se donne une liste \texttt{L} de nombres de longueur \texttt{n}, {triée dans l'ordre croissant}, et un nombre \texttt{x0}. 

Pour chercher \texttt{x0}, on va couper la liste en deux moitiés et chercher dans la moitié de tableau qui encadre \texttt{x0} et ainsi de suite...

On appelle \texttt{g} l'indice de l'élément du début de la sous-liste dans laquelle on travaille et \texttt{d} l'indice de l'élément de fin.

Au début, \texttt{g = 0} et \texttt{d = n - 1}.

%On souhaite construire un algorithme admettant l'invariant suivant:
%\bigskip
%
%\centerline{\fbox{si \t{x0} est dans \t{L} alors \t{x0} est dans la sous-liste \t{L[g:d]} (\t{g} inclus et \t{d} exclu).}}

%\bigskip
%\newpage
%\medskip 
On utilise la méthode suivante :
\begin{itemize}
\item on compare \texttt{x0} à << l'élément du milieu >>  \texttt{L[m]} avec \texttt{m = (g + d) // 2};
%son indice est \t{m} $=$\cache{\t{ n//2} (division euclidienne)}
\item si \texttt{x0 = L[m]}, on a trouvé \texttt{x0}, on peut alors s'arrêter;
\item si \texttt{x0} $<$ \texttt{L[m]}, c'est qu'il faut chercher entre \texttt{L[g]} et  \texttt{L[m-1]};% (\texttt{L[m]} exclu).
%dans \cache{la première moitié de la liste \t{L[g:m]}}\\
\item si \texttt{x0} $>$ \texttt{L[m]}, c'est qu'il faut chercher  entre \texttt{L[m+1]} et \texttt{L[d]}.% (\texttt{L[m]} exclu).
%}\\
%\finli
\end{itemize}

On poursuit jusqu'à ce qu'on a trouvé \texttt{x0} ou lorsque l'on a épuisé la liste \texttt{L}.

%\emph{Remarque} : On verra plus tard que l'algorithme est tellement rapide que cela ne change pas grand chose de s'arrêter avant et cela simplifie son écriture.

%En notant $g$ et $d$  les indices de gauche et de droite du morceau de la liste $l$ où l'on est en train de faire la %recherche, 

\question{Illustrer la méthode avec les deux exemples suivants :}
\begin{itemize}  \texttt{x0 = 5} et \texttt{L} $= \begin{array}{|c|c|c|c|c|c|c|c|c|} 
\hline -3 & 5 & 7 & 10 & 11 & 14 & 17 & 21 & 30 \\ \hline
\end{array}$
% g=0\\d=8\\m=4,L[m]>x0
% g=0\\d=3\\m=1,L[m]=x0
%\medskip 
\item \texttt{x0 = 11} et \texttt{L} $= \begin{array}{|c|c|c|c|c|c|c|c|c|} 
\hline -2 & 1 & 2 & 7 & 8 & 10 & 13 & 16 & 17  \\ \hline
\end{array}$.
\end{itemize}

\question{Si $x0$ n'est pas dans la liste $L$, donner un test d'arrêt du processus de dichotomie portant sur $g$ et $d$. }

\question{\'Ecrire une fonction \texttt{dichotomie(x0,L)} qui renvoie \texttt{True} ou \texttt{False} selon que \texttt{x0} figure ou non dans \texttt{L} par cette méthode. On utilisera une boucle \texttt{while} que l'on interrompra soit lorsque l'on a trouvé $x0$, soit lorsque l'on a fini de parcourir la liste.}

\question{Combien vaut $g-d$ au $i\ieme$ tour de boucle ? Si \texttt{x0} ne figure pas dans \texttt{L}, montrer que le nombre de tours de boucles nécessaires pour sortir de la fonction est de l'ordre de $\ln n$ où $\text{n=len(L)} (cela rend la fonction beaucoup plus efficace qu'une simple recherche séquentielle pour laquelle le nombre de comparaisons pour sortir de la boucle serait de l'ordre de $n$).}
%
%
%\subsection{Recherche d'un indice satisfaisant une condition donnée}
%
%On se donne dans cette question une liste \texttt{L} strictement décroissante de réels contenant des termes strictement positifs et des termes strictement négatifs. 
%
%\question{\'Ecrire une fonction \texttt{recherche\_dicho(L)}, d'argument la liste \texttt{L}, qui renvoie l'indice ainsi que la valeur du dernier élément positif ou nul par une méthode dichotomique adaptée de la précédente.}
%
%
%
%\subsection{Recherche d'un zéro d'une fonction}
%
%
%Soit une fonction $f:[a,b]\to\Rr$ ($a<b$) vérifiant : 
%$$\cond{$f$ continue sur $[a,b]$\\$f(a).f(b)\<0$ ie $f(a)$ et $f(b)$ de signes opposés.}.$$
%
%Le théorème des valeurs intermédiaires s'applique et assure que $f$ possède au moins un zéro $\ell$ entre $a$ et $b$. 
%
%%\medskip 
%
%\begin{center}
%	\begin{tikzpicture}[scale=2]
%		\shorthandoff{:};
%		\draw[->] (-2.5,0)--(2.5,0);
%		\draw[->] (-2.25,-1.5)--(-2.25,1.5);
%		\fenetre
%		\draw[domain=-2:2, samples=200, very thick]  plot ({\x},{((\x)^5+3*(\x)-7)/34});
%		\draw (-2,0)node{$\cdot$};
%		\draw (-1,0)node{$\cdot$};
%		\draw (1,0)node{$\cdot$};
%		\draw (0,0)node{$\cdot$};
%		\draw (2,0)node{$\cdot$};
%		\draw (-2 , 0) node[below] {$g_0=a$};
%		\draw (2 , 0) node[below] {$d_0=b$};
%		\draw (1 , 0.15) node[allow] {$g_2=m_1$};
%		\draw (2 , 0.15) node[allow] {$d_2=b$};
%		\draw (2 , -0.2) node[below] {$d_1=b$};
%		\draw (0 , 0) node[below] {$g_1=m_0$};
%		\draw (1.26 , 0) node[below] {$\ell$};
%	\end{tikzpicture}
%\end{center}
%
%L'idée consiste à créer une suite d'intervelles $[g_n,d_n]$ tels que pour tout entier naturel $n$, $$g_n\<\ell\<\d_n \textrm{ et } 0\< g_n-d_n=\Frac{g_{n-1}-d_{n-1}}2.$$
%
%%\medskip 
%
%On considère $m_0 = \dfrac{g_0+d_0}{2}$ et on évalue $f(m_0)$ : 
%%\bigskip 
%\begin{itemize}
%	\item si $f(m_0)\times f(d_0)\> 0$, on va poursuivre la recherche d'un zéro dans l'intervalle $[g_1,d_1]=[g_0,m_0]$ ;	\item sinon,  on poursuit la recherche dans l'intervalle $[g_1,d_1]=[m_0,d_0]$;
%	\item on recommence alors en considérant $m_1 = \dfrac{g_1+d_1}{2}$ ...\\
%\end{itemize}
%
%%\bigskip
%
%\question{Si l'on souhaite que $g_n$ et $d_n$ soient des solutions approchées de $\ell$ à une précision $\varepsilon$, quelle est la condition d'arrêt de l'algorithme ? Préciser alors la valeur approchée de $\ell$ qui sera renvoyée par la fonction.}
%	
%\question{\'Ecrire une fonction \texttt{recherche\_zero(f,a,b,epsilon)} qui renvoie une valeur approchée du zéro de \texttt{f} sur \texttt{[a,b]} a epsilon près.}
%	
%\question{Tester la fonction avec $f:x\mapsto x^2-2$ sur $[0,2]$ et $\varepsilon=0.001$.}
%	
%\question{Avec une erreur de $\varepsilon=\Frac 1{2^p}$, combien y a-t-il de comparaisons au final en fonction de $p$ ?}
%	
	