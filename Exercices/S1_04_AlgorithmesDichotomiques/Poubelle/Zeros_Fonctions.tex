Soit une fonction $f:[a,b]\to\mathbb{R}$ ($a<b$) vérifiant : 
$f$ continue sur $[a,b]$ et $f(a).f(b)\<0$ ie $f(a)$ et $f(b)$ de signes opposés.}.$$

Le théorème des valeurs intermédiaires s'applique et assure que $f$ possède au moins un zéro $\ell$ entre $a$ et $b$. 



\begin{center}
\begin{tikzpicture}[scale=2]
	\shorthandoff{:};
	\draw[->] (-2.5,0)--(2.5,0);
	\draw[->] (-2.25,-1.5)--(-2.25,1.5);
	\fenetre
	\draw[domain=-2:2, samples=200, very thick]  plot ({\x},{((\x)^5+3*(\x)-7)/34});
	\draw (-2,0)node{$\cdot$};
	\draw (-1,0)node{$\cdot$};
	\draw (1,0)node{$\cdot$};
	\draw (0,0)node{$\cdot$};
	\draw (2,0)node{$\cdot$};
	\draw (-2 , 0) node[below] {$g_0=a$};
	\draw (2 , 0) node[below] {$d_0=b$};
	\draw (1 , 0.15) node[allow] {$g_2=m_1$};
	\draw (2 , 0.15) node[allow] {$d_2=b$};
	\draw (2 , -0.2) node[below] {$d_1=b$};
	\draw (0 , 0) node[below] {$g_1=m_0$};
	\draw (1.26 , 0) node[below] {$\ell$};
\end{tikzpicture}
\end{center}

L'idée consiste à créer une suite d'intervalles $[g_n,d_n]$ tels que pour tout entier naturel $n$, $$g_n\le\ell\le\d_n \textrm{ et } 0\le g_n-d_n=\displaystyle\frac{g_{n-1}-d_{n-1}}2.$$

\medskip 

On considère $m_0 = \dfrac{g_0+d_0}{2}$ et on évalue $f(m_0)$ : 

\bigskip 

\begin{itemize}
\item Si $f(m_0)\times f(d_0)\ge 0$, on va poursuivre la recherche d'un zéro dans l'intervalle $[g_1,d_1]=[g_0,m_0]$

\item Sinon,  on poursuit la recherche dans l'intervalle $[g_1,d_1]=[m_0,d_0]$. 


\item On recommence alors en considérant $m_1 = \dfrac{g_1+d_1}{2}$ ...\\
\end{itemize}

\bigskip

\question{Si l'on souhaite que $g_n$ et $d_n$ soient des solutions approchées de $\ell$ à une précision $\varepsilon$, quelle est la condition d'arrêt de l'algorithme ? Préciser alors la valeur approchée de $\ell$ qui sera renvoyée par la fonction.}

\question{\'Ecrire une fonction \texttt{recherche\_zero(f,a,b,epsilon)} qui renvoie une valeur approchée du zéro de \texttt{f} sur \texttt{[a,b]} a epsilon près.}

\question{Tester la fonction avec $f:x\mapsto x^2-2$ sur $[0,2]$ et $\varepsilon=0.001$.}

\question{Avec une erreur de $\varepsilon=\Frac 1{2^p}$, combien y a-t-il de comparaisons au final en fonction de $p$ ?}