	
%Etant donné deux suites $(u_n)$ et $(v_n)$  de réels strictement positifs, on dit que $(u_n)$ est dominée par la suite $(v_n)$ lorsque $\left(\Frac{u_n}{v_n}\right)$ est une suite bornée. On note alors $u_n=\O{n\to +\infty}(v_n)$.
%
%Par exemple:
%
%\bi\q Si $(u_n)$ converge alors $u_n=\O{n\to +\infty}(1)$. Réciproque fausse.
%
%\q  $3n=\O{n\to +\infty}(n^2)$, $5n^2=\O{n\to +\infty}(n^2)$, $\ln(n)=\O{n\to +\infty}(n\ln(n)^2)$, $an^2+bn+c\ln(n)=\O{n\to +\infty}(n^2)$ ...
%
%\q Pour tout polynôme de degré $p$, $P=a_px^p+a_{p-1}x^{p-1}+...+a_1x+a_0$, on a $P(n)=\O{n\to +\infty}(n^p)$.
%
%\ei 
%
%En programmation, on dira qu'un programme a:
%
%- une complexité linéaire lorsque le nombre d'opérations effectuées est en $\O{n\to +\infty}(n)$
%
%- une complexité logarithmique lorsque le nombre d'opérations effectuées est en $\O{n\to +\infty}(\log(n))$
%
%- une complexité quadratique lorsque le nombre d'opérations effectuées est en $\O{n\to +\infty}(n^2)$


On se donne une liste \texttt{L} de nombres de longueur \texttt{n}, {triée dans l'ordre croissant}, et un nombre \texttt{x0}. 

Pour chercher \texttt{x0}, on va couper la liste en deux moitiés et chercher dans la moitié qui encadre \texttt{x0} et ainsi de suite...

On appelle \texttt{g} l'indice de l'élément du début de la sous-liste dans laquelle on travaille et \texttt{d} l'indice de l'élément de fin.

Au début, \texttt{g =} {0} et \texttt{d = } {n - 1}

%On souhaite construire un algorithme admettant l'invariant suivant:
%\bigskip
%
%\centerline{\fbox{si \t{x0} est dans \t{L} alors \t{x0} est dans la sous-liste \t{L[g:d]} (\t{g} inclus et \t{d} exclu).}}

%\bigskip
%\newpage
\medskip 
On utilise la méthode suivante :

\begin{itemize}
	\item On compare \texttt{x0} à "l'élément du milieu"  \texttt{L[m]} avec \texttt{m = (g + d) // 2}}
%son indice est \t{m} $=$\cache{\t{ n//2} (division euclidienne)}

\item Si \texttt{x0 = L[m]}, on a trouvé \texttt{x0}, on peut alors s'arrêter.
\item  Si \texttt{x0} $<$ \texttt{L[m]}, c'est qu'il faut chercher entre \texttt{L[g]} et  \texttt{L[m-1]}% (\texttt{L[m]} exclu).
}%dans \cache{la première moitié de la liste \t{L[g:m]}}\\
\item Si \texttt{x0} $>$ \texttt{L[m]}, c'est qu'il faut chercher  entre \texttt{L[m+1]} et \texttt{L[d]}% (\texttt{L[m]} exclu).
}\\
\end{enumerate}

On poursuit jusqu'à ce qu'on a trouvé \texttt{x0} ou lorsque l'on a épuisé la liste \texttt{L}.

%\emph{Remarque} : On verra plus tard que l'algorithme est tellement rapide que cela ne change pas grand chose de s'arrêter avant et cela simplifie son écriture.

%En notant $g$ et $d$  les indices de gauche et de droite du morceau de la liste $l$ où l'on est en train de faire la %recherche, 

\bigskip \question{Illustrer la méthode avec les deux exemples suivants en déterminant successivement les valeurs de $g$, de $d$ et de $m$:}


-   \texttt{x0 = 5} et \texttt{L} $= \begin{array}{|c|c|c|c|c|c|c|c|c|} 
\hline -3 & 5 & 7 & 10 & 11 & 14 & 17 & 21 & 30 \\ \hline
\end{array}$

% g=0\\d=8\\m=4,L[m]>x0
% g=0\\d=3\\m=1,L[m]=x0

\medskip 

- \texttt{x0 = 11} et \texttt{L} $= \begin{array}{|c|c|c|c|c|c|c|c|c|} 
\hline -2 & 1 & 2 & 7 & 8 & 10 & 13 & 16 & 17  \\ \hline
\end{array}$


%g=0\\d=8\\m=4,L[m]<x0
%g=5\\d=8\\m=6,L[m]>x0
%g=5\\d=5\\m=5,L[m]<x0
%g=6\\d=5

\bigskip \question{Si $x0$ n'est pas dans la liste $L$, donner un test d'arrêt du processus de dichotomie portant sur $g$ et $d$. }

\question{\'Ecrire une fonction \texttt{dichotomie(x0,L)} qui renvoie \texttt{True} ou \texttt{False} selon que \texttt{x0} figure ou non dans \texttt{L} par cette méthode. On utilisera une boucle \texttt{while} que l'on interrompra soit lorsque l'on a trouvé $x0$, soit lorsque l'on a fini de parcourir la liste.}

\question{Combien vaut $g-d$ au $i\ieme$ tour de boucle ? Si \texttt{x0} ne figure pas dans \texttt{L}, montrer que le nombre de tours de boucles nécessaires pour sortir de la fonction est de l'ordre de $\ln n$ où $n=len(L)$ (cela rend la fonction beaucoup plus efficace qu'une simple recherche séquentielle pour laquelle le nombre de comparaisons pour sortir de la boucle serait de l'ordre de $n$)}


%\subsection{Recherche d'un indice satisfaisant une condition donnée}
%
%On se donne dans cette question une liste \texttt{L} strictement décroissante de réels contenant des termes strictement positifs et des termes strictement négatifs. Par exemple $L=[6,5,3,1.5,0.6,-0.3,-1,-3]$
%
%On cherche l'indice et la valeur du dernier élément positif ou nul $x0$. Sur cet exemple, on doit trouver $[4,0.6]$. Pour cela, on va adapter la méthode de dichotomie ci-dessus de la manière suivante:
%
%- On pose au départ $g=0$, $d=n-1$ avec $n=len(L)$.
%
%- On définit $m=(g+d)//2$. 
%
%Si $L[m]>0$, c'est que $x0$ est situé entre $L[m]$ et $L[n-1]$. On pose donc $g=m$, $d=n-1$.
%
%Si $L[m]<=0$, c'est que $x0$ est situé entre $L[0]$ et $L[m]$. On pose donc $g=0$, $d=m$.
%
%- On continue ainsi le processus.
%
%\be\q Quelle est la condition d'arrêt à la dernière étape ?
%
%\q \'Ecrire une fonction \texttt{recherche\_dicho(L)}, d'argument la liste \texttt{L}, qui renvoie l'indice ainsi que la valeur du dernier élément positif ou nul. On utilisera là encore une boucle \texttt{while}.
%
%\ee