%\exer{Recherche d'un mot dans un texte}
%\setcounter{numques}{0}


\question{\'Ecrire une fonction \texttt{est\_ici(texte,motif,i)} qui, étant données deux chaines de caractères \texttt{texte}, \texttt{motif} et un indice \texttt{i}, renvoie \texttt{True} ou \texttt{False} selon que \texttt{motif} est ou n'est pas dans \texttt{texte} au rang \texttt{i}. On utilisera une boucle \texttt{while}.}

\question{\'Ecrire une fonction \texttt{est\_sous\_mot(texte,motif)} qui renvoie \texttt{True} ou \texttt{False} selon que \texttt{motif} est dans \texttt{texte} ou pas. On utilisera une boucle \texttt{while}.}

\question{\'Ecrire une fonction \texttt{position\_sous\_mot(texte,motif)} qui renvoie la liste de toutes les occurences de l'indice de position de la première lettre du mot \texttt{motif} dans \texttt{texte}. On utilisera une boucle \texttt{for}.}


%\end{multicols}
