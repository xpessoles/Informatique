\begin{itemize}\item  Un dictionnaire est une suite de couple (clé, valeur) non ordonnée. Chaque élément est repéré par sa clé qui est donc unique. Le type d'un dictionnaire est \texttt{dict}.
	
	\item  Le dictionnaire vide est \texttt{dico=\{\}}
	
	\item  On peut définir un dictionnaire globalement:
	
	\texttt{dico=\{”MPSI”:46,”PCSI”:47,”PTSI”,45\}}	ou “MPSI” est une clé, 46 sa valeur.
	
	\item  On peut ajouter des éléments à un dictionnaire:
	
	\texttt{dico[”MP”]=45}
	
	et afficher:
	
	\texttt{print(dico)} renvoie: \texttt{dico=\{”MPSI”:46,”PCSI”:47,”PTSI”,45,”MP”:45\}}

\texttt{print(dico[”MPSI”])} affiche 46

\item On peut supprimer un élément:

\texttt{del dico[”MP”]}	

\texttt{print(dico)} renvoie \texttt{dico=\{”MPSI”:46,”PCSI”:47,”PTSI”,45\}}

\item Test d'appartenance d'une clé:

\texttt{print(“PCSI” in dico)} qui renvoie ici \texttt{True} (ou \texttt{False} dans le cas général)

\item Nombre d'éléments d'un dictionnaire:

\texttt{len(dico)}

\item  Parcourir les éléments d'un dictionnaire avec \texttt{items()}:

\texttt{for elt in dico.items():}

\texttt{\ \qquad  a=elt[0]}  \# clé

\texttt{\ \qquad  b=elt[1]}   \# valeur

OU

\texttt{for clé, valeur in dico.items():}	

\texttt{\ \qquad  ...}

\item  Copie d'un dictionnaire:

\texttt{import copy}

\texttt{dico1=copy.deepcopy(dico)}	

(\textit{et non \texttt{dico1=dico} qui va créer un alias seulement.})

\end{itemize}

\subsection*{Exemple 1}

On dispose d'un dictionnaire associant à des noms de commerciaux d'une société le nombre de ventes qu'ils ont réalisées:

\medskip Par exemple:
\[
\texttt{ventes=$\{$”Dupont”:14,”Henry”:19,”Pierre”:15,”Lucas”:21$\}$}
\]

\question{Ecrire une fonction \texttt{Nb\_Ventes(ventes:dict)} qui prend en entrée un dictionnaire et renvoie le nombre total de 
ventes dans la société.}

\question{Ecrire une fonction \texttt{Nom\_vendeur(ventes:dict)} qui prend en entrée un dictionnaire et renvoie le nom du vendeur ayant réalisé le plus de ventes. Si plusieurs vendeurs sont ex-aequo, la fonction devra retourner le nom de l'un d'entre eux seulement.}


\subsection*{Exemple 2}

On se donne une liste (ou un tableau) \texttt{L} constituée d'entiers et contenant au moins deux éléments.

On cherche à identifier un couple \texttt{L[p]} et \texttt{L[q]} d'éléments les plus proches parmi ceux de la liste, c'est-à-dire vérifiant les conditions:

\begin{itemize} \item $0\le p<q\le \texttt{len(L)}-1$
	
	\item  $|L[p]-L[q]|=\min\{|L[i]-L[j]\tq 0\le i<j\le \texttt{len(L)}-1\}$.
	
\end{itemize}

\medskip \question{On appelle distance minimale, la distance entre deux éléments  les plus proches (éventuellement égaux) et d'indices distincts. 
	\'Ecrire une fonction \texttt{distance\_min(L)} qui renvoie la distance minimale de \texttt{L}.}

Par exemple, pour \texttt{T=[10, 10,10, 8, 8, 0, 5,1, 9, 5, 9, 9]}. la fonction devra renvoyer 0.

\medskip \question{\'Ecrire une fonction \texttt{indices\_distance\_min2(L)} qui renvoie un couple d'indices réalisant la distance minimale}

Par exemple, pour \texttt{T=[10, 10,10, 8, 8, 0, 5,1, 9, 5, 9, 9]}. la fonction pourra renvoyer $(0,8)$.

\bigskip 

Une autre méthode pour trouver un couple d'indices solution consiste à procéder de la façon suivante:

\begin{itemize}\item Réserver un dictionnaire \texttt{D}.
	
	\item  Pour chaque indice $0\le i\le \texttt{len (L)}-1$, placer dans \texttt{D[i]} le couple $(j,|L[j]-L[i]|)$ qui réalise le minimum de la distance $|L[i]-L[j]|$ pour $j>i$.
	
	\item Une fois ce dictionnaire construit, le parcourir pour déterminer un couple $(p,q)$ solution.
	
\end{itemize}

\medskip \question{\'Ecrire une fonction \texttt{indices\_distance\_min3(L)} qui, étant donnée une liste \texttt{L}, réalise ces opérations et renvoie un couple solution}

