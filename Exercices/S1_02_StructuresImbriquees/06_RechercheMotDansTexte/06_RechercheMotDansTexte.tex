\exer{Recherche d'un mot dans un texte}
\setcounter{numques}{0}


\question{\'Ecrire une fonction \texttt{est\_ici(texte,motif)} qui, étant données deux chaines de caractères \texttt{texte}  et \texttt{motif}, renvoie \texttt{True} ou \texttt{False} selon que mot est ou n'est pas dans texte. On n'utilisera pas de slicing mais on fera deux méthodes, l'une sans booléen explicite, l'autre avec.}

\question{\'Ecrire une fonction \texttt{est\_sous\_mot(texte,motif)} qui renvoie \texttt{True} ou \texttt{False} selon que motif est dans texte ou pas.}

\question{\'Ecrire une fonction \texttt{position\_sous\_mot(texte,motif)} qui renvoie la liste de toutes les occurences de l'indice de position du mot \texttt{motif} dans \texttt{texte}.}



%\end{multicols}
