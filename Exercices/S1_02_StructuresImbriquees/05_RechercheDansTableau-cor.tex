\question{On appelle distance minimale, la distance entre deux éléments  les plus proches (éventuellement égaux) et d'indices distincts. 
	\'Ecrire une fonction \texttt{distance\_min(L)} qui renvoie la distance minimale de \texttt{L}.}

\begin{lstlisting}
def distance_min(L):  # On cherche min |Ti-tj| pour i<j
    n=len(L)
    min=abs(L[1]-L[0])
    for i in range(n):
        for j in range(i+1,n):
            if abs(L[j]-L[i])<min:
                min=abs(L[j]-L[i])
    return(min)


\end{lstlisting}

\question{\'Ecrire une fonction \texttt{indices\_distance\_min2(L)} qui renvoie un couple d'indices réalisant la distance minimale}


\begin{lstlisting}
def indices_distance_min2(L):  # On cherche min |Ti-tj| pour i<j
    n=len(L)
    min=abs(L[1]-L[0])
    p,q=0,1
    for i in range(n):
        for j in range(i+1,n):
            if abs(L[j]-L[i])<min:
                min=abs(L[j]-L[i])
                p,q=i,j
    return p,q


\end{lstlisting}



\question{\'Ecrire une fonction \texttt{indices\_distance\_min3(L)} qui, étant donnée une liste \texttt{L}, réalise ces opérations et renvoie un couple solution}
 
 
 \begin{lstlisting}
 
 def indices_distance_min3(L):
    D={}
    n=len(L)
    for i in range(n-1):
        m=abs(L[i+1]-L[i])
        j=i+1
        for k in range(i+1,n):
            if abs(L[k]-L[i])<m:
                m=abs(L[k]-L[i])
                j=k
        D[str(i)]=[j,m]
    p,q=0,D["0"][0]
    min=D["0"][1]
    for elt in D.items():
    # elt de la forme ["3",[4,min {|Lk-L3|, k>4}]
        if elt[1][1]<min:
            p,q=int(elt[0]),elt[1][0]
            min=elt[1][1]
    return(p,q)


\end{lstlisting}

