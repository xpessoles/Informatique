\section*{Rappels sur les dictionnaires}

- Un dictionnaire est une suite de couple (clé, valeur) non ordonnée. Chaque élément est repéré par sa clé qui est donc unique. Le type d'un dictionnaire est \texttt{dict}.

- Le dictionnaire vide est \texttt{dico=\{\}}

- On peut définir un dictionnaire globalement:

\texttt{dico=\{”MPSI”:46,”PCSI”:47,”PTSI”,45\}}	ou “MPSI” est une clé, 46 sa valeur.

- On peut ajouter des éléments à un dictionnaire:

\texttt{dico[”MP”]=45}

et afficher:

\texttt{print(dico)} renvoie: \texttt{dico=\{”MPSI”:46,”PCSI”:47,”PTSI”,45,”MP”:45\}}

\texttt{print(dico[”MPSI”])} affiche 46

- On peut supprimer un élément:

\texttt{del dico[”MP”]}	

\texttt{print{dico}} renvoie \texttt{dico=\{”MPSI”:46,”PCSI”:47,”PTSI”,45\}}

 - Test d'appartenance d'une clé:

\texttt{print(“PCSI” in dico)} qui renvoie ici \texttt{True} (ou \texttt{False} dans le cas général)

- Nombre d'éléments d'un dictionnaire:

\texttt{len(dico)}

- Parcourir les éléments d'un dictionnaire avec \texttt{items()}:

\texttt{for elt in dico.items():}

\texttt{\ \qquad  a=elt[0]}  \# clé

\texttt{\ \qquad  b=elt[1]}   \# valeur

OU

\texttt{for clé, valeur in dico.items():}	

\texttt{\ \qquad  ...}

- Copie d'un dictionnaire:

\texttt{import copy}

\texttt{dico1=copy.deepcopy(dico)}	

(\textit{et non \texttt{dico1=dico} qui va créer un alias seulement.})


\bigskip

\textbf{Exercice:} On dispose d'un dictionnaire associant à des noms de commerciaux d'une société le nombre de ventes qu'ils ont réalisées:
Par exemple:

\texttt{ventes=$\{$”Dupont”:14,”Henry”:19,”Pierre”:15,”Lucas”:21$\}$}

\question{Ecrire une fonction \texttt{Nb\_Ventes(ventes:dict)} qui prend en entrée un dictionnaire et renvoie le nombre total de 
ventes dans la société.}

\question{Ecrire une fonction \texttt{Nom\_vendeur(ventes:dict)} qui prend en entrée un dictionnaire et renvoie le nom du vendeur ayant réalisé le plus de ventes. Si plusieurs vendeurs sont ex-aequo, la fonction devra retourner le nom de l'un d'entre eux seulement.}

