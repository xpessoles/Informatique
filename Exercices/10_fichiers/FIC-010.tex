Le fichier \texttt{body.csv} contient des données anatomiques de 507 personnes adultes. Pour chaque personne, plusieurs données sont disponibles : 
\begin{itemize}
  \item identifiant de la personne (nombre entre 1 et 507) ;
  \item 21 mesures de la corpulence de la personne (en centimètres) ;
  \item âge (en années) ;
  \item poids (en kilogrammes) ; 
  \item taille (en centimètres) ;
  \item sexe (1 : homme et 0 : femme).
\end{itemize}

On pensera d'abord à ouvrir ce fichier dans un éditeur de texte puis dans un tableur afin de bien visualiser ces données. 

L'indice de masse corporelle (IMC) est le rapport entre le poids d'un individu (exprimé en $kg$) et le carré de sa taille (exprimée en $m$). 
C'est un indicateur (simpliste) permettant de mesurer le sous/sur-poids d'un individu adulte. 
Ainsi, on (enfin, l'OMS) considère que si l'IMC d'un adulte est : 


\begin{itemize}
  \item dans $]0;18,5[$, la personne est en sous-poids ;
  \item dans $[18,5;25[$, la personne a un poids normal ; 
  \item dans $[25;30[$, la personne est en sur-poids ; 
  \item dans $[30;+\infty[$, la personne est obèse. 
\end{itemize}

Quelques rappels sur les chaînes. Il existe un moyen de \og découper \fg{} des chaînes de caractères en \python\ : c'est la méthode \texttt{split}. 
Réciproquement, la méthode \texttt{join(t)} appliquée à une chaîne \texttt{x} permet de concaténer toutes les chaînes du tableau \texttt{t}, séparées par \texttt{x}.
Il existe aussi des outils de conversion de nombres flottants en chaînes de caractère, et vice-versa.
Tout cela s'utilise comme suit.

\begin{pyconsole}
s = '123,45,2;1587,45,;45'
s.split(',')
s.split(';')
sep = '<'
t = ['GA','BU','ZO','MEU']
sep.join(t)
str(123.456)
float('456.123')
\end{pyconsole}

Enfin, on voudra bien entendu représenter les données contenues dans \texttt{body.csv} sous forme de \emph{tableau à double entrée}, c'est-à-dire comme une matrice. On représentera le tableau sous forme de liste \python, ligne par ligne, chaque ligne étant elle même une liste \python.

\begin{ex}
  Pour représenter le tableau $T = \begin{pmatrix} a & b & c \\ 4 & 5 & 6 \end{pmatrix}$, on fera comme suit. 
\begin{pyconsole}
T = [['a','b','c'],['4','5','6']]
T
T[1]
T[0][1]
T.append(['x','y','z'])
T
\end{pyconsole}
\end{ex}


\question{} \label{qu.lecture} \'Ecrire une fonction \texttt{lecture(nom\_de\_fichier)} qui prend en argument une chaîne de caractères \texttt{nom\_de\_fichier} contenant le chemin du fichier \texttt{body.csv} et qui renvoie un tableau à double entrées, où :
\begin{itemize}
  \item[\textbullet] la ligne $0$ contient le titre de chaque colonne ;
  \item[\textbullet] si $1\leq i \leq 507$, chaque ligne $i$ contient successivement:
    \begin{itemize}
      \item l'identifiant $i$ ;
      \item les $21$ mesures de corpulence pour l'individu \no $i$ ;
      \item l'âge de l'individu \no $i$ ;
      \item le poids de l'individu \no $i$ ;
      \item la taille de l'individu \no $i$ ; 
      \item le sexe de l'individu \no $i$.
    \end{itemize}  
\end{itemize}


\emph{Indication :} on choisira à chaque fois le type le plus convenable pour chaque donnée. 

\question{}\label{qu.calcul_imc} Écrire une fonction \texttt{calcul\_imc(T)} prenant en argument un tableau à double entrées que l'on supposera être celui renvoyé par la fonction \texttt{lecture} et qui renvoie un tableau à double entrées S possédant 26 colonnes, identique à T sur ses 25 premières colonnes et dont la dernière colonne contient l'IMC de chaque individu.    

\question{} \'Ecrire une fonction \texttt{catimc(nom\_de\_fichier)} qui prend en argument une chaîne de caractères \texttt{nom\_de\_fichier} contenant le chemin du fichier \texttt{body.csv} et qui renverra une liste \python\ de longueur 4 qui, pour chaque catégorie de poids, comptera le nombre de personnes du jeu de données dans cette catégorie. 
    
    Ainsi, le premier élément du tableau renvoyé par \texttt{catimc(nom\_de\_fichier)} sera le nombre d'individus en sous-poids dans \texttt{body.csv}. 

%
%\section{\'Ecriture de données.}

\question{} \label{q.write} \'Ecrire une fonction \pyv{ecrire(T,nom_de_fichier,sep=',')} qui écrit le contenu d'un tableau \texttt{T} analogue à celui obtenu en sortie de la fonction \texttt{calcul\_imc()} (voir question \ref{qu.calcul_imc}) dans le fichier \pyv{nom_de_fichier}. 

On veillera notamment à séparer les colonnes par le séparateur \texttt{sep}.
    
Vous enverrez le fichier produit à l'enseignant. Les instructions de rendu sont données dans le préambule.