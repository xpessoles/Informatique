Copier et coller le fichier "dipole$\_$electrique.txt" que vous trouverez sur le site de la classe dans votre répertoire personnel. Celui-ci contient des données numériques, obtenues en TP d’électronique, correspondant à différentes mesures de l'intensité traversant un dipôle et de la tension entre ses bornes. On souhaite récupérer ces données et les exploiter.

Ouvrir le fichier "dipole$\_$electrique.txt" à l'aide d'un éditeur de texte (de type bloc-notes) et observer son contenu.

\question{}Écrire un programme Python que vous appellerez \pyv{lit_dipole} et qui prendra en argument une variable de type chaine de caractère qui sera le nom du fichier de données. Cette fonction renverra deux listes notées \pyv{I} et \pyv{U},  l'une contenant les intensités mesurées, et l'autre contenant les
tensions mesurées.

\question{}Créer une fonction que l'on notera \pyv{tracer_dipole}  qui prendra en argument deux listes représentant respectivement l'intensité I et U. Cette fonction permettra de représenter graphiquement les couples de points (I,U). On sauvegardera la figure avec le nom 'tp06$\_$noms$\_$q02.png'



\question{}Conjecturer alors une relation reliant la tension aux bornes du dipôle à l'intensité le traversant.