\exer{[FIC-004]}
\setcounter{numques}{0}~\\

\question{Écrire une fonction \texttt{moyennes(fichier\_notes,fichier\_moyennes,sep=',')} lisant les notes écrites dans \texttt{fichier\_notes} et écrivant dans le \texttt{fichier\_moyennes} le prénom de chaque élève, suivi par la moyenne de ses notes.}

Par convention, la première ligne de chaque fichier comporte les titres des colonnes et la première colonne contient le prénom de chaque étudiant. Un exemple (fichiers \texttt{notes.csv} et \texttt{moyennes.csv}) sera mis en ligne sur le site de classe. 

On respectera le format  \texttt{.csv}, le séparateur par défaut sera donc les virgules (\texttt{','}). 