\exer{[FIC-006]}
\setcounter{numques}{0}~\\

Vous trouverez sur le site de classe un fichier \texttt{cats.csv}.
Il contient des mesures de corpulence de chats. 
Vous y trouverez trois colonnes : le sexe de chaque chat (colonne \texttt{Sex}), son poids en kilogrammes (colonne \texttt{Bwt}) et le poids de son cœur en grammes (colonne \texttt{Hwt}). 

Écrire un script \texttt{Python}{} permettant de lire ce fichier et d'écrire dans un autre fichier, pour chaque sexe, le poids moyen et le poids du cœur moyen. 
