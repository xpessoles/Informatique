\exer{[FIC-012]}
\setcounter{numques}{0}


Le fichier \texttt{embrunman2019.csv} contient les résultats de l'ultratriatlon de Embrun 2019. Les données concernent dans l'ordre :
\begin{itemize}
  \item le classement  ;
  \item le numéro de dossard ;
  \item
\end{itemize}

On pensera d'abord à ouvrir ce fichier dans un éditeur de texte puis dans un tableur afin de bien visualiser ces données. 



Quelques rappels sur les chaînes. Il existe un moyen de \og découper \fg{} des chaînes de caractères en \texttt{Python}\ : c'est la méthode \texttt{split}. 
Réciproquement, la méthode \texttt{join(t)} appliquée à une chaîne \texttt{x} permet de concaténer toutes les chaînes du tableau \texttt{t}, séparées par \texttt{x}.
Il existe aussi des outils de conversion de nombres flottants en chaînes de caractère, et vice-versa.
Tout cela s'utilise comme suit.

\begin{lstlisting}
>>> s = '123,45,2;1587,45,;45'
>>> s.split(',')
>>> s.split(';')
>>> sep = '<'
>>> t = ['GA','BU','ZO','MEU']
>>> sep.join(t)
>>> str(123.456)
>>> float('456.123')
\end{lstlisting}

Enfin, on voudra bien entendu représenter les données contenues dans \texttt{embrunman2019.csv} sous forme de \emph{tableau à double entrée}, c'est-à-dire comme une matrice. On représentera le tableau sous forme de liste \texttt{Python}, ligne par ligne, chaque ligne étant elle même une liste \texttt{Python}.
