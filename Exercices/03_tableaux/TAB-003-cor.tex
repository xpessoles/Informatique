L'algorithme suivant répond à la question. 
\begin{Verbatim}[gobble=0,numbers=left]
def pascal_rapide(n):
    t = [1]*(n+1)
    for m in range(n):
        t[1:m+1] = [t[i-1]+t[i] for i in range(1,m+1)]
    return t
\end{Verbatim}
Cependant, la structure de la ligne 4 n'est pas très détaillée. Nous allons donc justifier l'algorithme suivant. 
\begin{Verbatim}[gobble=0,numbers=left]
def pascal(n):
    """Calcule la n-ième ligne du triangle de Pascal"""
    t = [1]
    for m in range(n):
        # Invariant : t = pascal(m)
        nouv = [1]*(m+2)
        for i in range(1,m+1):
            # Invariant : nouv[:i] = pascal(m+1)[:i]
            nouv[i] = t[i-1]+t[i]
        t = nouv.copy()
    return t
\end{Verbatim}
\begin{rem}
  On ne vous demandait pas d'écrire les invariants, ils sont présents à titre d'exemple. 
\end{rem}
Notons, pour $n,p\in\Z$, $\displaystyle C(n,p) = \binom{n}{p}$.
\begin{itemize}
  \item[\textbullet] Ligne 3 : \pyv{t} vaut $[1]$, c'est bien le tableau $[C(0,0)]$.
  \item[\textbullet] Ligne 4 : soit $\texttt{m} \in \ii{0,n}$, supposons que \pyv{t} contienne bien le tableau des $C(\texttt{m},p)$, pour $0\leq p \leq \texttt{m}$. 
  \item[\textbullet] Ligne 5 : \pyv{nouv} est un tableau de longueur $\texttt{m}+2$, \pyv{nouv[0]} et \pyv{nouv[m+1]} valent $1$. 
    Ainsi, \pyv{nouv[0]} vaut $C(\texttt{m}+1,0)$ et \pyv{nouv[m+1]} vaut $C(\texttt{m}+1,\texttt{m}+1)$.
  \item[\textbullet] Ligne 9 : soit $\texttt{i} \in \ii{1,m+1}$, supposons que \pyv{nouv[:i]} soit le tableau des $C(\texttt{m}+1,j)$ pour $0\leq j < \texttt{i}$. 
    Alors, par la formule du triangle de Pascal, $C(\texttt{m}+1,\texttt{i}) = C(\texttt{m},\texttt{i}) + C(\texttt{m},\texttt{i}-1)$, ce qui vaut \pyv{t[i]+t[i-1]}. 
    On a donc bien \pyv{nouv[i]} qui vaut $C(\texttt{m}+1,\texttt{i})$. Ainsi, \pyv{nouv[:i+1]} est le tableau des $C(\texttt{m}+1,j)$ pour $0\leq j < \texttt{i}+1$. 
  \item[\textbullet] Boucle des lignes 7-9 : en sortie la boucle, on a $\texttt{i} = \texttt{m}$ donc \pyv{nouv[:m+1]} est le tableau des $C(\texttt{m}+1,j)$ pour $0\leq j < \texttt{m}+1$. 
  \item[\textbullet] Ligne 10 : comme \pyv{nouv[:m+1]} est le tableau des $C(\texttt{m}+1,j)$ pour $0\leq j < \texttt{m}+1$ et comme \pyv{nouv[m+1]} vaut $C(\texttt{m}+1,\texttt{m}+1)$, \pyv{t} est le tableau des $C(\texttt{m}+1,j)$ pour $0\leq j \leq \texttt{m}+1$
  \item[\textbullet] Boucle des lignes 4-10 : en sortie de boucle, on a $\texttt{m} = \texttt{n-1}$ donc  \pyv{t} est le tableau des des $C(\texttt{n},j)$ pour $0\leq j \leq \texttt{n}$. 
  \item[\textbullet] Ligne 11 - conclusion : la fonction renvoie bien le résultat demandé.
\end{itemize}