\exer{}
\setcounter{numques}{0}

 \question{\'Ecrire une fonction \texttt{indice(x, t)} renvoyant un indice
\texttt{i} tel que \texttt{t[i]==x} si \texttt{x} apparaît dans le tableau \texttt{t} et $-1$ sinon.}
%\end{exo}

%\begin{exo}
 \question{\'Ecrire une fonction \texttt{tous\_les\_indices(e,t)} renvoyant la liste de tous les indices des occurrences de \texttt{e} dans le tableau \texttt{t}.}
%\end{exo}

%\begin{exo}
 \question{Écrire une fonction \texttt{compte(e,t)} renvoyant le nombre d’occurrences de \texttt{e} dans le tableau \texttt{t}.}
%\end{exo}

%\begin{exo}
 \question{\'Ecrire une fonction \texttt{ind\_appartient\_dicho(e,t)} renvoyant l'indice d'une occurrence de \texttt{e} dans le tableau \texttt{t} (\texttt{None} si \texttt{e} n'est pas dans \texttt{t}), en supposant que \texttt{t} est trié par ordre croissant.}
%\end{exo}


%\begin{exo}
 \question{\'Ecrire une fonction \texttt{dec\_appartient\_dicho(e,t)} renvoyant un booléen indiquant si \texttt{e} est dans le tableau \texttt{t}, en supposant que \texttt{t} est trié par ordre décroissant.}
%\end{exo}

%\begin{exo}
 \question{Écrire une fonction \texttt{compte(e,t)} comptant le nombre d'occurences de l'élément \texttt{e} dans le tableau \texttt{t}.}