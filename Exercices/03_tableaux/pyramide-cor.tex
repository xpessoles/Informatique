Dans une pyramide de nombres, en partant du sommet, et en se dirigeant vers le bas à 
chaque étape, on cherche à maximiser le total de la somme des nombres traversés.

\begin{center}
$ \underline{\textbf{2}}$\\$4\ \underline{\textbf{5}}$\\$1\ 7\ 
\underline{\textbf{8}}$\\$2\ 3\ 1\ \underline{\textbf{6}}$\\$9\ 4\ 6\ 
\underline{\textbf{5}}\ 2$\end{center}

Sur cet exemple, la somme totale maximale est 26, obtenue en parcourant les nombres soulignés.


On peut voir la pyramide comme un graphe, parcourir les 16 chemins, et choisir celui qui 
a le plus grand total. Quand la pyramide a $n$ niveaux, il y a $2^{n-1}$ chemins, ce qui 
rend vite cet algorithme inexploitable.\\
Un algorithme récursif est aussi possible, mais alors certains calculs sont effectués 
plusieurs fois, et là encore l'algorithme est trop long pour des pyramides de taille 
conséquente.\\
La méthode la plus rapide est celle utilisant la \emph{programmation dynamique}. L'idée 
est résumée dans la séquence suivante :

\begin{center}
\begin{tabular}{ccccccccccccc}
 2 &\hspace{1cm}&& 2 &\hspace{1cm}&& 2 &\hspace{1cm}&& 2&\hspace{1cm}& &26\\
 4\ 5 &\hspace{1cm}&& 4\ 5 &\hspace{1cm}&& 4\ 5 &\hspace{1cm}&& 20\ 24 &\hspace{1cm}&&\\
 1\ 7\ 8 &\hspace{1cm}&& 1\ 7\ 8&\hspace{1cm}&& 12\ 16\ 19 &\hspace{1cm}&& 
&\hspace{1cm}&&\\
 2\ 3\ 1\ 6 &\hspace{1cm}&& 11\ 9\ 7\ 11 &\hspace{1cm}&& &\hspace{1cm}&& &\hspace{1cm}&& 
\\
 9\ 4\ 6\ 5\ 2 &\hspace{1cm}&& &\hspace{1cm}&& &\hspace{1cm}&& &\hspace{1cm}&&
\end{tabular}
\end{center}

À vous de comprendre cette idée et d'écrire un programme calculant ce total maximum pour 
des pyramides que l'on définit de la manière suivante : chaque pyramide est donnée dans 
un tableau, dont les éléments sont les lignes de la pyramide, représentées elles-mêmes 
dans un tableau. Par exemple, la pyramide de l'exemple sera représentée dans le tableau 
\texttt{[ [ 2 ], [ 4, 5 ], [ 1, 7, 8 ], [ 2, 3, 1, 6 ], [ 9, 4, 6, 5, 2 ] ]}. 

Vous pourrez utiliser la fonction suivante pour construire ces pyramides :
\begin{lstlisting}
def pyramide(alpha,n):
    p = []
    x = alpha
    for i in range(n):
        l = []
        for j in range(i+1) :
            y = x % 10
            l.append(y)
            x = (15091 * x) % 64007
        p.append(l)
    return p
\end{lstlisting}

Enfin, on rappelle que si $x\in\N$, on note $x\% 10$ le reste de la division euclidienne par 10.

\question{Donner la somme maximale pour la pyramide ayant $20$ lignes, définie par les $u_k \% 10$ (lus de haut en bas, de gauche à droite, la première ligne est $[u_0\%10]$, la seconde $[u_1\%10$, $u_2\%10]$ \emph{etc}.).}

\medskip

\question{Donner la somme maximale pour la pyramide ayant $100$ lignes, définie par les $u_k \% 10$ (lus de haut en bas, de gauche à droite, la première ligne est $[u_0\%10]$, la seconde $[u_1\%10$, $u_2\%10]$ \emph{etc}.).}