\exer{}
\setcounter{numques}{0}

Dans un tableau $t = [t_0,\dots,t_{n-1}]$, on dit que l'on a un record à une position $0 \leq k < n$ si $\forall i < k,~ t_i < t_k$. 
Par définition, on a toujours un record en position 0.


\question\ \'Ecrire une fonction \texttt{record(t,k)} qui, à un tableau de nombres \texttt{t} et un entier \texttt{k}, renvoie le booléen \texttt{True} si \texttt{t} admet un record en position \texttt{k}, \texttt{False} sinon.

\question\ \'Ecrire une fonction \texttt{nb\_records(t)} qui, à un tableau de nombres \texttt{t}, renvoie le nombre de records dans~\texttt{t}.