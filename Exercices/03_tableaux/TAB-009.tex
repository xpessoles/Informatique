\exer{}
\setcounter{numques}{0}

Le but de cet exercice est de construire la suite ordonnée des entiers
de la forme $2^p3^q5^r$, où $p$, $q$ et $r$ sont des entiers naturels.\\
Cette suite commence par :\\


\begin{tabular}{|l|c|c|c|c|c|c|c|c|c|c|c|c|c|c|c|c|}
 \hline
 valeurs&\bf 1&\bf 2&\bf 3&\bf 4&\bf 5&\bf 6&\bf 8&\bf 9&\bf 10&\bf
12&\bf 15&\bf 16&&&&\\ \hline
 \it indice&\it 1&\it 2&\it 3&\it 4&\it 5&\it 6&\it
7&\it 8&\it 9&\it 10&\it 11&\it 12&\it 13&\it 14&\it 15&\it 16 \rm\\ 
 \hline
\multicolumn{1}{c}{} &\multicolumn{1}{c}{}
&\multicolumn{1}{c}{} &\multicolumn{1}{c}{}
&\multicolumn{1}{c}{$\uparrow$}&\multicolumn{1}{c}{}
&\multicolumn{1}{c}{$\uparrow$}&\multicolumn{1}{c}{}
&\multicolumn{1}{c}{$\uparrow$} &\multicolumn{1}{c}{}&\multicolumn{1}{c}{}
&\multicolumn{1}{c}{}
&\multicolumn{1}{c}{} &\multicolumn{1}{c}{} &\multicolumn{1}{c}{}
&\multicolumn{1}{c}{} &\multicolumn{1}{c}{} \\
 \multicolumn{1}{c}{} &\multicolumn{1}{c}{} &\multicolumn{1}{c}{}
&\multicolumn{1}{c}{} &\multicolumn{1}{c}{$i_5$}&\multicolumn{1}{c}{}
&\multicolumn{1}{c}{$i_3$}&\multicolumn{1}{c}{}
&\multicolumn{1}{c}{$i_2$}&\multicolumn{1}{c}{} &\multicolumn{1}{c}{}
&\multicolumn{1}{c}{}
&\multicolumn{1}{c}{} &\multicolumn{1}{c}{} &\multicolumn{1}{c}{}
&\multicolumn{1}{c}{} &\multicolumn{1}{c}{} 
 \end{tabular}
\\

\noindent\bf Principe :\rm\\

Le tableau \texttt{t} étant en cours de construction, l'élément suivant du
tableau sera à choisir parmi les nombres suivants : \texttt{2*t[i\_2]},
\texttt{3*t[i\_3]} et \texttt{5*t[i\_5]}, où $i_2$ (respectivement $i_3$,
$i_5$) est l'indice du plus petit élément du tableau n'ayant pas son double
(respectivement triple, quintuple) présent dans le tableau.\\
Dans l'exemple, $i_2=8$, $i_3=6$ et $i_5=4$, et l'élément suivant est donc à
choisir parmi 2*9=18, 3*6=18 et 5*4=20.\\
Le plus petit élément étant 18, c'est l'élément à rajouter au tableau. Les
indices concernés par le choix (ici $i_2$ et $i_3$) sont à augmenter de 1, ce
qui donne le tableau suivant :\\

\begin{tabular}{|l|c|c|c|c|c|c|c|c|c|c|c|c|c|c|c|c|}
 \hline
 valeurs&\bf 1&\bf 2&\bf 3&\bf 4&\bf 5&\bf 6&\bf 8&\bf 9&\bf 10&\bf
12&\bf 15&\bf 16&\bf 18&&&\\ \hline
 \it indice&\it 1&\it 2&\it 3&\it 4&\it 5&\it 6&\it
7&\it 8&\it 9&\it 10&\it 11&\it 12&\it 13&\it 14&\it 15&\it 16 \rm\\ 
 \hline
\multicolumn{1}{c}{} &\multicolumn{1}{c}{}
&\multicolumn{1}{c}{} &\multicolumn{1}{c}{}
&\multicolumn{1}{c}{$\uparrow$}&\multicolumn{1}{c}{}
&\multicolumn{1}{c}{} &\multicolumn{1}{c}{$\uparrow$}&\multicolumn{1}{c}{}
&\multicolumn{1}{c}{$\uparrow$}&\multicolumn{1}{c}{} &\multicolumn{1}{c}{}
&\multicolumn{1}{c}{} &\multicolumn{1}{c}{} &\multicolumn{1}{c}{}
&\multicolumn{1}{c}{} &\multicolumn{1}{c}{} \\
 \multicolumn{1}{c}{} &\multicolumn{1}{c}{} &\multicolumn{1}{c}{}
&\multicolumn{1}{c}{} &\multicolumn{1}{c}{$i_5$}&\multicolumn{1}{c}{}
&\multicolumn{1}{c}{} &\multicolumn{1}{c}{$i_3$}&\multicolumn{1}{c}{}
&\multicolumn{1}{c}{$i_2$}&\multicolumn{1}{c}{} &\multicolumn{1}{c}{}
&\multicolumn{1}{c}{} &\multicolumn{1}{c}{} &\multicolumn{1}{c}{}
&\multicolumn{1}{c}{} &\multicolumn{1}{c}{} 
 \end{tabular}
\\



\begin{enumerate}
\item Écrire, selon ce principe, la fonction qui construit le tableau des $n$
premiers nombres de la forme $2^p3^q5^r$.
\item Donner l'invariant de boucle et la démonstration de ce théorème.
\end{enumerate}