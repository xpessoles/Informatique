\exer{}
\setcounter{numques}{0}

Le but de cet exercice est de trier un tableau \texttt{t} dont les éléments ne
peuvent prendre qu'un ``petit'' nombre de valeurs (ici, trois valeurs : 0, 1 ou
2).\\
Par exemple, à partir du tableau initial :\\

\begin{tabular}{|l|c|c|c|c|c|c|c|c|c|c|c|c|c|}
 \hline
 valeurs&\bf 2&\bf 0&\bf 1&\bf 0&\bf 2&\bf 1&\bf 1&\bf 0&\bf 1&\bf
2&\bf 2&\bf 1&\bf 0\\ \hline
 \it indice&\it 1&\it 2&\it 3&\it 4&\it 5&\it 6&\it
7&\it 8&\it 9&\it 10&\it 11&\it 12&\it 13\\\hline
 \end{tabular}
 \\
 
il faut obtenir le tableau final :\\

\begin{tabular}{|l|c|c|c|c|c|c|c|c|c|c|c|c|c|}
 \hline
 valeurs&\bf 0&\bf 0&\bf 0&\bf 0&\bf 1&\bf 1&\bf 1&\bf 1&\bf 1&\bf
2&\bf 2&\bf 2&\bf 2\\ \hline
 \it indice&\it 1&\it 2&\it 3&\it 4&\it 5&\it 6&\it
7&\it 8&\it 9&\it 10&\it 11&\it 12&\it 13\\\hline
 \end{tabular}
 \\
 
 Le programme sera itératif (i.e. on utilisera une boucle \texttt{for}). Pour
trouver comment écrire le corps de la boucle, on suppose que le tableau
\texttt{t} (de taille $n$) a été traité jusqu'au rang $i$, et que sont connus
$j$ et $k$ vérifiant :\\
\begin{itemize}
 \item tous les éléments du tableau d'indice inférieur ou égal à $j$ sont égaux
à 0 ;
\item tous les éléments du tableau d'indice supérieur strictement à $j$
et inférieur ou égal à $j$ sont égaux à 1 ;
\item tous les éléments du tableau d'indice supérieur strictement à $k$
sont égaux à 2.\\
\end{itemize}

\begin{tabular}{|ccc|ccc|ccc|c|c|}
 \hline 0&$\cdots$&0& 1&$\cdots$&1& 2&$\cdots$&2&x&$\cdots$\\
 \hline \multicolumn{1}{c}{$1$}& \multicolumn{1}{c}{}& \multicolumn{1}{c}{$j$}&
\multicolumn{1}{c}{$j+1$}& \multicolumn{1}{c}{}& \multicolumn{1}{c}{$k$}&
\multicolumn{1}{c}{$k+1$}& \multicolumn{1}{c}{}& \multicolumn{1}{c}{$i$}&
\multicolumn{1}{c}{$i+1$}& \multicolumn{1}{r}{$n$}
\end{tabular}
\\

Le prochain élément du tableau à traiter est \texttt{t[i+1]}, noté \texttt{x}.\\
$\bullet$ En supposant $1\leq j<k<i<n$ (c'est-à-dire qu'il y a au moins un 0, un
1 et un 2 placés), quelles sont les modifications à faire sur \texttt{t}, $i$ et
$j$ si $x=2$ ? si $x=1$ ? si $x=0$ ?\\

Il se peut qu'il n'y ait pas encore de 0, 1 ou 2 dans la partie de tableau déjà
triée. Il faut en tenir compte dans les modifications de \texttt{t}, $i$ et $j$
selon les valeurs de $x$.\\

\noindent$\bullet$  Écrire le traitement complet du tri.