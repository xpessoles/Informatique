\exer{}
\setcounter{numques}{0}

\question{} En entrée de boucle, on a toujours $m=0$ et le tableau $t[i:i+1]$ n'a qu'un élément, donc aucune ascensions, donc l'invariant est vérifié en entrée de boucle. 

Supposons que cette propriété soit vraie au début d'un tour de boucle. 
\begin{itemize}
  \item Si $t[j]  \leq t[i]$, alors il n'y a pas d'ascension entre $i$ et $j$. Les ascensions de $t[i:j+1]$ sont exactement les mêmes que celles de $t[i:j]$, donc si $\texttt{m}>0$, alors \texttt{m} est la hauteur de la plus haute ascension de \texttt{t[i:j+1]}, sinon $m=0$ 
  \item Si $t[j]  > t[i]$, alors il y a une ascension en plus dans $t[i:j+1]$ que dans $t[i:j]$ : c'est celle entre $i$ et $j$. 
    \begin{itemize}
      \item[\textbullet] Si $t[j]-t[i] > m$, au début de la ligne 10, $t[j]-t[i]$ est donc la hauteur de la plus grande ascension de $t[i:j+1]$. À la fin de la ligne 10, $m$ a alors pour valeur  $t[j]-t[i]$
      \item[\textbullet] Sinon, au début de la ligne 10, \texttt{m} est la hauteur de la plus haute ascension de \texttt{t[i:j+1]}, sinon $m=0$.
    \end{itemize}
    Dans tous les cas, à la fin de la ligne 10, \texttt{m} est la hauteur de la plus haute ascension de \texttt{t[i:j+1]}, sinon $m=0$.
\end{itemize}
Ainsi, cette propriété est vraie à la fin de ce tour de boucle. 

\fbox{C'est donc bien un invariant pour cette boucle for.}


\question{}
Exactement de la même manière, \og \texttt{M} est la hauteur de la plus haute ascension de \texttt{t[:i]} \fg{} est un invariant pour la boucle  for portant des lignes n°6 à 12.

En sortie de boucle, on a $i = n$, donc \texttt{M} est la hauteur de la plus haute ascension de \texttt{t}.

\fbox{%
\begin{minipage}{0.9\textwidth}%
  Un appel de la fonction \texttt{plus\_haute\_ascension(t)} renvoie bien la hauteur de la plus haute ascension de \texttt{t}.
\end{minipage}%
}


\question{}
\begin{lstlisting}
def nb_ascensions(t):
    """Nombre d'ascensions du tableau t.
    Préconditions : t est un tableau de nombres."""
    n = len(t)
    nb = 0
    for i in range(n):
        for j in range(i+1,n):
            if t[j] > t[i]:
                nb = nb + 1
    return nb
\end{lstlisting}
