\exer{}
\setcounter{numques}{0}



\question{Définir une fonction \texttt{Python}, \texttt{creationgrille(p, n)} à deux
paramètres : un nombre réel $p$ (qu'on supposera dans l'intervalle
$[0,1[$ et un entier naturel $n$, qui renvoie un tableau
$(n,n)$ dans lequel chaque case sera ouverte avec la probabilité
$p$ et fermée sinon.}

\begin{lstlisting}
def creation_grille(p, n):
    grille = np.zeros((n, n))
    for i in range(n):
        for j in range(n):
            if rand() < p:
                grille[i][j] = 1.
    return grille
\end{lstlisting}


\question{Écrire une fonction \texttt{afficher_grille(grille,nom_de_fichier)} qui prend en argument une variable grille qui correspond à une grille de percolation générée précédemment et ne renvoyant rien mais enregistrant dans \texttt{nom_de_fichier} le graphe obtenu. On pourra exporter une grille de $10\times 10$ cases avec l'échelle suggérée précédemment, l'enregistrer sous le nom "tp09$\_$q02$\_$vos$\_$noms.png" et l'envoyer à votre professeur.}

\begin{lstlisting}
echelle = ListedColormap(['black', 'aqua', 'white'])
def afficher_grille(grille,nom_de_fichier):
    plt.matshow(grille,cmap=echelle)
    plt.colorbar()
    plt.savefig(nom_de_fichier)
    return None
\end{lstlisting}



\question{Écrire une fonction \texttt{percolation(grille)} qui prend en argument une
grille et qui remplit de fluide celle-ci, en appliquant l'algorithme
exposé ci-dessous :
\begin{enumerate}
\item
  Créer une liste contenant initialement les coordonnées des cases
  ouvertes de la première ligne de la grille et remplir ces cases de
  liquide.
\item
  Puis, tant que cette liste n'est pas vide, effectuer les opérations suivantes :
  \begin{enumerate}
  \item extraire de cette liste les coordonnées d'une case quelconque ;
 \item ajouter à la liste les coordonnées des cases voisines qui sont encore
vides, et les remplir de liquide.
  \end{enumerate}
\end{enumerate}


L'algorithme se termine quand la liste est vide.

}

\begin{lstlisting}
def percolation(grille):
    n, p = grille.shape
    lst = []
    for j in range(p):
        if grille[0][j] == 1.: # les cases vides de la première ligne
            grille[0][j] = .5 # sont remplies et ajoutées à lst
            lst.append((0, j))
    while len(lst) > 0:
        (i, j) = lst.pop() # une case est extraite de lst
        if i > 0 and grille[i-1][j] == 1.: # si le voisin haut est vide, il est rempli
            grille[i-1][j] = .5
            lst.append((i-1, j))
        if i < n-1 and grille[i+1][j] == 1.: # si le voisin bas est vide, il est rempli
            grille[i+1][j] = .5
            lst.append((i+1, j))
        if j > 0 and grille[i][j-1] == 1.: # si le voisin gauche est vide, il est rempli
            grille[i][j-1] = .5
            lst.append((i, j-1))
        if j < p - 1 and grille[i][j+1] == 1.: # si le voisin droit est vide, il est rempli
            grille[i][j+1] = .5
            lst.append((i, j+1))

\end{lstlisting}


\question{Rédiger un script vous permettant de visualiser une grille avant et
après remplissage, et faire l'expérience avec quelques valeurs de
$p$ pour une grille de taille raisonnable (commencer avec $n
= 10$ pour vérifier visuellement que votre algorithme est correct, puis
augmenter la taille de la grille, par exemple avec $n = 64$). On pourra exporter et l'enregistrer sous le nom "tp09$\_$q04$\_$vos$\_$noms.png" et l'envoyer à votre professeur.}

\begin{lstlisting}
grille = creation_grille(.61, 64)
grille_percolee = grille.copy()
percolation(grille_percolee)
fig1 = plt.matshow(grille, cmap=echelle)
plt.axis('off')
fig2 = plt.matshow(grille_percolee, cmap=echelle)
plt.axis('off')
plt.savefig('tp09_Q04_durif.png')
\end{lstlisting}



\question{Écrire une fonction \texttt{teste_percolation(p,n)} qui prend en argument
un réel $p\in\left[0,1\right[$ et un entier $n\in \mathbb{N}^*$, crée une grille, effectue la percolation et
retourne :
\begin{itemize}
\item \texttt{True} lorsque la percolation est réussie, c'est-à-dire lorsque le bas
  de la grille est atteint par le fluide ;
\item \texttt{False} dans le cas contraire.
\end{itemize}
}

\begin{lstlisting}
def teste_percolation(p, n):
    grille = creation_grille(p, n)
    percolation(grille)
    for j in range(n):
        if grille[n-1][j] == .5:
            return(True)
    return(False)
\end{lstlisting}


\question{Rédiger la fonction \texttt{proba(p,k,n)} qui prend en argument le nombre d'essai $k$, la variable $p$ ainsi que le nombre de cases $n$ sur la largeur de la grille et qui renvoie $P(p)$.}

\begin{lstlisting}
def proba(p, k=20, n=128):
    s = 0
    for i in range(k):
        if teste_percolation(p, n):
            s += 1
    return s/k
\end{lstlisting}

\question{Ecrire une fonction \texttt{tracer_proba(n,nom_de_fichier)} qui prendre en argument une taille $n$  ne renvoyant rien mais enregistrant dans \texttt{nom_de_fichier} le graphe obtenu. On pourra traiter le cas d'une grille de $128\times 128$ cases et enregistrer la figure obtenue sous le nom "tp09$\_$q07$\_$vos$\_$noms.png" et l'envoyer à votre professeur.}

\begin{lstlisting}
def tracer_proba(n,nom_de_fichier):
    x=np.linspace(0,1,21)
    y=[]
    for p in x:
        y.append(proba(p,20,n))
    plt.clf()
    plt.plot(x,y)
    plt.savefig(nom_de_fichier)
    return None
\end{lstlisting}


%
%P(\emph{p})
%\end{quote}
%
%1
%
%\begin{quote}
%\emph{p}
%
%\emph{p}0 1
%
%Figure 3 -- \emph{L'allure théorique du graphe de la fonction} P\emph{.}
%\end{quote}
%
%
%Pour déterminer le seuil critique $p_0$, on peut procéder à une recherche dichotomique (assez grossière) en convenant que si
%$P(p) < 0,4$ alors $p < p_0$ et si $P(p) > 0,6$ alors $p > p_0$
%\question{ (Facultative) En utilisant une recherche dichotomique, chercher à estimer le plus précisément possible la valeur numérique du seuil $p_0$.}
%\end{enumerate}
%
%\section{Propriétés
%macroscopiques}\label{propriuxe9tuxe9s-macroscopiques}
%
%\begin{quote}
%A l'instar de la thermodynamique, on peut décrire le comportement d'un
%système lors d'une transition de phase en introduisant des propriétés
%macroscopiques. Dans le cas de la percolation, on peut par exemple
%définir la densité moyenne \emph{d}(\emph{p}) de cases ouvertes
%atteintes par le fluide.
%
%Question 5.
%\end{quote}
%
%\begin{enumerate}
%\def\labelenumi{\alph{enumi})}
%\item
%  Définir une fonction densite qui prend en argument une grille dans
%  laquelle la percolation a eu lieu et qui retourne la valeur de sa
%  densité.
%\item
%  À l'aide d'un nombre raisonnable d'échantillons, définir alors la
%  fonction d qui à une probabilité \emph{p} ∈ {[}0\emph{,} 1{[} associe
%  la densité moyenne de la percolation dans une grille 128 × 128.
%\item
%  Tracer le graphe de \emph{d}(\emph{p}) pour \emph{p} ∈ {[}0\emph{,}
%  1{[}.
%\end{enumerate}
%
%\section{Et pour les plus rapides}\label{et-pour-les-plus-rapides}
%
%\begin{quote}
%Recommencez toute cette étude, mais cette fois-ci avec une grille
%hexagonale. Il est possible de prouver que pour une grille hexagonale
%carrée le seuil critique \emph{p}\textsubscript{0} est exactement égal à
%\emph{p}\textsubscript{0} = 1\emph{/}2 ; le vérifier expérimentalement
%et chercher à le démontrer (en utilisant un argument de symétrie).
%\end{quote}

\end{document}
