\exer{}
\setcounter{numques}{0}

Dans un tableau $t$ contenant $2N$ ou $2N+1$ nombres, une médiane est un réel $m$ tel que, si l'on classe $t$ par ordre croissant,
\begin{enumerate}
  \item $m$ est supérieur ou égal aux $N$ premiers nombres de $t$ ;
  \item $m$ est inférieur ou égal aux $N$ derniers nombres de $t$.
\end{enumerate}


\begin{exemple}
  \begin{itemize}
    \item[\textbullet] La seule médiane de $[1,3,2]$ est $2$.
    \item[\textbullet] La seule médiane de $[5,1,2,5,5]$ est $5$. 
    \item[\textbullet] La seule médiane de $[0,1,0,0]$ est $0$. 
    \item[\textbullet] L'ensemble des médianes de $[-3,4,1,-1]$ est l'intervalle $]-1,1[$.
  \end{itemize}
\end{exemple}

\question \'Ecrire une fonction \texttt{mediane(t)} qui, à un tableau de nombres \texttt{t}, renvoie une médiane de \texttt{t}.

\emph{On pourra s'inspirer de l'une des deux idées suivantes.
\begin{itemize}
  \item Copier \texttt{t}, lui ôter tant que possible son maximum et son minimum.
  \item Trier \texttt{t} par ordre croissant.
\end{itemize}
}