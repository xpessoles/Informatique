\exer{}
\setcounter{numques}{0}

Dans un tableau de nombres $t = [t_0,\dots,t_{n-1}]$ de longueur $n$, la position $1 \leq i < n-1$ est dit \emph{sous l'eau} s'il existe $k \in \ii{0,i}$ et $\ell \in \ii{i+1,n}$ tels que 
\begin{equation*}
  t_k > t_i \quad\textrm{et}\quad t_\ell > t_i. 
\end{equation*}

\question{} Écrire une fonction \texttt{nb\_sousleau(t)} prenant en argument un tableau de nombres $t$ et renvoyant le nombres d'indices sous l'eau de ce tableau. 