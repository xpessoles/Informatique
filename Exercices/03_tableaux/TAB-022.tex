\exer{}
\setcounter{numques}{0}

On considère la fonction suivante.
\begin{lstlisting}%[gobble=0,numbers=left]
def mystere(n):
    """Préconditions : n entier positif"""
    L = []
    c = 0 
    while c ** 2 <= n : 
        L.append(c**2)
        c = c+1
    return L
\end{lstlisting}

\bigskip{}

\question{} Montrer que, si \texttt{n} est un entier, un  appel de la fonction \texttt{mystere(n)} renvoie un résultat, à l'aide d'un variant. 

\medskip{}

\question{} Que renvoie un appel de la fonction \texttt{mystere(n)} ? On justifiera la réponse, à l'aide notamment d'un invariant. 