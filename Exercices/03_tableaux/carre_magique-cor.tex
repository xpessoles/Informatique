\question{}

\renewcommand{\arraystretch}{1.2}
		\begin{tabular}{|*{5}{c|}}
			\hline
			$15$ & $22$ & $9$ & $16$ &$~3~$\\
			\hline
			$~2~$& $14$ & $21$ & $8$ & $20$ \\
			\hline
			 $19$  &$~1~$& $13$  & $25$  & $7$   \\
			\hline
			$~6~$& $18$ &$~5~$& $12$ & $24$ \\
			\hline
			 $23$ & $10$ & $17$ &$~4~$& $11$ \\
			\hline
		\end{tabular}
\renewcommand{\arraystretch}{1}

On vérifie les trois propriétés d'un carré magique avec La somme de chaque colonne, la somme de chaque et la somme de chaque diagonale est égale à la densité $d=65$.

\question{}

\begin{lstlisting}
def Carre_vide(n):
    if n%2==0:
        print("Erreur n doit être impair")
        return None
    else:
        carre=[]
        for i in range(n):
            carre.append([0]*n)
        return carre
\end{lstlisting}


\question{}

\begin{lstlisting}
def Remplir_carre(carre):
    n=len(carre)
    carre_magique=deepcopy(carre)
    x,y=(n-1)//2-1,(n-1)//2
    for i in range(1,n**2+1):
        carre_magique[y][x]=i
        print(carre_magique)
        if i%(n)==0:
            x=(x-2)%n
        else:
            x,y=(x-1)%n,(y-1)%n
    return carre_magique
\end{lstlisting}


\question{}

On vérifie les trois propriétés d'un carré magique avec les sommes suivantes égales à la densité :
\begin{itemize}
\item La somme de chaque colonne
\item La somme de chaque ligne
\item La somme de chaque diagonale
\end{itemize}

\begin{lstlisting}
def Verif_carre(carre_magique):
    n=len(carre)
    dens=int(n*(n**2 + 1)/2)
    for i in range(n):
        d1=0
        d2=0
        for j in range(n):
            d1+=carre_magique[j][i]
            d2+=carre_magique[i][j]
        if d1!=dens or d2!=dens:
            return False
    d1=0
    d2=0
    for i in range(n):
        d1+=carre_magique[i][i]
        d2+=carre_magique[n-1-i][i]
    if d1!=dens or d2!=dens:
        return False
    return True
\end{lstlisting}