\exer{}
\setcounter{numques}{0}

\begin{itemize}
\item Ce devoir est à faire de façon individuelle.
\item Utilisez Python, version 3.
\item Créez un répertoire pour faire ce DS.
\item Allez sur le site de la classe dans la rubrique Info/DS Tronc Commun.
\item Recopiez le fichier
  \texttt{ds.py} dans ce répertoire, ainsi
  que le fichier \texttt{zeta5.txt}.
  Vous aurez besoin de ces fichiers pour le DS mais vous ne devrez pas
 modifier que le fichier \texttt{ds.py} sans modifier les instructions déjà données (sous peine de risquer d'avoir tout faux).
\item Avec votre IDE (Pyzo ou IDLE) ouvrir le script \texttt{ds.py}. 
%  Dans \texttt{ds.py}, faites un \texttt{import sujet}
%  pour pouvoir utiliser les fonctions fournies dans \texttt{sujet.py}
\item On ne vous demande pas de rendre votre programme mais seulement
  de répondre aux questions posées, dont les réponses sont toutes
  numériques, \textbf{sur le formulaire de réponse joint}.
\item \textbf{Attention: toutes les questions posées dépendent d'un
  paramètre $\alpha$, qui vous est donné sur le formulaire de réponse. La valeur de
  $\alpha$ est différente pour chacun d'entre vous. Soyez attentif à
  sa valeur: s'il est faux toutes vos réponses seront fausses.}
  Dans toute la suite, on considère que la variable Python \texttt{alpha}
  contient la valeur de votre paramètre $\alpha$. Il vous est donc
  conseillé, au début de \texttt{ds.py}, de faire un \texttt{alpha =
    $\alpha$}.
\item Lorsque la réponse demandée est un réel, on attend que l'écart
  entre la réponse que vous donnez et la vraie valeur soit strictement
  inférieur à $10^{-4}$. Donnez donc des valeurs avec $5$ chiffres après la
  virgule.
%\item Vous avez le droit d'utiliser toutes les fonctions prédéfinies
%  de Python que vous voulez, en particulier celles définies dans les
%  modules \texttt{scipy.optimize} et \texttt{scipy.integrate}.
\end{itemize}


\section*{Analyse de tableaux}

\subsection*{Première partie}


Après avoir exécuter votre script \texttt{ds.py} (contenant la définition de la variable \texttt{alpha} avec votre valeur de $\alpha$)  exécutez \texttt{t = cree\_tableau(alpha)}.


\question{
  Quelle est la longueur du tableau \texttt{t}? Par la suite, on la
  note $N$.
}

\question{
  Combien d'éléments de \texttt{t} sont supérieurs ou égaux à $3000$?
}

\question{
  Combien d'éléments de ce tableau sont divisibles par $3$?
}

\question{
  Quel est le nombre de couples $(i,j)$ tels que $0\leq i < j < N$
  et \texttt{t[$i$]} $<$ \texttt{t[$j$]}?
}

On définit la suite $u$ par
\begin{align*}
\forall n\in \mathbb{N} \ u_{n+1} &= r(15091\cdot u_{n},64007)\\
\text{et }  u_{0} & = 10 + \alpha
\end{align*}
où $r(a,d)$ désigne le reste de la division euclidienne de $a$ par
$d$ (en python, on utilise l'opérateur \texttt{\%} pour calculer ce reste).

Construire un tableau \texttt{U} de taille $10^{4}$ tel que pour tout $i\in
\ii{0,10^{4}-1}$, \texttt{U[$i$]} contienne $u_{i}$.

\question{
  Que vaut $u_{42}$?
}

\question{
  Que vaut le dernier élément du tableau \texttt{U}?
}

%\question{
%  Combien d'éléments du tableau \texttt{U} sont-ils divisibles par $17$?
%}

\question{
  Quel est le nombre d'indices $i\in \ii{0,10^{4}-2}$ tels que
  $\abs{u_{i}-u_{i+1}}\leq 1000$?
}

\question{
  Quelle est la somme des valeurs de ce tableau?
}

\subsection{Autour de $\zeta(5)$}

Dans cette partie, on utilise le fichier \texttt{zeta5.txt}. Celui-ci
contient un million de décimales (après la virgule) de $\zeta(5) =
\sum_{n=1}^{+\infty} \frac{1}{n^{5}} \approx 1,0369$.

Voici les premières lignes du fichier (à gauche, on a noté en petit
les numéros des lignes pour pouvoir en parler; attention: \textbf{ils n'apparaissent pas
dans le fichier}):
\begin{lstlisting}%[numbers=left]
zeta5_huvent with 1000000 digits : 
zeta5_huvent = 1.
0369277551 4336992633 1365486457 0341680570 8091950191  : 50
2811974192 6779038035 8978628148 4560043106 5571333363  : 100
7962034146 6556609042 8009617791 5597084183 5110721800  : 150
8764486628 6337180353 5983639623 6512888898 1335276775  : 200
2398275032 0224368457 6644466595 8115993917 9777450392  : 250
4464391966 6615966401 6205325205 0215192267 1351256785  : 300
9748692860 1974479843 2006726812 9753091990 0774656558  : 350
6015265737 3003756153 2683149897 9719350398 3785813199  : 400
2288488642 5335104251 6025108499 0434640294 1172432757  : 450
6341508162 3322456186 4992714427 2264614113 0075808683  : 500

1691649791 8137769672 5145590158 0353093836 2260020230  : 550
4558560981 5265536062 6530883832 6130378691 7412255256  : 600
0507375081 3917876046 9541867836 6657122379 6259477937  : 650
8931344280 5560465115 0585291073 6964334642 8934143397  : 700
5231743713 3962434331 1485731093 6262213535 7253048207  : 750
\end{lstlisting}

À partir  de la  ligne 3,  les décimales sont  rangées par  paquets de
$10$, eux-mêmes rangés par lignes  de $5$ paquets, elles-mêmes rangées
dans des blocs de 10 lignes (qui contiennent donc $500$ décimales ; il
y a ainsi $2000$ blocs).  Attention: il y a des lignes vides (la ligne
13 sur cet  extrait puis de manière générale, tous  les paquets de 500
décimales). On va faire des statistiques sur le nombre d'apparitions de
$2000 + \alpha$ dans l'écriture  décimale de $\zeta(5)$.  Par exemple,
$92$ apparaît deux fois dans les $20$ premières décimales. Plus subtil:
$70$ apparaît  deux fois dans  les $40$ premières décimales,  dont une
fois  tronqué par  un  espace.  On peut  même  trouver des  occurrences
coincées entre deux lignes ($1004$,  en position $6199$), voire entre
deux blocs ($1039$, en position $15498$).

Au besoin, vous pouvez ouvrir ce fichier avec n'importe quel éditeur
de texte.

\question{
Donner le nombre d'occurrences de $2000 + \alpha$ dans les paquets de
$10$ décimales.
On pourra appliquer un algorithme du type: \og{}Pour chaque ligne $l$,
on casse $l$ selon les espaces. S’il y a le bon nombre de blocs,
alors pour chacun des blocs $b$, et pour chaque $i$ entre $0$ et $6$
(inclus), on regarde
si $b[i : i + 4]$ vaut la chaîne représentant $2000 +\alpha$\fg{}.
}

\question{
  Donner le nombre d'occurrences de $2000 + \alpha$ dans les paquets de
$50$ décimales obtenus en concaténant, pour chaque ligne, les $5$
paquets de $10$ décimales.
}

\question{
  Donner le nombre d'occurrences de $2000 + \alpha$ dans les paquets de
$500$ décimales obtenus en concaténant, pour chaque bloc de $10$
lignes, les $50$ paquets de $10$ décimales.
}

\question{
  Donner le nombre d'occurrences de $2000 + \alpha$ dans les $10^{6}$
  premières décimales de $\zeta(5)$.
}