\question\

\begin{Verbatim}[gobble=0,numbers=left]
def maxi(T):
    """INdice d'un maximum de T, T : tableau de nombres """
    m = 0
    for i in range(1,len(T)) : 
        if T[i] > T[m] :
            m = i
    return m 
\end{Verbatim}

\question\ Montrons qu'un invariant pour la boucle for des lignes 4 à 6 est : $P_i$ : \og \texttt{T[m]} est le maximum 
de \texttt{T[:i]} \fg. 

Au début du premier tour de boucle (initialisation), $i = 1$, $T[:i] = [T[0]]$ et $T[m] = T[0]$, donc l'invariant est  
bien vérifié. 

Au début du tour de boucle \no$i$, supposons que $P_i$ est vérifié. Alors, $T[m] = \max(T[:i])$. Si $T[i] > T[m]$, 
alors $T[i] = \max(T[:i+1])$ et sinon $T[m] = \max(T[:i])$. 
Ainsi, à la fin de ce tour de boucle, on a bien $T[m] = \max(T[:i+1])$. 

Au dernier tour de cette boucle, $i = \mathrm{len}(T)-1$, donc $T[m]$ est le maximum de $T$.

Cette fonction renvoie $m$, qui est bien un indice du maximum de $T$. 

\question\ Notons $n$ la longueur de $T$. 

On se place dans le modèle de complexité usuelle : les affectations, accès à un élément d'une liste et comparaisons de nombres se font en $O(1)$. 

La ligne 3 s'effectue en $O(1)$. 

Chaque tour de la boucle for (lignes 5 et 6) s'effectue aussi en $O(1)$. Il y a au plus $n-1$ tour de boucles, donc cette boucle for s'effectue en $O(n)$. 

La complexité de la fonction \texttt{maxi} est donc en $O(1)+O(n) = O(n)$. 