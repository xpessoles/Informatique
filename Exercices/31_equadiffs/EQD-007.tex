\exer{[EQD-007]}
\setcounter{numques}{0}~\\

\section{Pour commencer}

Avant toute chose, on prendra soin de recopier la fonction \texttt{euler(F,a,b,y0,h)} donnée dans le chapitre 11 du cours. 

\section{Oscillations libres d'un pendule}

On considère les oscillations libres d'un pendule. L'angle que fait ce pendule avec la verticale au temps $t$ sera noté $\theta(t)$ et l'on étudie les oscillations du pendule entre $0$ et $10$ secondes.
L'équation vérifiée par la fonction $\theta$ est alors 
\begin{equation}\label{eq:pendule}\tag{$\mathscr{P}$}
  \ddot{\theta} + \alpha\dot{\theta}+\omega_0^2\sin\theta = 0,
\end{equation}
où
\begin{itemize}
  \item $\alpha$  (proportionnel au coefficient de frottements) est exprimé en $s^{-1}$ ;
  \item $\omega_0$ (la pulsation propre) est exprimé en $s^{-1}$ . 
\end{itemize}
Si le coefficient de frottement est nul, l'équation devient 
\begin{equation}\label{eq:pendulesf}\tag{$\mathscr{P}_{sf}$}
  \ddot{\theta} + \omega_0^2\sin\theta = 0.
\end{equation}
Enfin, on sait qu'au voisinage de $0$ on a $\sin\theta = \theta + o(\theta^2)$. En supposant que les oscillations sont petites, on approche alors la dernière équation par 
\begin{equation}\label{eq:pendulepo}\tag{$\mathscr{P}_{po}$}
  \ddot{\theta} + \omega_0^2\theta = 0.
\end{equation}

\question\ Pour quelle variable $\Theta$ et quelles fonctions $F$, $F_{sf}$ et $F_{po}$ les équations respectives \eqref{eq:pendule}, \eqref{eq:pendulesf} et \eqref{eq:pendulepo} sont-elles équivalentes aux équations suivantes ?
\begin{equation*}
  \dot{\Theta} = F(\Theta, t),\quad \dot{\Theta} = F_{sf}(\Theta, t), \quad \dot{\Theta} = F_{po}(\Theta, t).
\end{equation*}

\subsection{Approximation des petites oscillations par la méthode d'Euler}

On suppose les oscillations petites et sans frottements, on prendra donc les valeurs numériques suivantes : $\alpha = 0 s^{-1}$, $\omega_0 = \sqrt{2\pi} s^{-1}$. 
On étudiera les solutions sur un intervalle de temps de $10$ secondes et l'on supposera que $\dot{\theta}(0) = 0s^{-1}$. 
Notamment, une méthode numérique effectuée avec $n$ segments correspondra à un pas de $\dfrac{10}{n}$ secondes. 

\question{} Résoudre littéralement l'équation \eqref{eq:pendulepo}.

\medskip

\question\ \'Ecrire une fonction \texttt{Fpo(Theta,t)} prenant en argument un vecteur \texttt{Theta} et un nombre \texttt{t} et renvoyant $F_{po}(\texttt{Theta},\texttt{t})$.

\medskip

\question{\label{qu:tracepo}} \'Ecrire une fonction \texttt{trace\_po(th0,n,nom\_de\_fichier)} enregistrant dans \texttt{nom\_de\_fichier} le tracé de la solution exacte de \eqref{eq:pendulepo} ($\theta(0) = \texttt{th0}$, $\dot{\theta}(0) = 0s^{-1}$) ainsi que, sur le même graphe, celui de la solution obtenue par la méthode d'Euler avec \texttt{n} segments.  

Vous enverrez à l'enseignant la figure produite pour $\theta_0 = 10^{-1}$ et $n=10^2$. 

\subsection{Facultatif : approximation des petites oscillations par la méthode de Runge-Kutta d'ordre 4}

\question{} Sur le modèle de la fonction \texttt{euler(F,a,b,y0,pas)}, écrire une fonction Python \texttt{rk4(F,a,b,y0,pas)} mettant en {\oe}uvre la méthode de Runge-Kutta d'ordre 4 pour la fonction \texttt{F}, sur le segment $[a,b]$, avec un pas \texttt{pas} et la condition initiale \texttt{y0}. 
  Cette fonction renverra un couple de tableaux. 
  
\medskip
  
\question{} Compléter la fonction \texttt{trace\_po(th0,n,nom\_de\_fichier)} pour y superposer le tracé obtenu par la méthode de Runge-Kutta.  Que remarque-t-on ?

\subsection{Oscillations sans frottements}

On suppose les oscillations sans frottements, mais pas forcément petites on prendra donc les valeurs numériques suivantes : $\alpha = 0 s^{-1}$, $\omega_0 = \sqrt{2\pi} s^{-1}$.
On étudiera les solutions sur un intervalle de temps de $10$ secondes et l'on supposera que $\dot{\theta}(0) = 0s^{-1}$. 
Notamment, une méthode numérique effectuée avec $n$ points correspondra à un pas de $\dfrac{10}{n}$ secondes. 


\question{} \'Ecrire une fonction \texttt{Fsf(Theta,t)} prenant en argument un vecteur \texttt{Theta} et un nombre \texttt{t} et renvoyant $F_{sf}(\texttt{Theta},\texttt{t})$.

\bigskip{}

On veut maintenant vérifier numériquent l'approximation des petites oscillations. On connait déjà la solution exacte de \eqref{eq:pendulepo}, il ne nous reste plus qu'à obtenir une approximation de la solution de \eqref{eq:pendulesf}. 

\medskip{}

\question{\label{qu:tracesf}} \'Ecrire une fonction \texttt{approx\_po(th0,n,nom\_de\_fichier)} enregistrant dans \texttt{nom\_de\_fichier} le tracé de la solution exacte de \eqref{eq:pendulepo} ($\theta(0) = \texttt{th0}$, $\dot{\theta}(0) = 0s$) ainsi que, sur le même graphe, celui de la solution de \eqref{eq:pendulesf} obtenue par la méthode d'Euler avec \texttt{n} segments.  
Que remarque-t-on ? 

Vous enverrez à l'enseignant la figure produite pour $\theta_0 = 1$ et $n=10^4$. 

\subsection{Facultatif : période des oscillations.}

En dehors de la situation des petites oscillations, le mouvement du pendule est toujours périodique, mais sa période n'est pas $T_0 = \dfrac{2\pi}{\omega_0}$. 
Plus exactement, cette période dépend de la position initiale du pendule (le calcul précis n'est pas à votre portée). Nous allons tracer cette période en fonction de la position initiale du pendule.

Nous allons donc estimer la période de ces oscillations à partir d'une approximation obtenue par une méthode numérique. 

\medskip{}

On appelle \emph{pic} d'un tableau une élément de ce tableau strictement plus grand que ses deux voisins. 
On considérera que le premier et le dernier élément d'un tableau ne sont pas des pics de ce tableau. 
On appelle alors \emph{période} d'un tableau la moyenne des distances entre deux pics consécutifs. 

\medskip{}

\question{} \'Ecrire une fonction \texttt{periode(L)} renvoyant la période du tableau \texttt{L}.

\medskip{}

On estimera la période du pendule simple à partir de la période du tableau des valeurs successives de $\theta$ obtenues par la méthode de Runge-Kutta. 
Plus exactement, si la période de ce tableau est $p$, si ce tableau comporte $m$ valeurs et si les oscillations ont été mesurées sur un intervalle de temps de longueur $T$, alors l'estimation de la pédiode du pendule simple sera 
\begin{equation*}
    T \times \dfrac{p}{m}.
\end{equation*}


\medskip{}

\question{} \'Ecrire une fonction \texttt{periode\_pendule(n)} prenant en argument un entier \texttt{n} et renvoyant le tableau des estimations de la période du pendule, pour les conditions initiales $\dot{\theta}(0) = 0s^{-1}$ et $\theta(0) = \dfrac{k\pi}{2n}$, pour chaque $k \in \iif{1,n}$.

\medskip{}

On réfléchira à l'intervalle de temps utilisé ainsi qu'à la discrétisation utilisée dans la méthode de Runge-Kutta.

\medskip{}

\question{\label{qu:periode}} \'Ecrire une fonction \texttt{trace\_periode(n,nom\_de\_fichier)} prenant en argument un entier \texttt{n} et enregistrant dans \texttt{nom\_de\_fichier} le tracé des points obtenus à la question précédente. 
  On placera en abscisse les $\theta(0) = \dfrac{k\pi}{2n}$ et en ordonnée les périodes estimées.

  Vous enverrez à votre enseignant la figure produite pour $n=100$.
  
\section{Oscillations forcées d'un pendule avec frottements}

On suppose les oscillations forcées, avec frottements. 
La variable $\theta$ suit alors l'équation différentielle suivante.
\begin{equation}\label{eq:penduleforce}\tag{$\mathscr{P}_f$}
  \ddot{\theta} + \alpha\dot{\theta}+\omega_0^2\sin\theta = \cos(\omega t).
\end{equation}
On prendra les valeurs numériques suivantes : $\alpha = 0,5 s^{-1}$, $\omega_0 = \sqrt{2\pi} s^{-1}$ et $\omega = 1s^{-1}$.
On étudiera les solutions sur un intervalle de temps de $10$ secondes. 
Notamment, une méthode numérique effectuée avec $n$ segments correspondra à un pas de $\dfrac{10}{n}$ secondes. 

\question{} Pour quelle variable $\Theta$ et quelle fonction $F_f$ l'équation  \eqref{eq:penduleforce} est-elle équivalente à l'équation suivante ? 
\begin{equation*}
  \dot{\Theta} = F_f(\Theta, t).
\end{equation*}

\question{} \'Ecrire une fonction \texttt{Ff(Theta,t)} prenant en argument un vecteur \texttt{Theta} et un nombre \texttt{t} et renvoyant $F_f(\texttt{Theta},\texttt{t})$.

% \question\label{qu:eulerf} \'Ecrire une fonction \texttt{euler_f(th0,thp0,n)} prenant en argument deux nombres \texttt{th0}, \texttt{thp0} ainsi qu'un entier \texttt{n} et renvoyant le couple de listes  \texttt{t_list,y_list} obtenu en appliquant la méthode d'Euler avec \texttt{n} segments à l'équation \eqref{eq:penduleforce}, avec les conditions initiales  $ \theta(0) = \texttt{th0}$ et $\dot{\theta}(0) = \texttt{thp0}$.

\medskip{}

\question{} \label{qu:trace_trajectoire_f} Écrire une fonction \texttt{trace\_trajectoire\_f(th0,thp0,n,nom\_de\_fichier)} prenant en argument deux nombres \texttt{th0}, \texttt{thp0}, un entier \texttt{n} ainsi qu'une chaîne \texttt{nom\_de\_fichier} et enregistrant dans \texttt{nom\_de\_fichier} la courbe de la trajectoire du pendule (c'est-à-dire des points $(t,\theta(t))$, obtenue en appliquant la méthode d'Euler avec \texttt{n} segments à l'équation \eqref{eq:penduleforce}, avec les conditions initiales  $ \theta(0) = \texttt{th0}$ et $\dot{\theta}(0) = \texttt{thp0}$.

Vous enverrez à l'enseignant les figures produites pour les conditions initiales $\theta(0) = 1$ et $\dot{\theta}(0) = 2$ et pour $n=10^2$. 

\medskip{}

\question{} \label{qu:trace_phase_f} Écrire une fonction \texttt{trace\_phase\_f(th0,thp0,n,nom\_de\_fichier)} prenant en argument deux nombres \texttt{th0}, \texttt{thp0}, un entier \texttt{n} ainsi qu'une chaîne \texttt{nom\_de\_fichier} et enregistrant dans \texttt{nom\_de\_fichier} le portrait de phase du pendule (c'est-à-dire des points $(\theta(t),\dot{\theta}(t))$, obtenu en appliquant la méthode d'Euler avec \texttt{n} segments à l'équation \eqref{eq:penduleforce}, avec les conditions initiales  $ \theta(0) = \texttt{th0}$ et $\dot{\theta}(0) = \texttt{thp0}$.

Vous enverrez à l'enseignant les figures produites pour les conditions initiales $\theta(0) = 1$ et $\dot{\theta}(0) = 2$ et pour $n=10^2$. 


\medskip

\question{\label{qu:odeintf}} \'Ecrire une fonction \texttt{odeint\_f(th0,thp0,n)} prenant en argument deux nombres \texttt{th0} et \texttt{thp0} ainsi qu'un entier \texttt{n} et renvoyant le couple de listes  \texttt{t\_list,y\_list} obtenu en appliquant la fonction \texttt{odeint} avec \texttt{n} segments à l'équation \eqref{eq:penduleforce}, avec les conditions initiales  $ \theta(0) = \texttt{th0}$ et $\dot{\theta}(0) = \texttt{thp0}$.

\medskip{}

\question{}\label{qu:trace_trajectoire_odeint} Écrire une fonction \texttt{trace\_trajectoire\_odeint(t\_list,y\_list,nom\_de\_fichier)} enregistrant dans \texttt{nom\_de\_fichier} la courbe de la trajectoire du pendule (c'est-à-dire des points $(t,\theta(t))$, où \texttt{t\_list} et \texttt{y\_list} sont supposées être les listes renvoyées par la fonction \texttt{odeint\_f}.

Vous enverrez à l'enseignant les figures produites pour les conditions initiales $\theta(0) = 1$ et $\dot{\theta}(0) = 2$ et pour $n=10^2$. 

\medskip{}

\question{}\label{qu:trace_phase_odeint} Écrire une fonction \texttt{trace\_phase\_odeint(t\_list,y\_list,nom\_de\_fichier)} enregistrant dans \texttt{nom\_de\_fichier} le portrait de phase du pendule (c'est-à-dire des points $(\theta(t),\dot{\theta}(t))$, où \texttt{t\_list} et \texttt{y\_list} sont supposées être les listes renvoyées par la fonction \texttt{odeint\_f}.

Vous enverrez à l'enseignant les figures produites pour les conditions initiales $\theta(0) = 1$ et $\dot{\theta}(0) = 2$ et pour $n=10^2$. 