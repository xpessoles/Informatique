\exer{[EQD-000]}
\setcounter{numques}{0}~\\

On considère une équation différentielle $(\mathscr{E})$ : $y' = F(y,t)$, où $F$ est une fonction de $\R^n \times \R$ dans 
$\R^n$, et où l'inconnue $y$ est une fonction de $\mathscr{C}^1(\R,\R^n)$, avec la condition initiale $y(t_0) = y_0$.

\medskip

\question\ Décrire le principe de la méthode d'Euler. On donnera clairement la relation de récurrence qui est au coeur de 
cette méthode, en expliquant bien ce que représente la suite vérifiant cette relation de récurrence.

\medskip 

\question\ Écrire en \texttt{Python} une fonction mettant en \oe uvre la méthode d'Euler  et permettant de résoudre numériquement $(\mathscr{E})$ sur le segment 
$[a,b]$, 
où $a,b\in\R$, $a<b$. Vous écrirez une docstring décrivant tous les arguments de cette fonction.

\medskip

On considère l'équation différentielle d'inconnue $y\in\mathscr{C}^2(\R)$ et les conditions initiales suivantes : 
\begin{equation}\tag{$\mathscr{E}$}\label{d03s:equadiff}
  y''+\cos(t) y' - t^2 y = \e^t,\qquad y(0) = 4,~ y'(0) = 2. 
\end{equation}

\question\ Donnez une fonction $F$ et une variable $X$ telle que \eqref{d03s:equadiff} soit équivalente à l'équation $X'=F(X,t)$. On précisera bien 
les ensembles de définition et d'arrivée de $F$, et l'ensemble auquel appartient la variable $X$, ainsi que la condition initiale à utiliser.