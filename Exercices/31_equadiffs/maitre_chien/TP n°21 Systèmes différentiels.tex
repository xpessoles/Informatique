\documentclass[11pt]{article}

\usepackage{amsmath,amsthm,amsfonts,amssymb,mathrsfs,mathtools,stmaryrd} % caract\`{e}res math\'{e}matiques

\usepackage[french]{babel} % texte en français (gestion des c\'{e}sures)

\usepackage[utf8]{inputenc} % gestion des accents - remplacer uft8 par applemac ?

\usepackage[T1]{fontenc} % codage des caract\`{e}res

\usepackage{geometry} % mise en page (gestion des marges)
\geometry{ hmargin=1.5cm, vmargin=2cm }

\usepackage[version=3]{mhchem}%pour écrire des formule chimique

\usepackage{multicol}

\usepackage{enumitem}

\usepackage{graphicx} % ins\'{e}rer des figures
\usepackage{epstopdf} % gestion du format eps
\usepackage{tikz} % dessins et courbes
\usepackage{tkz-tab} % tableaux de variations

\usepackage{fancyhdr} % entête et pied de page
\pagestyle{fancy}

\usepackage{fancybox} % encadrements
\usepackage{boxedminipage} % encadrement des minipages

\usepackage{xcolor}

\usepackage{listings}
%\usepackage{xkeyval}
\lstset{
	language={Python},
	columns=flexible,
	basicstyle=\ttfamily,
	keywordstyle=\color{blue}, % je voudrais mettre les mots-cl\'{e} en gras (\bfseries) mais c'est incompatible avec typewriter
	commentstyle=\color{gray}, % commentaires
	backgroundcolor=\color{green!15}, % couleur du fond
	frame=single, rulecolor=\color{green!15}, % encadrement + couleur de l'encadrement
	%numbers=left, numberstyle=\tiny, stepnumber=1, numbersep=5pt,
	showstringspaces=false, % supprime les espaces apparents dans les lignes de texte
	stringstyle=\color{red!60}\itshape,
}

\usepackage{fancyvrb}
\newsavebox{\BBbox}
\newenvironment{DDbox}[1]{
\begin{lrbox}{\BBbox}\begin{minipage}{9,2cm}}
{\end{minipage}\end{lrbox}\noindent\colorbox{gray!20}{\usebox{\BBbox}} \\
[.5cm]}

\newcommand{\tpIPT}[5]{
	\lhead{#1}
	\chead{\begin{minipage}{10cm} \begin{center} \textbf{#2} \end{center} \end{minipage}}
	\rhead{TP \texttt{info} n$\textsuperscript{o}$#3}
	\lfoot{#4}
	\rfoot{Ann\'{e}e scolaire #5}}

\usepackage{fourier-orns} % symbol "danger" par la commande \danger{}

\usepackage{ulem} % pour souligner du texte sur plusieurs lignes avec \uline{texte}

\parindent=0em

\usepackage{lastpage}

%%%%%%%%%%%%%%%%%%%%%%%%%%%%%%%%%%%%%%%%%%%%%%%%%%%%%%%

\begin{document}

%\fancyfoot{\thepage/\pageref{LastPage}}
\fancyfoot{}

\tpIPT{PCSI 2}{Systèmes différentiels du premier ordre}{21}{Lyc\'{e}e F. Roosevelt -- Reims}{2013-2014}

\section{Un marcheur et son chien.}

Un marcheur $M$ suit une trajectoire rectiligne à vitesse constante $V_M$. Son chien $C$, qui part d'un point éloigné, court pour le rejoindre à vitesse constante $V_C$. \`{A} chaque instant, sa course est dirigée vers son maître, \textit{i.e.} les vecteurs $\dfrac{\text{d}\overrightarrow{OC}}{\text{d}t}(t)$ et $\overrightarrow{CM}(t)$ sont colinéaires et de même sens :
\[ \dfrac{\text{d}\overrightarrow{OC}}{\text{d}t}(t)=V_C\,\frac{\overrightarrow{CM}(t)}{\vert\vert\overrightarrow{CM}(t)\vert\vert}. \]
Ainsi, les coordonnées $\big(x(t),y(t)\big)$ du chien vérifient le système différentiel :
\[ \left\{ \begin{aligned} x'(t)&=V_C\,\frac{V_M\,t-x(t)}{\sqrt{\big( V_M\,t-x(t) \big)^2+\big( -y(t) \big)^2}} \\ y'(t)&=V_C\,\frac{-y(t)}{\sqrt{\big( V_M\,t-x(t) \big)^2+\big( -y(t) \big)^2}} \end{aligned} \right. \]
\begin{itemize}
	\item[\textbullet] \'{E}crire un programme permettant de tracer la trajectoire du chien avec les données :
		\[ V_M=1{,}5\,\text{m}.\text{s}^{-1} \quad \text{et} \quad V_C=8\,\text{m}.\text{s}^{-1} \]
		et la condition initiale :
		\[ x(0)=100\,\text{m} \quad \text{et} \quad y(0)=300\,\text{m}. \]
\end{itemize}

\section{Cinétique chimique.}

On considère deux réactions successives :
\[A \xrightarrow{k_1} B\]
\[B \xrightarrow{k_2} C\]
dans lesquelles $k_1$ et $k_2$ désignent les constantes de vitesse. Les vitesses de réaction sont donc :
\[ v_1=-\dfrac{\text{d}[A]}{\text{d}t}=k_1[A] \quad \text{et} \quad v_2=\dfrac{\text{d}[C]}{\text{d}t}=k_2[B]. \]

$B$ intervient dans les deux réactions chimiques : il est produit par la réaction 1 et consommé par la réaction 2. On en déduit :
\[ \dfrac{\text{d}[B]}{\text{d}t}=v_1-v_2=k_1[A]-k_2[B]. \]

On obtient ainsi un système de trois équations différentielles du premier ordre couplées : 
\[ \left\{ \begin{aligned} \dfrac{\text{d}[A]}{\text{d}t}&=-k_1[A] \\
	\dfrac{\text{d}[B]}{\text{d}t}&=k_1[A]-k_2[B] \\
	\dfrac{\text{d}[C]}{\text{d}t}&=k_2[B] \end{aligned} \right. \]

\begin{itemize}
	\item[\textbullet] \'{E}crire un programme permettant de représenter l'évolution des concentrations $[A]$, $[B]$ et $[C]$ au cours du temps avec la donnée :
		\[ (k_1,k_2)=(8,1) \]
		et la condition initiale :
		\[ [A]_0=1\,\text{mol}.\text{L}^{-1}, \quad [B]_0=0\,\text{mol}.\text{L}^{-1} \quad \text{et} \quad [C]_0=0\,\text{mol}.\text{L}^{-1}. \]
\end{itemize}

\end{document}