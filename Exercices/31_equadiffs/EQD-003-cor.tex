\exer{[EQD-003]}
\setcounter{numques}{0}~\\

\question{} 
% Écrire la fonction \texttt{T(f,a,b,N)} permet de donner l'estimation de $\displaystyle{\int_{z=a}^bf(z)dz}$ avec la méthode des trapèzes en fonction du nombre d'intervalles ($N$) sur $\left[a,b\right]$.
\begin{lstlisting}
def T(f,a,b,N):
    h=(b-a)/N
    S=(f(a)+f(b))*0.5
    for k in range(1,N):
        S+=f(a+k*h)
    return S*h
\end{lstlisting}

\question{} 
% Écrire une fonction \texttt{volume_immerge(z)} qui renvoie le volume immergé du bouchon en fonction de la profondeur du bas du bouchon (noté $z$) en utilisant la fonction définie à la question précédente.

\begin{lstlisting}
from math import sin, pi

def r(z):
    return R*(1+0.1*sin(np.pi*z/H))
    
def r2(z):
    return r(z)**2

def volume_immerge(z):
    """Renvoie le volume immergé du bouchon.
    z : profondeur du bas du bouchon"""
    if z <= 0 :
        # bouchon hors de l'eau
        return 0
    elif z <= H :
        # bouchon immergé jusqu'à la hauteur z
        return pi*T(r2,0,z,N)
    else :
        # bouchon totalement immergé
        return pi*T(r2,0,H,N)
\end{lstlisting}

\question{} 
% Écrire une instruction qui affecte le volume du bouchon à \texttt V. 
\begin{lstlisting}
V = volume_immerge(H)
\end{lstlisting}


\question{} 
% Écrire une fonction \texttt{Fv(z,zp)} qui renvoie la force de frottement visqueux $F_v\left(z_A(t),\dfrac{\mathrm{d}z_A}{\mathrm{d}t}(t)\right)$.
\begin{lstlisting}
def Fv(z,zp):
    """renvoie la force de frottement visqueux"""
    if zp==0 or z<0:
        return 0
    else:
        if zp>0 : 
            signe = 1
        else :
            signe = -1
        alpha = -signe*0.5*Cx*rho_eau*zp**2
        if zp>0 : # Le bouchon descend
            if 0<=z<H/2:
                return alpha*pi*r(z)**2
            else: # z >= H/2
                return alpha*pi*r(H/2)**2
        else s: #Le bouchon remonte
            if 0<=z<H/2:
                return 0
            elif H/2<=z<=H:
                return alpha*pi*(r(H/2)**2-r(z)**2)
            else : # z>H:
                return alpha*pi*r(H/2)**2
\end{lstlisting}


\question{} 
% Exprimer $\dfrac{\mathrm{d}^2z_A}{\mathrm{d}t^2}(t)$ en fonction de $\rho_{e}$, $\rho_b$, $V_i(z_A(t))$, $F_v\left(z_A(t),\dfrac{\mathrm{d}z_A}{\mathrm{d}t}(t)\right)$ et $z_A(t)$. 



En combinant les différentes équations, on obtient : 

\begin{align*}
\rho_b\cdot V \cdot \dfrac{\mathrm{d}^2z_A}{\mathrm{d}t^2}(t)=-\rho_e\cdot g\cdot V_i(z_A(t))+\rho_b\cdot g\cdot V+F_v\left(z_A(t),\dfrac{\mathrm{d}z_A}{\mathrm{d}t}(t)\right)
\end{align*}

En simplifiant les équations, on obtient : 

\begin{align*}
\dfrac{\mathrm{d}^2z_A}{\mathrm{d}t^2}(t)=\dfrac{F_v\left(z_A(t),\dfrac{\mathrm{d}z_A}{\mathrm{d}t}(t)\right)-\rho_e\cdot g\cdot V_i(z_A(t))}{\rho_b\cdot V}+g
\end{align*}

\question{} 
% Définir l'expression de $Z(t)$ et de $F(Z(t),t)$ pour que le problème de Cauchy de la forme $Z'=F(Z(t),t)$ corresponde à l'équation différentielle \ref{eqn_diff}. Écrire les instructions permettant de définir la fonction \texttt{F(Z(t),t)}.

On peut poser 
\begin{equation*}
    Z(t) = \begin{pmatrix} z_A(t) \\ \dfrac{\mathrm{d}z_A}{\mathrm{d}t}(t)\end{pmatrix} \quad\textrm{et}\quad Z_0 = \begin{pmatrix} -0,2 \\ 0 \end{pmatrix},.
\end{equation*}
Ainsi, 
\begin{equation*}
    Z'(t) = \begin{pmatrix}\dfrac{\mathrm{d}z_A}{\mathrm{d}t}(t) \\ \dfrac{\mathrm{d}^2z_A}{\mathrm{d}t^2}(t) \end{pmatrix}
\end{equation*}
Avec 
\begin{equation*}
    F \p{\begin{pmatrix} a\\b \end{pmatrix},t} = \begin{pmatrix} b \\ g + \dfrac{1}{\rho_b V} \p{F_v(a,b) - \rho_e g V_i(a)} \end{pmatrix},
\end{equation*}
on a bien 
\begin{equation*}
    Z'(t) = F(Z(t),t) \quad\textrm{et}\quad Z(0) = Z_0.
\end{equation*}

\begin{lstlisting}
from numpy import array

def F(Z,t):
    """Avec frottement visqueux"""
    z,zp = Z
    zpp = (Fv(z,zp)- rho_eau*g*volume_immerge(z)) / (rho_b*volume_immerge(H)) + g
    return array([zp,zpp])
    
Z0 = array([-0.2,0])
\end{lstlisting}

\question{} 
% Écrire une fonction \texttt{euler(F, tmin, tmax, Z0, h)} prenant en argument la fonction \texttt F définie précédemment, \texttt{tmin} et \texttt{tmax} définissant l'intervalle de résolution, le vecteur \texttt{Z0} définissant les consitions initiales ainsi \texttt h le pas de discrétisation temporelle et permettant de résoudre de manière approchée l'équation~\eqref{EQD-003:eq:eqn_Z} par la méthode d'Euler.

\begin{lstlisting}
def euler(F, tmin, tmax, Z0, h):
    """Solution de Z'=F(Z,t) sur [tmin,tmax], Z(tmin) = Z0,"""
    Z = Z0
    t = tmin
    z_list = [Z0] # la liste des valeurs renvoyées
    t_list = [tmin] # la liste des temps
    while t+h <= tmax : 
        # Variant : floor((tmax-t)/h)
        # Invariant : au tour k, z_list = [Z_0,...,Z_k], t_list = [t_0,...,t_k]
        Z = Z + h * F(Z, t)
        z_list.append(Z)
        t = t + h
        t_list.append(t)
    return t_list, z_list
\end{lstlisting}

\question{} 
% Commenter brièvement la cohérence physique d'une telle courbe. Pouvez-vous fournir une explication et une solution au problème observé, tout en restant dans le cadre de la méthode d'Euler ? 
Le terme de frottement fluide induit une perte d'énergie pour le bouchon. Après la phase de chute libre, les oscillations du bouchon dans l'eau devraient donc décroître en amplitude.
Ce n'est pas ce que l'on observe (la second oscillation a même une amplitude légèrement plus grande que la première !).

Une explication : le pas temporel choisi pour la méthode d'Euler est trop grand. Il suffit de le diminuer.

