\exer{[EQD-001]}
\setcounter{numques}{0}~\\

\question{}

\begin{itemize}
\item $\left[1, 2, 3, 4, 5, 6\right]$ ;
\item $\left[1, 2, 3, 1, 2, 3\right]$
\end{itemize}

\question{}

\begin{lstlisting}
def smul(n,liste):
	"""
	Multiplie chaque élément de la liste par un nombre et renvoie une nouvelle liste
	"""
	liste2=[]
	for i in liste:
		liste2.append(i*n) 
	return liste2
	
\end{lstlisting}

\question{}

On effectue $len(liste)$ tours de boucle. La complexité pour chaque tour de boucle est en $O(1)$ (nombre borné/ fini d'opérations en $O(1)$) Ainsi la complexité est en $O(len(liste))$.


%\question{}
%
%\begin{lstlisting}
%def vsom(liste1,liste2):
%	"""
%	Prend en paramètre deux listes de nombres de même longueur et renvoie une nouvelle
%	liste constituée de la somme terme à terme de ces deux listes.
%	"""   
%	liste3=[]
%	for i in range(len(liste1)):
%		liste3.append(liste1[i]+liste2[i])
%	return liste3    
%\end{lstlisting}
%
%
%\question{}
%
%\begin{lstlisting}
%def vdif(liste1,liste2):
%	"""
%	Prend en paramètre deux listes de nombres de même longueur et qui renvoie une nouvelle
%	liste constituée de la différence terme à terme de ces deux listes.
%	"""
%	liste3=[]
%	for i in range(len(liste1)):
%		liste3.append(liste1[i]-liste2[i]) 
%	return liste3
%\end{lstlisting}

\question{} 


%On a par définition $y'(t)=z(t)$ or y est de classe $C^2$ donc on peut obtenir la deuxième expression par dérivation.

On peut poser :
$$Y=\left(
\begin{array}{c}
y(t)\\
z(t)
\end{array}
\right)$$

et $F$ peut se définir sous la forme : 

\begin{align*}
F:
\left(
\begin{array}{c}
y(t)\\
z(t)
\end{array}
\right)
\mapsto
\left(\begin{array}{c}
z(t)\\
f(y(t))
\end{array}
\right)
\end{align*} 

%\question{}
%
%D'après la question précédente, on a:
%\begin{eqnarray*}
%& y'(t)=z(t) & \\
%& \int_{t_i}^{t_{i+1}}y'(t)dt=\int_{t_i}^{t_{i+1}}z(t)dt \\
%& \boxed{y\left(t_{i+1}\right)=y\left(t_i\right)+\int_{t_i}^{t_{i+1}}z(t)dt}
%\end{eqnarray*}
%
%De la même manière:
%\begin{eqnarray*}
%& z'(t)=f\left(y(t)\right) & \\
%& \int_{t_i}^{t_{i+1}}z'(t)dt=\int_{t_i}^{t_{i+1}}f\left(y(t)\right)dt \\
%& \boxed{z\left(t_{i+1}\right)=z\left(t_i\right)+\int_{t_i}^{t_{i+1}}f\left(y(t)\right)dt}
%\end{eqnarray*}




\question{}



\begin{lstlisting}
import numpy as np
def euler(F, tmin, tmax, Y0, n):
    """Solution de Y'=F(Y,t) sur [a,b], y(a) = y0, pas h"""
    Y = Y0
    t = tmin
    les_y = [Y0[0]] 
    les_z = [Y0[1]] 
    h=(tmax-tmin)/(n-1)
    les_t = np.arange(tmin,tmax,h)
    while t+h <= tmax:
        Y = Y + h * F(Y, t) # surtout pas += !
        les_y.append(Y[0])
        les_z.append(Y[1])
        t += h
    return les_t, les_y, les_z
\end{lstlisting}

%\question{}
%$y''(t)=-\omega^2.y(t)$
%En multipliant chaque membre de l'égalité par $y'(t)$, on a:
%
%$$y'(t).y''(t)=-\omega^2.y(t).y'(t)$$
%
%$$\frac{1}{2}\left(2y'(t).y''(t)\right) =(-\omega^2)\frac{1}{2}.\left(2y(t).y'(t)\right)$$
%
%donc en intégrant :
%
%$$\exists E\;tq\quad \frac{1}{2}y'(t)^2+\frac{\omega^2}{2} y(t)^2 =E $$
%
%
%Ainsi en posant $g : x \mapsto \dfrac{\omega^2}{2} x^2$ avec $g'=-f$, on obtient :
%
%$$
%\boxed{\frac{1}{2}y'(t)^2+g(y(t))= E
%}
%$$

\question{}

Mathématiquement, on écrit 
\begin{align*}
f:\fct{\R}{\R}{x}{-\omega^2x}
\end{align*}

\begin{align*}
F : \begin{pmatrix} y \\ z \end{pmatrix} \longmapsto \begin{pmatrix} z \\ f(y) \end{pmatrix}.
\end{align*}

Informatiquement, cela donne :
\begin{lstlisting}
def f(x):
    return -w**2*x
    

def F(Y,t):
    y=Y[0]
    z=Y[1]
    return np.array([z,f(y)])
\end{lstlisting}

\question{}
\begin{lstlisting}
tmin=0
tmax=3
w=2*np.pi
n=100
y0=3
z0=0
vt,vy,vz=euler(F, tmin, tmax, np.array([y0,z0]), n)

plt.clf()
plt.plot(vy,vz,'ko-')
plt.xlabel('y(t)')
plt.ylabel('z(t)')
plt.grid()
plt.savefig('figure1.png')
\end{lstlisting}

%
%$$E(t_i)=\frac{1}{2} y'(t_i)^2 + g(y(t_i))=\frac{1}{2} z(t_i)^2 + g(y(t_i))$$
%$$E_i = \frac{1}{2} z_i^2 + g(y_i)= \frac{1}{2} z_i^2 + \frac{\omega^2}{2} (y_i)^2$$
%
%$$E_{i+1} = \frac{1}{2} z_{i+1}^2 + \frac{\omega^2}{2} (y_{i+1})^2$$
%
%Soit $$E_{i+1}-E_i=\frac{1}{2} z_{i+1}^2 + \frac{\omega^2}{2} (y_{i+1})^2-\frac{1}{2} z_i^2 - \frac{\omega^2}{2} (y_i)^2$$
%
%En utilisant les relations de récurrence:
%
%$$E_{i+1}-E_i=\frac{1}{2} (z_i-h \omega^2 y_i)^2 + \frac{\omega^2}{2} (y_i+h z_i)^2-\frac{1}{2} z_i^2 - \frac{\omega^2}{2} (y_i)^2$$
%
%Après simplification:
%
%$$E_{i+1}-E_i=\frac{1}{2} h^2 \omega^2 (z_i^2 +\omega^2 y_i^2)$$
%
%
%$$\boxed{E_{i+1}-E_i= h^2 \omega^2 E_i}$$
%
%\question{}
%
%Pour conserver l'énergie il faut: $$E_{i+1}-E_i= 0$$
%
%Tous les $E_i$ sont égaux à $E$ or $E_i=\frac{1}{2} (z_i^2 +\omega^2 y_i^2)$ donc $ z_i^2 +\omega^2 y_i^2 = constante$.
%
%On obtient l'équation d'une ellipse.
%
%Exemples, circuit RLC avec R nulle (non dissipatif), barre en liaison pivot sans frottement. 
%
%\question{}
%
%Allure d'une ellipse centrée en $y_i = z_i = 0$. 
%\begin{figure}[!h]
%\centering
%\includegraphics[scale=0.5]{ellipse.png}
%\end{figure}

%\question{} 
%
%L'allure n'est pas une ellipse. Il y a divergence des $y_i$ et des $z_i$. Le système ainsi modélisé n'est pas conservatif en énergie. 
%En calculant le rapport $\dfrac{E_{i+1}}{E_i}$, on trouve : 
%$$\dfrac{E_{i+1}}{E_i} = 1 + h^2 \omega^2 $$
%
%qui est strictement supérieur à 1. Donc, la raison de la suite est supérieur à 1, donc divergente. Cela signifie que l'énergie n'est pas finie, et donc diverge. 
%Par conséquent, les $y_i$ et les $z_i$ divergent aussi. D'où l'allure de la figure de gauche. 

\question{}

Comme pour la méthode d'Euler, pour calculer tous les $y_i$ et $z_i$, il est nécessaire de connaître: $f,y_0,z_0,i$ et $h$. 


\begin{lstlisting}
def verlet(f, tmin, tmax, Y0, n):
    y,z=Y0[0],Y0[1]
    valeursy =[ y ]
    valeursz =[ z ]
    h=(tmax-tmin)/(n-1)
    h2sur2=h**2/2. # Hors de la boucle pour la complexité
    les_t = np.arange(tmin,tmax,h)
    for k in range ( n-1 ):
        fi=f(valeursy[-1])
        y = y + h * valeursz[ -1]+h2sur2*fi   
        z = z + h/2. * (fi+f(y))
        valeursy . append ( y )
        valeursz . append ( z )
    return [les_t,valeursy , valeursz ]
\end{lstlisting}



\question{}



\begin{lstlisting}
vt,vy,vz=verlet(f, tmin, tmax, np.array([y0,z0]), n)

plt.clf()
plt.plot(vy,vz,'ko-')
plt.xlabel('y(t)')
plt.ylabel('z(t)')
plt.axis([-15,20,-80,100])
plt.grid()
plt.savefig('fig2.png')
\end{lstlisting}

\question{} Le schéma de Verlet semble plus stable que le schéma d'Euler. 
En effet l'équation différentielle traduit le comportement d'un système physique conservatif, c'est-à-nom\_de\_fichier que la quantité $\dfrac{(y')^2}{2} + \dfrac{\omega^2}{2}y^2$ doit être constante. 
On devrait alors avoir un portrait de phase décrivant une ellipse. 
On observe bien que c'est le cas du portrait de phase de Verlet mais pas celui obtenu avec la méthode d'Euler.
La solution approchée donnée par la méthode d'Euler est absurde physiquement : il y aurait augmentation de l'énergie dans le système. 
