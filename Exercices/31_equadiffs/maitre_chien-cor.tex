\exer{[maitre-chien]}
\setcounter{numques}{0}~\\

Un marcheur $M$ suit une trajectoire rectiligne à vitesse constante $V_M$. Son chien $C$, qui part d'un point éloigné, court pour le rejoindre à vitesse constante $V_C$. \`{A} chaque instant, sa course est dirigée vers son maître, \textit{i.e.} les vecteurs $\dfrac{\text{d}\overrightarrow{OC}}{\text{d}t}(t)$ et $\overrightarrow{CM}(t)$ sont colinéaires et de même sens :
\[ \dfrac{\text{d}\overrightarrow{OC}}{\text{d}t}(t)=V_C\,\frac{\overrightarrow{CM}(t)}{\vert\vert\overrightarrow{CM}(t)\vert\vert}. \]
Ainsi, les coordonnées $\big(x(t),y(t)\big)$ du chien vérifient le système différentiel :
\[ \left\{ \begin{aligned} x'(t)&=V_C\,\frac{V_M\,t-x(t)}{\sqrt{\big( V_M\,t-x(t) \big)^2+\big( -y(t) \big)^2}} \\ y'(t)&=V_C\,\frac{-y(t)}{\sqrt{\big( V_M\,t-x(t) \big)^2+\big( -y(t) \big)^2}} \end{aligned} \right. \]
\begin{itemize}
	\item[\textbullet] \'{E}crire un programme permettant de tracer la trajectoire du chien avec les données :
		\[ V_M=1{,}5\,\text{m}.\text{s}^{-1} \quad \text{et} \quad V_C=8\,\text{m}.\text{s}^{-1} \]
		et la condition initiale :
		\[ x(0)=100\,\text{m} \quad \text{et} \quad y(0)=300\,\text{m}. \]
\end{itemize}