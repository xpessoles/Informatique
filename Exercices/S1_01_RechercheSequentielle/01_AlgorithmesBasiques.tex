%
%\subsection*{Recherche séquentielle}

%\exer{Exercices d'échauffement}
%\setcounter{numques}{0}~\\

\begin{obj}
Rechercher séquentiellement un élément dans un tableau unidimensionnel ou dans un dictionnaire.
\end{obj}

\subsection*{Création et manpulation de listes}
Rappels : 
\begin{itemize}
\item Une liste est une structure de données contenant une série de valeurs. Les valeurs sont séparées par des virgules et encadrées par des crochets. Par exemple \texttt{L = [1, 2, 3]}.
\item Pour accéder au \ieme élément d'une liste on procède ainsi \texttt{L[i]}. Attention l'indexation commence à 0. 
\item Pour accéder à la taille de liste on utilise \texttt{len[L]}.
\item Pour ajouter un élément à une liste, on utilise \texttt{L.append(4)}.
\end{itemize}

\question{\'Ecrire une fonction de signature \texttt{generer\_liste\_entiers\_01(n: int) -> list} renvoyant la liste des entiers compris entre 0 (inclus) et n (exclus). On utilisera une boucle \texttt{for}.}

\question{\'Ecrire une fonction de signature \texttt{generer\_liste\_entiers\_02(n: int) -> list} renvoyant la liste des entiers compris entre 0 (inclus) et \texttt{n} (exclus). On utilisera une boucle \texttt{while}.}

\question{\'Ecrire une fonction de signature \texttt{generer\_liste\_entiers\_03(deb: int, fin:int) -> list} renvoyant la liste des entiers compris entre \texttt{deb} et \texttt{fin}  inclus. On utilisera une boucle \texttt{for}.}

\question{\'Ecrire une fonction de signature \texttt{generer\_liste\_entiers\_04(deb: int, fin:int) -> list} renvoyant la liste des entiers compris entre \texttt{deb} et \texttt{fin}  inclus. On utilisera une boucle \texttt{while}.}

\question{\'Ecrire une fonction de signature \texttt{generer\_liste\_pairs(n:int) -> list} renvoyant la liste des entiers pairs compris entre 0 et \texttt{n} exclus.}

\question{\'Ecrire une fonction de signature \texttt{generer\_liste\_impairs(n:int) -> list} renvoyant la liste des entiers impairs compris entre 0 et \texttt{n} exclus.}

\question{\'Ecrire une fonction de signature \texttt{is\_multiple(n:int, m:int) -> bool} renvoyant \texttt{True} si \texttt{n} est multiple de de \texttt{m}.}


\question{\textbf{En utilisant la fonction précédente}, écrire une fonction de signature \texttt{genere\_liste\_multiple(n:int, m:int) -> list} renvoyant les \texttt{n} premiers multiples de \texttt{m}.}


\subsection*{Recherche d'un nombre dans une liste}


Nous allons rechercher si un nombre est dans une liste. 

Commençons par générer une liste de nombre alétoires. Pour cela recopier les lignes suivantes. 

\begin{lstlisting}
import random as rd
def generer_alea(nb: int) -> list :
    """
    Génération d'une liste nb de nombres aléatoires compris entre 0 inclus et nb exclus.
    """
    res = []
    for i in range(nb):
        res.append(rd.randrange(0,nb))
    return res
\end{lstlisting}



\question{Écrire une fonction de signature  \texttt{recherche\_nb\_01(nb: int, L: list) -> bool} qui renvoie \texttt{True} si \texttt{nb} est dans \texttt{L}, \texttt{False} sinon. On utilisera une boucle \texttt{for}.}

\question{Écrire une fonction de signature  \texttt{recherche\_nb\_02(nb: int, L: list) -> bool} qui renvoie \texttt{True} si \texttt{nb} est dans \texttt{L}, \texttt{False} sinon. On utilisera une boucle \texttt{while}.}

\question{Écrire une fonction de signature  \texttt{recherche\_nb\_03(nb: int, L: list) -> bool} qui renvoie \texttt{True} si \texttt{nb} est dans \texttt{L}, \texttt{False} sinon. On n'utilisera pas explicitement de boucles \texttt{for} ou \texttt{while}.}

\question{Écrire une fonction de signature  \texttt{recherche\_first\_index\_nb\_01(nb: int, L: list) -> int} qui renvoie l'index de la première appartion du nombre \texttt{nb} dans la liste \texttt{L}. La fonction renverra \texttt{-1} si \texttt{nb} n'est pas dans la liste. On utilisera une boucle \texttt{for}.}

\question{Écrire une fonction de signature  \texttt{recherche\_first\_index\_nb\_02(nb: int, L: list) -> int} qui renvoie l'index de la première appartion du nombre \texttt{nb} dans la liste \texttt{L}. La fonction renverra \texttt{-1} si \texttt{nb} n'est pas dans la liste. On utilisera une boucle \texttt{while}.}

\question{Écrire une fonction de signature  \texttt{recherche\_last\_index\_nb\_01(nb: int, L: list) -> int} qui renvoie l'index de la dernière appartion du nombre \texttt{nb} dans la liste \texttt{L}. La fonction renverra \texttt{-1} si \texttt{nb} n'est pas dans la liste.}


\question{Écrire une fonction de signature  \texttt{recherche\_index\_nb\_01(nb: int, L: list) -> list} qui renvoie la liste des index du nombre \texttt{nb} dans la liste \texttt{L}. La fonction renverra une liste vide si \texttt{nb} n'est pas dans la liste.}

\subsection*{Recherche d'un caractère dans une chaîne (de caractères)}

\question{Écrire une fonction de signature  \texttt{is\_char\_in\_str\_01(lettre: str, mot: str) -> int} qui renvoie \texttt{True} si \texttt{lettre} est dans \texttt{mot}, \texttt{False} sinon. On utilisera une boucle \texttt{for} ou \texttt{while}.}

\question{Écrire une fonction de signature  \texttt{is\_char\_in\_str\_02(lettre: str, mot: str) -> int} qui renvoie \texttt{True} si \texttt{lettre} est dans \texttt{mot}, \texttt{False} sinon. On n'utilisera ni boucle \texttt{for} ni \texttt{while} explicite.}

\question{Écrire une fonction de signature  \texttt{compte\_lettre\_01(lettre: str, mot: str) -> int} qui renvoie le nombre d’occurrences de \texttt{lettre} dans le \texttt{mot}.}

Les instructions suivantes permettent de charger l'ensemble des mots du dictionnaire dans la variable \texttt{dictionnaire}. \texttt{dictionnaire} est une liste de mots. Chacune des lettres de l'alphabet sont stockées dans la variable \texttt{alphabet}.

\begin{lstlisting}
alphabet = 'abcdefghijklmnopqrstuvwxyz'
def load_fichier(file):
    fid = open(file,'r')
    mots = fid.readlines()
    fid.close()
    return mots
liste_mots = load_file('liste_francais.txt')
\end{lstlisting}

\question{Écrire une fonction de signature  \texttt{compte\_lettre\_02(lettre: str, mots: list) -> int} qui renvoie le nombre d’occurrences de \texttt{lettre} dans une liste de mots \texttt{mots}.}

\question{Quelle consonne apparaît le plus souvent ? Quelle consonne apparaît le moins souvent ? Indiquer le nombre d’occurrences dans chacun des mots}

\question{Écrire une fonction de signature  \texttt{mots\_plus\_long(mots: list) -> str} qui renvoie le mot le plus long.}

%\question{Écrire une fonction de signature  \texttt{cherche\_mot\_in\_chaine\_01(mot: str, chaine: str) -> int} qui renvoie \texttt{True} si \texttt{mot} est dans \texttt{chaine}, \texttt{False} sinon. On utilisera des boucles \texttt{for} ou \texttt{while}.}
%
%\question{Écrire une fonction de signature  \texttt{cherche\_mot\_in\_chaine\_02(mot: str, chaine: str) -> int} qui renvoie \texttt{True} si \texttt{mot} est dans \texttt{chaine}, \texttt{False} sinon. On n'utilisera ni boucle \texttt{for} ni \texttt{while}.}
%
%\question{Écrire une fonction de signature  \texttt{cherche\_mot\_in\_dico(nb: int, dico: lst) -> str} qui permet de trouver le mot de \texttt{nb} lettres qui est le plus contenu dans d'autres mots.}


\subsection*{Recherche dans un dictionnaire}

Un dictionnaire (\texttt{dict}) est un type composite (au même titre que les chaînes, les listes ou les tuples). Les éléments d'un dictionnaire sont constitués d'une \textbf{clé} (alphabétique ou numérique par exemple) et d'une valeur. À la différence d'une liste par exemple, les éléments ne sont pas ordonnés.

\begin{exemple}%[Création d'un dictionnaire]
Dans le but de faire un comptage du nombre de lettres des mots du dictionnaire, nous allons créer un dictionnaire constitué des lettres de l'alphabet (clés) et de leur nombre d'apparitions (valeurs). 
~\\

\begin{minipage}[c]{.45\linewidth}
\begin{lstlisting}
nb_lettres = {}
nb_lettres['a']=0
nb_lettres['b']=0
print(nb_lettres)
    {'a': 0, 'b': 0}
\end{lstlisting}
\end{minipage}
\hfill
\begin{minipage}[c]{.45\linewidth}
\begin{lstlisting}
nb_lettres = {}
for lettre in alphabet :
    nb_lettres[lettre]=0
print(nb_lettres["a"])
    0
\end{lstlisting}

\end{minipage}

Tester l'appartenance d'une clé à un dictionnaire : \texttt{"a" in nb\_lettres}.

Supprimer une clé d'un dictionnaire : \texttt{del nb\_lettres["a"]}.

Parcourir un dictionnaire : 
\begin{lstlisting}
for clef in nb_lettres :
    print(clef)
    print(clef,nb_lettres[clef])

for clef,valeur in nb_lettres.items() :
    print(clef,valeur)
\end{lstlisting}
\end{exemple}



%Il est possible de définir la fonction \texttt{generate\_tab\_alea} différemment.
%\begin{lstlisting}
%def generate_tab_alea_02(deb: int, fin: int,nb: int) -> list :
%    """
%	Génération d'une liste de nb entiers compris entre deb (inclus) et fin (exclus).
%    """
%    return [rd.randrange(deb,fin) for i in range(nb)]
%
%\end{lstlisting}
%
%
%Pour cela commençons par générer une liste d'entiers aléatoires.
%
%
%\begin{lstlisting}
%import random as rd # Permet de charger une bibliothèque permettant de générer des nombres aléatoires.
%
%def generate_tab_alea_01(deb: int,fin: int,nb: int) -> list :
%    """
%	Génération d'une liste de nb entiers compris entre deb (inclus) et fin (exclus).
%    """
%    res = []
%    for i in range(nb):
%        res.append(rd.randrange(deb,fin))
%    return res
%\end{lstlisting}
%
%
%Il est possible de définir la fonction \texttt{generate\_tab\_alea} différemment.
%\begin{lstlisting}
%def generate_tab_alea_02(deb: int,fin: int,nb: int) -> list :
%    """
%	Génération d'une liste de nb entiers compris entre deb (inclus) et fin (exclus).
%    """
%    return [rd.randrange(deb,fin) for i in range(nb)]
%
%\end{lstlisting}
%
%
%Recherche d'un élément 
%Recherche du maximum
%Recherche du second maximum
\question{Écrire une fonction \texttt{init\_dictionnaire(chaine:str) -> dict} que crée, initialise et renvoie le dictionnaire dont les clés sont les lettres de l'alphabet et les valeurs sont initialisées à 0.}
%% --  %%

\question{Écrire une fonction \texttt{remplir\_dictionnaire(dico:dict, liste\_mots:list) -> dict} qui compte le nombre de lettres de chacun des mots de la liste \texttt{liste\_mots} chargée précédemment en incrémentant chacune des valeurs du dictionnaire.}

\question{Écrire une fonction \texttt{cherche\_podium(dico:dict) ->list} qui renvoie la liste des trois lettres les plus utilisées. La liste renvoyée sera sous la forme \texttt{[['a',5],['b',4],['c',3]].}}