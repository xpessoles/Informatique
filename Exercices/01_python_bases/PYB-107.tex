\exer{}
\setcounter{numques}{0}

\question{}
\begin{enumerate}%[label = \emph{\alph*)}]
  \item Affecter à \texttt{v} la liste \texttt{[2,5,3,-1,7,2,1]}
  \item Affecter à \texttt{L} la liste vide.
  \item Vérifier le type des variables crées.
  \item Calculer la longueur de \texttt{v}, affectée à \texttt{n} et celle de \texttt{L}, affectée à \texttt{m}.
  \item Tester les expressions suivantes : \texttt{v[0]}, \texttt{v[2]}, \texttt{v[n]}, \texttt{v[n-1]}, \texttt{v[-1]} et \texttt{v[-2]}.
  \item Changer la valeur du quatrième élément de \texttt{v}.
  \item Que renvoie \texttt{v[1:3]} ? Remplacer dans \texttt{v} les trois derniers éléments par leurs carrés.
  \item Que fait \texttt{v[1] = [0,0,0]} ? Combien d'éléments y a-t-il alors dans \texttt{v} ?
\end{enumerate}