\exer{}
\setcounter{numques}{0}

\question Un banquier vous  propose un prêt de $400\,000$ euros  sur $40$ ans «à
$3\%$ par  an» ---  ce qui, dans  le langage commercial  des banquiers,
veut  dire $0,25\%$  par mois  ---  avec des  mensualités de  $1431,93$
euros.  Autrement  dit,  vous   contractez  une  dette  de  $400\,000$
euros. Chaque mois, cette dette  augmente de $0,25\%$ puis est diminuée
du  montant  de  votre  mensualité.  À  la  fin  des  $40  \times  12$
mensualités, il  ne vous  reste plus qu'à  vous acquitter  d'une toute
petite dette, que vous rembourserez aussitôt.
\begin{enumerate}[label=\emph{\alph*)}]
  \item Écrire  une fonction  \texttt{reste\_a\_payer(p, t,  m, d)}
renvoyant  le montant  de cette  somme à  rembourser  immédiatement
après le paiement de la dernière
mensualité, où  $p$ est le  montant total du  prêt en euros (dans l'exemple, $400\,000$), $t$ son  taux mensuel (dans l'exemple, 
$0,25 \times 10^{-2}$), $m$ le montant d'une mensualité en euros (dans l'exemple, $1431,93$) et $d$ la
durée en années (dans l'exemple, $40$).

\emph{Indice : dans le cas donné dans cet énoncé, vous devez trouver un montant
restant d'un peu moins de $7,12$ euros.}
  \item Écrire une fonction \texttt{somme\_totale\_payee(p, t, m,
  d)} renvoyant la somme totale (mensualités plus le dernier
paiement) que vous aurez payé au banquier.
  \item Écrire une fonction \texttt{cout\_total(p, t, m, d)} renvoyant
  le coût total  du crédit, c'est-à-dire le total de  ce que vous avez
  payé moins le montant du prêt.
\end{enumerate}