\exer{}
\setcounter{numques}{0}

%\question{Prévoir les résultats des expressions suivantes, puis le vérifier grâce à l'interpréteur interactif.}

%\begin{multicols}{3}
%  \begin{enumerate}[label=\emph{\alph*)}]
%    \item \texttt{(1,2)}
%    \item \texttt{(1)}
%    \item \texttt{(1,)}
%    \item \texttt{(,)}
%    \item \texttt{()}
%    \item \texttt{()+()}
%    \item \texttt{()+() == ()}
%    \item \texttt{(1,2)+3}
%    \item \texttt{(1,2)+(3)}
%    \item \texttt{(1,2)+(3,)}
%    \item \texttt{(1,2)+(3,4,5)}
%    \item \texttt{len((1,7,2,"zzz",[]))}
%    \item \texttt{len(())}
%    \item \texttt{len(("a","bc")+("cde",""))}
%  \end{enumerate}
%\end{multicols}
%
%\question{Pour chaque séquence d'instruction, prévoir son résultat puis le vérifier grâce à l'interpréteur interactif.}
%
%
%
%
%\begin{enumerate}[label=\emph{\alph*)}]
%\item 
%\begin{lstlisting}
%t = (2,"abra",9,6*9,22)
%print(t)
%t[0]
%t[-1]
%t[1]
%t[1] = "cadabra" 
%\end{lstlisting}
%\end{enumerate}
%\begin{enumerate}[label=\emph{\alph*)}]
%\setcounter{enumi}{1}
%\item 
%\begin{lstlisting}
%res = (45,5)
%x,y = res
%(x,y) == x,y
%(x,y) == (x,y)
%print x
%print(y)
%x,y = y,x
%print(y)
%\end{lstlisting}
%\end{enumerate}
%\begin{enumerate}[label=\emph{\alph*)}]
%\setcounter{enumi}{2}
%\item 
%\begin{lstlisting}
%v = 7
%ex = (-1,5,2,"","abra",8,3,v)
%5 in ex
%abra in ex
%(2 in ex) and ("abr" in ex)
%v in ex
%\end{lstlisting}
%\end{enumerate}
%
%\question{Prévoir les résultats des expressions suivantes, puis le vérifier grâce à l'interpréteur interactif.}
%
%\begin{multicols}{3}
%  \begin{enumerate}[label=\emph{\alph*)}]
%    \item \texttt{"abba"}
%    \item \texttt{abba}
%    \item \texttt{""}
%    \item \texttt{"" == " "}
%    \item \texttt{""+""}
%    \item \texttt{""+"" == ""}
%    \item \texttt{"May"+" "+"04th"}
%    \item \texttt{"12"+3}
%    \item \texttt{"12"+"trois"}
%    \item \texttt{len("abracadabra")}
%    \item \texttt{len("")}
%    \item \texttt{len("lamartin"+"2015")}
%  \end{enumerate}
%\end{multicols}
%
%\question{Pour chaque séquence d'instruction, prévoir son résultat puis le vérifier grâce à l'interpréteur interactif.}
%
%\begin{enumerate}[label=\emph{\alph*)}]
%\item 
%\begin{lstlisting}
%t = "oh oui youpi !"
%print(t)
%t[0]
%t[-1]
%t[1]
%t[2]
%t[1] = "o" 
%\end{lstlisting}
%\end{enumerate}
%\begin{enumerate}[label=\emph{\alph*)}]
%\setcounter{enumi}{1}
%\item
%\begin{lstlisting}
%ex = "abdefgh"
%"a" in ex
%a in ex
%"def" in ex
%"adf" in ex
%\end{lstlisting}
%\end{enumerate}

\question{Prévoir les résultats des expressions suivantes, puis le vérifier grâce à l'interpréteur interactif d'IDLE.}

\begin{multicols}{2}
  \begin{enumerate}[label=\emph{\alph*)}]
    \item \texttt{[1,2,3,"a"]}
    \item \texttt{123a}
    \item \texttt{[]}
    \item \texttt{[]+[]}
    \item \texttt{[]+[] == []}
    \item \texttt{[1,2] + [5,7,9]}
    \item \texttt{[0,0]+[0]}
    \item \texttt{len(["a","b"])}
    \item \texttt{len([])}
    \item \texttt{len([[]])}
    \item \texttt{len([[[]]])}
    \item \texttt{len([0,0]+[1])}
  \end{enumerate}
\end{multicols}

%\question{Pour chaque séquence d'instruction, prévoir son résultat puis le vérifier grâce à l'interpréteur interactif d'IDLE.}
%
%\begin{enumerate}[label=\emph{\alph*)}]
%\item 
%\begin{lstlisting}
%t = [1,2,3,4,5,6]
%u = ["a","b","c","d"]
%print(t+u)
%t[0]
%t[-1]
%z = t[3]
%print(z)
%t.append(7)
%print(t)
%\end{lstlisting}
%\end{enumerate}
%\begin{enumerate}[label=\emph{\alph*)}]
%\setcounter{enumi}{2}
%\item
%\begin{lstlisting}
%ex = ["sin","cos","tan","log","exp"]
%"log" in ex
%log in ex
%"l" in ex
%z = ex.pop()
%print(z)
%z in ex
%print(ex)
%\end{lstlisting}
%\end{enumerate}
%\begin{enumerate}[label=\emph{\alph*)}]
%\setcounter{enumi}{3}
%\item 
%\begin{lstlisting}
%  u = [1,2,3,4,5,6]
%  L = u
%  u = [1,2,3,42,5,6]
%  print(L)
%\end{lstlisting}
%\end{enumerate}
%\begin{enumerate}[label=\emph{\alph*)}]
%\setcounter{enumi}{4}
%\item
%\begin{lstlisting}
%  u = [1,2,3,4,5,6]
%  L = u
%  u[3] = 42
%  print(L)
%\end{lstlisting}
%\end{enumerate}


\question{Calculer cette suite d'expressions.}
\begin{lstlisting}
[i for i in range(10)]
[compt**2 for compt in range(7)]
[j+1 for j in range(-2,8)]
\end{lstlisting}
Sur ce modèle, obtenir de manière synthétique : 
\begin{enumerate}[label=\emph{\alph*)}]
  \item la liste des 20 premiers entiers naturels impairs ;
  \item la liste de tous les multiples de 5 entre 100 et 200 (inclus) ;
  \item La liste de tous les cubes d'entiers naturels, inférieurs ou égaux à 1000. 
  \item une liste contenant tous les termes entre $-20$ et $5$ d'une progression arithmétique de raison $0,3$ partant de $-20$.
\end{enumerate}

%\question{}
%\begin{enumerate}[label = \emph{\alph*)}]
%  \item Affecter à \texttt{v} la liste \texttt{[2,5,3,-1,7,2,1]}
%  \item Affecter à \texttt{L} la liste vide.
%  \item Vérifier le type des variables crées.
%  \item Calculer la longueur de \texttt{v}, affectée à \texttt{n} et celle de \texttt{L}, affectée à \texttt{m}.
%  \item Tester les expressions suivantes : \texttt{v[0]}, \texttt{v[2]}, \texttt{v[n]}, \texttt{v[n-1]}, \texttt{v[-1]} et \texttt{v[-2]}.
%  \item Changer la valeur du quatrième élément de \texttt{v}.
%  \item Que renvoie \texttt{v[1:3]} ? Remplacer dans \texttt{v} les trois derniers éléments par leurs carrés.
%  \item Que fait \texttt{v[1] = [0,0,0]} ? Combien d'éléments y a-t-il alors dans \texttt{v} ?
%\end{enumerate}


%\question{Quel type choisiriez-vous pour représenter les données suivantes ?}
%Vous justifierez brièvement chaque réponse. 
%
%\begin{enumerate}[label = \emph{\alph*)}]
%  \item Le nom d'une personne.
%  \item L'état civil d'une personne : nom, prénom, date de naissance, nationalité.
%  \item Les coordonnées d'un point dans l'espace.
%  \item L'historique du nombre de $5/2$ dans la classe de MP du lycée. 
%  \item Un numéro de téléphone. 
%  \item \emph{Plus difficile :} l'arbre généalogique de vos ancêtres. 
%\end{enumerate}
%=======
%% \question
%\'Evaluer les expressions suivantes.
%\begin{multicols}{2}
%  \begin{enumerate}[label=\emph{\alph*)}]
%    \item \texttt{4.3+2}
%    \item \texttt{2.5-7.3}
%    \item \texttt{42+4.}
%    \item \texttt{42+4}
%    \item \texttt{42.+4}
%    \item \texttt{12*0.}
%    \item \texttt{11.7*0}
%    \item \texttt{2,22/(1.6-2*0.8)}
%    \item \texttt{42/6}
%    \item \texttt{1,8/7}
%    \item \texttt{(447+3*6)/5}
%    \item \texttt{0/0}
%  \end{enumerate}
%\end{multicols}

