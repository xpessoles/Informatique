%%\exer{}
%\setcounter{numques}{0}

\question{Voici des affectations successives des variables $a$ et $b$. Dresser un tableau donnant les valeurs 
de $a$ et $b$ à chaque étape.}

\begin{lstlisting}
>>>  a = 1
>>>  b = 5
>>>  a = b-3 
>>>  b = 2*a 	
>>>  a = a
>>>  a = b	
\end{lstlisting}

\question{Écrire une séquence d'instructions qui échange les valeurs de deux variables $x$ et $y$.}

\question{Écrire, sans variable supplémentaire, une suite d'affectation qui permute circulairement vers la 
gauche les valeurs des variables $x$, $y$, $z$ : $x$ prend la valeur de $y$ qui prend celle de $z$ 
qui prend celle de $x$.}

%\question{Dans les cas où c'est possible, affecter les valeurs aux variables correspondantes à l'aide de l'interpréteur interactif. 
%On notera \texttt{var} $\leftarrow$ \texttt{a} pour dire que l'on affecte la valeur \texttt{a} à la variable \texttt{var}.}
%
%\begin{multicols}{3}
%  \begin{enumerate}[label=\emph{\alph*)}]
%    \item \texttt{ArthurDent} $\leftarrow$ \texttt{42}
%    \item \texttt{4} $\leftarrow$ \texttt{0.}
%    \item \texttt{L} $\leftarrow$ \texttt{[]}
%    \item \texttt{list} $\leftarrow$ \texttt{[1,2,3]}
%    \item \texttt{int} $\leftarrow$ \texttt{5}
%    \item \texttt{s} $\leftarrow$ \texttt{""}
%    \item \texttt{True} $\leftarrow$ \texttt{1}
%    \item \texttt{ok} $\leftarrow$ \texttt{ok}
%    \item \texttt{x} $\leftarrow$ \texttt{"x"}
%    \item \texttt{a} $\leftarrow$ \texttt{1<0}
%    \item \texttt{lam} $\leftarrow$ \texttt{1/0}
%    \item \texttt{or} $\leftarrow$ \texttt{"xor"}
%  \end{enumerate}
%\end{multicols}
%
%\question{On part des affectations suivantes : \texttt{a} $\leftarrow$ \texttt{5} et \texttt{b} $\leftarrow$ \texttt{0}. 
%Pour la suite d'instructions suivante, prévoir ligne à ligne le résultat affiché par l'interpréteur interactif de Python ainsi que l'état des variables.
%Le vérifier grâce à l'interpréteur interactif d'IDLE, en prenant soin de partir d'une nouvelle session.}
%\begin{lstlisting}
%a*b
%x = a**b + a
%print(x)
%print(y)
%z = x
%x = 5
%print(z)
%a = a+a**b
%print(a)
%\end{lstlisting}
%
%\question{Affecter des valeurs toutes différentes aux variables \texttt{a}, \texttt{b}, \texttt{c} et \texttt{d}.}
%
%\'A chaque fois, effectuer les permutations suivantes de manière naïve (c'est-à-dire, sans utiliser de \texttt{tuple}).
%\begin{enumerate}[label=\emph{\alph*)}]
%  \item \'Echanger les contenus de \texttt{a} et de \texttt{b}.
%  \item Placer le contenu de \texttt{b} dans \texttt{a}, celui de \texttt{a} dans \texttt{c} et celui de \texttt{c} dans \texttt{b}.
%  \item Placer le contenu de \texttt{a} dans \texttt{d}, celui de \texttt{d} dans \texttt{c}, celui de \texttt{c} dans \texttt{b} et celui de \texttt{b} dans \texttt{a}.
%\end{enumerate}
%Reprendre cet exercice en effectuant chaque permutation en une instruction à l'aide d'un \texttt{tuple}.
%
%\question{Combien d'affectations sont suffisantes pour permuter circulairement les valeurs des variables 
%$x_1,\cdots ,x_n$ sans utiliser de variable supplémentaire ? Et en utilisant autant de variables 
%supplémentaires que l'on veut ?}
%
%\question{Mêmes questions en remplaçant suffisantes par nécessaires.}
%
%\question{Supposons que la variable $x$ est déja affectée, et soit $n\in\N$. On veut calculer $x^n$ sans 
%utiliser la puissance, avec uniquement des affectations, autant de variables que l'on veut, mais 
%avec le moins de multiplications possible. Par exemple, avec les 4 instructions :}
%
%%\begin{lstlisting}
%%>>> y1 = x * x
%%>>> y2 = y1 * x
%%>>> y3 = y2 * x
%%>>> y4 = y3 * x
%%\end{lstlisting}
%
%on calcule $x^5$, qui est la valeur de \texttt{y4}.\\
%Mais 3 instructions suffisent :
%\begin{lstlisting}
%>>> y1 = x * x
%>>> y2 = y1 * y1
%>>> y3 = y2 * x
%\end{lstlisting}

%\question{En fonction de $n$, et avec les contraintes précédentes, quel est le nombre minimum d'instructions
%pour calculer $x^n$ ?}

\question{Calculer, sans utiliser la fonction \texttt{sqrt} ni la division flottante \texttt{/}, les nombres suivants.}
%\begin{multicols}{2}
  \begin{enumerate}[label=\emph{\alph*)}]
    \item $\dfrac{1}{7,9}$
    \item $\sqrt{6,2}$
    \item $\dfrac{1}{\sqrt{3,5}}$
    \item $2\sqrt{2}$
  \end{enumerate}
%\end{multicols}
De base, on ne peut réaliser que des calculs élémentaires avec Python. Cependant, il est possible d'avoir accès à des possibilités de calcul plus avancées en utilisant une \emph{bibliothèque}. 
Par exemple, la bibliothèque \texttt{math} permet d'avoir accès à de nombreux outils mathématiques. 
On peut donc taper
\begin{lstlisting}
  from math import sqrt, log, exp, sin, cos, tan, pi, e 
\end{lstlisting}

pour avoir accès à toutes ces fonctions. \\
Calculer les nombres suivants (on n'hésitera pas à consulter l'aide en ligne).
%\begin{multicols}{2}
  \begin{enumerate}[label=\emph{\alph*)}]
    \item $e^2$
    \item $\sqrt{13}$
    \item $\cos\left(\dfrac{\pi}{5}\right)$
    \item $e^{\sqrt{5}}$
    \item $\ln 2$
    \item $\ln 10$
    \item $\log_{2} 10$
    \item $\tan\left(\dfrac{\pi}{2}\right)$
  \end{enumerate}
%\end{multicols}

