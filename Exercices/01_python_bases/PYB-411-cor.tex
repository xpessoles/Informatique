\exer{}
\setcounter{numques}{0}

\question{} 
Indentation vraie.
\begin{lstlisting}

x=0
y=5
t=False
if x>=1:
    t=True
if y<=6:
    t=True
\end{lstlisting}

Indentation fausse.
\begin{lstlisting}

x=0
y=5
t=False
if x>=1:
    t=True
    if y<=6:
        t=True
\end{lstlisting}




\medskip{}


\question{} 
\begin{lstlisting}

from random import randrange
n= randrange(100) # Un entier aléatoire entre 0 et 99

if n <= 10:
    print("Trop petit")
elif n >= 50:
    print("Trop grand")
else:
    print("Juste comme il faut")
\end{lstlisting}



\medskip{}

\question{} 
\begin{lstlisting}

def inv(n):
    """Somme les inverses des n premiers entiers naturels non nuls"""
    s = 0
    for k in range(n):
        x = 1/(k+1)
        s = s+x
    return s
\end{lstlisting}
