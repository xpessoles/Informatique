\exer{}
\setcounter{numques}{0}

\question{Dans chaque cas, indiquez le type que vous utiliseriez pour modéliser les grandeurs suivantes dans leur contexte scientifique usuel.
Vous justifierez brièvement chaque réponse. }

\begin{enumerate}[label=\emph{\alph*)}]
  \item La taille d'un individu en mètres. 
  \item Le tour de taille d'un manequin, en millimètres.
  \item Le nombre d'Avogadro.
  \item Le nombre de Joules dans une calorie.
  \item Le nombre de secondes dans une année.
  \item Le plus grand nombre premier représentable avec 20 chiffres en écriture binaire.
\end{enumerate}