\exer{Factorielle $n!$}
On donne l'algorithme suivant. 

\begin{lstlisting}
for i in range(1,n+1):
    # en entrant dans le ième tour de boucle, p = (i-1)!
    p=p*i
    # en sortant du ième tour de boucle, p = i!
\end{lstlisting}


\question{Montrer que l'algorithme précédent permet de calculer $n!$.}
\ifprof
\begin{corrige}
Ici, l'invariant de boucle est << $p$ contient $(i-1)!$ >> : 
\begin{enumerate}
\item c'est bien une propriété qui est vraie pour $i=1$;
\item supposons qu'au rang i, $p=(i-1)!$ à l'entrée de la boucle. Au cours de la boucle, $p$ va prendre la valeur $p=(i-1)!\times i=i!=((i+1)-1)!$ donc la propriété est vérifiée en sortie de boucle;
\item enfin, au dernier tour de boucle, $i$ vaut $n$ donc $p=n!$ ce qui répond à la question.\\
\end{enumerate}
\end{corrige}
\else
\fi