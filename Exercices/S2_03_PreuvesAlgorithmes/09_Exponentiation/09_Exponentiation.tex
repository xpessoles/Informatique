\exer{Exponentiation rapide}
\begin{lstlisting}
def expo_rapide(x,n):
    y = x
    z = 1
    m = n
    # Invariant : x*y^m = x^n
    while m>0:
        q,r=m//2,m%2
        if r==1:
            z=z*y
        y=y*y
        m=q
    return z
\end{lstlisting}

\question{Montrer que la fonction permet de calculer $x^n$.}
\ifprof
\begin{corrige}
\textbf{A vérifier}

Juste avant la boucle, on a $y = x$; $z = 1$ et $m = n$. Remarquez qu’ainsi, $z\times y^m = x^n$. Vérifions que si cette propriété est vraie en haut du corps de boucle, alors elle est vérifiée en bas du corps de boucle
(là où se trouve le second). Si on se trouve en haut du corps de boucle, ceci signifie que la variable $m$ contient un entier
strictement positif. On a deux cas à examiner, suivant la parité de $m$. Notons $x'$, $y'$ et $m'$ les valeurs des variables
$z$; $y$; $m$ en bas de la boucle.
\begin{itemize}
\item si $m$ est pair, alors $q = \dfrac{m}{2}$, $r=0$, $y'=y^2$, et $m'=q=\dfrac{m}{2}z'=z$ est inchangé. On a alors
$z'\times y'^{m'}=z\times \left(y^2\right)^{\dfrac{m}{2}} = z\times y^m$. Puisque $x\times y^m$ valait $x^n$ en haut de la boucle, c’est toujours le cas en bas du corps de boucle.
\item si $m$ est impair, alors $q=\dfrac{m-1}{2}$, $r=1$. Dans ce cas, $z'$ prend la valeur $z\times y$, $y'=y^2$ et 
$m'=\dfrac{m-1}{2}$. On a alors $z' \times z'^{m'} = z\times y \times \left(y^2)^{\dfrac{m-1}{2}}  = z\times y^m$.
De même, puisque $z\times y^m$ valait $x^n$ en haut de la boucle, c’est toujours le cas en bas du corps de boucle.
\end{itemize}
Ainsi, la propriété $x\times y^m = x^n$ est maintenue à chaque passage dans la boucle, c’est donc un invariant de boucle.
On en déduit en particulier que cette propriété est vérifiée également après la sortie de la boucle. Rappelons que l’on
a montré qu’en sortie de boucle, m était nul. Or comme l’invariant est vérifié, cela signifie que $z=x^n$. Comme on renvoie $z$, l'algorithme renvoie bien $x^n$, et est donc correct.
\end{corrige}
\else
\fi