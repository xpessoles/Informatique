\exer{Multiplication}

L’objectif est de calculer le produit de deux nombres entiers positifs $a$ et $b$ sans utiliser de multiplication.
\begin{lstlisting}
p = 0
m = 0
while m < a :
    m = m + 1
    p = p + b
\end{lstlisting}

\question{Montrer la terminaison de l'algorithme.}
\ifprof
\begin{corrige}
Le programme se termine car la suite des valeurs de $m$ est une suite
d’entiers consécutifs strictement croissante, et atteint la valeur $a$ en $a$ étapes.
\end{corrige}
\else
\fi

\question{Proposer une propriété d'invariance.}
\ifprof
\begin{corrige}
Un invariant de boucle est ici : $p = m.b$.
\end{corrige}
\else
\fi


\question{Montrer la correction de l'algorithme.}
\ifprof
\begin{corrige}

Un invariant de boucle est ici : $p = m.b$.

\begin{itemize}
\item Initialisation : avant le premier passage dans la boucle, $p = 0$ et $m = 0$, donc $p = mb$.
\item Hérédité : supposons que $p = mb$ avant une itération; les valeurs de $p$ et $m$ après l’itération sont
$p'= p + b$ et $m' = m + 1$. Or $p'= (p + b) = m.b + b = (m + 1)b = m'b$. Donc la propriété
reste vraie.
\item Conclusion : à la sortie de la boucle $p = m.b$.
\end{itemize}
Puisqu’à la sortie de la boucle $m = a$, on a bien $p = ab$.
\end{corrige}
\else
\fi