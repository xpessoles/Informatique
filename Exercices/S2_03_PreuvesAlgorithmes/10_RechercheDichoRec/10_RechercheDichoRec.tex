\exer{Recherche dichotomique dans un tableau trié -- Formulation récursive}
% Svartz Massena
\begin{lstlisting}
def dicho_rec(L,x):
    """L liste triée dans l'ordre croissant, x un élément. On renvoie True si x est dans L, False sinon"""
    n=len(L)
    if n==0:
        return False
    m=n//2
    if L[m]==x:
        return(True)
    elif L[m]<x:
        return dicho_rec(L[m+1:],x) #la partie à droite de L[m].
    else:
        return dicho_rec(L[:m],x) #la partie à gauche.
\end{lstlisting}

\question{Montrer que la fonction termine.}
\ifprof
\begin{corrige}
La taille de la liste \texttt{L} est un variant de boucle.
\end{corrige}
\else
\fi

Soit la proposition suivante : 
\textit{<< Si \texttt{L} est une liste triée dans l’ordre croissant et \texttt{x} un élément comparable à ceux de \texttt{L}, alors \texttt{dicho\_rec(L,x)} retourne \texttt{True} si et seulement si \texttt{x} est dans \texttt{L}, \texttt{False} sinon.>>}.

\question{Montrer que la fonction termine.}
\ifprof
\begin{corrige}
\begin{itemize}
\item Initialisation : si la longueur du tableau est 1, alors la sortie de la fonction est correcte.
\item Hérédité : si \texttt{L} est de longueur au moins 2, un et un seul des cas suivants se produit :
\begin{itemize}
\item \texttt{L[m]} est égal à \texttt{x}, auquel cas la sortie de la fonction est correcte.
\item \texttt{L[m]} est inférieur strictement à \texttt{x}, auquel cas \texttt{x} ne peut se trouver qu’après \texttt{m} puisque \texttt{L} est trié dans l’ordre
croissant. Le tableau \texttt{L[m:]} correspond aux éléments placés après m (inclus), triés également dans l’ordre
croissant. Par hypothèse de récurrence, la sortie de la fonction \texttt{dicho\_rec} sur l’instance \texttt{(L[m:],x)} est
correcte, donc également sur \texttt{(L,x)}.
\item On procède de même,si \texttt{L[m]} est strictement supérieur à x avec le tableau \texttt{L[:m]}.
\end{itemize}
\item Par principe de récurrence, la fonction \texttt{dicho\_rec} est correcte.
\end{itemize}
\end{corrige}
\else
\fi