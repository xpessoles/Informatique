\exer{Recherche dichotomique dans une liste triée}
%Jules Svartz Masséna
%
Soit la fonction suivante.
\begin{lstlisting}
def recherche_dicho(L,x):
    """ La fonction prend en entrée une liste L triée dans l'ordre croissant et un élément x,
    et retourne True si x est dans L, False sinon."""
    g=0
    d=len(L)
    while g<d:
        #Inv: x ne se trouve ni dans L[0:g] ni dans L[d:len(L)].
        m=(g+d)//2
        if L[m]==x:
            return True
        elif L[m]<x:
            g=m+1
        else:
            d=m
        #Inv: x ne se trouve ni dans L[0:g] ni dans L[d:len(L)].
    return False
\end{lstlisting}

\question{Montrer la correction de la fonction \texttt{recherche\_dicho}.}
\ifprof
\begin{corrige}
\begin{itemize}
\item La terminaison de l’algorithme repose sur celle de la boucle \texttt{while} : la quantité $d-g$ est à valeurs dans $\mathbb{N}$ et
décroît strictement à chaque itération de la boucle : l’algorithme termine.
\item La correction repose elle aussi sur celle de la boucle \texttt{while}. On se rend compte facilement qu’elle admet l’invariant
indiqué : si \texttt{L[m]} est égal à \texttt{x}, on renvoie simplement \texttt{True} et la fonction est correcte. Sinon, si \texttt{L[m]<x}, comme
la liste est triée cela signifie que \texttt{x} ne peut se trouver qu’à un indice strictement supérieur à \texttt{m} et strictement
inférieur si \texttt{L[m]>x}.
\item  Après la boucle, comme $g\geq d$ (en fait, $g = d$), l’invariant assure que \texttt{x} ne se trouve ni dans \texttt{L[0:g]} ni dans
\texttt{L[d:len(L)]} donc en fait pas dans \texttt{L}. On renvoie \texttt{False} et la fonction est correcte.
\end{itemize}
\end{corrige}
\else
\fi