\exer{Maximum d'une liste}
Soit la fonction suivante.
\begin{lstlisting}
def maximum(L):
    """ La fonction prend en entrée une liste L non vide de flottants ou d'entiers,
    et retourne le maximum de ses éléments. """
    m=L[0]
    for i in range(1,len(L)):
        #Inv(i): m est le plus grand élément de L[0:i].
        if L[i]>m:
            m=L[i]
        #Inv(i+1): m est le plus grand élément de L[0:i+1].
    return m
\end{lstlisting}

\question{Montrer la terminaison et la correction de la fonction \texttt{maximum}}
\ifprof
\begin{corrige}~\\
\begin{lstlisting}
#Inv(i): m est le plus grand élément de L[0:i].
#Inv(i+1): m est le plus grand élément de L[0:i+1].
\end{lstlisting}
\end{corrige}
\else
\fi