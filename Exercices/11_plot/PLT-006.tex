\exer{[PLT-005]}
\setcounter{numques}{0}~\\



On s'intéresse maintenant à l'approximation d'un signal périodique par des fonctions trigonométriques. 
On considère sur $\R$ la fonction créneau, impaire et périodique de période 2, définie par $C(1) = 0$ et sur $\left]0,1\right[$ par 
\begin{equation*}
  C:t\mapsto 1.
\end{equation*}
Soit aussi la fonction triangle, définie sur $\R$, paire et périodique de période $2$, définie sur $[0,1]$ par  
\begin{equation*}
 T:t\mapsto 1-2t.
\end{equation*}
Ces deux fonctions sont représentées sur la figure~\ref{TP05:fig:creneau_triangle}.
% \begin{minipage}{0.5\textwidth}
\begin{figure}[!h]
  \begin{center}
    \begin{tikzpicture}
      \draw[->] (-2.5,0) -- (2.5,0);
      \draw[->] (0,-1.5) -- (0,1.5);
      \draw (-0.1,1) node[anchor=east] {$1$} -- (0.1,1);
      \draw (0.1,-1) node[anchor=west] {$-1$} -- (-0.1,-1);
      \draw (1,0) node[anchor = north] {$1$};
      \draw (-1,0) node[anchor = north] {$-1$};
      \draw[-(,blue,thick] (-2.5,-1) -- (-2,-1);
      \draw[)-(,blue,thick] (-2,1) -- (-1,1);
      \draw[)-(,blue,thick] (-1,-1) -- (0,-1);
      \draw[)-(,blue,thick] (0,1) -- (1,1);
      \draw[)-(,blue,thick] (1,-1) -- (2,-1);
      \draw[)-,blue,thick] (2,1) -- (2.5,1);
      \draw[blue,thick] (-2,0) node {\textbullet};
      \draw[blue,thick] (-1,0) node {\textbullet};
      \draw[blue,thick] (0,0) node {\textbullet};
      \draw[blue,thick] (1,0) node {\textbullet};
      \draw[blue,thick] (2,0) node {\textbullet};
      \draw[->] (4.5,0) -- (9.5,0) ;
      \draw[->] (7,-1.5) -- (7,1.5) ;
      \draw (6.9,1) node[anchor=east] {$1~$} -- (7.1,1);
      \draw (7.1,-1) node[anchor=west] {$-1$} -- (6.9,-1);
      \draw (8,0.1) node[anchor = south] {$1$} -- (8,-0.1);
      \draw (6,0.1) node[anchor = south] {$-1$} -- (6,-0.1);
      \draw[red,thick] (4.5,0) -- (5,1) -- (6,-1) -- (7,1) -- (8,-1) -- (9,1) -- (9.5,0);
    \end{tikzpicture}
    \caption{Signal créneau et signal triangulaire.}
    \label{TP05:fig:creneau_triangle}
  \end{center}
\end{figure}
% \end{minipage}


On peut montrer que, en tout réel $t$ où $C$ est continue, 
\begin{equation*}
  C(t) = \dfrac{4}{\pi}\sum_{p=0}^{+\infty} \dfrac{1}{2p+1} \sin((2p+1)\pi t).
\end{equation*}
De même, pour tout $t\in\R$, 
\begin{equation*}
  T(t) = \dfrac{8}{\pi^2}\sum_{p=0}^{+\infty} \dfrac{1}{(2p+1)^2} \cos((2p+1)\pi t).
\end{equation*}
Ce symbole $\displaystyle \sum_{p=0}^{+\infty}$ doit être vu comme une limite et sera défini dans le cours de mathématiques. 
On approche alors, pour tout $t\in\R$, $C(t)$ et $T(t)$ respectivement par les sommes 
\begin{equation*}
  C_n(t) = \dfrac{4}{\pi}\sum_{p=0}^{n} \dfrac{1}{2p+1} \sin((2p+1)\pi t)
\end{equation*}
et 
\begin{equation*}
   T_n(t) = \dfrac{8}{\pi^2}\sum_{p=0}^{n} \dfrac{1}{(2p+1)^2} \cos((2p+1)\pi t).
\end{equation*}
Les physiciens appellent le terme en $p=0$ le \emph{fondamental} et les autres termes les \emph{harmoniques}. 


\medskip{}

\question{} \'Ecrire une fonction \texttt{creneau(t)} renvoyant la valeur de $C(\texttt{t})$ (pour \texttt{t} réel). 

\emph{Indication :} on pourra considérer la parité de la partie entière de \texttt{t}.


\medskip{}

\question{} \'Ecrire une fonction \texttt{sp\_creneau(n,t)} renvoyant la valeur de $C_{\texttt{n}}(\texttt{t})$ (pour \texttt{n} entier et \texttt{t} réel). 


\question{}\label{tp05:qu:creneau} \'Ecrire une fonction \texttt{fourier\_creneau(nom\_de\_fichier)} ne renvoyant rien 
et enregistrant dans \texttt{nom\_de\_fichier} le graphe sur l'intervalle $[0,4]$ de $C$, celui 
de son fondamental (\emph{i.e}, $C_0$), ainsi que ceux de $C_3$, $C_5$ et $C_{100}$.
Vous produirez une image, que vous enverrez à votre enseignant. 

\bigskip

\emph{Pour les plus rapides, voici des questions supplémentaires.}

\bigskip



\question{} \'Ecrire une fonction \texttt{triangle(t)} renvoyant la valeur de $T(\texttt{t})$ (pour 
\texttt{t} réel).


\medskip{}

\question{} \'Ecrire une fonction \texttt{sp\_triangle(n,t)} renvoyant la valeur de $T_{\texttt{n}}(\texttt{t})$ (pour \texttt{n} entier et \texttt{t} réel). 


\medskip{}

\question{}\label{tp05:qu:triangle} \'Ecrire une fonction \texttt{fourier\_triangle(nom\_de\_fichier)} ne renvoyant rien et enregistrant dans \texttt{nom\_de\_fichier} le graphe sur l'intervalle $[0,4]$ de $T$, celui de son fondamental (\emph{i.e}, $T_0$), ainsi que ceux de $T_3$, $T_5$ et $T_{100}$.
Vous produirez une image, que vous enverrez à votre enseignant. 