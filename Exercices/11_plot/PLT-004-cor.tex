%\exer{[PLT-004]}
%\setcounter{numques}{0}~\\

En préambule : 
\begin{lstlisting}
import matplotlib.pyplot as plt
from numpy import exp, linspace, pi, sin, cos
from math import floor
\end{lstlisting}


\question{}
\texttt x est une liste python, donc de type \texttt{list}.

\question{}
Python représente une fonction comme une ligne brisée. 
On indique les coordonnées des extrémités des segments en passant en argument la liste des abscisses et celle des ordonnées à la fonction plot.

\question{}
Le tracé s'arrete au point d'abscisse 9,5. C'est bien le dernier élément de x.

\question{}
\begin{lstlisting}
def ex_sin(nom\_de\_fichier):
    """Trace la courbe du sinus sur [0,10] et l'enregistre dans nom\_de\_fichier"""
    x = linspace(0,10,200)
    y = [sin(t) for t in x]

    plt.clf()
    plt.plot(x,y,label='sin(x)')
    plt.xlabel('x')
    plt.legend(loc=0)
    plt.title('Tracé du sinus sur [0,10]')
    plt.savefig(nom\_de\_fichier)
\end{lstlisting}
\question{}
\begin{lstlisting}
def transitoire(A,nom\_de\_fichier):
    """Trace les graphes de t->A(1-exp(t/tau)) pour tau = 2,4,6,8 sur [0,10].
       Les sauvegarde dans nom\_de\_fichier."""
    x = linspace(0,10,200)
    tau = [0.5,1,2,4,8]
    style = ['g-','b-','r-','b--','r--']
    plt.clf()
    for k in range(5):
       y = [A*(1-exp(-t/tau[k])) for t in x]
       plt.plot(x,y,style[k],label='$\\tau='+str(tau[k])+'$') 
    plt.xlabel('$t$')
    plt.ylabel('$'+str(A)+'\\times\\exp(-t / \\tau)$')
    plt.title('Régime transitoire, A={}'.format(A))
    plt.axis([0,10,0,A])
    plt.legend(loc=0)
    plt.savefig(nom\_de\_fichier)
    return None
\end{lstlisting}