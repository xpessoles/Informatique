\begin{Verbatim}[gobble=0,numbers=left]
import matplotlib.pyplot as plt
import numpy as np

x=np.linspace(0,6*np.pi,100)
y1=np.sin(x)
y2=np.exp(-0.1*x)
y3=y1*y2

plt.clf()
plt.plot(x,y1,'b--',label='$\\sin(x)$ en bleu',linewidth=2)
plt.plot(x,y2,'r-.',label='$\\exp(-0,01\cdot x)$ en rouge',linewidth=2)
plt.plot(x,-y2,'g-.',label='$-\\exp(-0,01\cdot x)$ en vert',linewidth=2)
plt.plot(x,y3,'k-',label='$\\sin(x)\\cdot \\exp(0,1\\cdot x)$ en noir',linewidth=2)
plt.xlabel('x')
plt.ylabel('y')
plt.legend(loc='0')
plt.title('Fonction pseudo-périodique')
plt.savefig('pseudo-harmonique.png')
\end{Verbatim}
On pouvait aussi définir les variables \texttt{y1}, \texttt{y2} \emph{etc.} de la manière suivante. 
\begin{verbatim}
y1 = [np.sin(t) for t in x]
y21 = [np.exp(-0.1*t) for t in x]
y22 = [-np.exp(-0.1*t]) for t in x]
y3 = [np.sin(t)*np.exp(-0.1*t) for t in x]
\end{verbatim}
Les fonctions sinus, exponentielle et la constante $\pi$ pouvaient aussi être importées depuis la bibliothèque \texttt{math}.