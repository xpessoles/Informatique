\usepackage[utf8]{inputenc}   % permet l'utilisation de caractres ISO
\usepackage[french]{babel}    % choix du francais
\usepackage{fancyhdr}       % ncessaire pour \pagestyle{fancy}
\usepackage{fancybox}
\usepackage[dvips]{graphicx}       % gestion des images
\usepackage{placeins}   % gestion de la position des images
\usepackage{amsmath}        % modification des polices mathmatiques
\usepackage{amssymb}
\usepackage{mathrsfs}
\usepackage[cyr]{aeguill}
\usepackage{enumitem}
\usepackage{pifont}
\usepackage{psfrag}
\usepackage{epsfig}
\usepackage{supertabular}
\usepackage{array}
\usepackage{multicol}
\usepackage{multirow} % Pour faire des tableaux contenant des lignes fusionnes
\usepackage{hyperref}
\usepackage{color}
\usepackage[dvipsnames]{pstricks}
\usepackage{pstricks-add,pst-plot,pst-node}
\usepackage[dvipsnames]{xcolor}
\usepackage{enumerate}
\usepackage{float}
\usepackage{titlesec}
\usepackage{textcomp}
\usepackage{subfigure}
\usepackage{sectsty}
\usepackage{pdfpages}
\usepackage{etex} % permet de charger plus de package 
\usepackage{listings}

%===============================
% Mise en page chapitre ligne verticale TD
%------------------------------------------
\makeatletter
\def\thickhrulefill{\leavevmode \leaders \hrule height 1ex \hfill \kern \z@}
\def\@makechapterhead#1{%
  \vspace*{10\p@}%
  {\parindent \z@ 
        \raggedleft
        \reset@font\huge\bfseries
        \begin{tabular}{l|p{13cm}}
         \textcolor{rltred}{ { TD  }}
          &\
          \Huge #1
        \end{tabular}
        \par\nobreak
    \vskip 70\p@
  }}

%===============================
% Definition des couleurs 
%------------------------------------------
\definecolor{rltred}{rgb}{0.75,0,0}    
\definecolor{rltgreen}{rgb}{0,0.5,0}
\definecolor{oneblue}{rgb}{0,0,0.6}
\definecolor{marron}{rgb}{0.64,0.16,0.16}
\definecolor{forestgreen}{rgb}{0.13,0.54,0.13}
\definecolor{purple}{rgb}{0.62,0.12,0.94}
\definecolor{dockerblue}{rgb}{0.11,0.56,0.98}
\definecolor{freeblue}{rgb}{0.25,0.41,0.88}
\definecolor{freeblue2}{rgb}{0.25,0.41,0.77}
\definecolor{myblue}{rgb}{0,0.2,0.4}

%===============================
% Definition de l'affichage des sections et sous sections
%------------------------------------------
\renewcommand{\sectionmark}[1]{\markright{\thesection\ #1}}
\renewcommand{\headrulewidth}{0.4pt}   % trait horizontal en dessous de l'entete
\renewcommand{\footrulewidth}{0.4pt}     % trait horizontal au dessus du pied de page
\renewcommand{\baselinestretch}{1.1} % augmente l'interligne  1.1
    
\renewcommand{\thesection}{\Roman{section}}
\renewcommand{\thechapter}{\arabic{chapter}}

\usepackage{geometry}
\geometry{ hmargin=3cm, vmargin=2.5cm }
\headheight=15pt
\textwidth=170mm \oddsidemargin=-4mm %\evensidemargin=0mm

%% Profondeur de \subsubsection = 3
\setcounter{tocdepth}{2} % pour que les subsubsection apparaissent dans la table des matieres
\setcounter{secnumdepth}{3} % pour que les subsubsection soient numrotes

% Pouvoir afficher des URL correctement
\usepackage{url}
\let\urlorig\url
\renewcommand{\url}[1]{%
  \begin{otherlanguage}{english}\urlorig{#1}\end{otherlanguage}%
}


%===============================
% Définition de nouvelles commandes de listes
%------------------------------------------
\newlist{coche}{itemize}{2}
\setlist[coche, 1]{font=\color{rltred} , label=\ding{51}}
\setlist[coche, 2]{font=\color{rltred}}

\newlist{point}{itemize}{2}
\setlist[point, 1]{label=\textbullet}
\setlist[point, 2]{label=--}

\newlist{tiret}{itemize}{2}
\setlist[tiret, 1]{label=--}
\setlist[tiret, 2]{label=-}

%===============
% Environnements 
%--------------------
\newcommand{\Obj}[1]{\noindent
\rule[0.1cm]{1cm}{0.8pt} \textbf{Objectif} \rule[0.1cm]{14.1cm}{0.8pt}\\
#1\\
\rule[0.1cm]{17cm}{0.8pt} }

\newcommand{\Pbm}[1]{\noindent
\textcolor{rltred}{\rule[0.1cm]{1cm}{0.8pt}} \textbf{Problématique technique} \textcolor{rltred}{\rule[0.1cm]{11cm}{0.8pt}}\\
#1\\
\textcolor{rltred}{\rule[0.1cm]{17cm}{0.8pt}}}
   
\newcounter{Qu}
\addtocounter{Qu}{+1}
\newcommand{\Q}{\noindent \textbf{Q\theQu .\hspace{0,7mm}}\addtocounter{Qu}{+1}}

\newcounter{Exem}
\addtocounter{Exem}{+1}
% \newcommand{\Exem}{\noindent \textbf{Exemple: \hspace{0,7mm}}\addtocounter{Exem}{+1}}
\newcommand{\Exem}[1]{\noindent \fcolorbox{lightgray}{lightgray}{\begin{minipage}[c]{16.7cm}\textbf{Exemple\hspace{0,8mm} \theExem \hspace{0,8mm}}\addtocounter{Exem}{+1}:  \textit{#1}\end{minipage}}}


\newcounter{Ex}
\addtocounter{Ex}{+1}
\newcommand{\E}{\noindent \textbf{Ex\theEx .\hspace{0,7mm}}\addtocounter{Ex}{+1}}

\newcounter{Prop}
\addtocounter{Prop}{+1}
\newcommand{\Prop}[1]{\noindent \fcolorbox{lightgray}{lightgray}{\begin{minipage}[c]{16.7cm}\textbf{Propriété\hspace{0,8mm} \theProp \hspace{0,8mm}}\addtocounter{Prop}{+1} \textit{#1}\end{minipage}}}

\newcounter{Def}
\addtocounter{Def}{+1}
\newcommand{\Def}[1]{\noindent \fcolorbox{lightgray}{lightgray}{\begin{minipage}[c]{16.75cm}\textbf{Définition\hspace{0,8mm} \theDef \hspace{0,8mm}}\addtocounter{Def}{+1} \textit{#1}\end{minipage}}}

\newcounter{Thm}
\addtocounter{Thm}{+1}
\newcommand{\Thm}[1]{\noindent \fcolorbox{lightgray}{lightgray}{\begin{minipage}[c]{16.7cm}\textbf{Théorème\hspace{0,8mm} \theThm \hspace{0,8mm}}\addtocounter{Thm}{+1} \textit{#1}\end{minipage}}}

\newcommand{\Man}[1]{\noindent
\textcolor{blue}{\rule[0.1cm]{1cm}{0.8pt}} \textcolor{blue}{\textbf{Manipulation}} \textcolor{blue}{\rule[0.1cm]{13.2cm}{0.8pt}}\\ \textcolor{blue}{#1}\\
\textcolor{blue}{\rule[0.1cm]{17cm}{0.8pt}}}


\definecolor{gris25}{gray}{0.75}
\definecolor{bleu}{RGB}{18,33,98}
\definecolor{bleuf}{RGB}{42,94,171}
\definecolor{bleuc}{RGB}{231,239,247}
\definecolor{rougef}{RGB}{185,18,27}
\definecolor{rougec}{RGB}{255,230,231}
\definecolor{vertf}{RGB}{103,126,82}
\definecolor{vertc}{RGB}{220,255,191}
\definecolor{violetf}{RGB}{112,48,160}
\definecolor{violetc}{RGB}{230,224,236}
%

\newenvironment{py}[1][\hsize]%
{%
    \def\FrameCommand%
    {%
%\rotatebox{90}{\textit{\textsf{Python}}} 
\rotatebox{90}{\includegraphics[height=.6cm]{../images/logo_python}} 
        {\color{violetf}\vrule width 3pt}%
        \hspace{0pt}%must no space.
        \fboxsep=\FrameSep\colorbox{violetc}%
    }%
    \MakeFramed{\hsize #1 \advance\hsize-\width\FrameRestore}%
}%
{\endMakeFramed}%


%  Code python
%%%%%%%%%%%%%%%%%%%%%%%%%%%%

\definecolor{fond}{rgb}{1,1,1}			%Couleur du fond
\definecolor{commentaires}{rgb}{0.5,0.5,0.5}	%Couleur des commentaires
\definecolor{chaines}{rgb}{0,0.63,0}		%Couleur des chaines de caracteres
\definecolor{fonctions}{rgb}{0,0,1}		%Couleur des fonctions de bases
\definecolor{decoration}{rgb}{0.5,0.5,0.5}		%Couleur des fonctions autre (??)
\definecolor{self}{rgb}{0,0,0}		%je sais pas. J'ai recopié d'internet
\definecolor{numeros}{rgb}{0.7,0.7,0.7}		%numéros de ligne
\definecolor{chevrons}{rgb}{1,0.,0.}		%chevrons


\lstset{
	%Langage
		language=Python,	%Langage par défaut
	%Mise en forme de l'environnement
		xleftmargin=2em,	%marge a gauche
		%frame=trBL,		%Cadre (double cadre)
		backgroundcolor=\color{fond},	%Couleur d'arriere plan
		framexleftmargin=5mm,
		frame=shadowbox,
		rulesepcolor=\color{black},
		texcl=true,	%Activle les commandes LaTex dans le code
		%escapechar=ø,	%Caractere a echapper
		escapeinside={(*}{*)},
	%Numeros de ligne
		numbers=left,	%Rajoute des numéros de ligne
		numberstyle=\footnotesize\color{numeros},	%Taille des numéros de ligne
		numbersep=1em,			%Marge entre le code et les numéros de ligne
	%Mise en forme du texte
		columns=fixed,		%Largeur des lettres (fixed, flexible, fullflexible)
		basicstyle=\ttfamily,	%Mise en forme de base
		showspaces=false,	%Pour mettre des especes d'underscore sur les espaces
		tabsize=8,	%Taille des tabulations	
	%Commentaires
		commentstyle=\color{commentaires}\slshape,	%Couleur des commentaires
	%Chaine de caracteres
		stringstyle=\color{chaines},
		showstringspaces=false,
		morecomment=[s][\color{chaines}]{"""}{"""},
		morecomment=[s][\color{chaines}]{'''}{'''},
	%Fonctions et mot clé de python
		morekeywords={import,from,class,def,for,while,if,is,in,elif,else,not,and,or,print,break,continue,return,True,False,None,access,as,del,except,exec,finally,global,import,lambda,pass,print,raise,try,assert},
		keywordstyle=\color{fonctions}\bfseries,
		morekeywords={[3]>>>},
		keywordstyle={[3]\color{chevrons}\bfseries},
		morekeywords={[2]@invariant},
		keywordstyle={[2]\color{decoration}\slshape},
		emph={self},
		emphstyle={\color{self}\slshape}
	}
\lstnewenvironment{codePython}
		{	\footnotesize\setbox1=\vbox
			\bgroup
		}
		{	\egroup	
			\begin{center}
				\begin{minipage}{0.8\linewidth}
					%\begin{bclogo}[couleur=white,logo=\bcoutil,noborder = true]{Code Python}
						\box1
					%\end{bclogo}
				\end{minipage}
			\end{center}
		}

%%%%%%%%%%%%%%%%%%%%%%%%%%%%%%%%%%%%%%%%%%
% PseudoCode
%%%%%%%%%%%%%%%%%%%%%%%%%%%%%%%%%%%%%%%%%%

\definecolor{fond_pseudo}{RGB}{250,250,250} %Couleur du fond
\definecolor{bord_pseudo}{RGB}{0,0,0} %Couleur du fond

\newcounter{cptPseudo}

\newenvironment{pseudoCode}[1][]	{\refstepcounter{cptPseudo}
				\begin{center}
					\begin{minipage}{0.95\linewidth}
						\begin{bclogo}[couleur=fond_pseudo,couleurBord=bord_exemple,arrondi=0.2,logo=\bcbook]{pseudo code \ : \emph{#1}}}
				{		\end{bclogo}
					\end{minipage}
				\end{center}}		
		
		
% \lstnewenvironment{codepseudo}
% 		{	\footnotesize\setbox1=\vbox
% 			\bgroup
% 		}
% 		{	\egroup	
% 			\begin{center}
% 			\begin{minipage}{0.05\linewidth}
% 			 {\includegraphics[angle=90,width=0.45\linewidth]{../image/pseudo_code_image.png}}
% 			\end{minipage}\begin{minipage}{0.7\linewidth}
% 					%\begin{bclogo}[couleur=white,logo=\bcoutil,noborder = true]{Code Python}
% 						\box1
% 					%\end{bclogo}
% 				\end{minipage}
% 			\end{center}
% 		}		
\lstnewenvironment{codePseudo}
		{	\footnotesize\setbox1=\vbox
			\bgroup
		}
		{	\egroup	
			\begin{center}
				\begin{minipage}{0.8\linewidth}
					%\begin{bclogo}[couleur=white,logo=\bcoutil,noborder = true]{Code Python}
						\box1
					%\end{bclogo}
				\end{minipage}
			\end{center}
		}
		
		
		
		
		
		
\lstnewenvironment{pseudo}[1][]{
% 		\rotatebox{90}{\textit{\textsf{Pseudo Code}}} 
\lstset{
language=pseudocode,
basicstyle=\sffamily\footnotesize, 	
stringstyle=\color{red}, 
showstringspaces=false, 
alsoletter={1234567890},
otherkeywords={\ , \}, \{},
keywordstyle=\color{blue},
emph={pour, faire, varie, si, tant que, et, ou, Fonction, alors, sinon, fin si},
emphstyle=\color{black}\bfseries,
emph={[2]True, False, None, self},
emphstyle=[2]\color{olive},
emph={[3]from, import, as},
emphstyle=[3]\color{blue},
upquote=true,
columns=flexible, % pour empecher d'avoir un espacement mono
morecomment=[s]{"""}{"""},
commentstyle=\color{Hellbraun}\backslash\backslash,
%emph={[4]1, 2, 3, 4, 5, 6, 7, 8, 9, 0},
emphstyle=[4]\color{blue},
literate=*{:}{{\textcolor{blue}:}}{1}
{=}{{\textcolor{blue}=}}{1}
{-}{{\textcolor{blue}-}}{1}
{+}{{\textcolor{blue}+}}{1}
{*}{{\textcolor{blue}*}}{1}
{!}{{\textcolor{blue}!}}{1}
{(}{{\textcolor{blue}(}}{1}
{)}{{\textcolor{blue})}}{1}
{[}{{\textcolor{blue}[}}{1}
{]}{{\textcolor{blue}]}}{1}
{<}{{\textcolor{blue}<}}{1}
{>}{{\textcolor{blue}>}}{1},
%framexleftmargin=1mm, framextopmargin=1mm, frame=shadowbox, rulesepcolor=\color{blue},#1
%backgroundcolor=\color{SourceHintergrund}, 
%framexleftmargin=1mm, framexrightmargin=1mm, framextopmargin=1mm, frame=single, framerule=1pt, rulecolor=\color{black},#1
}}{}

				