\question{} Donner une approximation de $\int_{\alpha}^{\alpha+1} \cos(\sqrt{t})\, \dd t$ avec la méthode des trapèzes et 1000 subdivisions.


\question{}  Donnez une approximation (à $10^{-5}$ près) de l'unique réel positif solution de
  l'équation $x^{2}+\sqrt{x}-10 = \alpha$ avec la méthode de votre choix.
  
  \question{}
  Donnez le nombre d'itérations nécessaires pour obtenir ce résultats avec la méthode de Newton en prenant pour valeur initiale $\alpha$.
  
\question{}
  Donnez le nombre d'itérations nécessaires pour obtenir ce résultats avec la méthode de Dichotomie sur l'intervalle $\left[0,12+\alpha\right]$.



\question{}
  Donnez à l'aide une approximation (à $10^{-5}$ près) de l'unique réel positif $t$ tel que $    \int_{\alpha}^{\alpha+t} (2+\sqrt{x}+\cos x)\, \dd x = 10$.
  \textit{Pour l'intégration numérique, on pourra utiliser la méthode des trapèzes avec 1000 subdivisions.}
  


\question{}
  Donner une valeur approchée de $x(1)$ avec $x$ l'unique fonction
  vérifiant $x(0)=\alpha$ et pour tout $t\in \R$, $x'(t) = 3\cos(x(t))
  + t$.


\question{}
  Donner une valeur approchée de $x\left(1+\frac{\alpha}{10}\right)$ avec $x$ l'unique fonction
  vérifiant $x(0)=0$, $x'(0)=0$ et pour tout $t\in \R$, $x''(t) = 1 + \sin(t+x(t))$.

%%\question{}
%%  Donner une valeur approchée de $x(1+\frac{\alpha}{10})$ avec $x$ l'unique fonction
%%  vérifiant $x(0)=0$, $x'(0)=1$ et pour tout $t\in \R$, $x''(t) = 1 + \sin(t+x(t))$.
%%
\question{}
  Donner une approximation de $\beta\in \R$, pour que l'unique
  solution de l'équation différentielle non linéaire $x''(t)=
  1+\arctan(t+x(t))$ avec les conditions initiales $x(0)=0$ et
  $x'(0)=\beta$ vérifie    $x\left(1+\frac{\alpha}{10}\right) = 1 + \frac{2}{3}\alpha$.
