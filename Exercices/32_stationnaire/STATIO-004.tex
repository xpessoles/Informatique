\exer{[SATIO-004]}
\setcounter{numques}{0}~\\
\subsection*{Introduction}

On s'intéresse ici au problème de régression par moindres carrés, que l'on envisagera dans le cas particulier d'un problème de cinétique chimique. 

On souhaite modéliser l'évolution de la concentration d'un réactif, en fonction du temps. On a 
mesuré, expérimentalement, les données contenues dans la table~\ref{table:donnees}.
\begin{figure}[!h]
  \begin{center}
    \setlength\extrarowheight{2.5pt}
    \begin{tabular}{|c|c|c|c|c|c|c|c|}
      \hline
      Temps ($s$) & 0 & 7 & 18 & 27 & 37 & 56 & 102 \\[2.5pt]
      \hline
      Concentration ($\textrm{mol}.\textrm{L}^{-1}$) & 34,83 & 32,14 & 28,47 & 25,74 & 23,14 & 18,54 & 11,04\\[2.5pt]
      \hline
    \end{tabular}
    \caption{Données expérimentales : concentration du réactif en fonction du temps.}
    \label{table:donnees}
  \end{center}
\end{figure}

\begin{rem}
  En préambule de votre script, vous aurez intérêt à créer deux listes, une pour les temps, une pour les concentrations, de la manière suivante. 
\begin{lstlisting}
T = [0, 7, 18, 27, 37, 56, 102]
C = [34.83, 32.14, 28.47, 25.74, 23.14, 18.54, 11.04]
\end{lstlisting}

\end{rem}


On se donne donc une suite de sept mesures de temps $(t_i)_{i=0}^6$ ainsi que de sept mesures de concentration $(C_i)_{i=0}^6$. 
L'objectif est de trouver une fonction $f$ telle que les points $f(t_i)$ sont proches des $C_i$. 
Pour quantifier cela, on considèrera le critère des moindres carrés, ce qui revient à considérer la quantité 
\begin{equation}\label{eq.mc}\tag{MC}
  \cL(f) = \dfrac{1}{7}\sum_{i=0}^6 (C_i - f(t_i))^2 .
\end{equation}
Il exite une infinité de fonctions qui annulent cette quantité (penser par exemple à l'interpolation de Lagrange) et qui n'ont aucun rapport avec notre problème. 
Nous allons donc nous restreindre à des ensembles réduits de fonctions : c'est ce que l'on appellera le \emph{modèle}.

\subsubsection*{Modèle cinétique d'ordre 1.}

Dans ce modèle, on considère que la concentration vérifie l'équation différentielle 
\begin{equation*}
   C'(t) = -k C(t) \textrm{ avec }k>0 \textrm{, qui admet pour solution } C:t\mapsto C(0) \e^{-kt}.
\end{equation*}

Pour simplifier, on suppose que $C(0) = C_0 = 34,83~\textrm{mol}.\textrm{L}^{-1}$. On considère donc le modèle
\begin{equation*}
  \mathscr{M}_1 = \left\{ f^1_k :t\mapsto C_0 e^{-kt} ~\big|~ k>0 \right\}.   
\end{equation*}
On cherche donc une fonction dans $\mathscr{M}_1$ minimisant \eqref{eq.mc}, ce qui revient à chercher un paramètre $k>0$ minimisant 
\begin{equation*}
  L^1(k) = \dfrac{1}{7} \sum_{i=0}^6 \left(C_i - C_0 \e^{-k t_i}\right)^2.
\end{equation*}

\subsubsection*{Modèle cinétique d'ordre 2.}

Dans ce modèle, on considère que la concentration vérifie l'équation différentielle 
\begin{equation*}
   C'(t) = -k C^2(t) \textrm{ avec }k>0 \textrm{, qui admet pour solution } C:t\mapsto \dfrac{C(0)}{C(0)kt + 1}.
\end{equation*}

Pour simplifier, on suppose que $C(0) = C_0 = 34,83~\textrm{mol}.\textrm{L}^{-1}$. On considère donc le modèle
\begin{equation*}
  \mathscr{M}_2 = \left\{ f^2_k :t\mapsto  \dfrac{C_0}{C_0kt + 1} ~\big|~ k>0 \right\}.   
\end{equation*}
On cherche donc une fonction dans $\mathscr{M}_2$ minimisant \eqref{eq.mc}, ce qui revient à chercher un paramètre $k>0$ minimisant 
\begin{equation*}
  L^2(k) = \dfrac{1}{7} \sum_{i=0}^6 \left(C_i -\dfrac{C_0}{C_0kt_i + 1}\right)^2.
\end{equation*}

\subsection*{Rendu graphique.}


\question{} \'Ecrire une fonction \texttt{trace\_fonction(xmin,xmax,f,nom\_de\_fichier)} qui trace la courbe de la fonction \texttt{f} de \texttt{xmin} à \texttt{xmax}, puis sauvegarde le résultat dans le fichier \texttt{nom\_de\_fichier}.
  
    Par exemple, les commandes successives
\begin{lstlisting}
xmin = 0
xmax = 20
def f(x) : 
    return 0.02* x*(x-5)
trace_fonction(xmin,xmax,f,'fig_ex_fonction.png')
\end{lstlisting}
 devront produire (et sauvegarder) un graphique semblable (vous n'essaierez pas dans un premier temps de reproduire les titres, légendes etc.) à \texttt{fig\_ex\_fonction.png}, que vous trouverez sur le site de la classe.
    
\question{} \'Ecrire une fonction \texttt{trace\_ajustement(X,Y,f,nom)} qui trace un nuage de points 
dont les abscisses sont données dans le vecteur \texttt{X} et les ordonnées dans le vecteur 
\texttt{Y}, qui superpose la courbe de la fonction \texttt{f} pour des arguments allant de 
\texttt{min(X)} à \texttt{max(X)} et qui sauvegarde l'image produite dans le fichier \texttt{nom}.
    
    Par exemple, les commandes successives 
\begin{lstlisting}
import numpy as np
T = [0, 7, 18, 27, 37, 56, 102]
C = [34.83, 32.14, 28.47, 25.74, 23.14, 18.54, 11.04]
def g(x) : 
    return 34 - 0.2*x
trace_ajustement(T,C,g,'fig_ex_ajustement.png')
\end{lstlisting}
    devront produire (et sauvegarder) un graphique semblable (vous n'essaierez pas dans un premier temps de reproduire les titres, légendes etc.) à \texttt{fig\_ex\_ajustement.png}, que vous trouverez sur le site de la classe.


\subsection*{Régression pour le modèle d'ordre 1.}
\label{sec.reg.o1}


\question{} Écrire une fonction \texttt{L1(k)} prenant en argument un flottant positif \texttt{k} et renvoyant la valeur de $L^1(k)$.

\medskip{}

\question{\label{qu:traceL1}} Représenter $L^1$ sur un intervalle convenablement choisi. Semble-t-il y avoir un minimum à cette fonction ? Quelle est la régularité de $L^1$ sur $\R^\ast_+$ ? 

  Vous enverrez la figure tracée à votre enseignant. 

\medskip{}

\question{} Déterminer une fonction $dL^1$ sur laquelle appliquer une méthode de Newton permet d'obtenir le lieu du minimum de $L^1$. 
Écrire une fonction \texttt{dL1(k)} prenant en argument un flottant positif \texttt{k} et renvoyant la valeur de $dL^1(\texttt k)$.

\medskip{}

\question{} Écrire une fonction \texttt{ddL1(k)} prenant en argument un flottant positif \texttt{k} et renvoyant la valeur de $(dL^1)'(\texttt k)$.

\medskip{}

\question{} \'Ecrire une fonction \texttt{newton(f, fp, x0, eps)} implémentant la méthode de Newton pour la fonction \texttt{f}, ou \texttt{fp} est supposée être la dérivée de \texttt{f}, à partir du point \texttt{x0} et s'arrêtant dès que deux points consécutifs sont distants d'au plus \texttt{eps}.

\medskip{}

\question{} Appliquer la méthode de Newton pour trouver le lieu du minimum de $L^1$, noté $k_1$. Converge-t-elle pour toutes les valeurs initiales ? 

Dans le script, vous écrirez une fonction \texttt{val\_k1()}, sans argument, effectuant ce calcul et renvoyant la valeur trouvée. On pourra affecter à la variable \texttt{k1} cette valeur pour la suite du script.

\question{\label{qu:tracef1}} Représenter les mesures expérimentales superposées à la fonction $f^1_{k_1}$.

  Vous enverrez la figure tracée à votre enseignant. 


\subsection*{Régression pour le modèle d'ordre 2.}
\label{sec.reg.o2}

\question{} Écrire une fonction \texttt{L2(k)} prenant en argument un flottant positif \texttt{k} et renvoyant la valeur de $L^2(\texttt k)$.

\medskip{}

\question{\label{qu:traceL2}} Représenter $L^2$ sur un intervalle convenablement choisi. Semble-t-il y avoir un minimum à cette fonction ? Quelle est la régularité de $L^2$ sur $\R^\ast_+$ ?

  Vous enverrez la figure tracée à votre enseignant. 

\medskip{}  
  
\question{} Déterminer une fonction $dL^2$ sur laquelle appliquer la méthode de la sécante (ou de Newton) permet d'obtenir le lieu du minimum de $L^2$.
Écrire une fonction \texttt{dL2(k)} prenant en argument un flottant positif \texttt{k} et renvoyant la valeur de $dL^2(\texttt k)$.

\medskip{}

\question{}\'Ecrire une fonction \texttt{secante(f, x0, x1, eps)} implémentant la méthode de la sécante pour la fonction \texttt{f}, à partir du point \texttt{x0} et \texttt{x1} et s'arrêtant dès que deux points consécutifs sont distants d'au plus \texttt{eps}.

\medskip{}  

 
\question{} Appliquer la méthode de la sécante pour trouver le lieu du minimum de $L^2$, noté $k_2$. 

Dans le script, vous écrirez une fonction \texttt{val\_k2()}, sans argument, effectuant ce calcul et renvoyant la valeur trouvée. On pourra affecter à la variable \texttt{k2} cette valeur pour la suite du script.

\medskip{}

\question{\label{qu:tracef2}} Représenter les mesures expérimentales superposées à la fonction $f^2_{k_2}$.

  Vous enverrez la figure tracée à votre enseignant. 


\medskip{}  
 
\question{} Facultatif : appliquer la méthode de Newton pour trouver le lieu du minimum de $L^2$, noté $k_2$. Converge-t-elle pour toutes les valeurs initiales ?

Dans le script, vous écrirez une fonction \texttt{val\_k2\_bis()}, sans argument, effectuant ce calcul et renvoyant la valeur trouvée. 


\subsection*{Comparaison qualitative des deux modèles.}


\question{} Commenter les résultats des deux parties précédentes, ainsi que l'utilisation des méthodes de Newton et de la sécante.


\subsection*{Facultatif : régression linéaire.}
\label{sec.papl.reglin}
% La suite  est moins guidée : à vous de prendre des initiatives !  

Dans le modèle d'ordre 1, on peut remarquer que l'on a $\displaystyle -\log\left(\dfrac{C(t)}{C(0)}\right) = kt$. 
On a donc ici une relation linéaire entre le temps et une transformation des concentrations. On peut donc effectuer une \emph{régression linéaire par moindres carrés}, qui consiste à minimiser le critère 
\begin{equation*}
  L^1_{\mathrm{lin}}(k) = \dfrac{1}{7} \sum_{i=0}^6 \left( k t_i + \log\left(\dfrac{C_i}{C(0)}\right) \right)^2.
\end{equation*}
On effectue en réalité une régression linéaire par moindres carrés sur les points de coordonnées 
\begin{equation*}
  \left(t_i,\log\left(\dfrac{C_i}{C(0)}\right)\right).
\end{equation*}

\question{} Montrer qu'il existe un unique paramètre $k_1'$ minimisant $L^1_{\mathrm{lin}}$, puis le déterminer explicitement (en fonction des données du problème).

\medskip{}

\question{} \'Ecrirer une fonction \texttt{val\_kp1()}, sans argument, et renvoyant la valeur de $k_1'$. On pourra affecter ensuite à la variable \texttt{kp1} cette valeur pour la suite du script.

\medskip{}

\question{\label{qu:traceflin}} Représenter les mesures expérimentales superposées à la fonction $f^1_{k_1'}$.

  Vous enverrez la figure tracée à votre enseignant.

\medskip{}
  
\question{} Comparer cette méthode avec celle utilisée dans la partie \ref{sec.reg.o1}. Quels sont les avantages de chacune ?


