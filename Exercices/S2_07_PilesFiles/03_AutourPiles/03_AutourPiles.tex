\section*{Autour des piles}

Dans le cadre cet exercice, les piles dont des \lstinline{deque} importé du module \lstinline{collections}\sidenote{\lstinline{from collections import deque}}.
On ne s'autorise que les opérations suivantes : 
\begin{itemize}
 \item \lstinline{pile = deque()} pour créer une pile;
 \item \lstinline{len(pile) == 0} pour vérifier si une pile est vide;
 \item \lstinline{pile.append('Truc')} pour ajouter un élément dans la pile;
 \item \lstinline{pile.pop()} pour supprimer et renvoyer le sommet de la pile.
\end{itemize}

\question{Implémenter la fonction \lstinline{somme(pile)} renvoyant la somme des éléments d'une pile d'entiers. La pile initiale devra rester inchangée.}

\question{Implémenter la fonction \lstinline{maximum(pile)} renvoyant le maximum des éléments d'une pile d'entiers. La pile initiale devra rester inchangée.}

\question{Implémenter la fonction \lstinline{pop_maximum(pile)} renvoyant le maximum des éléments d'une pile d'entiers. La pile initiale sera donc la même à l'exception du maximum le plus haut.}