\exer{Évaluation d’une expression postfixée}

\begin{flushright}
\textit{Jean-Pierre Becirspahic}
\end{flushright}

La notation postfixée d’une expression algébrique consiste à placer les opérateurs après son ou ses opérandes. Par
exemple, l’addition de \texttt{a} et de \texttt{b} sera écrite \texttt{a b +} en notation postfixée, la racine carrée de \texttt{a} sera écrite \texttt{a
 $\sqrt{}$}.
 
L’intérêt majeur de cette notation est qu’une expression postfixe n’est jamais ambiguë : alors que expression
infixe \texttt{1 + 2 $\times$ 3} peut avoir deux significations : \texttt{(1 + 2) 
$\times$ 3} ou \texttt{1 + (2 $\times$ 3)}, ce n’est jamais le cas d’une
expression postfixe, ce qui rend l’usage des parenthèses superflu : \texttt{1 2 + 3 $\times$} ne peut être compris que de cette
façon : \texttt{(1 2 +) 3 $\times$} et \texttt{1 2 3 $\times$ +} de cette façon : \texttt{1 (2 3 $\times$) +}.
Nous allons montrer comment, à l’aide d’une pile, on peut évaluer une expression algébrique postfixe.
Dans cet exercice, les expressions algébriques seront représentée par les listes qui pourront contenir des nombres
(de type \texttt{int} ou \texttt{float}) ou des chaînes de caractères représentant des opérateurs unaires ou binaires (comme par
exemple \texttt{sqrt} ou \texttt{+}).
Par exemple, l’expression  $\dfrac{1+2\sqrt{3}}{4}$ sera représentée par la liste \texttt{[1, 2, 3, 'sqrt', '*', '+', 4, '/']}.
On suppose donné deux dictionnaires répertoriant pour l’un les opérateurs unaires, pour l’autre les opérateurs
binaires, et qui associent à chaque chaîne de caractère la fonction correspondante. On peut par exemple définir
ces deux dictionnaires à l’aide du script suivant, et les compléter en suivant le même modèle.

\begin{lstlisting}
from numpy import sqrt, exp, log
op_uni = {'sqrt': sqrt, 'exp': exp, 'ln':log}
def add(x, y):
    return x + y
    
def sous(x, y):
    return x - y
    
def mult(x, y):
    return x * y
    
def div(x, y):
    return x / y

op_bin = {'+': add, '-': sous, '*': mult, '/': div}
\end{lstlisting}


L’évaluation d’une expression postfixe consiste à utiliser une pile initialement vide et à parcourir les éléments
de la liste représentant l’expression à évaluer en appliquant les règles suivantes :
\begin{itemize}
\item si l’élément est un nombre, il est empilé ;
\item si l’élément est un opérateur unaire, le sommet de la pile est dépilé, l’opérateur lui est appliqué et le
résultat ré-empilé ;
\item si l’élément est un opérateur binaire, deux éléments de la pile sont dépilés, l’opérateur leur est appliqué
et le résultat ré-empilé.
\end{itemize}

Si l’expression postfixe est correcte sur le plan syntaxique (et mathématique), à la fin du traitement de la liste la
pile ne contient plus qu’un seul élément égal au résultat de l’évaluation de l’expression.
On suppose donnés les deux dictionnaires \texttt{op\_uni} et \texttt{op\_bin}.

\question{Rédiger une fonction qui évalue une expression postfixe donnée sous forme de liste \texttt{evalue(lst:liste)->float}. Dans un premier temps, on
pourra supposer que l'expression est syntaxiquement correcte.}

\question{Rédiger une seconde fonction d’évaluation qui détecte les erreurs de syntaxe. On l'a notera \texttt{evalue2(lst:liste)->float}}