\exer{Calcul de distance}

Soient $n$ points de coordonnées $P_1(x_1,y_1,z_1)$,  ...$P_n(x_n,y_n,z_n)$. On cherche à déterminer la distance entre chacun de ces points.

On donne la fonction \texttt{distance} permettant de déterminer la distance entre deux points. 

\begin{lstlisting}
def distance(p1,p2):
    d0 = p2[0]-p1[0]
    d1 = p2[1]-p1[1]
    d2 = p2[2]-p1[2]
    d = (d0**2+d1**2+d2**2)**(1/2)
    return d
\end{lstlisting}


\question{Préciser la signature de cette fonction sous forme de commentaires. On ajoutera aussi les annotations de type de cette fonction.}
\ifprof
\begin{lstlisting}
def distance(p1:list, p2:list) -> float:
    """
    Déterminer les distances entre deux points de l'espace. 
    Entrées : 
     - p1:list : liste des coordonnées (x1,y1,z1) du point 1
     - p2:list : liste des coordonnées (x2,y2,z2) du point 2
    Sortie :
     - distance entre p1 et p2
    """
    d0 = p2[0]-p1[0]
    d1 = p2[1]-p1[1]
    d2 = p2[2]-p1[2]
    d = (d0**2+d1**2+d2**2)**(1/2)
    return d
\end{lstlisting}
\else
\fi

\question{Donner un (ou des) test(s) sous la forme d'assertions(\texttt{assert}) permettant de valider les entrées de la fonction.}
\ifprof
\begin{lstlisting}
def distance(p1:list, p2:list) -> float:
    """ (...)  """
    assert len(p1)==3 and len(p2)==3

    
    return d
\end{lstlisting}

\else
\fi


\question{Proposer des tests permettant de vérifier les sorties de la fonction.}
\ifprof
\begin{lstlisting}
def test_distance():
    p1 = [0,0,0]
    p2 = [0,0,0]
    assert distance(p1,p2) == 0
    p1 = [1,0,0]
    p2 = [0,0,0]
    assert distance(p1,p2) == 1
\end{lstlisting}

\else
\fi

On donne maintenant la fonction suivante permettant de calculer la longueur d'un chemin constitué de $n$ points. 
\begin{lstlisting}
def longueur(L:list)->float:
    """
    Déterminer a longueur du chemin L
    Entrée : 
     - L:list : liste des points constitués de leurs cordonnées : [[x0,y0,z0],...[xn,yn,zn]]
    Sortie :
     - longueur du chemin
    """
    l = 0
    for i in range(len(L)):
        l = l+distance(L[i],L[i+1])
    return l
\end{lstlisting}

Soient \texttt{p1 = [0,0,0]}, \texttt{p2 = [1,0,0]} et  \texttt{p3 = [2,0,0]} trois points.


\question{Comment utiliser la fonction \texttt{longueur} pour déterminer la distance du chemin \texttt{p1 - p2 - p3} ?}

\question{La fonction \texttt{longueur} fonctionne-t-elle pour ce chemin ? Si elle ne fonctionne pas, modifier la fonction.}

\question{Donner un (ou des) test(s) sous la forme d'assertions(\texttt{assert}) permettant de valider les entrées de la fonction \texttt{longueur}.}

\ifprof
\begin{lstlisting}
def longueur(L:list)->float:
    """
    Déterminer a longueur du chemin L
    Entrée : 
     - L:list : liste des points constitués de leurs cordonnées : [[x0,y0,z0],...[xn,yn,zn]]
    Sortie :
     - longeur du chemin
    """
    # Test à l'entrée
    for pt in L :
        assert len(pt)==3
    
    l = 0
    for i in range(len(L)-1):
        l = l+distance(L[i],L[i+1])
    return l
\end{lstlisting}

\else
\fi

