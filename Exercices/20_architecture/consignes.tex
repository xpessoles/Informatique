\vspace{0.5cm}
\begin{Large}
{
\color{ocre}
\textbf{Consignes}
}
\end{Large}
\vspace{0.5cm}

\begin{enumerate}
\item  \textbf{Lisez attentivement  tout l'énoncé
    avant de commencer.}
% \item Commencez la séance en créant un dossier au nom du TP dans le répertoire dédié à l'informatique de votre compte. 
\item Après la séance, vous devez rédiger un compte-rendu de TP et
l'envoyer au format électronique à votre enseignant.
\item Le seul format accepté pour l'envoi d'un texte de compte-rendu est le
format PDF. Votre fichier s'appellera impérativement \texttt{tp02\_kleim\_durif.pdf}, où \og \texttt{kleim}\fg\ et \og \texttt{durif}\fg\ sont à remplacer par les noms des membres du binôme. 
\item Ce TP est à faire en binôme, vous ne rendrez donc qu'un compte-rendu pour deux.
\item Vous préciserez en préambule de votre compte-rendu les noms des membres du binôme ainsi que le système d'exploitation sur lequel vous avez travaillé. 
\item Ayez toujours un crayon et un papier sous la main. Quand vous réfléchissez à une question, utilisez-les !
\item Vous devez être autonome. Ainsi, avant de poser une question à l'enseignant, merci de commencer par :
\begin{itemize}
  \item relire l'énoncé du TP (beaucoup de réponses se trouvent dedans) ;
  \item relire les passages du cours\footnote{Dans le cas fort 
improbable où vous ne vous en souviendriez pas.} relatifs à votre problème ;
  \item effectuer une recherche dans l'aide disponible sur votre ordinateur (ou sur internet) concernant votre question.
\end{itemize}
  Il est alors raisonnable d'appeler votre enseignant pour lui demander des explications ou une confirmation !
\end{enumerate}