\begin{minipage}{0.5\textwidth}
\question{}

\begin{itemize}
\item Universelle ; 
\item Servant à traiter de l'information ;
\item Programmable.
\end{itemize}
\end{minipage}
\begin{minipage}{0.5\textwidth}
\question{}

\begin{enumerate}[label=(\alph*)]
\item \sout{Mémoire virtuelle} ;
\item \sout{Processus} ;
\item Processeur ;
\item Canal de communication ;
\item Mémoire vive ;
\item Mémoire de masse.
\end{enumerate}
\end{minipage}





\question{}

Un OS est un système d'exploitation (Operating System en Anglais). 
Ce programme a pour but de gérer les accès aux processeurs et à la mémoire et aux périphériques, ce qui permet effectivement que l'ordinateur exécute plusieurs programmes à la fois.

Par ailleurs l'OS permet de manière générale de gérer l'organisation des données sur le disque dur ainsi que leur droit d'accès. Il gère aussi les différentes ressources et sert de garde-fou en cas de tentative de mauvaise utilisation des ressources de l'ordinateur.

Pour finir, l'OS permet à un ou plusieurs utilisateurs de s'identifier leur permettant ainsi d'utiliser un seul ordinateur sans nécessairement partager les données et les programmes. 