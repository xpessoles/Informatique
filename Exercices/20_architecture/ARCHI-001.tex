\exer{[ARCHI-001]}
\setcounter{numques}{0}~\\

\begin{enumerate}
\item  \textbf{Lisez attentivement  tout l'énoncé
    avant de commencer.}
% \item Commencez la séance en créant un dossier au nom du TP dans le répertoire dédié à l'informatique de votre compte. 
\item Après la séance, vous devez rédiger un compte-rendu de TP et
l'envoyer au format électronique à votre enseignant.
\item Le seul format accepté pour l'envoi d'un texte de compte-rendu est le
format PDF. Votre fichier s'appellera impérativement \texttt{tp02\_kleim\_durif.pdf}, où \og \texttt{kleim}\fg\ et \og \texttt{durif}\fg\ sont à remplacer par les noms des membres du binôme. 
\item Ce TP est à faire en binôme, vous ne rendrez donc qu'un compte-rendu pour deux.
\item Vous préciserez en préambule de votre compte-rendu les noms des membres du binôme ainsi que le système d'exploitation sur lequel vous avez travaillé. 
\item Ayez toujours un crayon et un papier sous la main. Quand vous réfléchissez à une question, utilisez-les !
\item Vous devez être autonome. Ainsi, avant de poser une question à l'enseignant, merci de commencer par :
\begin{itemize}
  \item relire l'énoncé du TP (beaucoup de réponses se trouvent dedans) ;
  \item relire les passages du cours\footnote{Dans le cas fort 
improbable où vous ne vous en souviendriez pas.} relatifs à votre problème ;
  \item effectuer une recherche dans l'aide disponible sur votre ordinateur (ou sur internet) concernant votre question.
\end{itemize}
  Il est alors raisonnable d'appeler votre enseignant pour lui demander des explications ou une confirmation !
\end{enumerate}

Le but de ce TP est de vous faire de manipuler un système d'exploitation par lignes de commandes, puis de prendre en main \texttt{Python}{}.

{\bf Attention :} les étudiants travaillant sous Unix (Linux et MacOS) répondront d'abord aux questions des parties \ref{tp02:sec:unix1} et \ref{tp02:sec:unix2}, ceux travaillant sous Windows répondront d'abord aux questions des parties \ref{tp02:sec:win1} et \ref{tp02:sec:win2}. Tous les étudiants finiront par répondre aux questions de la partie~\ref{tp02:sec:python}. 

Pour les manipulations de fichier, vous utiliserez le code source du logiciel
\texttt{Python}{}, que vous pourrez trouver dans le dossiers groupes. à l'adresse suivante (où \texttt{X} 
est à remplacer par \texttt{1} ou \texttt{2})  :
\begin{center}
  $\sim$\texttt{/groupes/mpsX/données/TP02/cpython-4243df51fe43}
\end{center}




\subsection{Introduction : terminal et shell} 

Unix a été inventé à un moment où l'utilisateur avait la possibilité
d'interagir avec l'ordinateur via un \emph{terminal}, c'est-à-dire la
combinaison d'un clavier et d'un écran pouvant écrire (en général) 25
lignes de 80 caractères (en une seule couleur, généralement vert ou
orange sur fond noir). Cette façon d'interagir avec la machine peut
paraître archaïque de nos jours mais elle est pourtant d'une puissance
diabolique.

Vous trouverez un émulateur de terminal dans
le menu \og{}Applications\fg{}, sous-menu \og{}Accessoires\fg{}, et
sélectionnez \og{}LXTerminal\fg{}. Ceci démarre un programme, appelé
\textit{shell}
ou \emph{interprète de commandes}. Ce \textit{shell} vous donne
quelques informations, et
affiche un symbole \texttt{\$}, appelé invite (ou \textit{prompt} en
anglais) signe qu'il attend vos ordres.

Pour lui donner un ordre, il suffit de taper le nom de la commande
désirée, éventuellement suivie d'un espace puis
d'options ou d'arguments séparés par des espaces, puis de valider par la touche
Entrée.

Sous Windows, le programme \texttt{cmd.exe} permet d'ouvrir une console fonctionnant sous le même principe.
Une remarque : les noms de dossiers comportant des espaces devront être écrits entre guillemets (par exemple : \texttt{"Mes Documents"}). 
De manière générale, c'est une très mauvaise idée de placer des espaces dans des noms de dossiers (ou de fichiers). 

Vous trouverez dans la table~\ref{TP02:tab:comlinwin} les équivalents Windows de quelques commandes Unix. 

\begin{table}[!h]
  \begin{center}
  \begin{tabular}{|c|c|l|}
    \hline 
    Commande Unix & Commande Windows & Description rapide\\
    \hline
    \texttt{cd}    & \texttt{cd}      & Change le répertoire courant\\
    \hline
    \texttt{cp}    & \texttt{copy}    & Copie un ou des fichiers \\
    \hline
    \texttt{exit}  & \texttt{exit}    & Ferme la console \\
    \hline
    \texttt{less}  & \texttt{more}    & Lit un document texte \\
    \hline
    \texttt{ls}    & \texttt{dir}     & Liste les fichiers dans un répertoire \\
    \hline
    \texttt{man}   & \texttt{help}    & Ouvre une page de documentation \\
    \hline
    \texttt{mkdir} & \texttt{mkdir}   & Crée un répertoire\\
    \hline
    \texttt{mv}    & \texttt{move}    & Déplace un ou des fichiers \\
    \hline
    \texttt{pwd}   & \texttt{cd}      & Affiche le répertoire courant\\
    \hline
    \texttt{rm}    & \texttt{del}     & Supprime  un ou des fichiers \\
    \hline
  \end{tabular}
  \caption{Correspondances des commandes Unix et Windows}
  \label{TP02:tab:comlinwin}
  \end{center}
\end{table}

\begin{table}[!h]
  \begin{center}
  \begin{tabular}{|c|c|}
    \hline 
    Variable Unix & Variable Windows \\
    \hline
    $\sim$ & \texttt{\%UserProfile\%} \\
    \hline
    \texttt{/} & {\bf\textbackslash} \\
    \hline
  \end{tabular}
  \caption{Correspondances des variables Unix et Windows}
  \label{TP02:tab:varlinwin}
  \end{center}
\end{table}


\subsection{Unix : utilisation d'un terminal.}  \label{tp02:sec:unix1}

Une commande très pratique est \texttt{man}: elle permet d'obtenir le
manuel de quasiment toutes les commandes. On l'utilise sous la
forme \texttt{man} \textit{page} où \textit{page} est la page de
manuel désirée.

\medskip

\question\ Tapez \texttt{man less}. Que se passe t-il ?

\medskip

Vous pouvez faire défiler le texte ligne par ligne avec Entrée ou page
par page avec la barre d'espace et quitter \texttt{man} avec la touche
q.

\medskip

\question\ Quelle commande permet d'afficher la page suivante dans \texttt{less} ? de quitter \texttt{less} ? 



\subsection{Unix : fichiers et répertoires} \label{tp02:sec:unix2}

Sous Unix (dont la distribution GNU/Linux est un représentant) les
fichiers sont organisés hiérarchiquement en une arborescence unique de
répertoires. La racine de cette arborescence, c'est-à-dire le
répertoire supérieur de la hiérarchie contenant tous les fichiers
auxquels à accès le système, est noté \texttt{/}. Ses sous-répertoires
directs (de l'ordre de la dizaine ou quelques dizaines de
répertoires), comme \texttt{home}, \texttt{media}, \ldots{}
sont notés \texttt{/home}, \texttt{/media}, \ldots

Le \emph{chemin absolu} d'un fichier est l'adresse complète de son emplacement,
débutant de la racine et passant par tous les sous-répertoires requis pour
atteindre le fichier visé.

Le \emph{chemin relatif} d'un fichier est l'adresse de son emplacement, écrite à partir d'un emplacement de l'arborescence que l'on appelle \emph{répertoire courant} (en anglais : \emph{current working directory}). 
Ce répertoire courant est initialisé par défaut à un point prédéterminé de l'arborescence (répertoire \og maison \fg, ou \emph{home}), mais peut ensuite être modifié.

\medskip

\question\ Que fait précisément la commande \texttt{ls} ?

\medskip

La commande \texttt{cd} permet de changer de répertoire
courant, \texttt{pwd} permet d'afficher le répertoire courant. En
particulier, la commande \texttt{cd} \textit{d},où \textit{d} est le nom absolu ou relatif
d'un répertoire, change le répertoire courant en
\textit{d}. Essayez avec \texttt{cd /usr/bin} par exemple.

\medskip

\question\ Que fait \texttt{cd} sans argument (\emph{i.e.} \texttt{cd} non
  suivi du nom d'un répertoire) ?

\question\ Changer le répertoire courant (par exemple en \texttt{/usr/bin}). Que fait \texttt{cd }$\sim$ ?  De 
quoi $\sim$ est-il l'abréviation ? Quel est le chemin absolu du répertoire $\sim$ ?

\medskip

\question\ Que fait la commande \texttt{mkdir } ? En utilisant une ligne de commande, créer dans le répertoire $\sim$ un sous-répertoire nommé \texttt{TP02}. Qu'observe-t-on en exécutant la commande \texttt{ls }$\sim$ ? 

\medskip

\question\ Après avoir éxécuter la commande \texttt{cd TP02} à partir du répertoire $\sim$, qu'affiche la commande
  \texttt{pwd}?
  
\medskip  

Copier le dossier \texttt{cpython-4243df51fe43} depuis le dossier \texttt{groupes/mpsX/TP02} dans le répertoire $\sim$\texttt{/TP02} en passant par un explorateur de fichiers. Exécuter la commande \texttt{cd} dans le terminal. 

% \emph{Indication : si vous devez recommencer les étapes suivantes, veillez à bien réinitialiser les fichiers.}

\medskip 

Pour Unix : dans un répertoire, les fichiers et répertoires dont le nom
commence par un point sont dits \emph{cachés}.

\medskip

\question\ {En consultant le manuel de \texttt{ls}, trouver la commande qui
permet d'afficher les fichiers et répertoires cachés.}

\medskip

\question\ Dans $\sim$\texttt{/TP02}, vous pouvez alors voir deux
  répertoires cachés. Quels sont leurs noms ? 
  
\medskip

En fait, dans chaque répertoire du système, il existe deux répertoires cachés
avec ces deux mêmes noms.

\medskip

\question{} Que désignent les répertoires \texttt{.} et \texttt{..} ?
Que donne un \texttt{cd} sur chacun de ces répertoires ?

\medskip

\question\ Changez le répertoire courant en 
$\sim$\texttt{/TP02/cpython-4243df51fe43/Lib/test/capath}. Que
fait   alors \texttt{cd ../../multiprocessing} ?

\medskip

\question\ Comment obtenir grâce à la commande
\texttt{ls} et l'option \texttt{-l} la taille de tous les
fichiers de \texttt{TP02/cpython-4243df51fe43/Modules}, en les triant par ordre
croissant de taille ?

\medskip

\question{}\emph{Facultatif :} Avec la commande précédente, que remarquez-vous quant à la taille des
sous-répertoires de \texttt{TP02/cpython-4243df51fe43/Modules} ?

\medskip

\question{}\emph{Facultatif :} En utilisant la commande \texttt{du}, donner la taille du répertoire
\texttt{cjkcodecs}. En comparant ce résultat à celui de la question
précédente, que pouvez-vous dire de la manière dont Linux considère les
répertoires ?

\subsection{Windows : utilisation d'un terminal.}  \label{tp02:sec:win1}

Une commande très pratique est \texttt{help}: elle permet d'obtenir le
manuel de quasiment toutes les commandes. On l'utilise sous la
forme \texttt{help} \textit{page} où \textit{page} est la page de
manuel désirée.

\medskip

\question\ Tapez \texttt{help more}. Que se passe t-il ?

\medskip

Vous pouvez faire défiler le texte ligne par ligne avec Entrée ou page
par page avec la barre d'espace et quitter \texttt{man} avec la touche
q.

\medskip

\question\ Quelle commande permet d'afficher la page suivante dans \texttt{more} ? de quitter \texttt{more} ? 

\subsection{Windows : fichiers et répertoires} \label{tp02:sec:win2}

Sous Windows, les fichiers sont organisés hiérarchiquement en plusieurs arborescences de
répertoires. La racine de chacune de ces arborescences, c'est-à-dire le
répertoire supérieur de la hiérarchie contenant tous les fichiers
auxquels à accès le système, est nommé par une lettre (\texttt{A:}, \texttt{B:} \textit{etc.}). 

Le séparateur de nom de dossier est alors {\bf\textbackslash}. Par exemple, le sous-répertoire \texttt{Home} de la racine \texttt{C:} est noté \texttt{C:\textbackslash Home}. 

Le \emph{chemin absolu} d'un fichier est l'adresse complète de son emplacement,
débutant de la racine et passant par tous les sous-répertoires requis pour
atteindre le fichier visé.

Le \emph{chemin relatif} d'un fichier est l'adresse de son emplacement, écrite à partir d'un emplacement de l'arborescence que l'on appelle \emph{répertoire courant} (en anglais : \emph{current working directory}). 
Ce répertoire courant est initialisé par défaut à un point prédéterminé de l'arborescence (répertoire \og maison \fg, ou \emph{home}), mais peut ensuite être modifié.

\medskip

\question\ Que fait précisément la commande \texttt{dir} ?

\medskip

La commande \texttt{cd}, avec un argument, permet de changer de répertoire
courant. Sans argument, elle permet d'afficher le répertoire courant. En
particulier, la commande \texttt{cd} \textit{d},où \textit{d} est le nom absolu ou relatif
d'un répertoire, change le répertoire courant en
\textit{d}. Essayez avec par exemple.

\medskip

\question\ Changer le répertoire courant. Que fait \texttt{cd \%UserProfile\%} ?  De 
quoi \texttt{\%UserProfile\%} est-il l'abréviation ? Quel est le chemin absolu du répertoire \texttt{\%UserProfile\%} ?

\medskip

\question\ Que fait la commande \texttt{mkdir} ? En utilisant une ligne de commande, créer dans le répertoire \texttt{\%UserProfile\%} un sous-répertoire nommé \texttt{TP02}. Qu'observe-t-on en exécutant la commande \texttt{dir }\texttt{\%UserProfile\%} ? 

\medskip

\question\ Après avoir éxécuter la commande \texttt{cd TP02} à partir du répertoire $\sim$, qu'affiche la commande
  \texttt{cd}?
  
\medskip  

Copier le dossier \texttt{cpython-4243df51fe43} depuis le dossier \texttt{groupes/mpsX/TP02} dans le répertoire \texttt{\%UserProfile\%\textbackslash TP02} en passant par un explorateur de fichiers.  

% \emph{Indication : si vous devez recommencer les étapes suivantes, veillez à bien réinitialiser les fichiers.}

\medskip 

Pour Unix : dans un répertoire, les fichiers et répertoires dont le nom
commence par un point sont dits \emph{cachés}. Ce n'est pas la même chose sous Windows.

\medskip

\question\ En consultant le manuel de \texttt{dir}, trouver la commande qui
permet d'afficher les fichiers et répertoires cachés.

\medskip

\question\ Faire un inventaire des fichiers de \texttt{\%UserProfile\%\textbackslash TP02}? Y en a-t-il des cachés ? 
  
\medskip

\question{} Que désignent les répertoires \texttt{.} et \texttt{..} ?
Que donne un \texttt{cd} sur chacun de ces répertoires ?

\medskip

\question\ Changez le répertoire courant en 
\texttt{\%UserProfile\%\textbackslash TP02\textbackslash cpython-4243df51fe43\textbackslash Lib\textbackslash test\textbackslash capath}. Que
fait   alors \texttt{cd ..\textbackslash ..\textbackslash multiprocessing} ?

\medskip

\question\ Comment obtenir grâce à la commande
\texttt{dir} et l'option \texttt{-l} la taille de tous les
fichiers de \texttt{TP02\textbackslash cpython-4243df51fe43\textbackslash Modules}, en les triant par ordre
croissant de taille ?
