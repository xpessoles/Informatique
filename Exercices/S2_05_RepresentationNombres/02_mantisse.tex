\exer{Détermination du nombre de bit de la mantisse d'un flottant}

On se propose de vérifier que le stockage de la mantisse d'un flottant \texttt{python} s'effectue sur 52 bits.\\

On note $mc=m-1$, \texttt{m} étant la mantisse du flottant.
\texttt{mc} est la valeur stockée en mémoire en binaire.\\

L'idée est de se servir du nombre $0.5$ dont on connait parfaitement la décomposition binaire :

\begin{center}
$0.5=\frac{1}{2}=1 \times \frac{1}{2} + 0 \times \frac{1}{4} + 0 \times \frac{1}{8}... $
\end{center}

On observe qu'en divisant \texttt{mc} successivement par 2 on obtient :
\begin{center}
$
\begin{array}{l l l l}
mc=0.5 & m=1.5 & \mbox{stockée en mémoire sous la forme} & 1000...000\\
mc=0.25 & m=1.25 & \mbox{stockée en mémoire sous la forme} & 0100...000\\
mc=0.125 & m=1.125 & \mbox{stockée en mémoire sous la forme} & 0010...000\\
...\\
mc=0.00...1 & m=1.00...1 & \mbox{stockée en mémoire sous la forme} & 0000...001\\
\end{array}
$
\end{center}

Au bout d'un nombre suffisamment grand de divisions par 2 le chiffre 1 disparait complètement. En comptant le nombre de divisions par 2 nécessaires pour aboutir à 0, on a accès au nombre de bits disponibles pour coder \texttt{mc}.\\
On propose l'algorithme suivant :

\includegraphics[scale=0.67]{pseudo.png}

\begin{qexo}
Commenter ou compléter l'algorithme proposé :
\begin{itemize}
\item compléter la partie initialisation
\item justifier le type de boucle choisie
\item repérer la condition d'arrêt
\item invariant de boucle
\item la boucle a-t-elle une fin ?
\item l'algorithme effectue t-il ce que l'on attend ?
\end{itemize}
\end{qexo}


\begin{qexo}
Implémenter cet algorithme dans \texttt{python}. Conclure quant au nombre de bits disponibles pour coder la mantisse d'un flottant.
\end{qexo}

Il est possible d'appliquer la méthode \texttt{hex()} sur un flottant pour avoir sa représentation en hexadécimal.

\begin{pythonshell}
\invite f=5.25\\
\invite f.hex()\\
’0x1.5000000000000p+2’
\end{pythonshell}

Cela dit que 5.25 est représentée par le nombre $1.(5000000000000)_{16} \times 2^{2}$
dont la mantisse est 1.5000000000000 et l'exposant 2
.
%\invite rad = 2.0 ** .5\\
%\invite rad.hex ()\\
%'0x1.6a09e667f3bcdp+0'
%La mantisse est 1.6a09e667f3bcdp et l'exposant 0

\begin{qexo}
Déterminer la mantisse de $\sqrt{2}$ à partir de son expression hexadécimale.
\end{qexo}
