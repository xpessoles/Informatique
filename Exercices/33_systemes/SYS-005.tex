 
 
 On considère les points de coordonnées 
  \begin{equation*}
    M_1 \bpm -5 \\ 11,67 \epm,\quad M_2 \bpm -2 \\ 4,52 \epm,\quad M_3 \bpm 1\\-0,15 \epm,\quad M_4\bpm 2 \\ -3,31 \epm.
  \end{equation*}
  On cherche à ajuster une droite affine sur le nuage de points $(M_1,M_2,M_3,M_4)$ par le critère des moindres carrés. Avec 
  \begin{equation*}
    A = \bpm 1 & -5 \\ 1 & -2 \\ 1 & 1 \\ 1 & 2 \epm, \quad X = \bpm \alpha \\ \beta \epm \quad\textrm{et} \quad B = \bpm 11,67 \\ 4,52 \\ -0,15 \\ -3,31 \epm,
  \end{equation*}
  
\question{}  déterminer $X$ minimisant la quantité 
  \begin{equation*}
    \norm{AX-B}. 
  \end{equation*}

\question{} Produire une figure superposant les points $(M_1,M_2,M_3,M_4)$ et la droite d'équation $y = \alpha + \beta x $. 
%\end{exo}

%\begin{exo}
 