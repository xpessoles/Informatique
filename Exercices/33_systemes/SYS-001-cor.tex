\exer{[SYS-001]}
\setcounter{numques}{0}~\\

\setcounter{question}{0}

\question\

L'algorithme suivant répond à la question. 
\begin{Verbatim}[gobble=0,numbers=left]
from numpy import zeros, array

def produit(P,Q,n):
    """Produit PQ tronqué à l'ordre n"""
    L = [0.]*(n+1)
    for k in range(n+1): # invariant : L[:k] contient les
			 # coefficients de PQ de degré < k
        for i in range(k+1): # invariant : L[k] vaut la somme 
			 # des P[j]*Q[k-j] pour j de 0 à i-1
            if k-len(Q) < i < len(P) :
                L[k] = L[k] + P[i]*Q[k-i]
    return L
\end{Verbatim}

\medskip\

\question\ Soit $k\in\llbr 0,n\rrbr$. La boucle \texttt{for} des lignes 8 à 11 contient $k$ tours de boucle. Chacun de 
ces tours de boucle contient au plus une instruction, celle de la ligne 11, qui elle-même contient trois opérations 
(une somme, un produit et une affectation). La complexité de cette boucle est donc $O(k)$.\\
La boucle des lignes 6 à 11 contient $n$ tours de boucle, et chacun de ces tours de boucle a une complexité en 
$O(k)$, et donc en $O(n)$. La complexité de cette boucle est donc $O(n^2)$.\\
La fonction \texttt{produit} ne comporte que l'instruction de la ligne 5 en plus de la boucle précédente, et la ligne 5 
a une complexité en $O(n)$.\\
La complexité de \texttt{produit} est donc un $O(n^2)$.


\medskip\

\question\ Soit $k\in\llbr 0,n\rrbr$. Puisque la valuation du DL de $f$ est 1, celle du DL de $f^k$ est $k$.\\
Notons alors $f^k(t)\underset{t\to0}{=}\displaystyle\sum_{i=k}^n a_{i,k}t^i + o(t^n)$.\\
Il vient donc :
\begin{align*}
 (f^{-1}\circ f)(t)&\underset{t\to0}{=}  \displaystyle\sum_{k=1}^n x_kf^k(t) + o(t^n)\\
 &\underset{t\to0}{=} \displaystyle\sum_{k=1}^n x_k\displaystyle\sum_{i=k}^na_{i,k}t^i + o(t^n)\\
 &\underset{t\to0}{=} \displaystyle\sum_{i=1}^n \displaystyle\sum_{k=1}^i x_k a_{i,k}t^i + o(t^n)
\end{align*}
Puisque $f^{-1}\circ f=\mathrm{Id}$, nous obtenons donc le système suivant :\\
\begin{itemize}
 \item $x_1 a_{1,1} = 1$ ;
 \item pour tout $i\in\llbr 2,n\rrbr$, $\displaystyle\sum_{k=1}^i x_k a_{i,k} = 0$.
\end{itemize}

Ce système est bien triangulaire inférieur.

\medskip\

\question\ 
\begin{Verbatim}[gobble=0,numbers=left]
def matrice(P):
    """Matrice du systeme donnant les coefficients du DL de la réciproque de f
       P : coefficients du DL de f"""
    n = len(P)-1
    B = zeros((n,n))
    Q = P.copy()
    for j in range(n): 	# invariant : les lignes de la matrice du système 
			# sont construites jusqu'à la (j-1)-ème.
        for i in range(j,n): # invariant : les coefficients de la j-ème  
			# ligne de la matrice du système sont construits
			# jusqu'au la (i-1)-ème.
            B[i,j] = Q[i+1]
        Q = produit(P,Q,n)
    return B
\end{Verbatim}


\medskip\

\question\ Soit $n$ le degré du polynôme $P$. La fonction \texttt{matrice} contient ligne 5 une instruction en 
$O(n^2)$, et ligne 6 une instruction en $O(n)$. Le reste de la fonction est constitué de deux boucles imbriquées.\\
La boucle des lignes 9 à 12 comporte $n-j$ tours de boucle, et chacun des ces tours est constitué d'une affectation. 
La complexité de cette boucle est donc un $O(n-j)$, soit un $O(n)$.\\
La boucle des lignes 7 à 13 comporte $n$ tours de boucle. Dans chaque tour de boucle, il y a $O(n)$ opérations 
contenues dans la boucle intérieure, et $O(n^2)$ opérations résultant de l'appel à \texttt{produit(P,Q,n)}. Cette 
boucle a donc une complexité en $O(n^3)$\\
Finalement, la complexité de \texttt{matrice} est un $O(n^3)$.

\medskip\

\question\ 
\begin{Verbatim}[gobble=0,numbers=left]
def resoutTI(T,Y):
    """Donne X solution de TX = Y, avec T triangulaire inférieure"""
    n,_ = T.shape
    X = zeros((n,1))
    for i in range(n): # invariant : les valeurs des X[j] pour j de 
		      # 0 à (i-1) ont été calculées
        X[i,0] = Y[i,0]
        for j in range(i):
            X[i,0] = X[i,0] - T[i,j]*X[j,0]
        X[i,0] = X[i,0] / T[i,i]
    return X
\end{Verbatim}

\medskip\

\question\ Pour $i\in\llbr 0,n-1\rrbr$, la complexité de la boucle intérieure est un $O(i)$. On dénombre facilement que 
chaque tour de la boucle initiée ligne 5 est en $O(i)$, donc cette boucle est en $O(n^2)$. La ligne 4 étant en $O(n)$, 
la complexité de \texttt{resout} est un $O(n^2)$.

\medskip\

\question\
\begin{Verbatim}[gobble=0,numbers=left]
def DLreciproque(P):
    """Donne le DL de la reciproque de f(x)+o(x n)
       Précondition : f(0)=0"""
    T = matrice(P)
    n = len(P)-1
    Y =zeros((n,1))
    Y[0,0] = 1.
    X = resoutTI(T,Y)
    Q = [0.]*(n+1)
    for i in range(1,n+1): # invariant : les coefficients du DL de 
			# l'inverse de f sont calculés jusqu'au degré (i-1)
        Q[i] = X[i-1,0]
    return Q

\end{Verbatim}
 
\medskip\

\question\ Il est aisé de constater qu'en dehors de l'appel à \texttt{matrice}, toutes les instructions sont au pire en 
$O(n^2)$, donc la complexité de \texttt{DLreciproque} est un $O(n^3)$.

\medskip\

\question\
\begin{Verbatim}[gobble=0,numbers=left]
def DLtan(n):
    """Donne la partie principale du DL de la tangente à l'ordre n"""
    atan = [0.]*(n+1)
    i = -1
    s = -1
    while i+2 <= n : # invariant : les coefficients du DL de 
		# atan sont calculés jusqu'au degré (2*(i+1))
        i = i+2
        s = -s
        atan[i] = s / i
    return DLreciproque(atan)
\end{Verbatim}

L'appel à \texttt{DLtan(7)} donne bien le résultat voulu.