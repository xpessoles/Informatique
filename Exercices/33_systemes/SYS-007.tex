\exer{[SYS-007]}
\setcounter{numques}{0}~\\

%\subsection{Algèbre linéaire}

Dans cette partie, on travaille avec la matrice
\begin{equation*}
A=  \begin{pmatrix}
  0 & 1 & 32 & 243\\
  1 & 32 & 243 & 1024\\
  32 & 243 & 1024 & 3125\\
  243 & 1024 & 3125 & 7776
  \end{pmatrix}
\end{equation*}
de terme général $a_{ij}=(i+j-2)^{5}$ pour $1\leq i,j\leq n$
(attention en Python, les indices commencent à $0$ et le terme général
est alors $(i+j)^{5}$).


\question{} Résoudre

  \begin{equation*}
    A
    \begin{pmatrix}
      x_{1}\\x_{2}\\x_{3}\\x_{4}
    \end{pmatrix}
    =
    \begin{pmatrix}
      1\\ 2\\ 3\\ \alpha
    \end{pmatrix}
  \end{equation*}
  Donner la valeur de $x_{1}$.
  
  
  


\question{}
  Calculer $B = A^{3}$ et donner le reste du coefficient de $B$ situé
  sur la première ligne et la première colonne de $B$ (donc d'indices
  $0$ et $0$ en \texttt{numpy}) dans la division par $10\,000+\alpha$.


%\question{}
%  On note $A'$ la matrice obtenue à partir de $A$ en remplaçant par
%  $\alpha$ le zéro situé en haut à
%  gauche. Le déterminant $\Delta$ de $A'$ est évidemment
%  un entier. Donner le reste de $\Delta$ dans la division par $1523$.
