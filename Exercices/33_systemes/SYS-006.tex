 On considère les points de coordonnées 
  \begin{equation*}
    M_1 \bpm -2 \\ 7,62 \epm,\quad M_2 \bpm -1 \\ 3,87 \epm,\quad M_3 \bpm 0 \\ 0,94 \epm,\quad M_4 \bpm 1\\1,56 \epm,\quad M_5\bpm 2 \\ 2,66 \epm.
  \end{equation*}
  On cherche à ajuster une courbe polynomiale de degré $2$ sur le nuage de points $(M_1,M_2,M_3,M_4,M_5)$ par le critère des moindres carrés. Avec 
  \begin{equation*}
    A = \bpm 1 & -2 & 4 \\ 1 & -1 & 1 \\ 1 & 0 & 0 \\ 1 & 1 & 1\\ 1 & 2 & 4\epm, \quad X = \bpm \alpha \\ \beta \\ \gamma \epm \quad\textrm{et} \quad B = \bpm 7,62 \\ 3,87 \\ 0.94 \\ 1,56 \\ 2,66 \epm,
  \end{equation*}
\question{}  déterminer $X$ minimisant la quantité 
  \begin{equation*}
    \norm{AX-B}. 
  \end{equation*}
  
  
\question{}  Produire une figure superposant les points $(M_1,M_2,M_3,M_4,M_5)$ et la courbe d'équation $y = \alpha + \beta x + \gamma x^2$. 