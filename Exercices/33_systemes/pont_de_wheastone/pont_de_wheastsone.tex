\documentclass[11pt]{article}

\usepackage{amsmath,amsthm,amsfonts,amssymb,mathrsfs,mathtools,stmaryrd} % caract\`{e}res math\'{e}matiques

\usepackage[french]{babel} % texte en français (gestion des c\'{e}sures)

\usepackage[utf8]{inputenc} % gestion des accents - remplacer uft8 par applemac ?

\usepackage[T1]{fontenc} % codage des caract\`{e}res

\usepackage{geometry} % mise en page (gestion des marges)
\geometry{ hmargin=1.5cm, vmargin=2cm }

\usepackage[version=3]{mhchem}%pour écrire des formule chimique

\usepackage{multicol}

\usepackage{enumitem}

\usepackage{graphicx} % ins\'{e}rer des figures
\usepackage{epstopdf} % gestion du format eps
\usepackage{tikz} % dessins et courbes
\usepackage{circuitikz} % l est un "L" minuscule
\usepackage{tkz-tab} % tableaux de variations

\usepackage{fancyhdr} % entête et pied de page
\pagestyle{fancy}

\usepackage{fancybox} % encadrements
\usepackage{boxedminipage} % encadrement des minipages

\usepackage{xcolor}

\usepackage{listings}
%\usepackage{xkeyval}
\lstset{
	language={Python},
	columns=flexible,
	basicstyle=\ttfamily,
	keywordstyle=\color{blue}, % je voudrais mettre les mots-cl\'{e} en gras (\bfseries) mais c'est incompatible avec typewriter
	commentstyle=\color{gray}, % commentaires
	backgroundcolor=\color{green!15}, % couleur du fond
	frame=single, rulecolor=\color{green!15}, % encadrement + couleur de l'encadrement
	%numbers=left, numberstyle=\tiny, stepnumber=1, numbersep=5pt,
	showstringspaces=false, % supprime les espaces apparents dans les lignes de texte
	stringstyle=\color{red!60}\itshape,
}

\usepackage{fancyvrb}
\newsavebox{\BBbox}
\newenvironment{DDbox}[1]{
\begin{lrbox}{\BBbox}\begin{minipage}{10cm}}
{\end{minipage}\end{lrbox}\noindent\colorbox{gray!20}{\usebox{\BBbox}} \\
[.5cm]}

\newcommand{\tpIPT}[5]{
	\lhead{#1}
	\chead{\begin{minipage}{10cm} \begin{center} \textbf{#2} \end{center} \end{minipage}}
	\rhead{TP \texttt{info} n$\textsuperscript{o}$#3}
	\lfoot{#4}
	\rfoot{Ann\'{e}e scolaire #5}}

\usepackage{fourier-orns} % symbol "danger" par la commande \danger{}

\usepackage{ulem} % pour souligner du texte sur plusieurs lignes avec \uline{texte}

\parindent=0em

\usepackage{lastpage}

\usepackage{systeme}

%%%%%%%%%%%%%%%%%%%%%%%%%%%%%%%%%%%%%%%%%%%%%%%%%%%%%%%

\begin{document}

%\fancyfoot{\thepage/\pageref{LastPage}}
\fancyfoot{}

\tpIPT{PCSI 2}{Résolution de systèmes}{27}{Lyc\'{e}e F. Roosevelt -- Reims}{2013-2014}

\section*{Résolution des systèmes avec NumPy.}

\begin{multicols}{2}
La résolution numérique avec NumPy du système de Cramer :
\begin{center}
	$A\,X=B \; \Longleftrightarrow $ \; \systeme[xy]{2x+y=0, x-3y=7}
\end{center}
s'effectue comme ci-contre :
\begin{center}
	\begin{minipage}{8cm}
		\lstinputlisting{programme_1.py}
	\end{minipage}
\end{center}
\end{multicols}
\textbf{Remarque.} Pour le second membre, on peut prendre une liste ou un vecteur (tableau unidimensionnel) :
\begin{center}
	\texttt{B=[0,7]} \quad ou \quad \texttt{B=np.array([0,7])}.
\end{center}
Le résultat retourné \texttt{X} sera alors un vecteur. \\

\begin{enumerate}
	\item Déterminer l'équation $y=a\,x^2+b\,x+c$ de la parabole passant par les points $(-1,9)$, $(1,3)$ et $(2,3)$.
\end{enumerate}

\section*{Une application en physique.}

\begin{multicols}{2}
On considère le circuit électrique ci-contre appelé \textit{pont de Wheatstone}. Les valeurs des résistances $R_1$, $R_2$, $R_3$, $R_4$, $R$ et de la tension $E$ étant connues, les inconnues sont les $5$ intensités $i_1$, $i_2$, $i_3$, $i_4$, et $i$ qui vérifient les $5$ équations :
\begin{itemize}
\item loi des noeuds en $B$ : $i_1 = i_2 + i_5$ ; \smallskip
\item loi des noeuds en $D$ : $i_4 + i_5 = i_3$ ; \smallskip
\item loi des mailles : $E = R_1 \times i_1 + R_2 \times i_2$ ; \smallskip
\item loi des mailles : $E = R_4 \times i_4 + R_3 \times i_3$ ; \smallskip
\item loi des mailles : $0 = R_1 \times i_1 + R \times i_5 - R_4 \times i_4$. \\
\end{itemize}

\begin{center}\begin{circuitikz}[european resistors,scale=0.8]
\draw (8,0) -- (5,0) to[V, v=$E$] (3,0) -- (0,0) -- (0,2) to[R, l=$R_4$, i=$i_4$] (4,2) to[R, l=$R_3$, i=$i_3$] (8,2) -- (8,0) ;
\draw (0,2) -- (0,5) to[R, l=$R_1$, i=$i_1$] (4,5) to[R, l=$R_2$, i=$i_2$] (8,5) -- (8,2) ;
\draw (4,5) to[R, l=$R$, i=$i$] (4,2) ;
\path (0,2) node{\small \textbullet} ; \path (-0.4,2) node{$A$} ;
\path (4,5) node{\small \textbullet} ; \path (4,5.4) node{$B$} ;
\path (8,2) node{\small \textbullet} ; \path (8.4,2) node{$C$} ;
\path (4,2) node{\small \textbullet} ; \path (4,1.6) node{$D$} ;
\end{circuitikz}\end{center}
\end{multicols}

\begin{enumerate}[resume]
	\item Calculer la valeur de l'intensité $i$ dans les deux cas suivants :
		\begin{itemize}
			\medskip\item[\textbullet] $E = 10$ V ; $R_1 = R_3 = 10 \text{ k} \Omega$ ; $R_2 = R_4 = R = 1 \text{ k} \Omega$ ;
			\medskip\item[\textbullet] $E = 10$ V ; $R_1 = R_3 = 4 \text{ k} \Omega$ ; $R_2 = R = 2 \text{ k} \Omega$ ; $R_4= 8 \text{ k} \Omega$.
		\end{itemize}
\end{enumerate}
Dans le second cas, le pont de Wheatstone est dit \textit{équilibré}. L'intensité $i$ est nulle et cela correspond à l'égalité :
\[ R_1\times R_3=R_2\times R_4. \]
\begin{enumerate}[resume]
	\item On suppose désormais connues les valeurs : $E = 10$ V, $R_1 = R = 1 \text{ k} \Omega$ et $R_4= 2 \text{ k} \Omega$. La valeur de la résistance $R_2$ est inconnue. On la fixera aléatoirement ainsi :
		\begin{center}
			\texttt{R2=(random.randrange(11)+5)*500}
		\end{center}
		après avoir importé le module \texttt{random}.
		\begin{enumerate}
			\item \'{E}crire un programme permettant :
				\begin{itemize}
					\medskip\item[\textbullet] de résoudre le système pour des valeurs de $R_3$ comprises entre $1\text{ k} \Omega$ et $20\text{ k} \Omega$ (on prendra $100$ valeurs) ;
					\medskip\item[\textbullet] d'afficher la courbe de l'intensité $i$ en fonction de la résistance $R_3$.
				\end{itemize}
			\smallskip\item \`{A} l'aide d'une lecture graphique, en déduire la valeur de la résistance $R_2$.
		\end{enumerate}
\end{enumerate}

\end{document}