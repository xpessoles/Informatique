\exer{Conversion d'entier en binaire}
\setcounter{numques}{0}

\question{ \'Etudier la complexité théorique de la fonction \texttt{conv\_b2}}

\begin{lstlisting}
def conv_b2(p):
 """Convertit l'entier p en base 2 (renvoie une chaîne)"""
    x = p
    s = ""
    while x > 1 :
        s = str(x%2) + s
        x = x // 2
    return str(x)+s
\end{lstlisting}    

\question{\'Etudier les complexités théoriques des fonctions \texttt{calc\_b2\_naif} 
et \texttt{calc\_b2\_horner}. Les comparer.}

\begin{lstlisting}
def calc_b2_naif(s):
    """Renvoie l'entier p représeté en binaire par s"""
    p = 0
    x = 1 ## 2**0
    for i in range(len(s)):
        p = p+int(s[len(s)-i-1])*x
        x = 2*x
    return p
\end{lstlisting}


\begin{lstlisting}
def calc_b2_horner(s):
    """Renvoie l'entier p représenté en binaire par s"""
    p = int(s[0])
    for i in range(1,len(s)):
        p = int(s[i])+2*p
    return p
\end{lstlisting}