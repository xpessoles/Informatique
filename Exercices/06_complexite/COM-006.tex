On considère le code suivant, qui crée une liste aléatoire \texttt L et qui la trie ensuite par ordre croissant par la méthode du «tri par insertion». 
\begin{Verbatim}[gobble=0,numbers=left]
from random import randrange

def liste_triee(n):
    """Renvoie une liste d'éléments de range(100), de longueur n,
       triée par ordre croissant """
    L = []
    for i in range(n):
        L.append(randrange(100))
    for i in range(1,n):
        # Inv : L[:i] est triée par ordre croissant
        # Idée : on fait redescendre L[i] pour l'insérer au bon endroit
        j = i
        while j >= 1 and L[j] < L[j-1]:
            # On échange L[j-1] et L[j]
            L[j], L[j-1] = L[j-1], L[j]
            j = j-1
    return L
\end{Verbatim}
Un appel de \texttt{randrange(100)} renvoie un nombre tiré aléatoirement et uniformément dans $\ii{0,100}$ et a une complexité en $O(1)$. 

\medskip{}

\question{} Peut-on donner explicitement et exactement une complexité pour la boucle \texttt{while} des lignes 13 à 16 ? 
Que peut-on quand même proposer comme type de complexité pour cette boucle ? 

Donner dans ce cas la complexité de cette boucle en fonction de \texttt{i}. 

\question{} Donner dans ce cas la complexité d'un appel de \texttt{liste\_triee(n)} en fonction de \texttt n. 
