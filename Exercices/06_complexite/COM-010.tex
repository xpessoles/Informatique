\exer{}
\setcounter{numques}{0}

\question{\'Etudier la complexité théorique dans le pire des cas de la fonction \texttt{recherche}. On pourra être amené à la reformuler légèrement.}

\begin{lstlisting}
def recherche(m,s):
    """Recherche le mot m dans la chaîne s
       Préconditions : m et s sont des chaînes de caractères"""
    long_s = len(s) # Longueur de s
    long_m = len(m) # Longueur de m
    for i in range(long_s-long_m+1): 
        # Invariant : m n'a pas été trouvé dans s[0:i+long_m-1]
        if s[i:i+long_m] == m: # On a trouvé m
            return True
    return False
\end{lstlisting}