\exer{}
\setcounter{numques}{0}

On considère la suite $u$ définie par 
\begin{equation*}
  u_0 = 2,\quad \forall n \in \N,~ u_{n+1} = u_n^2.
\end{equation*}
On se considère la fonction suivante, permettant de calculer les valeurs de $u$. 
\begin{lstlisting}
def u(n):
    """u_n, n : entier naturel"""
    v = 2
    # Inv : v = u_0
    for k in range(n):
        # Inv : v = u_k
        v = v*v
        # Inv : v = u_k**2 = u_k+1
    # Inv : au dernier tour, 
    # k = n-1, donc v = u_n
    return v
\end{lstlisting}

Pour étudier le temps d'exécution d'une fonction, on pourra utiliser le morceau de code suivant. 
\begin{lstlisting}
import timeit

REPEAT=3

def duree(f, x):
  """Calcule le temps mis par Python pour 
  calculer f(x).   Cette fonction effectue 
  en fait le calcul de f(x) REPEAT fois et
  garde la valeur la plus petite (l'idée est
  d'éliminer les éventuelles perturbations 
  provoquées par d'autres processus
  tournant sur la machine)"""
  t = timeit.Timer(stmt=lambda : f(x))
  time = min(t.repeat(REPEAT, number=1))
  return time
\end{lstlisting}

\begin{enumerate}
  \item Étudier en fonction de $n$ la complexité asymptotique de la fonction \texttt{u}, dans le modèle standard.
  \item Tracer les temps de calculs de $u_k$ pour $k \in \ii{0;30}$ par la fonction \texttt{u}. Discuter le résultat.
  
    \emph{Indication :} on pourra utiliser une échelle semi-logarithmique.
  \item Proposer un modèle de complexité plus réaliste et étudier dans ce modèle $n$ la complexité asymptotique de la fonction \texttt{u}.
    On pourra déterminer explicitement $u_n$. 
\end{enumerate}
