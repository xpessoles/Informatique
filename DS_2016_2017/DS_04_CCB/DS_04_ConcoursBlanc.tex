\documentclass[10pt,fleqn]{article} % Default font size and left-justified equations
\usepackage[%
    pdftitle={Informatique : Transfert thermique},
    pdfauthor={Xavier Pessoles}]{hyperref}

%%%%%%%%%%%%%%%%%%%%%%%%%%%%%%%%%%%%%%%%%
% Original author:
% Mathias Legrand (legrand.mathias@gmail.com) with modifications by:
% Vel (vel@latextemplates.com)
% License:
% CC BY-NC-SA 3.0 (http://creativecommons.org/licenses/by-nc-sa/3.0/)
%%%%%%%%%%%%%%%%%%%%%%%%%%%%%%%%%%%%%%%%%



%----------------------------------------------------------------------------------------
%	MAIN TABLE OF CONTENTS
%----------------------------------------------------------------------------------------


% Part text styling
\titlecontents{part}[0cm]
{\addvspace{20pt}\centering\large\bfseries}
{}
{}
{}

% Chapter text styling
\titlecontents{chapter}[1.25cm] % Indentation
{\addvspace{12pt}\large\sffamily\bfseries} % Spacing and font options for chapters
{\color{bleuxp!60}\contentslabel[\Large\thecontentslabel]{1.25cm}\color{bleuxp}} % Chapter number
{\color{bleuxp}}  
{\color{bleuxp!60}\normalsize\;\titlerule*[.5pc]{.}\;\thecontentspage} % Page number

% Section text styling
\titlecontents{section}[1.25cm] % Indentation
{\addvspace{3pt}\sffamily\bfseries} % Spacing and font options for sections
{\color{bleuxp!60}\contentslabel[\thecontentslabel]{1.25cm} \color{bleuxp}} % Section number
{\color{bleuxp}}
{\hfill\color{bleuxp!60}\thecontentspage} % Page number
[]

% Subsection text styling
\titlecontents{subsection}[1.25cm] % Indentation
{\addvspace{1pt}\sffamily\small} % Spacing and font options for subsections
{\contentslabel[\thecontentslabel]{1.25cm}} % Subsection number
{}
{\ \titlerule*[.5pc]{.}\;\thecontentspage} % Page number
[]


% Subsection text styling
\titlecontents{subsubsection}[1.25cm] % Indentation
{\addvspace{1pt}\sffamily\small} % Spacing and font options for subsections
{\contentslabel[\thecontentslabel]{1.25cm}} % Subsection number
{}
{\ \titlerule*[.5pc]{.}\;\thecontentspage} % Page number
[]

% List of figures
\titlecontents{figure}[0em]
{\addvspace{-5pt}\sffamily}
{\thecontentslabel\hspace*{1em}}
{}
{\ \titlerule*[.5pc]{.}\;\thecontentspage}
[]

% List of tables
\titlecontents{table}[0em]
{\addvspace{-5pt}\sffamily}
{\thecontentslabel\hspace*{1em}}
{}
{\ \titlerule*[.5pc]{.}\;\thecontentspage}
[]

%----------------------------------------------------------------------------------------
%	MINI TABLE OF CONTENTS IN PART HEADS
%----------------------------------------------------------------------------------------

% Chapter text styling
\titlecontents{lchapter}[0em] % Indenting
{\addvspace{15pt}\large\sffamily\bfseries} % Spacing and font options for chapters
{\color{bleuxp}\contentslabel[\Large\thecontentslabel]{1.25cm}\color{bleuxp}} % Chapter number
{}  
{\color{bleuxp}\normalsize\sffamily\bfseries\;\titlerule*[.5pc]{.}\;\thecontentspage} % Page number

% Section text styling
\titlecontents{lsection}[0em] % Indenting
{\sffamily\small} % Spacing and font options for sections
{\contentslabel[\thecontentslabel]{1.25cm}} % Section number
{}
{}

% Subsection text styling
\titlecontents{lsubsection}[.5em] % Indentation
{\normalfont\footnotesize\sffamily} % Font settings
{}
{}
{}

%----------------------------------------------------------------------------------------
%	PAGE HEADERS
%----------------------------------------------------------------------------------------




\pagestyle{fancy}
 \renewcommand{\headrulewidth}{0pt}
 \fancyhead{}
 
 % ENTETES de page
 \fancyhead[L]{%
 \begin{tikzpicture}[overlay]
\node(logo) at (1,0)
    {\includegraphics[width=2cm]{logo_lycee.png}};
\end{tikzpicture}
 %\noindent\begin{minipage}[c]{2.6cm}%
 %\includegraphics[width=2cm]{logo_lycee.png}%
 %\end{minipage}
}

\fancyhead[C]{\rule{8cm}{.5pt}}

 \fancyhead[R]{%
 \noindent\begin{minipage}[c]{3cm}
 \begin{flushright}
 \footnotesize{\textit{\textsf{\xxtete}}}%
 \end{flushright}
 \end{minipage}
}

 \fancyfoot{}
 % PIEDS de page
\fancyfoot[C]{\rule{12cm}{.5pt}}
\renewcommand{\footrulewidth}{0.2pt}
\fancyfoot[C]{\footnotesize{\bfseries \thepage}}
\fancyfoot[L]{ 
\begin{minipage}[c]{.4\linewidth}
\noindent\footnotesize{{\xxauteur}}
\end{minipage}}

\fancyfoot[R]{\footnotesize{\xxpied}
\ifthenelse{\isodd{\value{page}}}{
\begin{tikzpicture}[overlay]
\node[shape=rectangle, 
      rounded corners = .25 cm,
	  draw= bleuxp,
	  line width=2pt, 
	  fill = bleuxp!10,
	  minimum width  = 2.5cm,
	  minimum height = 3cm,] at (\xxposongletx,\xxposonglety) {};
\node at (\xxposonglettext,\xxposonglety) {\rotatebox{90}{\textbf{\large\color{bleuxp}{\xxonglet}}}};
%{};
\end{tikzpicture}}{}
}



%
%
%
% Removes the header from odd empty pages at the end of chapters
\makeatletter
%\renewcommand{\cleardoublepage}{
%\clearpage\ifodd\c@page\else
%\hbox{}
%\vspace*{\fill}
%\thispagestyle{empty}
%\newpage
%\fi}

%\fancypagestyle{plain}{%
%\fancyhf{} % vide l’en-tête et le pied~de~page.
%%\fancyfoot[C]{\bfseries \thepage} % numéro de la page en cours en gras
%% et centré en pied~de~page.
%\fancyfoot[R]{\footnotesize{\xxpied}}
%\fancyfoot[C]{\rule{12cm}{.5pt}}
%\renewcommand{\footrulewidth}{0.2pt}
%\fancyfoot[C]{\footnotesize{\bfseries \thepage}}
%\fancyfoot[L]{ 
%\begin{minipage}[c]{.4\linewidth}
%\noindent\footnotesize{{\xxauteur}}
%\end{minipage}}}

\fancypagestyle{plain}{%
\fancyhf{} % vide l’en-tête et le pied~de~page.
\fancyfoot[C]{\rule{12cm}{.5pt}}
\renewcommand{\footrulewidth}{0.2pt}
\fancyfoot[C]{\footnotesize{\bfseries \thepage}}
\fancyfoot[L]{ 
\begin{minipage}[c]{.4\linewidth}
\noindent\footnotesize{{\xxauteur}}
\end{minipage}}
\fancyfoot[R]{\footnotesize{\xxpied}}
}







%----------------------------------------------------------------------------------------
%	SECTION NUMBERING IN THE MARGIN
%----------------------------------------------------------------------------------------
\setcounter{secnumdepth}{3}
\setcounter{tocdepth}{2}



\makeatletter
\renewcommand{\@seccntformat}[1]{\llap{\textcolor{bleuxp}{\csname the#1\endcsname}\hspace{1em}}}                    
\renewcommand{\section}{\@startsection{section}{1}{\z@}
{-4ex \@plus -1ex \@minus -.4ex}
{1ex \@plus.2ex }
{\normalfont\large\sffamily\bfseries}}
\renewcommand{\subsection}{\@startsection {subsection}{2}{\z@}
{-3ex \@plus -0.1ex \@minus -.4ex}
{0.5ex \@plus.2ex }
{\normalfont\sffamily\bfseries}}
\renewcommand{\subsubsection}{\@startsection {subsubsection}{3}{\z@}
{-2ex \@plus -0.1ex \@minus -.2ex}
{.2ex \@plus.2ex }
{\normalfont\small\sffamily\bfseries}}                        
\renewcommand\paragraph{\@startsection{paragraph}{4}{\z@}
{-2ex \@plus-.2ex \@minus .2ex}
{.1ex}
{\normalfont\small\sffamily\bfseries}}

%----------------------------------------------------------------------------------------
%	PART HEADINGS
%----------------------------------------------------------------------------------------


%----------------------------------------------------------------------------------------
%	CHAPTER HEADINGS
%----------------------------------------------------------------------------------------

% \newcommand{\thechapterimage}{}%
% \newcommand{\chapterimage}[1]{\renewcommand{\thechapterimage}{#1}}%
% \def\@makechapterhead#1{%
% {\parindent \z@ \raggedright \normalfont
% \ifnum \c@secnumdepth >\m@ne
% \if@mainmatter
% \begin{tikzpicture}[remember picture,overlay]
% \node at (current page.north west)
% {\begin{tikzpicture}[remember picture,overlay]
% \node[anchor=north west,inner sep=0pt] at (0,0) {\includegraphics[width=\paperwidth]{\thechapterimage}};
% \draw[anchor=west] (\Gm@lmargin,-9cm) node [line width=2pt,rounded corners=15pt,draw=bleuxp,fill=white,fill opacity=0.5,inner sep=15pt]{\strut\makebox[22cm]{}};
% \draw[anchor=west] (\Gm@lmargin+.3cm,-9cm) node {\huge\sffamily\bfseries\color{black}\thechapter. #1\strut};
% \end{tikzpicture}};
% \end{tikzpicture}
% \else
% \begin{tikzpicture}[remember picture,overlay]
% \node at (current page.north west)
% {\begin{tikzpicture}[remember picture,overlay]
% \node[anchor=north west,inner sep=0pt] at (0,0) {\includegraphics[width=\paperwidth]{\thechapterimage}};
% \draw[anchor=west] (\Gm@lmargin,-9cm) node [line width=2pt,rounded corners=15pt,draw=bleuxp,fill=white,fill opacity=0.5,inner sep=15pt]{\strut\makebox[22cm]{}};
% \draw[anchor=west] (\Gm@lmargin+.3cm,-9cm) node {\huge\sffamily\bfseries\color{black}#1\strut};
% \end{tikzpicture}};
% \end{tikzpicture}
% \fi\fi\par\vspace*{270\p@}}}

%-------------------------------------------

\def\@makeschapterhead#1{%
\begin{tikzpicture}[remember picture,overlay]
\node at (current page.north west)
{\begin{tikzpicture}[remember picture,overlay]
\node[anchor=north west,inner sep=0pt] at (0,0) {\includegraphics[width=\paperwidth]{\thechapterimage}};
\draw[anchor=west] (\Gm@lmargin,-9cm) node [line width=2pt,rounded corners=15pt,draw=bleuxp,fill=white,fill opacity=0.5,inner sep=15pt]{\strut\makebox[22cm]{}};
\draw[anchor=west] (\Gm@lmargin+.3cm,-9cm) node {\huge\sffamily\bfseries\color{black}#1\strut};
\end{tikzpicture}};
\end{tikzpicture}
\par\vspace*{270\p@}}
\makeatother



%----------------------------------------------------------------------------------------
%	
%----------------------------------------------------------------------------------------

\newcommand{\thechapterimage}{}%
\newcommand{\chapterimage}[1]{\renewcommand{\thechapterimage}{#1}}%
\def\@makechapterhead#1{%
{\parindent \z@ \raggedright \normalfont
\begin{tikzpicture}[remember picture,overlay]
\node at (current page.north west)
{\begin{tikzpicture}[remember picture,overlay]
\node[anchor=north west,inner sep=0pt] at (0,0) {\includegraphics[width=\paperwidth]{\thechapterimage}};
%\draw[anchor=west] (\Gm@lmargin,-9cm) node [line width=2pt,rounded corners=15pt,draw=bleuxp,fill=white,fill opacity=0.5,inner sep=15pt]{\strut\makebox[22cm]{}};
%\draw[anchor=west] (\Gm@lmargin+.3cm,-9cm) node {\huge\sffamily\bfseries\color{black}\thechapter. #1\strut};
\end{tikzpicture}};
\end{tikzpicture}
\par\vspace*{270\p@}
}}


%% Questions et exercices
\newcounter{numques}%Création d'un compteur qui s'appelle numques
\setcounter{numques}{0}%initialisation du compteur
\newcommand{\question}[1]{%Création d'une macro ayant un paramètre
\addtocounter{numques}{1}%chaque fois que cette macro est appelée, elle ajoute 1 au compteur numexos
\textbf{Question\, \textcolor{bleuxp}{\thenumques}\,}\,\textit{#1}}

\newcounter{numexo}%Création d'un compteur qui s'appelle numques
\setcounter{numexo}{0}%initialisation du compteur
\newcommand{\exer}[1]{%Création d'une macro ayant un paramètre
\refstepcounter{numexo} % incrément compteur et label
%\addtocounter{numexo}{1}%chaque fois que cette macro est appelée, elle ajoute 1 au compteur numexo
\noindent\textsf{\textbf{Exercice\, \textcolor{bleuxp}{\thenumexo}\, -- \, #1}}}



% \makeatletter             
% \renewcommand{\subparagraph}{\@startsection{exo}{5}{\z@}%
                                    % {-2ex \@plus-.2ex \@minus .2ex}%
                                    % {0ex}%               
% {\normalfont\bfseries Question \hspace{.7cm} }}
% \makeatother
% \renewcommand{\thesubparagraph}{\arabic{subparagraph}} 
% \makeatletter


%%%%%%%%%%%%
% Définition des vecteurs 
%%%%%%%%%%%%
\newcommand{\vect}[1]{\overrightarrow{#1}}
\newcommand{\axe}[2]{\left(#1,\vect{#2}\right)}
\newcommand{\couple}[2]{\left(#1,\vect{#2}\right)}
\newcommand{\angl}[2]{\left(\vect{#1},\vect{#2}\right)}

\newcommand{\rep}[1]{\mathcal{R}_{#1}}
\newcommand{\quadruplet}[4]{\left(#1;#2,#3,#4 \right)}
\newcommand{\repere}[4]{\left(#1;\vect{#2},\vect{#3},\vect{#4} \right)}
\newcommand{\base}[3]{\left(\vect{#1},\vect{#2},\vect{#3} \right)}


\newcommand{\vx}[1]{\vect{x_{#1}}}
\newcommand{\vy}[1]{\vect{y_{#1}}}
\newcommand{\vz}[1]{\vect{z_{#1}}}

\newcommand{\norm}[1]{\ensuremath{\left\Vert {#1}\right\Vert}}
\newcommand{\Ker}{\mathop{\mathrm{Ker}}\nolimits}

% d droit pour le calcul différentiel
\newcommand{\dd}{\text{d}}

\newcommand{\inertie}[2]{I_{#1}\left( #2\right)}
\newcommand{\matinertie}[7]{
\begin{pmatrix}
#1 & #6 & #5 \\
#6 & #2 & #4 \\
#5 & #4 & #3 \\
\end{pmatrix}_{#7}}
%%%%%%%%%%%%
% Définition des torseurs 
%%%%%%%%%%%%

\newcommand{\ec}[2]{%
\mathcal{E}_c\left(#1/#2\right)}

\newcommand{\pext}[3]{%
\mathcal{P}\left(#1\rightarrow#2/#3\right)}

\newcommand{\pint}[3]{%
\mathcal{P}\left(#1 \stackrel{\text{#3}}{\leftrightarrow} #2\right)}


 \newcommand{\torseur}[1]{%
\left\{{#1}\right\}
}

\newcommand{\torseurcin}[3]{%
\left\{\mathcal{#1} \left(#2/#3 \right) \right\}
}

\newcommand{\torseurci}[2]{%
\left\{\sigma \left(#1/#2 \right) \right\}
}
\newcommand{\torseurdyn}[2]{%
\left\{\mathcal{D} \left(#1/#2 \right) \right\}
}


\newcommand{\torseurstat}[3]{%
\left\{\mathcal{#1} \left(#2\rightarrow #3 \right) \right\}
}


 \newcommand{\torseurc}[8]{%
%\left\{#1 \right\}=
\left\{
{#1}
\right\}
 = 
\left\{%
\begin{array}{cc}%
{#2} & {#5}\\%
{#3} & {#6}\\%
{#4} & {#7}\\%
\end{array}%
\right\}_{#8}%
}

 \newcommand{\torseurcol}[7]{
\left\{%
\begin{array}{cc}%
{#1} & {#4}\\%
{#2} & {#5}\\%
{#3} & {#6}\\%
\end{array}%
\right\}_{#7}%
}

 \newcommand{\torseurl}[3]{%
%\left\{\mathcal{#1}\right\}_{#2}=%
\left\{%
\begin{array}{l}%
{#1} \\%
{#2} %
\end{array}%
\right\}_{#3}%
}

% Vecteur vitesse
 \newcommand{\vectv}[3]{%
\vect{V\left( {#1} \in {#2}/{#3}\right)}
}

% Vecteur force
\newcommand{\vectf}[2]{%
\vect{R\left( {#1} \rightarrow {#2}\right)}
}

% Vecteur moment stat
\newcommand{\vectm}[3]{%
\vect{\mathcal{M}\left( {#1}, {#2} \rightarrow {#3}\right)}
}




% Vecteur résultante cin
\newcommand{\vectrc}[2]{%
\vect{R_c \left( {#1}/ {#2}\right)}
}
% Vecteur moment cin
\newcommand{\vectmc}[3]{%
\vect{\sigma \left( {#1}, {#2} /{#3}\right)}
}


% Vecteur résultante dyn
\newcommand{\vectrd}[2]{%
\vect{R_d \left( {#1}/ {#2}\right)}
}
% Vecteur moment dyn
\newcommand{\vectmd}[3]{%
\vect{\delta \left( {#1}, {#2} /{#3}\right)}
}

% Vecteur accélération
 \newcommand{\vectg}[3]{%
\vect{\Gamma \left( {#1} \in {#2}/{#3}\right)}
}

% Vecteur omega
 \newcommand{\vecto}[2]{%
\vect{\Omega\left( {#1}/{#2}\right)}
}
% }$$\left\{\mathcal{#1} \right\}_{#2} =%
% \left\{%
% \begin{array}{c}%
%  #3 \\%
%  #4 %
% \end{array}%
% \right\}_{#5}}

\newcommand{\N}{\mathbb{N}}
\newcommand{\Z}{\mathbb{Z}}
\newcommand{\R}{\mathbb{R}}
\newcommand{\C}{\mathbb{C}}
\newcommand{\K}{\mathbb{K}}

\newcommand{\cA}{\mathscr{A}}
\newcommand{\cM}{\mathscr{M}}
\newcommand{\cL}{\mathscr{L}}
\newcommand{\cS}{\mathscr{S}}

\newcommand{\python}{\texttt{Python}}

\newcommand{\z}[1]{\Z_{#1}}
\newcommand{\ztimes}[1]{\Z_{#1}^{\times}}
\newcommand{\ii}[1]{[\![#1[\![}
\newcommand{\iif}[1]{[\![#1]\!]}
\newcommand{\llbr}{\ensuremath{\llbracket}}
\newcommand{\rrbr}{\ensuremath{\rrbracket}}
%\newcommand{\p}[1]{\left(#1\right)}
\newcommand{\ens}[1]{\left\{ #1 \right\}}
\newcommand{\croch}[1]{\left[ #1 \right]}
%\newcommand{\of}[1]{\lstinline{#1}}
% \newcommand{\py}[2]{%
%   \begin{tabular}{|l}
%     \lstinline+>>>+\textbf{\of{#1}}\\
%     \of{#2}
%   \end{tabular}\par{}
% }
\newcommand{\floor}[1]{\left\lfloor#1\right\rfloor}
\newcommand{\ceil}[1]{\left\lceil#1\right\rceil}
\newcommand{\abs}[1]{\left|#1\right|}


% Binaire, octal, hexa
\newcommand{\hex}[1]{\underline{\text{\texttt{#1}}}_{16}}
\newcommand{\oct}[1]{\underline{\text{\texttt{#1}}}_{8}}
\newcommand{\bin}[1]{\underline{\text{\texttt{#1}}}_{2}}
\DeclareMathOperator{\mmod}{\texttt{\%}}


% Fonctions et systèmes
\newcommand{\fct}[5][t]{%
  \begin{array}[#1]{rcl}
    #2 & \rightarrow & #3\\
    #4 & \mapsto     & #5\\
  \end{array}
}
\newcommand{\fonction}[5]{#1 : \left\{\begin{array}{rcl} #2& \longrightarrow &#3 \\ #4 &\longmapsto & #5\end{array}\right.}
\newenvironment{systeme}{\left\{ \begin{array}{rcl}}{\end{array}\right.}

% Matrices
\newcommand{\mat}[1]{
  \begin{pmatrix}
    #1
  \end{pmatrix}
}
\newcommand{\inv}{\ensuremath{^{-1}}}
\newcommand{\bpm}{\begin{pmatrix}}
\newcommand{\epm}{\end{pmatrix}}


% bases de données
\newcommand{\relat}[1]{\textsc{#1}}
\newcommand{\attr}[1]{\emph{#1}}
\newcommand{\prim}[1]{\uline{#1}}
\newcommand{\foreign}[1]{\#\textsl{#1}}


% Bases de données

\newcommand{\att}{\ensuremath{\mathbf{att}}}
\newcommand{\dom}{\ensuremath{\mathbf{dom}}}
\newcommand{\sort}{\ensuremath{\mathbf{sort}}}
\newcommand{\relname}{\ensuremath{\mathbf{relname}}}
\newcommand{\var}{\ensuremath{\mathbf{var}}}
\newcommand{\FILM}{\ensuremath{\mathtt{FILM}}}
\newcommand{\JOUE}{\ensuremath{\mathtt{JOUE}}}
\newcommand{\PERSONNE}{\ensuremath{\mathtt{PERSONNE}}}
\newcommand{\PERSONNAGE}{\ensuremath{\mathtt{PERSONNAGE}}}

\newcommand{\ttid}{\ensuremath{\mathtt{id}}}
\newcommand{\tttitre}{\ensuremath{\mathtt{titre}}}
\newcommand{\ttdate}{\ensuremath{\mathtt{date}}}
\newcommand{\ttidr}{\ensuremath{\mathtt{idrealisateur}}}
\newcommand{\ttdatenais}{\ensuremath{\mathtt{datenaissance}}}
\newcommand{\ttnom}{\ensuremath{\mathtt{nom}}}
\newcommand{\ttprenom}{\ensuremath{\mathtt{prenom}}}
\newcommand{\ttidacteur}{\ensuremath{\mathtt{idacteur}}}
\newcommand{\ttidfilm}{\ensuremath{\mathtt{idfilm}}}
\newcommand{\ttidpersonnage}{\ensuremath{\mathtt{idpersonnage}}}

\newcommand{\fv}{\mathrm{libre}}
\newcommand{\sem}[1]{[\![ #1 ]\!]}


\fichetrue
%\fichefalse

%\proftrue
\proffalse

%\tdtrue
\tdfalse

%\courstrue
\coursfalse

% -------------------------------------
% Déclaration des titres
% -------------------------------------

\def\discipline{Informatique \ifprof \\ Corrigé \else \fi}
\def\xxtete{Informatique}

\def\classe{PT -- PT $\star$}
\def\xxnumpartie{CB 2017}
\def\xxpartie{Concours Blanc 2017}

\def\xxnumchapitre{Autour de données météorologiques }
\def\xxchapitre{}%\textit{$\;$ \\ D'après Concours CCP -- PC 2015}}

\def\xxtitreexo{Prothèse Active Transtibiale}
\def\xxsourceexo{\hspace{.2cm} D'après X 2015 -- MP/PC.}

\def\xxposongletx{2}
\def\xxposonglettext{1.45}
\def\xxposonglety{20}
\def\xxonglet{\textsf{CB 2017 }}

\def\xxactivite{}
\def\xxauteur{\textsl{La Martinière Monplaisir -- Lyon\\Étienne Mimard -- Saint-Étienne }}

\def\xxcompetences{%
\texttt{%
\textbf{Savoirs et compétences :}\\
\noindent \textbf{Résoudre :} à partir des modèles retenus :
\begin{itemize}[label=\ding{112},font=\color{ocre}] 
\item choisir une méthode de résolution analytique, graphique, numérique;
\item mettre en \oe{}uvre une méthode de résolution.
\end{itemize}
\begin{itemize}[label=\ding{112},font=\color{ocre}] 
\item \textit{Rés -- C1.1 :} Loi entrée sortie géométrique et cinématique -- Fermeture géométrique.
\end{itemize}
%
%\noindent \textit{Mod2 -- C4.1 :} Représentation par schéma bloc.
}}

\def\xxfigures{
%\includegraphics[width=.8\textwidth]{images/prot_01}
}%figues de la page de garde

\def\xxpied{%
Concours Blanc 2017}


\setcounter{secnumdepth}{5}
%---------------------------------------------------------------------------


\begin{document}

%\chapterimage{png/Fond_Cin}
\pagestyle{empty}


%%%%%%%% PAGE DE GARDE COURS
\ifcours
% ==== BANDEAU DES TITRES ==== 
\begin{tikzpicture}[remember picture,overlay]
\node at (current page.north west)
{\begin{tikzpicture}[remember picture,overlay]
\node[anchor=north west,inner sep=0pt] at (0,0) {\includegraphics[width=\paperwidth]{\thechapterimage}};
\draw[anchor=west] (-2cm,-8cm) node [line width=2pt,rounded corners=15pt,draw=ocre,fill=white,fill opacity=0.6,inner sep=40pt]{\strut\makebox[22cm]{}};
\draw[anchor=west] (1cm,-8cm) node {\huge\sffamily\bfseries\color{black} %
\begin{minipage}{1cm}
\rotatebox{90}{\LARGE\sffamily\textsc{\color{ocre}\textbf{\xxnumpartie}}}
\end{minipage} \hfill
\begin{minipage}[c]{14cm}
\begin{titrepartie}
\begin{flushright}
\renewcommand{\baselinestretch}{1.1} 
\Large\sffamily\textsc{\textbf{\xxpartie}}
\renewcommand{\baselinestretch}{1} 
\end{flushright}
\end{titrepartie}
\end{minipage} \hfill
\begin{minipage}[c]{3.5cm}
{\large\sffamily\textsc{\textbf{\color{ocre} \discipline}}}
\end{minipage} 
 };
\end{tikzpicture}};
\end{tikzpicture}
% ==== FIN BANDEAU DES TITRES ==== 


% ==== ONGLET 
\begin{tikzpicture}[overlay]
\node[shape=rectangle, 
      rounded corners = .25 cm,
	  draw= ocre,
	  line width=2pt, 
	  fill = ocre!10,
	  minimum width  = 2.5cm,
	  minimum height = 3cm,] at (18.3cm,0) {};
\node at (17.7cm,0) {\rotatebox{90}{\textbf{\Large\color{ocre}{\classe}}}};
%{};
\end{tikzpicture}
% ==== FIN ONGLET 


\vspace{3.5cm}

\begin{tikzpicture}[remember picture,overlay]
\draw[anchor=west] (-2cm,-6cm) node {\huge\sffamily\bfseries\color{black} %
\begin{minipage}{2cm}
\begin{center}
\LARGE\sffamily\textsc{\color{ocre}\textbf{\xxactivite}}
\end{center}
\end{minipage} \hfill
\begin{minipage}[c]{15cm}
\begin{titrechapitre}
\renewcommand{\baselinestretch}{1.1} 
\Large\sffamily\textsc{\textbf{\xxnumchapitre}}

\Large\sffamily\textsc{\textbf{\xxchapitre}}
\vspace{.5cm}

\renewcommand{\baselinestretch}{1} 
\normalsize\normalfont
\xxcompetences
\end{titrechapitre}
\end{minipage}  };
\end{tikzpicture}
\vfill

\begin{flushright}
\begin{minipage}[c]{.3\linewidth}
\begin{center}
\xxfigures
\end{center}
\end{minipage}\hfill
\begin{minipage}[c]{.6\linewidth}
\startcontents
%\printcontents{}{1}{}
\printcontents{}{1}{}
\end{minipage}
\end{flushright}

\begin{tikzpicture}[remember picture,overlay]
\draw[anchor=west] (4.5cm,-.7cm) node {
\begin{minipage}[c]{.2\linewidth}
\begin{flushright}
\includegraphics[width=2cm]{logoCC}
\end{flushright}
\end{minipage}
\begin{minipage}[c]{.2\linewidth}
\textsl{\xxauteur} \\
\textsl{\classe}
\end{minipage}
 };
\end{tikzpicture}

\newpage
\pagestyle{fancy}

%\newpage
%\pagestyle{fancy}

\else
\fi
%% FIN PAGE DE GARDE DES COURS

%%%%%%%% PAGE DE GARDE TD
\iftd
%\begin{tikzpicture}[remember picture,overlay]
%\node at (current page.north west)
%{\begin{tikzpicture}[remember picture,overlay]
%\draw[anchor=west] (-2cm,-3.25cm) node [line width=2pt,rounded corners=15pt,draw=ocre,fill=white,fill opacity=0.6,inner sep=40pt]{\strut\makebox[22cm]{}};
%\draw[anchor=west] (1cm,-3.25cm) node {\huge\sffamily\bfseries\color{black} %
%\begin{minipage}{1cm}
%\rotatebox{90}{\LARGE\sffamily\textsc{\color{ocre}\textbf{\xxnumpartie}}}
%\end{minipage} \hfill
%\begin{minipage}[c]{13.5cm}
%\begin{titrepartie}
%\begin{flushright}
%\renewcommand{\baselinestretch}{1.1} 
%\Large\sffamily\textsc{\textbf{\xxpartie}}
%\renewcommand{\baselinestretch}{1} 
%\end{flushright}
%\end{titrepartie}
%\end{minipage} \hfill
%\begin{minipage}[c]{3.5cm}
%{\large\sffamily\textsc{\textbf{\color{ocre} \discipline}}}
%\end{minipage} 
% };
%\end{tikzpicture}};
%\end{tikzpicture}

%%%%%%%%%% PAGE DE GARDE TD %%%%%%%%%%%%%%%
%\begin{tikzpicture}[overlay]
%\node[shape=rectangle, 
%      rounded corners = .25 cm,
%	  draw= ocre,
%	  line width=2pt, 
%	  fill = ocre!10,
%	  minimum width  = 2.5cm,
%	  minimum height = 2.5cm,] at (18.5cm,0) {};
%\node at (17.7cm,0) {\rotatebox{90}{\textbf{\Large\color{ocre}{\classe}}}};
%%{};
%\end{tikzpicture}

% PARTIE ET CHAPITRE
%\begin{tikzpicture}[remember picture,overlay]
%\draw[anchor=west] (-1cm,-2.1cm) node {\large\sffamily\bfseries\color{black} %
%\begin{minipage}[c]{15cm}
%\begin{flushleft}
%\xxnumchapitre \\
%\xxchapitre
%\end{flushleft}
%\end{minipage}  };
%\end{tikzpicture}

% BANDEAU EXO
\iflivret % SI LIVRET
\begin{tikzpicture}[remember picture,overlay]
\draw[anchor=west] (-2cm,-3.3cm) node {\huge\sffamily\bfseries\color{black} %
\begin{minipage}{5cm}
\begin{center}
\LARGE\sffamily\color{ocre}\textbf{\textsc{\xxactivite}}

\begin{center}
\xxfigures
\end{center}

\end{center}
\end{minipage} \hfill
\begin{minipage}[c]{12cm}
\begin{titrechapitre}
\renewcommand{\baselinestretch}{1.1} 
\large\sffamily\textbf{\textsc{\xxtitreexo}}

\small\sffamily{\textbf{\textit{\color{black!70}\xxsourceexo}}}
\vspace{.5cm}

\renewcommand{\baselinestretch}{1} 
\normalsize\normalfont
\xxcompetences
\end{titrechapitre}
\end{minipage}};
\end{tikzpicture}
\else % ELSE NOT LIVRET
\begin{tikzpicture}[remember picture,overlay]
\draw[anchor=west] (-2cm,-4.5cm) node {\huge\sffamily\bfseries\color{black} %
\begin{minipage}{5cm}
\begin{center}
\LARGE\sffamily\color{ocre}\textbf{\textsc{\xxactivite}}

\begin{center}
\xxfigures
\end{center}

\end{center}
\end{minipage} \hfill
\begin{minipage}[c]{12cm}
\begin{titrechapitre}
\renewcommand{\baselinestretch}{1.1} 
\large\sffamily\textbf{\textsc{\xxtitreexo}}

\small\sffamily{\textbf{\textit{\color{black!70}\xxsourceexo}}}
\vspace{.5cm}

\renewcommand{\baselinestretch}{1} 
\normalsize\normalfont
\xxcompetences
\end{titrechapitre}
\end{minipage}};
\end{tikzpicture}

\fi

\else   % FIN IF TD
\fi


%%%%%%%% PAGE DE GARDE FICHE
\iffiche
\begin{tikzpicture}[remember picture,overlay]
\node at (current page.north west)
{\begin{tikzpicture}[remember picture,overlay]
\draw[anchor=west] (-2cm,-2.25cm) node [line width=2pt,rounded corners=15pt,draw=ocre,fill=white,fill opacity=0.6,inner sep=40pt]{\strut\makebox[22cm]{}};
\draw[anchor=west] (1cm,-2.25cm) node {\huge\sffamily\bfseries\color{black} %
\begin{minipage}{1cm}
\rotatebox{90}{\LARGE\sffamily\textsc{\color{ocre}\textbf{\xxnumpartie}}}
\end{minipage} \hfill
\begin{minipage}[c]{14cm}
\begin{titrepartie}
\begin{flushright}
\renewcommand{\baselinestretch}{1.1} 
\large\sffamily\textsc{\textbf{\xxpartie} \\} 

\vspace{.2cm}

\normalsize\sffamily\textsc{\textbf{\xxnumchapitre -- \xxchapitre}}
\renewcommand{\baselinestretch}{1} 
\end{flushright}
\end{titrepartie}
\end{minipage} \hfill
\begin{minipage}[c]{3.5cm}
{\large\sffamily\textsc{\textbf{\color{ocre} \discipline}}}
\end{minipage} 
 };
\end{tikzpicture}};
\end{tikzpicture}

\iflivret
\begin{tikzpicture}[overlay]
\node[shape=rectangle, 
      rounded corners = .25 cm,
	  draw= ocre,
	  line width=2pt, 
	  fill = ocre!10,
	  minimum width  = 2.5cm,
	  minimum height = 2.5cm,] at (18.5cm,1.1cm) {};
\node at (17.9cm,1.1cm) {\rotatebox{90}{\textsf{\textbf{\large\color{ocre}{\classe}}}}};
%{};
\end{tikzpicture}
\else
\begin{tikzpicture}[overlay]
\node[shape=rectangle, 
      rounded corners = .25 cm,
	  draw= ocre,
	  line width=2pt, 
	  fill = ocre!10,
	  minimum width  = 2.5cm,
%	  minimum height = 2.5cm,] at (18.5cm,1.1cm) {};
	  minimum height = 2.5cm,] at (18.6cm,1cm) {};
\node at (18cm,1cm) {\rotatebox{90}{\textsf{\textbf{\large\color{ocre}{\classe}}}}};
%{};
\end{tikzpicture}

\fi

\else
\fi



\vspace{1.7cm}
\pagestyle{fancy}
\thispagestyle{plain}


%\Huge
%\begin{center} \fbox{{Autour de données météorologiques}}\end{center}
%\normalsize
\begin{obj}
Le service des prévisions et d'observations météorologiques de météo-France relève 
toute l'année les conditions météorologiques dans chaque ville de France.\\
Cela permet de déterminer les évolutions de climat et d'aider les prévisions saisonnières.
 
Étudier le climat passé permet de mieux comprendre le fonctionnement du système climatique, 
clé pour anticiper ses évolutions futures.
 Pour cela, les climatologues doivent disposer de séries d'observations sur la période 
 la plus longue possible.\\
 
Pour assurer sa mission de conservation de la mémoire du climat, Météo-France assure 
la collecte, le contrôle et l'archivage des données climatiques dans une base nationale.\\ 
Cette base de données contient les données de métropole, d'outre-mer et des TAAF 
(Terres australes et antarctiques françaises) observées au sol, en mer ou en altitude. \\
Les principales informations recueillies concernent la température, les précipitations, l'humidité,
 la pression atmosphérique, le vent et le rayonnement.\\
 
 L'objectif de ce sujet est :
\begin{itemize}
 \item partie 1 : de créer quelques outils (autour du minimum) pour le traitement de données
  météorologiques;
 \item partie 2 : d'exploiter les mesures de température;
 \item partie 3 : d'extraire des données de la base de données de météo-France;
 \item partie 4 : de manipuler des données particulières récupérées dans un fichier;
 \item partie 5 : d'étudier une modélisation physique des variations annuelles de température.
 \end{itemize}
Remarque : chaque partie présente des questions de difficultés variées. 
  
\end{obj}



%Évolution de la température dans une partie de la France}

\section*{Partie 1 : quelques outils pour le traitement des données}

Dans toutes cette partie, on s'interdit l'usage de la fonction \textbf{min}
préprogrammée en Python et permettant d'obtenir directement le minimum d'une 
liste donnée en argument. En revanche, on se donne la fonction 
\textbf{mini} suivante, écrite en Python :

\begin{py}
\begin{lstlisting}
def mini(t):
    '''Calcule le minimum d'un tableau d'entiers ou de flottants.'''
    if len(t) == 0:
        return None
    p = t[0]
    for i in range(1,len(t)):
        if t[i] <= p:
            p = t[i]
    return p
\end{lstlisting}
\end{py}

\subparagraph{}
\textit{Expliquer le déroulement pas à pas (évolution de la valeur des
  variables) lors de l'appel   \textbf{mini([6,2,15,1,15,1])}, puis donner la valeur   renvoyée.}
	

\ifprof
\begin{corrige}~\
\begin{python}
mini([6,2,15,1,15,1])
pour i=1 mini=2
pour i=2 mini=2
pour i=3 mini=1
pour i=4 mini=1
pour i=5 mini=1
\end{python}
\end{corrige}
\else
\fi

\subparagraph{}
\textit{Évaluer la complexité temporelle de l'appel \textbf{mini(t)} en fonction du nombre $n$ 
d'éléments de  \texttt{t}.\\
Vous calculerez pour cela le nombre de comparaisons et d'opérations (addition, soustraction, division, multiplication) réalisées.}

\ifprof
\begin{corrige}
il y a (n-1) comparaisons et aucune opération soit une complexité en O(n)
\end{corrige}
\else
\fi

\subparagraph{}
 \textit{Proposer une modification de la fonction \textbf{mini} pour que la valeur
  renvoyée soit le \emph{maximum} et non le \emph{minimum}.}
	
\ifprof
\begin{corrige}~\
\begin{python}
def max(t):
    if len(t)==0:
        return None
    m=t[0]
    for i in range(1,len(t)):
        if t[i]>=m:
            m=t[i]
    return (m)
\end{python}
\end{corrige}
\else
\fi
	
\subparagraph{}
\textit{Définir, du point de vue de la programmation, ce qu'est une fonction récursive. Donner les étapes de l'écriture d'une fonction récursive en donnant comme exemple la recherche du minimum d'une liste.}

\ifprof
\begin{corrige}
Il est nécessaire d'avoir une condition d'arrêt et une boucle de récursivité. La condtion d'arrêt pour la fonction \textbf{mini\_recursive} est le mini de la liste à un seul élément et la boucle de récursivité est la comparaison du mini de la liste tronquée avec l'élément suivant.
\end{corrige}
\else
\fi
	
\subparagraph{}
\textit{Définir une fonction \textbf{mini\_recursive} qui a pour argument une liste de valeurs entières ou flottants et qui renvoie par une méthode récursive le \emph{minimum} de cette liste.}

\ifprof
\begin{corrige}~\
\begin{python}
def mini_recursive(t):
    n=len(t)
    if n==1:
        return t[0]
    else:
        p=mini_recursive(t[0:n-1])
        if t[-1]<p:
            p=t[-1]
        return p
\end{python}
\end{corrige}
\else
\fi

\ifprof
\begin{corrige}
\begin{figure}[h]
	\centering
		\includegraphics[width=0.5\textwidth]{images/recursivite.png}
\end{figure}
\end{corrige}
\else
\fi
  
 \subparagraph{}
\textit{ Écrire une fonction \textbf{position\_mini} d'argument une liste et renvoyant \textbf{une} position
  du minimum de cette liste.\\}
  
 Dans  l'exemple vu plus haut,  il y a deux positions pour lesquelles
  le minimum est atteint : $3$ et $5$.
	
\ifprof
\begin{corrige}~\
\begin{python}
def position_mini(t):
    indice=0
    p=t[0]
    for i in range(1,len(t)):
        if t[i]<=p: #ou t[i]<p
            p=t[i]
            indice=i
    return i
\end{python}
\end{corrige}
\else
\fi
  
On souhaite maintenant déterminer la valeur minimale d'un tableau
  bidimensionnel d'entiers ou de flottants (c'est une liste de listes).\\
  Par exemple le minimum de $[[10,3,15],[5,13,10]]$ est $3$
  
\subparagraph{}
\textit{Écrire une fonction \textbf{min2D} d'argument un tableau \textbf{t} d'entiers ou flottants 
et renvoyant le minimum de \textbf{t}. On utilisera la fonction \textbf{mini}.}

\ifprof
\begin{corrige}~\
\begin{python}
def mini2D(t):
    p=mini(t[0])
    for i in range(1,len(t)):
        p1=mini(t[i])
        if p1<p:
            p=p1
    return p
\end{python}
\end{corrige}
\else
\fi

\textit{Évaluer la complexité temporelle de cette fonction.
Vous calculerez pour cela le nombre de comparaisons et d'opérations (addition, soustraction, division, multiplication) réalisées.}
  
On souhaite, partant d'une liste constituée de couples
%\footnote{Ou liste de listes à deux éléments.}
  (chaîne, entier), déterminer la (les) chaîne(s) pour laquelle(s) l'entier/le flottant 
  associé est minimal, voici un exemple d'utilisation de cette fonction :
\begin{py}
\begin{lstlisting}
>>> chaine_mini([['Tokyo',7000],['Paris',6000],['Londres',8000]]) 
>>> 'Paris'
\end{lstlisting}
\end{py}
\subparagraph{}
\textit{Écrire une fonction \textbf{chaine\_mini} réalisant effectivement cette opération.}


\ifprof
\begin{corrige}~\
\begin{python}
def chaine_mini(L):
    nom=L[0][0]
    valeur=L[0][1]
    for i in range(1,len(t)):
        if L[i][1]<=valeur:
            valeur=L[i][1]
            nom=L[i][0]
    return nom
\end{python}
\end{corrige}
\else
\fi



On souhaite enfin, à partir d'une liste \textbf{L}
 d'entiers (ou flottants) et d'un entier (ou flottant) appelé \texttt{seuil} obtenir le nombre 
 d'éléments de \textbf{L} majorés (au sens strict) par \texttt{seuil}, voici un exemple d'utilisation de cette fonction : 
 \begin{py}
\begin{lstlisting}
>>> majores_par([12,-5,10,9],10)
>>> 2
\end{lstlisting}
\end{py}
\subparagraph{} 
\textit{Écrire enfin une fonction \textbf{majores\_par} réalisant cette opération.}

\ifprof
\begin{corrige}~\
\begin{python}
def majores_par(t,seuil):
    k=0
    for i in range(len(t)):
        if t[i]<seuil:
            k+=1
    return k
\end{python}
\end{corrige}
\else
\fi

\subparagraph{}
\textit{Modifier la fonction précédente pour obtenir une fonction \textbf{elements\_majores\_par}
  retournant la liste des éléments majorés par le seuil.}
	
\ifprof
\begin{corrige}~\
\begin{python}
def elements_majores_par(t,seuil):
    L1=[]
    for i in range(len(t)):
        if t[i]<seuil:
            L1.append(t[i])
    return L1
\end{python}
\end{corrige}
\else
\fi



\section*{Partie 2 : Exploitation des mesures de température}
La problématique de Météo France est de fournir des valeurs mesurées les plus justes possibles. Les écarts sont essentiellement dus :
\begin{itemize}
	\item à la configuration de l'installation de la station météo qui doit respecter les normes MF et OMM (à une distance minimum de 2 fois la hauteur des obstacles dans un secteur le plus ensoleillé possible même en hiver, positionné à 1,5m d'un sol herbeux, à, au moins, 100m de sources de chaleur artificielles ou réfléchissantes et d'étendues d'eau). Ces conditions permettent de limiter les erreurs de mesures dues à l'environnement. Ces dernières pouvant facilement atteindre un écart de 50\% dans des configurations défavorables;
	\item à la qualité des appareils de mesure.
\end{itemize}

\vspace{0.5cm}
Météo-France utilise, dans ses stations météos automatiques, des thermomètres conçus sur la variation de résistance nommés Pt100 (fil de platine dont la résistance vaut 100 Ohms à 0°C). Ces sondes possèdent une sensibilité égale à $0,39 \Omega.°C^{-1}$.

\vspace{0.5cm}
Les sondes de platine utilisées possèdent les caractéristiques suivantes :
\begin{itemize}
	\item pour $\theta=0°$, $R(0^{\circ}C)=100\Omega$
	\item pour $\theta$ température ambiante en (°C) et $R$ en Ohms :
	
	\begin{center}
	$R(\theta)=R(0^{\circ}C)\times(1+a\theta+b{\theta}^2+c{\theta}^3\times(\theta-100))$
	\end{center}
	
\end{itemize}
Les coefficients \texttt{a}, \texttt{b} et \texttt{c} sont donnés par une norme internationale EIT90.\\

On désire connaitre la température mesurée lorsque la résistance de la sonde est connue.\\
On souhaite donc résoudre l'équation \textbf{R($\theta$)-$R_{\text{mesure}}$=0} d'inconnue $\theta$ pour une valeur de résistance $R_{\text{mesure}}$ connue.

\textbf{On suppose pour la suite que les valeurs de $a$, $b$ et $c$ sont définies comme 
variables globales dans le programme principal.} \\

\subparagraph{}
\textit{Définir la fonction \textbf{g($\theta$)=R($\theta$)-$R_{\text{mesure}}$}.}

\ifprof
\begin{corrige}~\
\begin{python}
def g(x):
    return 100*(1+a*x+b*x**2+c*x**3*(x-100))-R
\end{python}
\end{corrige}
\else
\fi

\subparagraph{}
\textit{Écrire les instructions python permettant de tracer \textbf{g($\theta$)} pour $\theta\in[0°,50°]$ et $R=104\Omega$ par pas de $0.5°$.}

\ifprof
\begin{corrige}~\
\begin{python}
les_theta=[i*0.5 for i in range(101)]
les_g=[g(i) for i in les_theta]
plt.plot(les_theta,les_g)
plt.xlabel('les theta')
plt.ylabel('g(theta)')
plt.grid()
plt.show()
\end{python}
\end{corrige}
\else
\fi

\begin{figure}[h]
	\centering
		\includegraphics[width=0.5\textwidth]{images/courbe.png}
\end{figure}

\subparagraph{}
\textit{Écrire une fonction \textbf{dichotomie} d'arguments les bornes d'étude \texttt{a} et \texttt{b}, la fonction \texttt{g} ainsi que la précision \texttt{epsilon} et qui renvoie la valeur de $\theta$ à epsilon près solution de \textbf{g($\theta$)=0}. Proposer une valeur pour \texttt{a} et une valeur pour \texttt{b}. Justifier votre choix.}

\ifprof
\begin{corrige}~\
\begin{python}
def dichotomie(a, b, f, epsilon):
    g = a
    d = b
    while (d - g) > 2 * epsilon:
        m = (g + d)/2
        if f(g) * f(m) <= 0:
            d = m
        else:
            g = m
    return((d + g) / 2)
\end{python}
\end{corrige}
\else
\fi

\vspace{0.5cm}
Remarque : \texttt{b} et \texttt{c} étant très faibles, on utilise généralement la relation approchée suivante :
\begin{itemize}
	\item $R(\theta)=R(0°C)\times(1+a\theta)$ avec $a=3,90802.10^{-3}°C^{-1}$
\end{itemize}

\vspace{0.5cm}
Le réseau RADOME fournit aux abonnés les données "en temps réel" mesurées aux stations automatiques du réseau français au pas de temps infrahoraire (6mn).\\
Ces données stockées dans un tableur sont obtenues après traitement des données brutes des différents appareils de mesure.\\
Pour les mesures de température, deux listes sont générées : la liste des temps \texttt{les\_t} et la liste des températures \texttt{les\_theta} telles que \texttt{les\_theta}[i] correspond à la mesure de température à l'instant \texttt{les\_t}[i].

\texttt{les\_t=[220.254,220.279,223.254,223.279,.....1012.368]}\\
\texttt{les\_theta=[26,26,25.9,25.9,25.8,.......25.5]}

\subparagraph{}
\textit{Écrire une fonction \textbf{tempInfrahoraire} ayant pour arguments une liste des temps en seconde sur un intervalle de 6 minutes et la liste des températures en degré celsius correspondantes et qui renvoie deux valeurs : la moyenne des températures sur cet intervalle de temps et l'heure en seconde correspondant au relevé de température.\
Les pas de temps ne sont pas constants dans la liste des temps. Vous appliquerez pour l'intégration la méthode des rectangles à "droite".}

\ifprof
\begin{corrige}~\
\begin{python}
def tempinfrahoraire(Ltemps,Ltheta):
    s=0
    for i in range(1,len(Ltheta)):
        s=s+Ltheta[i]*(Ltemps[i]-Ltemps[i-1]) #intégration à droite, c'est la valeur de température mesurée à l'instant t{i+1} qui correspond à la température mesurée sur l'intervalle [t{i},t{i+1}]
    return s/(Ltemps[-1]-Ltemps[0]),Ltemps[-1]
\end{python}
\end{corrige}
\else
\fi

\subparagraph{}
\textit{Écrire les instructions python pour construire les deux listes \texttt{les\_t\_6min} et \texttt{les\_theta\_6min} à partir des deux listes des relevés pour une journée. Vous utiliserez la fonction \textbf{tempInfrahoraire}.}

\ifprof
\begin{corrige}~\
\begin{python}
les listes de temps en secondes et de theta ont la même longueur
les_t_6min=[]
les_theta_6min=[]
for i in range(len(theta)):
    s=0
    Ltemps=[]
    Ltheta=[]
    while s<=360:
        Ltemps.append(temps[i])
        Ltheta.append(theta[i])
        s=s+temps[i]-temps[i-1]
    moy_theta,heure=tempinfrahoraire(Ltemps,Ltheta)
    les_t_6min.append(heure)
    les_theta_6min.append(moy_theta)
\end{python}
\end{corrige}
\else
\fi



\section*{Partie 3 : Recherche dans les archives de Météo-France}
La base de données de Météo-France concernant l'année écoulée est 
constituée de deux tables :\\


	\begin{itemize}
		\item	la table \textbf{villes} stocke les informations sur chaque ville 
		de France et contient les colonnes (attributs) suivantes:
		
			\begin{description}
				\item[\textbf{insee}:] Chaîne de caractères représentant le 
				numéro INSEE (caractéristique : deux villes différentes ont 
				toujours deux numéros INSEE différents).

				\item[\textbf{nom}:] Chaîne de caractères représentant le nom de la ville.

				\item[\textbf{dpt}:] Chaîne de caractères représentant le département de la ville.\\
				Ce n'est pas un simple entier du fait des départements corses numérotés
				2A et 2B.

				\item[\textbf{lat}:] Flottant représentant la latitude de la 
				ville en degrés (positif si la ville est dans l'hémisphère 
				nord).\\ On prendra $47$ degre de latitude Nord comme étant 
				le milieu de la France.

				\item[\textbf{lon}:] Flottant représentant la longitude de la 
				ville en degrés (positif si la ville est à l'Est du méridien 
				de Greenwich). On prendra $2$ degre de longitude Est comme 
				étant le milieu de la France.
				
				\item[\textbf{pop}:] Entier représentant la population de la ville.
				
			\end{description}


\begin{exemple}
Extrait de la table \textbf{villes}
\begin{center}\begin{tabular}{*{6}{c}}
\textbf{insee} & \textbf{nom} & \textbf{dpt} & \textbf{lat} & \textbf{lon} & \textbf{pop}\\ 
\hline \hline
$2B033$&Bastia&2B&42.7&9.45&$42912$\\ 
$23067$&Saint Denis&23&45.7&2.25&$766$\\ 
$25056$&Besançon&25&47.23&6.02&$115879$\\ 
$46175$&Saint Denis&46&44.62&1.98&$940$\\ 
$67482$&Strasbourg&67&48.58&7.75&$761042$\\ 
$93066$&Saint Denis&93&48.93&2.35&$107762$\\ 
$97411$&Saint Denis&97&-20.87&55.43&$145347$\\ 
...\end{tabular}\end{center}
\end{exemple}

		
		\item	la table \textbf{mesures} enregistre l'évolution des températures
		au cours de l'année et contient les colonnes (attributs) suivantes :
		
			\begin{description}
				\item[\textbf{ville}:] Chaîne de caractères représentant le numéro INSEE 
				identifiant la ville.
				
				\item[\textbf{jour}:] Entier (de valeur comprise entre 1 et 365 
				pour l'année 2013) représentant le jour $j$ de l'année où est 
				faite la mesure (1 correspond au premier janvier alors que 365 
				correspond au 31 décembre).
				
				\item[\textbf{Tmin}:] Flottant représentant la température 
				minimale mesurée le jour $j$ en degrés celsius.
				
				\item[\textbf{Tmax}:] Flottant représentant la température 
				maximale mesurée le jour $j$ en degrés celsius.
				
			\end{description}
		
\begin{exemple}
Extrait de la table \textbf{mesures}
\begin{center}\begin{tabular}{*{4}{c}}
\textbf{ville} & \textbf{jour} & \textbf{Tmin} & \textbf{Tmax} \\ \hline \hline
$$25056$$&$110$&$5.2$&$9.7$\\ 
$$97411$$&$110$&$25.4$&$36.8$\\ 
$$25056$$&$216$&$16.7$&$33.9$\\ 
$$67482$$&$363$&$-2.7$&$3.3$\\ 
...\end{tabular}\end{center}
\end{exemple}

	\end{itemize}


	
%\subparagraph{}
%Définir rapidement ce qu'est une clef primaire. Expliquer si la colonne 
%	(attribut) \textbf{nom} de la première table peut être une clef primaire et, 
%	dans le cas contraire, si une autre colonne (attribut) peut jouer ce rôle. 
%	Même question pour la seconde table concernant la colonne (attribut) 
%	\textbf{ville}.
\subparagraph{} 	\textit{Qu'appelle-t-on \textbf{clé primaire} pour une table ? 
	Peut-on facilement en définir une pour la table \textbf{villes} ? 
	Pour la table \textbf{mesures} ?}
	
\ifprof
\begin{corrige}~\
\begin{python}
une clé primaire est la donnée qui permet d'identifier de manière unique un enregistrement dans une table
pour la table ville, la clé primaire est le numéro d'insee
pour la table mesures, il n'y a pas de clé primaire simple, il faut la construire à partir de deux colonnes ville,jour
\end{python}
\end{corrige}
\else
\fi
	
\subparagraph{} 	
\textit{ Écrire une requête en langage SQL qui récupère depuis la table 
	\textbf{mesures} les relevés dont la température moyenne définie par 
	$T_{\text{moy}}=\frac{T_{\text{min}}+T_{\text{max}}}{2}$ est strictement inférieure à $0$ degré celsius
		 (avec 	toutes les colonnes disponibles). }
		 
\ifprof
\begin{corrige}~\
\begin{python}
SELECT * FROM mesures WHERE (Tmax+Tmin)/2<0;
\end{python}
\end{corrige}
\else
\fi
	
\subparagraph{} 
\textit{Écrire une requête en langage SQL qui récupère depuis la table 
	\textbf{villes} le numéro INSEE, le nom et le département de toutes les villes du quart 
	nord-est de la France, c'est-à-dire celles dont la latitude est supérieure 
	à $47$ degrés et dont la longitude est supérieure à $2$ degrés.}

\ifprof
\begin{corrige}~\
\begin{python}
SELECT INSEE,nom,dpt  FROM villes WHERE lat>47 AND lon>2;
\end{python}
\end{corrige}
\else
\fi

\subparagraph{}
\textit{Écrire une requête SQL qui récupère le nom des villes dont la température maximale est supérieure strictement à 30°C.}

\ifprof
\begin{corrige}~\
\begin{python}
SELECT DISTINCT nom FROM villes INNER JOIN mesures ON INSEE=ville WHERE Tmax>30;
\end{python}
\end{corrige}
\else
\fi

\section*{Partie 4 : Manipulation des données}

Depuis la base de données, on a récupéré les 
relevés météorologiques  concernant la ville de Saint-Étienne pour l'année 2016 dans
 un fichier nommé \textbf{SaintEtienne\_2016.txt} .\\
  Ce fichier contient 365 lignes (une pour chaque mesure et donc jour de l'année) du 
type 

\begin{exemple}
Extrait du fichier \textbf{SaintEtienne\_2016.txt}
\begin{verbatim}
# Jour;Tmin;Tmax
1;4.5;12
2;5.1;10.2
3;2.5;8.8
...
168;11.8;22.3
169;12.3;21.1
...
366;-4.7;4.7
\end{verbatim}

NB : les trois petits points ne sont bien sûr pas présents dans 
le fichier mais signalent juste que l'on a omis de recopier un certain nombre 
de lignes.
\end{exemple}


Sur chaque ligne, la chaîne de caractère correspond à trois champs séparés par des point-virgules, à savoir
	\begin{itemize}
		\item	le numéro correspondant au jour de la mesure (entier naturel);
		\item	la température minimale mesurée ce jour (en degrés celsius, flottant);
		\item	la température maximale mesurée ce jour (en degrés celsius, flottant).
	\end{itemize}


Pour lire des fichiers de ce type,  on a écrit  :
 \begin{py}
\begin{lstlisting}
def lecture_fichier(fichier):
    f = open(fichier,mode='r')
    liste=f.readlines()
    f.close()
    
    jours,Tmin,Tmax = [ ],[ ],[ ]
    for ligne in liste:
        if ligne[0] != '\#':
            t,T1,T2 = ligne.split(';')
            jours.append(int(t))
            Tmin.append(float(T1))
            Tmax.append(float(T2))
    
    return jours,Tmin,Tmax

jours,Tmin,Tmax = lecture_fichier('SaintEtienne_2016.txt')
\end{lstlisting}
\end{py}

%
%
%
%Aide Python concernant \textbf{split}, méthode agissant sur une chaîne de caractères
%\begin{verbatim}
% |  split(...)
% |      S.split(sep=None, maxsplit=-1) -> list of strings
% |      
% |      Return a list of the words in S, using sep as the
% |      delimiter string.  If maxsplit is given, at most maxsplit
% |      splits are done. If sep is not specified or is None, any
% |      whitespace string is a separator and empty strings are
% |      removed from the result.
%\end{verbatim}

On admet que l'appel de la fonction a permis de récupérer trois listes : la liste \textbf{jours}
contenant le numéros du jour et  les deux listes \textbf{Tmin} et 
\textbf{Tmax} contenant les températures minimales et maximales concernant le jour 
correspondant.


\subparagraph{} 
\textit{Proposer un ordre de grandeur du nombre d'octets utiles du fichier 
\textbf{saintEtienne\_2016.txt} dont chaque caractère est codé en ASCII.}

\ifprof
\begin{corrige}~\
\begin{python}
nb de jours codage de 1 à 365 soit (9+2*90+3*265)caractères
nb de ';' 2*365
4 caractères maxi pour la température mini et maxi : 4*2*365
soit 4634 octets (ASCII 1 octet)
\end{python}
\end{corrige}
\else
\fi
	
\subparagraph{} 
	\textit{Écrire une fonction \textbf{moyenne} qui prend en entrée deux listes 
	\textbf{a} et \textbf{b} de même taille (condition qui ne doit \textbf{pas} être vérifiée)
	 et renvoie une liste de même taille contenant dans 
	la case d'indice \textbf{i} la valeur moyenne des valeurs des flottants 
	stockés dans les deux listes \textbf{a} et \textbf{b} à l'indice \textbf{i}.}
	
\ifprof
\begin{corrige}~\
\begin{python}
def moyenne(a,b):
    Lm=[]
    for i in range(len(a)):
        Lm.append((a[i]+b[i])/2)
    return Lm
\end{python}
\end{corrige}
\else
\fi
	
\subparagraph{} 
\textit{En appliquant la fonction précédente, écrire l'instruction Python 
	qui stocke dans la variable \textbf{Tmoy} la liste des températures moyennes 
	journalières à partir des données stockées dans les listes \textbf{Tmin} et~\textbf{Tmax}.}
	
\ifprof
\begin{corrige}~\
\begin{python}
Tmoy=moyenne(Tmin,Tmax)
\end{python}
\end{corrige}
\else
\fi

	
\subparagraph{} 
\textit{On considère qu'il est nécessaire de couper les arrivées d'eau 
	extérieures pour risque de gel quand la température moyenne sur la journée 
	est strictement inférieure à $0$ degré celsius. En utilisant une des fonctions 
	programmées dans la première partie, stocker dans la variable 
	\textbf{nb\_jours\_gel} le nombre de jours où il a fallu couper l'eau des 
	conduites extérieures pour la ville de Saint Étienne.}
	
\ifprof
\begin{corrige}~\
\begin{python}
nb_jours_gel=majores_par(Tmoy,0)
\end{python}
\end{corrige}
\else
\fi

\subparagraph{} 
\textit{On s'intéresse ici aux écarts entre les températures minimale et maximale au cours d'une même
 journée, écrire les lignes de commande qui permettent de créer une liste \textbf{ecart} 
 contenant les écarts $Tmax-Tmin$ pour chaque jour et de créer la variable mini\_ecart contenant 
 le minimum des écarts pour l'année $2016$.}

\ifprof
\begin{corrige}~\
\begin{python}
ecart=[Tmax[i]-Tmin[i] for i in range(len(Tmin))]
mini_ecart=mini(ecart)
\end{python}
\end{corrige}
\else
\fi
 
 \subparagraph{} 
\textit{On souhaite étudier les variations d'un jour sur l'autre de la température maximale.\\
 Donner les lignes de commandes permettant d'obtenir, \textbf{sans créer de nouvelles listes}, 
 la plus grande variation de la température maximale entre deux jours consécutifs.}

\ifprof
\begin{corrige}~\
\begin{python}
variation=0
for i in range(0,len(Tmax)-1):
    e=abs(Tmax[i+1]-Tmax[i])
    if e>variation:
        variation=e
print (variation)
\end{python}
\end{corrige}
\else
\fi
 

\section*{Partie 5 : une modélisation physique}

On souhaite modéliser les variations annuelles de 
température en les couplant aux variations saisonnières d'ensoleillement.

On admet que l'équation régissant l'évolution de la température (notée $\theta$ exprimée
en degrés Celsius) s'écrit

$$	\frac{d\theta(t)}{dt} + \alpha\left( \theta(t)+T_0\right)^4 = \mu\left(1 + \varepsilon
\cos(\omega t)\right)	$$

où $\alpha$, $T_0$, $\mu$, $\varepsilon$ et $\omega$ sont des constantes définies de manières 
globales et accessibles dans toute fonction.\\

L'équation différentielle vérifiée par $\theta$ peut se mettre sous la forme :
$\frac{d\theta(t)}{dt} =f(\theta(t),t)$ où $f$ est une fonction.

On souhaite résoudre cette équation différentielle par la méthode d'Euler.\\
On travaille sur l'intervalle de temps $[0,t_{max}]$ avec le pas de temps $d t$.\\
On pose $t_0=0$, $t_1=t_0+dt$, $\cdots$, $t_N=t_{N-1}+dt$ jusqu'à ce que $t_N+dt>t_{max}$
et, $\theta_k$ une valeur approchée de la température au temps $t_k$ obtenu par la méthode d'Euler.\\

\subparagraph{}
\textit{Écrire la  fonction \textbf{f} d'arguments deux flottants $t$ et $\theta$ et retournant 
$f(\theta,t)$.}

\ifprof
\begin{corrige}~\
\begin{python}
def f(theta,t):
    return (mu*(1+eps*np.cos(omega*t))-alpha*(theta+T0)**4)
\end{python}
\end{corrige}
\else
\fi

\subparagraph{} 
\textit{Justifier que pour tout $k$ tel que $1 \leq k \leq N$, on a 
$\theta_{k}=\theta_{k-1}+f(\theta_{k-1},t_{k-1}) \times dt$.}

\subparagraph{} 
\textit{Écrire la fonction \textbf{euler} d'argument trois flottants $theta0$, $dt$ et $tmax$ qui retourne 
deux listes : \textbf{temps} et \textbf{theta} contenant respectivement les valeurs 
$t_k$ et $\theta_k$ pour $0\leq k \leq N$.}

\ifprof
\begin{corrige}~\
\begin{python}
def euler(theta0,dt,tmax):
    temps=[0]
    theta=[theta0]
    t=dt
    while t<=tmax:
        temps.append(t)
        theta.append(theta[-1]+dt*f2(theta[-1],t))
        t=t+dt
    return temps,theta
\end{python}
\end{corrige}
\else
\fi

\section*{V \quad Annexe - documentation partielle}

On supposera que les fonctions ont \'et\'e import\'ees et sont directement utilisables.


\subsection*{V.1 \quad Tracer une courbe {\tt plot(*args,**kwargs)}}

Plot lines and/or markers to the Axes. {\tt args} is a variable length argument, allowing for multiple {\tt x}, {\tt y} pairs with an optional format string ({\tt fmt}). For example, each of the following is legal :

\fbox{ \parbox{0.97\textwidth}{\tt
plot (x, y)        \# plot x and y using default linestyle and color\\
plot (x, y, 'bo')  \# plot x and y using blue circle markers}}

An arbitrary number of {\tt x}, {\tt y}, {\tt fmt} groups can be specified, as in :


\fbox{ \parbox{0.97\textwidth}{\tt
plot (x1, y1, 'g\^ ', x2, y2, 'g-' )}}

Return value is a list of lines that were added.

By default, each line is assigned a different style specified by a 'style cycle'.

The following format string characters are accepted to control the line style or marker :

\hfil
\begin{tabular}{|cc|c|cc|}
\hline
character &description & \quad &  character & description \\
\hline
'-' &  solid line style & & '3' &  tri\_left marker\\
'' &  dashed line style & &  '4' &  tri\_right marker\\
'-.' &  dash-dot line style & & 's' &  square marker\\
' :' &  dotted line style & & 'p' &  pentagon marker\\
'.' &  point marker & & '*' &  star marker\\
',' &  pixel marker & & 'h' &  hexagon1 marker\\
'o' &  circle marker & & 'H' &  hexagon2 marker\\
'v' &  triangle\_down marker & & '+' &  plus marker\\
' ' &  triangle\_up marker & & 'x' &  x marker\\
'<' &  triangle\_left marker & & 'D' &  diamond marker\\
'>' &  triangle\_right marker & & 'd' &  thin\_diamond marker\\
'1' &  tri\_down marker & & '|' &  vline marker\\
'2' &  tri\_up marker & & '\_' &  hline marker \\
\hline
\end{tabular}

The following color abbreviations are supported :

\hfil
\begin{tabular}{|cc|}
\hline
character & color\\
\hline
b' &  blue \\
g' & green \\
r' & red \\
c' & cyan \\
m' & magenta \\
y'  & yellow \\
k'  & black \\
w'  & white \\
\hline
\end{tabular}


Line styles and colors are combined in a single format string, as in {\tt 'bo'} for blue circles.


\subsection*{V.2 \quad  Titres et l\'egende}

{\tt xlabel(s)} : Set the x axis label of the current axis with the string {\tt s}.

{\tt ylabel(s)} : Set the y axis label of the current axis with the string {\tt s}.

{\tt title(s)} : Set a title of the current axes with the string {\tt s}.

{\tt legend(*args)} : Places a legend on the axes. args must be a list of string.

\end{document}
