\documentclass[10pt]{article}
\input{style/coursHeadings}
\usepackage{algorithm}
\usepackage{algorithmic}


% Python sources
\usepackage{listings}
\usepackage{textcomp}
\usepackage{setspace}
%\usepackage{palatino}

%\usepackage{color}
\definecolor{Bleu}{rgb}{0.1,0.1,1.0}
\definecolor{Noir}{rgb}{0,0,0}
\definecolor{Grau}{rgb}{0.5,0.5,0.5}
\definecolor{DunkelGrau}{rgb}{0.15,0.15,0.15}
\definecolor{Hellbraun}{rgb}{0.5,0.25,0.0}
\definecolor{Magenta}{rgb}{1.0,0.0,1.0}
\definecolor{Gris}{gray}{0.5}
\definecolor{Vert}{rgb}{0,0.5,0}
\definecolor{SourceHintergrund}{rgb}{1,1.0,0.95}

%
\renewcommand{\lstlistlistingname}{Listings}
\renewcommand{\lstlistingname}{Listing}

\lstnewenvironment{python}[1][]{
\lstset{
language=python,
basicstyle=\ttfamily\footnotesize\setstretch{1}, 	
stringstyle=\color{red}, 
showstringspaces=false, 
alsoletter={1234567890},
otherkeywords={\ , \}, \{},
keywordstyle=\color{blue},
emph={access,and,break,class,continue,def,del,elif ,else,
except,exec,finally,for,from,global,if,import,in,i s,
lambda,not,or,pass,print,raise,return,try,while},
emphstyle=\color{black}\bfseries,
emph={[2]True, False, None, self},
emphstyle=[2]\color{green},
emph={[3]from, import, as},
emphstyle=[3]\color{blue},
upquote=true,
morecomment=[s]{"""}{"""},
commentstyle=\color{Hellbraun}\slshape, 
%emph={[4]1, 2, 3, 4, 5, 6, 7, 8, 9, 0},
emphstyle=[4]\color{blue},
literate=*{:}{{\textcolor{blue}:}}{1}
{=}{{\textcolor{blue}=}}{1}
{-}{{\textcolor{blue}-}}{1}
{+}{{\textcolor{blue}+}}{1}
{*}{{\textcolor{blue}*}}{1}
{!}{{\textcolor{blue}!}}{1}
{(}{{\textcolor{blue}(}}{1}
{)}{{\textcolor{blue})}}{1}
{[}{{\textcolor{blue}[}}{1}
{]}{{\textcolor{blue}]}}{1}
{<}{{\textcolor{blue}<}}{1}
{>}{{\textcolor{blue}>}}{1},
%framexleftmargin=1mm, framextopmargin=1mm, frame=shadowbox, rulesepcolor=\color{blue},#1
backgroundcolor=\color{SourceHintergrund}, 
framexleftmargin=1mm, framexrightmargin=1mm, framextopmargin=1mm, frame=single, framerule=1pt, rulecolor=\color{black},#1
}}{}
%%%%%%%%%%%%
% Définition des vecteurs 
%%%%%%%%%%%%
\newcommand{\vect}[1]{\overrightarrow{#1}}
\newcommand{\axe}[2]{\left(#1,\vect{#2}\right)}
\newcommand{\couple}[2]{\left(#1,\vect{#2}\right)}
\newcommand{\angl}[2]{\left(\vect{#1},\vect{#2}\right)}

\newcommand{\rep}[1]{\mathcal{R}_{#1}}
\newcommand{\quadruplet}[4]{\left(#1;#2,#3,#4 \right)}
\newcommand{\repere}[4]{\left(#1;\vect{#2},\vect{#3},\vect{#4} \right)}
\newcommand{\base}[3]{\left(\vect{#1},\vect{#2},\vect{#3} \right)}


\newcommand{\vx}[1]{\vect{x_{#1}}}
\newcommand{\vy}[1]{\vect{y_{#1}}}
\newcommand{\vz}[1]{\vect{z_{#1}}}

\newcommand{\norm}[1]{\ensuremath{\left\Vert {#1}\right\Vert}}
\newcommand{\Ker}{\mathop{\mathrm{Ker}}\nolimits}

% d droit pour le calcul différentiel
\newcommand{\dd}{\text{d}}

\newcommand{\inertie}[2]{I_{#1}\left( #2\right)}
\newcommand{\matinertie}[7]{
\begin{pmatrix}
#1 & #6 & #5 \\
#6 & #2 & #4 \\
#5 & #4 & #3 \\
\end{pmatrix}_{#7}}
%%%%%%%%%%%%
% Définition des torseurs 
%%%%%%%%%%%%

\newcommand{\ec}[2]{%
\mathcal{E}_c\left(#1/#2\right)}

\newcommand{\pext}[3]{%
\mathcal{P}\left(#1\rightarrow#2/#3\right)}

\newcommand{\pint}[3]{%
\mathcal{P}\left(#1 \stackrel{\text{#3}}{\leftrightarrow} #2\right)}


 \newcommand{\torseur}[1]{%
\left\{{#1}\right\}
}

\newcommand{\torseurcin}[3]{%
\left\{\mathcal{#1} \left(#2/#3 \right) \right\}
}

\newcommand{\torseurci}[2]{%
\left\{\sigma \left(#1/#2 \right) \right\}
}
\newcommand{\torseurdyn}[2]{%
\left\{\mathcal{D} \left(#1/#2 \right) \right\}
}


\newcommand{\torseurstat}[3]{%
\left\{\mathcal{#1} \left(#2\rightarrow #3 \right) \right\}
}


 \newcommand{\torseurc}[8]{%
%\left\{#1 \right\}=
\left\{
{#1}
\right\}
 = 
\left\{%
\begin{array}{cc}%
{#2} & {#5}\\%
{#3} & {#6}\\%
{#4} & {#7}\\%
\end{array}%
\right\}_{#8}%
}

 \newcommand{\torseurcol}[7]{
\left\{%
\begin{array}{cc}%
{#1} & {#4}\\%
{#2} & {#5}\\%
{#3} & {#6}\\%
\end{array}%
\right\}_{#7}%
}

 \newcommand{\torseurl}[3]{%
%\left\{\mathcal{#1}\right\}_{#2}=%
\left\{%
\begin{array}{l}%
{#1} \\%
{#2} %
\end{array}%
\right\}_{#3}%
}

% Vecteur vitesse
 \newcommand{\vectv}[3]{%
\vect{V\left( {#1} \in {#2}/{#3}\right)}
}

% Vecteur force
\newcommand{\vectf}[2]{%
\vect{R\left( {#1} \rightarrow {#2}\right)}
}

% Vecteur moment stat
\newcommand{\vectm}[3]{%
\vect{\mathcal{M}\left( {#1}, {#2} \rightarrow {#3}\right)}
}




% Vecteur résultante cin
\newcommand{\vectrc}[2]{%
\vect{R_c \left( {#1}/ {#2}\right)}
}
% Vecteur moment cin
\newcommand{\vectmc}[3]{%
\vect{\sigma \left( {#1}, {#2} /{#3}\right)}
}


% Vecteur résultante dyn
\newcommand{\vectrd}[2]{%
\vect{R_d \left( {#1}/ {#2}\right)}
}
% Vecteur moment dyn
\newcommand{\vectmd}[3]{%
\vect{\delta \left( {#1}, {#2} /{#3}\right)}
}

% Vecteur accélération
 \newcommand{\vectg}[3]{%
\vect{\Gamma \left( {#1} \in {#2}/{#3}\right)}
}

% Vecteur omega
 \newcommand{\vecto}[2]{%
\vect{\Omega\left( {#1}/{#2}\right)}
}
% }$$\left\{\mathcal{#1} \right\}_{#2} =%
% \left\{%
% \begin{array}{c}%
%  #3 \\%
%  #4 %
% \end{array}%
% \right\}_{#5}}

\newcommand{\N}{\mathbb{N}}
\newcommand{\Z}{\mathbb{Z}}
\newcommand{\R}{\mathbb{R}}
\newcommand{\C}{\mathbb{C}}
\newcommand{\K}{\mathbb{K}}

\newcommand{\cA}{\mathscr{A}}
\newcommand{\cM}{\mathscr{M}}
\newcommand{\cL}{\mathscr{L}}
\newcommand{\cS}{\mathscr{S}}

\newcommand{\python}{\texttt{Python}}

\newcommand{\z}[1]{\Z_{#1}}
\newcommand{\ztimes}[1]{\Z_{#1}^{\times}}
\newcommand{\ii}[1]{[\![#1[\![}
\newcommand{\iif}[1]{[\![#1]\!]}
\newcommand{\llbr}{\ensuremath{\llbracket}}
\newcommand{\rrbr}{\ensuremath{\rrbracket}}
%\newcommand{\p}[1]{\left(#1\right)}
\newcommand{\ens}[1]{\left\{ #1 \right\}}
\newcommand{\croch}[1]{\left[ #1 \right]}
%\newcommand{\of}[1]{\lstinline{#1}}
% \newcommand{\py}[2]{%
%   \begin{tabular}{|l}
%     \lstinline+>>>+\textbf{\of{#1}}\\
%     \of{#2}
%   \end{tabular}\par{}
% }
\newcommand{\floor}[1]{\left\lfloor#1\right\rfloor}
\newcommand{\ceil}[1]{\left\lceil#1\right\rceil}
\newcommand{\abs}[1]{\left|#1\right|}


% Binaire, octal, hexa
\newcommand{\hex}[1]{\underline{\text{\texttt{#1}}}_{16}}
\newcommand{\oct}[1]{\underline{\text{\texttt{#1}}}_{8}}
\newcommand{\bin}[1]{\underline{\text{\texttt{#1}}}_{2}}
\DeclareMathOperator{\mmod}{\texttt{\%}}


% Fonctions et systèmes
\newcommand{\fct}[5][t]{%
  \begin{array}[#1]{rcl}
    #2 & \rightarrow & #3\\
    #4 & \mapsto     & #5\\
  \end{array}
}
\newcommand{\fonction}[5]{#1 : \left\{\begin{array}{rcl} #2& \longrightarrow &#3 \\ #4 &\longmapsto & #5\end{array}\right.}
\newenvironment{systeme}{\left\{ \begin{array}{rcl}}{\end{array}\right.}

% Matrices
\newcommand{\mat}[1]{
  \begin{pmatrix}
    #1
  \end{pmatrix}
}
\newcommand{\inv}{\ensuremath{^{-1}}}
\newcommand{\bpm}{\begin{pmatrix}}
\newcommand{\epm}{\end{pmatrix}}


% bases de données
\newcommand{\relat}[1]{\textsc{#1}}
\newcommand{\attr}[1]{\emph{#1}}
\newcommand{\prim}[1]{\uline{#1}}
\newcommand{\foreign}[1]{\#\textsl{#1}}


% Bases de données

\newcommand{\att}{\ensuremath{\mathbf{att}}}
\newcommand{\dom}{\ensuremath{\mathbf{dom}}}
\newcommand{\sort}{\ensuremath{\mathbf{sort}}}
\newcommand{\relname}{\ensuremath{\mathbf{relname}}}
\newcommand{\var}{\ensuremath{\mathbf{var}}}
\newcommand{\FILM}{\ensuremath{\mathtt{FILM}}}
\newcommand{\JOUE}{\ensuremath{\mathtt{JOUE}}}
\newcommand{\PERSONNE}{\ensuremath{\mathtt{PERSONNE}}}
\newcommand{\PERSONNAGE}{\ensuremath{\mathtt{PERSONNAGE}}}

\newcommand{\ttid}{\ensuremath{\mathtt{id}}}
\newcommand{\tttitre}{\ensuremath{\mathtt{titre}}}
\newcommand{\ttdate}{\ensuremath{\mathtt{date}}}
\newcommand{\ttidr}{\ensuremath{\mathtt{idrealisateur}}}
\newcommand{\ttdatenais}{\ensuremath{\mathtt{datenaissance}}}
\newcommand{\ttnom}{\ensuremath{\mathtt{nom}}}
\newcommand{\ttprenom}{\ensuremath{\mathtt{prenom}}}
\newcommand{\ttidacteur}{\ensuremath{\mathtt{idacteur}}}
\newcommand{\ttidfilm}{\ensuremath{\mathtt{idfilm}}}
\newcommand{\ttidpersonnage}{\ensuremath{\mathtt{idpersonnage}}}

\newcommand{\fv}{\mathrm{libre}}
\newcommand{\sem}[1]{[\![ #1 ]\!]}

\input{style/macros_Titres}
\input{style/macros_Frames}


%Si le boolen xp est vrai : compilation pour xabi
%Sinon compilation Damien
\newboolean{xp}
\setboolean{xp}{true}

\newboolean{prof}
\setboolean{prof}{true}

\usepackage[%
    pdftitle={Problèmes stationnaires},
    pdfauthor={Xavier Pessoles},
    colorlinks=true,
    linkcolor=blue,
    citecolor=magenta]{hyperref}


\def\discipline{Informatique}
\def\xxtitre{\ifthenelse{\boolean{xp}}{
CI 4 : Bases de données}{
Chapitre  -- }}

\def\xxsoustitre{\ifthenelse{\boolean{xp}}{
Chapitre 1 -- Introduction aux bases de données}{
Partie  -- }}

\def\xxauteur{\ifthenelse{\boolean{xp}}{
Xavier \textsc{Pessoles}}{
Damien \textsc{Iceta} \\ Xavier \textsc{Pessoles}}}

\def\xxpied{\ifthenelse{\boolean{xp}}{
CI 4 : Bases de données \\
Ch. 1 : Introduction aux bases de données -- Cours}{
\xxtitre}}

\def\xxcathegorie{\ifthenelse{\boolean{xp}}{
2013 -- 2014 \\
Xavier \textsc{Pessoles}}{
Informatique - Cours}}





%---------------------------------------------------------------------------


\begin{document}

\ifthenelse{\boolean{xp}}{\usepackage[%
    pdftitle={Représentation des nombres},
    pdfauthor={Xavier Pessoles},
    colorlinks=true,
    linkcolor=blue,
    citecolor=magenta]{hyperref}

\usepackage{pifont}
%\usepackage{lastpage}

% \makeatletter \let\ps@plain\ps@empty \makeatother
%% DEBUT DU DOCUMENT
%% =================
\sloppy
\hyphenpenalty 10000


\colorlet{shadecolor}{orange!15}

\newtheorem{theorem}{Theorem}


\begin{document}


%\newboolean{prof}
%\setboolean{prof}{true}
% \makeatletter \let\ps@plain\ps@empty \makeatother
%% DEBUT DU DOCUMENT
%% =================




%------------- En tetes et Pieds de Pages ------------


\pagestyle{fancy}
\ifthenelse{\boolean{xp}}{%
\renewcommand{\headrulewidth}{0pt}}{%
\renewcommand{\headrulewidth}{0.2pt}} %pour mettre le trait en haut
%\renewcommand{\headrulewidth}{0.2pt}

\fancyhead{}
\fancyhead[L]{%
\noindent\begin{minipage}[c]{2.6cm}%
\includegraphics[width=2cm]{png/logo_ptsi.png}%
\end{minipage}}


\fancyhead[C]{\rule{12cm}{.5pt}}



\fancyhead[R]{%
\noindent\begin{minipage}[c]{3cm}
\begin{flushright}
\footnotesize{\textit{\textsf{Informatique}}}%
\end{flushright}
\end{minipage}
}



\fancyhead[C]{\rule{12cm}{.5pt}}

\renewcommand{\footrulewidth}{0.2pt}

\fancyfoot[C]{\footnotesize{\bfseries \thepage}}
\fancyfoot[L]{%
\begin{minipage}[c]{.2\linewidth}
\noindent\footnotesize{{\xxauteur}}
\end{minipage}
\ifthenelse{\boolean{xp}}{}{%
\begin{minipage}[c]{.15\linewidth}
\includegraphics[width=2cm]{png/logoCC.png}
\end{minipage}}
}

\ifthenelse{\boolean{prof}}{%
\fancyfoot[R]{\footnotesize{\xxpied}}}

\begin{center}
 \huge\textsc{\xxtitre}
\end{center}

\begin{center}
 \LARGE\textsc{\xxsoustitre}
\end{center}

\vspace{.5cm}
}{\input{style/enteteDI}}

%\begin{minipage}[c]{.2\linewidth}
%\begin{center}
%%\includegraphics[width=.95\textwidth]{images/swing}
%\end{center}
%\end{minipage}\hfill
%\begin{minipage}[c]{.33\linewidth}
%\begin{center}
%%\includegraphics[width=.9\textwidth]{images/situation}
%\end{center}
%\end{minipage}\hfill
%\begin{minipage}[c]{.45\linewidth}
%\begin{center}
%%\includegraphics[width=.95\textwidth]{images/tir_alpha}
%\end{center}
%\end{minipage}
\vspace{.5cm}

\begin{savoir}

\end{savoir}



\setlength{\parskip}{0ex plus 0.2ex minus 0ex}
 \renewcommand{\contentsname}{}
 \renewcommand{\baselinestretch}{1}

\tableofcontents

 \renewcommand{\baselinestretch}{1.2}
\setlength{\parskip}{2ex plus 0.5ex minus 0.2ex}


%http://www.imdb.com/interfaces/#plain


\section{Présentation}
\subsection{Exemples de bases de données}
Si on considère l'ensemble des données présentes sur un disque dur, semblent aisément gérées par un ordinateur. Le temps pour accéder et ouvrir un fichier semble en effet assez court. Mais qu'en est-il lorsqu'il s'agit de trouver un fichier sur le disque ? Qu'en est-il lorsqu'il s'agit de trouver une information dans chacun des fichiers du disque ?

Lorsqu'il s'agit de faire de nombreuses recherche sur un grand nombre de fichiers, un stockage << à plat >> ne permet plus un temps d'accès satisfaisant. Il s'agit alors d'organiser les données sous une autre forme. On parle de base de données. 

Les bases de données sont omniprésentes dans l'industrie en général et il s'agit de présenter comment ces données sont traitées. 

On peut commencer par recenser des bases de données libres. Musicbrainz ou freedb sont par exemples des bases de données qui recensent des informations sur les disques de musiques (Auteurs, compositeurs, interprètes, titres des albums, titres des chansons, dates de sorties ...). IMDb est une base de donnée cinématographique. OpenStreetMap met à disposition des internautes des données cartographiques. 

La FNAC, ou d'autres sites commerciaux disposent d'une base de données de leurs produits. Ainsi, grâce à un champ de recherche, l'internaute peut interroger la base. Il peut avoir des informations sur la disponibilité d'un produit, le délai de livraison ...

Une organisation des données permet aux utilisateurs d'avoir un accès rapide à tout type d'informations. On appelle requête la demande d'un utilisateur formulée auprès d'une base de donnée. 

\subsection{Bases de données à <<plat>>}
Une première solution envisageable pour stocker des données est l'utilisation de bases de données dites plates. Les informations sont par exemple stockées dans un tableau. Prenons par exemple une base de données contenant la liste des aéroports du monde ainsi que diverses informations : 
\begin{center}
\begin{tabular}{p{4cm}llll}
\hline
Nom & Ville & Pays & Continent  & Type \\
\hline
\hline
Biarritz-Anglet-Bayonne Airport & Biarritz/Anglet/Bayonne & France & Europe & medium\_airport \\
Milhaud Heliport & Toulon & France & Europe & heliport \\
Toulon-Hyères Airport & Toulon/Hyères/Le Palyvestre & France & Europe & medium\_airport \\
Lake Hood Seaplane Base & Anchorage & États-Unis & Amérique du Nord & seaplane\_base\\
Ted Stevens Anchorage International Airport & Anchorage & États-Unis & Amérique du Nord & large\_airport\\
Mandalay International Airport & Mandalay & Myanmar & Asie & large\_airport\\
\hline
\end{tabular}
\end{center}

Plusieurs remarques peuvent déjà être formulées : 
\begin{itemize}
\item plusieurs informations sont stockées à plusieurs reprises : ainsi, l'information <<France>> est stockée à multiple reprise. Il en est de même pour le champ Continent;
\item des couples d'informations sont redondants : le couple (Pays, Continent) sera toujours identique pour un pays donné. Ainsi, stocker une seule fois que les États-Unis sont en Amérique du Nord.
\end{itemize}

Dans le cas d'une telle table, des requêtes simples sont aisées. Ainsi, faire la liste de tous les aéroports français ne pose pas de problème. Faire la liste de tous les héliports français est un peu plus difficile. 

Si maintenant une seconde table comprend l'ensemble des fréquences sur lesquelles chacun des aéroports peut communiquer, les tables à plat vont rapidement trouver leur limite. 

Enfin, lorsque les bases de données deviennent importantes (on recense plus de 40 000 installations aéroportuaires), l'accès aux données situées dans un tableau peut devenir considérablement lent. 

\subsection{Système de Gestion de Base de Données -- SGBD}
Pour stocker les données, on utilise des systèmes de gestion de base de données (SGBD). Le marché des SGBD est dominé par les entreprises Oracle, IBM ou Microsoft. Il existe par ailleurs des solutions libres telles que PostgreSQL ou MySQL. 

Une SGBD permettent d'assurer le stockage et l'organisation des informations ainsi que les gestions d'accès par des utilisateurs ayant des droits différents. La quantité de données peut dépasser plusieurs TéraOctets.


\subsection{Structure client -- serveur}


Classiquement, les données ne sont pas stockées sur l'ordinateur de l'utilisateur (appelé client) utilisant la base de données mais sur un serveur, voire même un <<nuage>> (\textit{cloud computing}). 

Pour simplifier, pour faire du (\textit{cloud computing}) les entreprises répartissent les informations sur plusieurs ordinateurs en réseau. Suivant les performances nécessaires ou suivant la quantité de stockage nécessaire, la base de donnée peur donc être répartie sur plusieurs ordinateurs physiques, la limite de répartition pouvant être modifié dynamiquement en fonction du besoin. 

\section{Structure des base de données : le modèle relationnel}
\subsection{Tables}
Dans une première approche, une base de données est constituée de tables. Une table est elle-même constituée de lignes rassemblant les informations (valeurs) que l'on désire stocker. On appellera entité chacune des lignes de cette table. Lorsque les valeurs d'une ligne ont les même propriétés, on les regroupe par colonnes. 

Lors de la conception de la base de données, on définit, pour une table, chacune des colonnes. On peut alors renseigner chacune des lignes. 

\begin{exemple}
Base de données des installations aéroportuaires
\begin{center}
\begin{tabular}{lp{4cm}p{3cm}ll}
\hline
\multicolumn{5}{c}{Aéroports} \\
\hline
Identifiant & Nom & Ville & iso\_country & Type \\
\hline
\hline
4077 &Biarritz-Anglet-Bayonne Airport & Biarritz / Anglet / Bayonne & FR &  medium\_airport \\
43537 &Milhaud Heliport & Toulon & FR  & heliport \\
4241 &Toulon-Hyères Airport & Toulon/Hyères/Le Palyvestre & FR &  medium\_airport \\
21567 &Lake Hood Seaplane Base & Anchorage & US &   seaplane\_base\\
5388 &Ted Stevens Anchorage International Airport & Anchorage & US & large\_airport\\
26727 &Mandalay International Airport & Mandalay & MM  & large\_airport\\
\hline
\end{tabular}
\end{center}

\begin{center}
\begin{tabular}{lll}
\hline
\multicolumn{3}{c}{Pays} \\
\hline
Identifiant & Code & Nom \\
\hline
\hline
302 687 & FR & France \\
302 649 & MM & Myanmar \\
302 755 & US & United States \\
\hline
\end{tabular}
\end{center}
\end{exemple}

\begin{rem}
Dans une table il n'y a pas de notion d'ordre a priori. Les données d'une ligne ne peuvent donc pas être désignées par un numéro de ligne.
\end{rem}

\subsection{Définitions \cite{2}}

\begin{defi}
\textbf{Attributs -- Domaines}

On considère donné un ensemble infini $\mathcal{A}$, dont les éléments sont appelés des attributs, un ensemble $D$, et une application dom de $\mathcal{A}$ dans les sous-ensembles de $D$. 

Si $A\in \mathcal{A}$, $dom(A)$ est appelé domaine de $A$.

\end{defi}

\begin{exemple}
Le type d'aéroport est un attribut. Son domaine est l'ensemble des types d'aéroport. 

La ville est un attribut. Son domaine est l'ensemble des villes du monde muni d'un équipement aéroportuaire. 

\end{exemple}

\begin{defi}
\textbf{Schéma relationnel}

Un schéma relationnel est un $n$-uplet de la forme $S=(A_1,...,A_n) \in \mathcal{A}^n$ où les $A_i$ sont distincts deux à deux. 
\end{defi}

\begin{exemple}
La table des aéroports est un schéma relationnel. 
Sous une forme formelle, on pourrait noter :
$$ \text{Aeroport} = \left(\left( \text{Identifiant}, \mathbb{N}\right),\left(\text{Ville}, \left( ...\right)\right), \left(\text{iso\_country},\left(\text{FR},\text{US},... \right)  \right),... \right)$$
\end{exemple}

\begin{defi}
\textbf{Relation -- table}

On appelle relation associé à un schéma relationnel $\left(A_i,...,A_n \right)$ est un ensemble fini de $n$-uplets de $dom(A_1)\times \cdot\cdot\cdot \times dom(A_n)$.

On note $R(S)$ la relation $R$ pour signifier qu'elle est associée au schéma relationnel $S$. 

Les éléments de $R$ sont appelés les \textit{valeurs}, ou encore les enregistrements, de la relation et leur nombre est appelé son \textit{cardinal} et est note \#R.
\end{defi}

\begin{exemple}
Les tables présentées précédemment sont des relations. 
\end{exemple}

\begin{defi}
\textbf{Valeurs}
Si $e\in R(S)$ et $A\in S$, on note $e.A$ la composante du $n$-uplet $e$ associée à l'attribut $A$. 

$R$ étant un ensemble, deux valeurs distinctes $e$ diffèrent forcément au moins sur un attribut.  Formellement, on a : 
$$
\forall e,e' \in R(S), \text{ si } e\neq e', \text{ alors } \exists A \in S, e.A \neq e'.A
$$

\end{defi}

\subsection{Notion de clef}
Afin de ne pas stocker des des doublons dans une base de donnée, on a recours au concept de clef primaire. 

Dans certains cas, si on est certain que la valeur d'un attribut sera différent pour chaque ligne de la table, cet attribut peut tenir compte de clef primaire. Dans d'autres cas, si on est persuadé que pour une ligne donnée, la combinaison de $n$ attributs est unique, la combinaison de ces attributs peut constituer une clef primaire. 

Enfin, dans certains cas, on définit un attribut de type <<identifiant>>. Lorsqu'on ajoute une ligne dans la table, on s'assure qu'un identifiant unique (un nombre entier par exemple) sera affecté.


\section{Algèbre relationnelle}
La raison d'être d'une base de données étant de disposer de données, il est nécessaire de disposer d'outils d'interrogation. 

\begin{defi}
\textbf{Requêtes -- Algèbre relationnelle}

On entend par algèbre relationnelle, une collection d'opérations (requêtes=) formelles qui agissent sur des relations et produisent une relation en résultat : $R_3 \rightarrow R_2 Op R_1$.
\end{defi}

Ceci signifie que dans l'algèbre relationnelle, le résultat des requêtes effectuées sur les relations (tables) sera toujours une nouvelle relation. 

\subsection{Opérations sur les ensembles}
\begin{defi}
\begin{minipage}[c]{.75\linewidth}
\textbf{Union}

L'union de deux relations $R_1(S)$ et $R_2(S)$ est l'ensemble des valeurs comprises dans $R_1$ ou $R_2$. 

On peut donc noter la relation $R_3(S)$ définie par : $R_3(S)\leftarrow R_1(S)\cup R_2(S)$
\end{minipage}\hfill
\begin{minipage}[c]{.2\linewidth}
\begin{center}
\includegraphics[width=.95\textwidth]{images/union}
\end{center}
\end{minipage}
\end{defi}

\begin{defi}
\begin{minipage}[c]{.75\linewidth}
\textbf{Intersection}

L'intersection de deux relations $R_1(S)$ et $R_2(S)$ est l'ensemble des valeurs comprises dans $R_1$ et dans $R_2$. 

On peut donc noter la relation $R_3(S)$ définie par : $R_3(S)\leftarrow R_1(S)\cap R_2(S)$
\end{minipage}\hfill
\begin{minipage}[c]{.2\linewidth}
\begin{center}
\includegraphics[width=.95\textwidth]{images/inter}
\end{center}
\end{minipage}
\end{defi}


\begin{defi}
\begin{minipage}[c]{.75\linewidth}
\textbf{Différence}

La différence de deux relations $R_1(S)$ et $R_2(S)$ est l'ensemble des valeurs comprises dans $R_1$ et qui ne sont pas comprises dans $R_2$. 

On peut donc noter la relation $R_3(S)$ définie par : $R_3(S)\leftarrow R_1(S)-R_2(S)$
\end{minipage}\hfill
\begin{minipage}[c]{.2\linewidth}
\begin{center}
\includegraphics[width=.95\textwidth]{images/diff}
\end{center}
\end{minipage}
\end{defi}


\subsection{Opérations relationnelles}

\begin{defi}

\textbf{Projection}

Opération notée $\pi$ au cours de laquelle on sélectionne certaines des colonnes (on élimine donc des attributs). 
 $$
 R_2 \leftarrow \pi_{\text{attribut 1, attribut 2, ...}}(R_1)
 $$

\end{defi}


\begin{defi}
\textbf{Sélection}

On appelle sélection de $R_1$ selon $A=a$, et on note $\sigma_{A=a}(R_1)$, la relation obtenue en sélectionnant dans $R_1$ uniquement les valeurs $e$ telles que $e.A = a$.
$$
R_2 \leftarrow \sigma_{\text{attribut}=\text{condition}}(R_1)
$$
\end{defi}



\begin{defi}
\textbf{Renommage d'attribut}

Cette opération est utilisée pour des raisons pratiques pour lever une ambigüité ou pour simplifier le nom d'un attribut de façon temporaire. 

Soit $S = (A_1,... ,A_n)$ un schéma, $ i \in[1;n]$ et $B$ un attribut tel que
$dom(B) = dom(A_i)$. On note :

$$
R_2 \leftarrow \rho_{\text{ancien attribut} \rightarrow \text{nouvel attribut}}
$$
\end{defi}



\begin{defi}

\textbf{Produit cartésien}

Soient $R_1(S_1)$ et $R_2(S_2)$ deux relations de schémas disjoints. L'opération produit cartésien est noté $\times$. 

$$
R_3 \leftarrow R_2 \times R_1
$$

La relation $R_3$ contient toutes les combinaisons d'association possibles entre les valeurs de $R_1$ et de $R_2$.

\end{defi}


\begin{thebibliography}{2}
\bibitem{1}{Serge Abiteboul, Benjamin Nguyen, Yannick Le Bras, \textit{Introduction aux Bases de Données Relationnelles Programme de Classes Préparatoires Scientifiques, Première année}.}
\bibitem{2}{Wack et Al., \textit{L’informatique pour tous en classes préparatoires aux grandes écoles, Editions Eyrolles}.}
\bibitem{3}{Christope Revy, \textit{Concepts des bases de données}, Cours de STS IRIS, Lycée Janot de SENS.}
\bibitem{4}{Patrick Beynet, \textit{Supports de cours de TSI 2}, Supports de cours de TSI 2, Lycée Rouvière, Toulon.}
%\bibitem{3}{Olivier Guindet, Résolution d'un problème dynamique par la méthode d'Euler, UPSTI.}
%\bibitem{4}{Alain Caignot, Marc Derumaux, Résolution des équations différentielles, UPSTI.}
%\bibitem{1}{Adrien Petri, \textit{Analyse numérique : Intégration numérique}, Notes de cours de TSI 1, Lycée Rouvière, Toulon.}
\end{thebibliography}
\end{document}


