\documentclass[10pt]{article}
\input{style/coursHeadings}
\usepackage{algorithm}
\usepackage{algorithmic}


% Python sources
\usepackage{listings}
\usepackage{textcomp}
\usepackage{setspace}
%\usepackage{palatino}

%\usepackage{color}
\definecolor{Bleu}{rgb}{0.1,0.1,1.0}
\definecolor{Noir}{rgb}{0,0,0}
\definecolor{Grau}{rgb}{0.5,0.5,0.5}
\definecolor{DunkelGrau}{rgb}{0.15,0.15,0.15}
\definecolor{Hellbraun}{rgb}{0.5,0.25,0.0}
\definecolor{Magenta}{rgb}{1.0,0.0,1.0}
\definecolor{Gris}{gray}{0.5}
\definecolor{Vert}{rgb}{0,0.5,0}
\definecolor{SourceHintergrund}{rgb}{1,1.0,0.95}

%
\renewcommand{\lstlistlistingname}{Listings}
\renewcommand{\lstlistingname}{Listing}

\lstnewenvironment{python}[1][]{
\lstset{
language=python,
basicstyle=\ttfamily\footnotesize\setstretch{1}, 	
stringstyle=\color{red}, 
showstringspaces=false, 
alsoletter={1234567890},
otherkeywords={\ , \}, \{},
keywordstyle=\color{blue},
emph={access,and,break,class,continue,def,del,elif ,else,
except,exec,finally,for,from,global,if,import,in,i s,
lambda,not,or,pass,print,raise,return,try,while},
emphstyle=\color{black}\bfseries,
emph={[2]True, False, None, self},
emphstyle=[2]\color{green},
emph={[3]from, import, as},
emphstyle=[3]\color{blue},
upquote=true,
morecomment=[s]{"""}{"""},
commentstyle=\color{Hellbraun}\slshape, 
%emph={[4]1, 2, 3, 4, 5, 6, 7, 8, 9, 0},
emphstyle=[4]\color{blue},
literate=*{:}{{\textcolor{blue}:}}{1}
{=}{{\textcolor{blue}=}}{1}
{-}{{\textcolor{blue}-}}{1}
{+}{{\textcolor{blue}+}}{1}
{*}{{\textcolor{blue}*}}{1}
{!}{{\textcolor{blue}!}}{1}
{(}{{\textcolor{blue}(}}{1}
{)}{{\textcolor{blue})}}{1}
{[}{{\textcolor{blue}[}}{1}
{]}{{\textcolor{blue}]}}{1}
{<}{{\textcolor{blue}<}}{1}
{>}{{\textcolor{blue}>}}{1},
%framexleftmargin=1mm, framextopmargin=1mm, frame=shadowbox, rulesepcolor=\color{blue},#1
backgroundcolor=\color{SourceHintergrund}, 
framexleftmargin=1mm, framexrightmargin=1mm, framextopmargin=1mm, frame=single, framerule=1pt, rulecolor=\color{black},#1
}}{}
%%%%%%%%%%%%
% Définition des vecteurs 
%%%%%%%%%%%%
\newcommand{\vect}[1]{\overrightarrow{#1}}
\newcommand{\axe}[2]{\left(#1,\vect{#2}\right)}
\newcommand{\couple}[2]{\left(#1,\vect{#2}\right)}
\newcommand{\angl}[2]{\left(\vect{#1},\vect{#2}\right)}

\newcommand{\rep}[1]{\mathcal{R}_{#1}}
\newcommand{\quadruplet}[4]{\left(#1;#2,#3,#4 \right)}
\newcommand{\repere}[4]{\left(#1;\vect{#2},\vect{#3},\vect{#4} \right)}
\newcommand{\base}[3]{\left(\vect{#1},\vect{#2},\vect{#3} \right)}


\newcommand{\vx}[1]{\vect{x_{#1}}}
\newcommand{\vy}[1]{\vect{y_{#1}}}
\newcommand{\vz}[1]{\vect{z_{#1}}}

\newcommand{\norm}[1]{\ensuremath{\left\Vert {#1}\right\Vert}}
\newcommand{\Ker}{\mathop{\mathrm{Ker}}\nolimits}

% d droit pour le calcul différentiel
\newcommand{\dd}{\text{d}}

\newcommand{\inertie}[2]{I_{#1}\left( #2\right)}
\newcommand{\matinertie}[7]{
\begin{pmatrix}
#1 & #6 & #5 \\
#6 & #2 & #4 \\
#5 & #4 & #3 \\
\end{pmatrix}_{#7}}
%%%%%%%%%%%%
% Définition des torseurs 
%%%%%%%%%%%%

\newcommand{\ec}[2]{%
\mathcal{E}_c\left(#1/#2\right)}

\newcommand{\pext}[3]{%
\mathcal{P}\left(#1\rightarrow#2/#3\right)}

\newcommand{\pint}[3]{%
\mathcal{P}\left(#1 \stackrel{\text{#3}}{\leftrightarrow} #2\right)}


 \newcommand{\torseur}[1]{%
\left\{{#1}\right\}
}

\newcommand{\torseurcin}[3]{%
\left\{\mathcal{#1} \left(#2/#3 \right) \right\}
}

\newcommand{\torseurci}[2]{%
\left\{\sigma \left(#1/#2 \right) \right\}
}
\newcommand{\torseurdyn}[2]{%
\left\{\mathcal{D} \left(#1/#2 \right) \right\}
}


\newcommand{\torseurstat}[3]{%
\left\{\mathcal{#1} \left(#2\rightarrow #3 \right) \right\}
}


 \newcommand{\torseurc}[8]{%
%\left\{#1 \right\}=
\left\{
{#1}
\right\}
 = 
\left\{%
\begin{array}{cc}%
{#2} & {#5}\\%
{#3} & {#6}\\%
{#4} & {#7}\\%
\end{array}%
\right\}_{#8}%
}

 \newcommand{\torseurcol}[7]{
\left\{%
\begin{array}{cc}%
{#1} & {#4}\\%
{#2} & {#5}\\%
{#3} & {#6}\\%
\end{array}%
\right\}_{#7}%
}

 \newcommand{\torseurl}[3]{%
%\left\{\mathcal{#1}\right\}_{#2}=%
\left\{%
\begin{array}{l}%
{#1} \\%
{#2} %
\end{array}%
\right\}_{#3}%
}

% Vecteur vitesse
 \newcommand{\vectv}[3]{%
\vect{V\left( {#1} \in {#2}/{#3}\right)}
}

% Vecteur force
\newcommand{\vectf}[2]{%
\vect{R\left( {#1} \rightarrow {#2}\right)}
}

% Vecteur moment stat
\newcommand{\vectm}[3]{%
\vect{\mathcal{M}\left( {#1}, {#2} \rightarrow {#3}\right)}
}




% Vecteur résultante cin
\newcommand{\vectrc}[2]{%
\vect{R_c \left( {#1}/ {#2}\right)}
}
% Vecteur moment cin
\newcommand{\vectmc}[3]{%
\vect{\sigma \left( {#1}, {#2} /{#3}\right)}
}


% Vecteur résultante dyn
\newcommand{\vectrd}[2]{%
\vect{R_d \left( {#1}/ {#2}\right)}
}
% Vecteur moment dyn
\newcommand{\vectmd}[3]{%
\vect{\delta \left( {#1}, {#2} /{#3}\right)}
}

% Vecteur accélération
 \newcommand{\vectg}[3]{%
\vect{\Gamma \left( {#1} \in {#2}/{#3}\right)}
}

% Vecteur omega
 \newcommand{\vecto}[2]{%
\vect{\Omega\left( {#1}/{#2}\right)}
}
% }$$\left\{\mathcal{#1} \right\}_{#2} =%
% \left\{%
% \begin{array}{c}%
%  #3 \\%
%  #4 %
% \end{array}%
% \right\}_{#5}}

\newcommand{\N}{\mathbb{N}}
\newcommand{\Z}{\mathbb{Z}}
\newcommand{\R}{\mathbb{R}}
\newcommand{\C}{\mathbb{C}}
\newcommand{\K}{\mathbb{K}}

\newcommand{\cA}{\mathscr{A}}
\newcommand{\cM}{\mathscr{M}}
\newcommand{\cL}{\mathscr{L}}
\newcommand{\cS}{\mathscr{S}}

\newcommand{\python}{\texttt{Python}}

\newcommand{\z}[1]{\Z_{#1}}
\newcommand{\ztimes}[1]{\Z_{#1}^{\times}}
\newcommand{\ii}[1]{[\![#1[\![}
\newcommand{\iif}[1]{[\![#1]\!]}
\newcommand{\llbr}{\ensuremath{\llbracket}}
\newcommand{\rrbr}{\ensuremath{\rrbracket}}
%\newcommand{\p}[1]{\left(#1\right)}
\newcommand{\ens}[1]{\left\{ #1 \right\}}
\newcommand{\croch}[1]{\left[ #1 \right]}
%\newcommand{\of}[1]{\lstinline{#1}}
% \newcommand{\py}[2]{%
%   \begin{tabular}{|l}
%     \lstinline+>>>+\textbf{\of{#1}}\\
%     \of{#2}
%   \end{tabular}\par{}
% }
\newcommand{\floor}[1]{\left\lfloor#1\right\rfloor}
\newcommand{\ceil}[1]{\left\lceil#1\right\rceil}
\newcommand{\abs}[1]{\left|#1\right|}


% Binaire, octal, hexa
\newcommand{\hex}[1]{\underline{\text{\texttt{#1}}}_{16}}
\newcommand{\oct}[1]{\underline{\text{\texttt{#1}}}_{8}}
\newcommand{\bin}[1]{\underline{\text{\texttt{#1}}}_{2}}
\DeclareMathOperator{\mmod}{\texttt{\%}}


% Fonctions et systèmes
\newcommand{\fct}[5][t]{%
  \begin{array}[#1]{rcl}
    #2 & \rightarrow & #3\\
    #4 & \mapsto     & #5\\
  \end{array}
}
\newcommand{\fonction}[5]{#1 : \left\{\begin{array}{rcl} #2& \longrightarrow &#3 \\ #4 &\longmapsto & #5\end{array}\right.}
\newenvironment{systeme}{\left\{ \begin{array}{rcl}}{\end{array}\right.}

% Matrices
\newcommand{\mat}[1]{
  \begin{pmatrix}
    #1
  \end{pmatrix}
}
\newcommand{\inv}{\ensuremath{^{-1}}}
\newcommand{\bpm}{\begin{pmatrix}}
\newcommand{\epm}{\end{pmatrix}}


% bases de données
\newcommand{\relat}[1]{\textsc{#1}}
\newcommand{\attr}[1]{\emph{#1}}
\newcommand{\prim}[1]{\uline{#1}}
\newcommand{\foreign}[1]{\#\textsl{#1}}


% Bases de données

\newcommand{\att}{\ensuremath{\mathbf{att}}}
\newcommand{\dom}{\ensuremath{\mathbf{dom}}}
\newcommand{\sort}{\ensuremath{\mathbf{sort}}}
\newcommand{\relname}{\ensuremath{\mathbf{relname}}}
\newcommand{\var}{\ensuremath{\mathbf{var}}}
\newcommand{\FILM}{\ensuremath{\mathtt{FILM}}}
\newcommand{\JOUE}{\ensuremath{\mathtt{JOUE}}}
\newcommand{\PERSONNE}{\ensuremath{\mathtt{PERSONNE}}}
\newcommand{\PERSONNAGE}{\ensuremath{\mathtt{PERSONNAGE}}}

\newcommand{\ttid}{\ensuremath{\mathtt{id}}}
\newcommand{\tttitre}{\ensuremath{\mathtt{titre}}}
\newcommand{\ttdate}{\ensuremath{\mathtt{date}}}
\newcommand{\ttidr}{\ensuremath{\mathtt{idrealisateur}}}
\newcommand{\ttdatenais}{\ensuremath{\mathtt{datenaissance}}}
\newcommand{\ttnom}{\ensuremath{\mathtt{nom}}}
\newcommand{\ttprenom}{\ensuremath{\mathtt{prenom}}}
\newcommand{\ttidacteur}{\ensuremath{\mathtt{idacteur}}}
\newcommand{\ttidfilm}{\ensuremath{\mathtt{idfilm}}}
\newcommand{\ttidpersonnage}{\ensuremath{\mathtt{idpersonnage}}}

\newcommand{\fv}{\mathrm{libre}}
\newcommand{\sem}[1]{[\![ #1 ]\!]}

\input{style/macros_Titres}
\input{style/macros_Frames}


%Si le boolen xp est vrai : compilation pour xabi
%Sinon compilation Damien
\newboolean{xp}
\setboolean{xp}{true}

\newboolean{prof}
\setboolean{prof}{true}

\newif\ifprof
\proftrue
%\proffalse

\usepackage[%
    pdftitle={Pivot de Gauss},
    pdfauthor={Xavier Pessoles},
    colorlinks=true,
    linkcolor=blue,
    citecolor=magenta]{hyperref}


\def\discipline{Informatique}
\def\xxtitre{\ifthenelse{\boolean{xp}}{
CI 3 : Ingénierie Numérique \& Simulation}{
Chapitre  -- }}

\def\xxsoustitre{\ifthenelse{\boolean{xp}}{
Chapitre 4 -- Résolution d'un système linéaire inversible par la méthode de Gauss}{
Partie  -- }}

\def\xxauteur{\ifthenelse{\boolean{xp}}{
Xavier \textsc{Pessoles}}{
Damien \textsc{Iceta} \\ Xavier \textsc{Pessoles}}}

\def\xxpied{\ifthenelse{\boolean{xp}}{
CI 3 : Ingénierie Numérique \& Simulation\\
Ch. 4 : Pivot de Gauss -- Cours}{
\xxtitre}}

\def\xxcathegorie{\ifthenelse{\boolean{xp}}{
2013 -- 2014 \\
Xavier \textsc{Pessoles}}{
Informatique - Cours}}





%---------------------------------------------------------------------------


\begin{document}

\ifthenelse{\boolean{xp}}{\usepackage[%
    pdftitle={Représentation des nombres},
    pdfauthor={Xavier Pessoles},
    colorlinks=true,
    linkcolor=blue,
    citecolor=magenta]{hyperref}

\usepackage{pifont}
%\usepackage{lastpage}

% \makeatletter \let\ps@plain\ps@empty \makeatother
%% DEBUT DU DOCUMENT
%% =================
\sloppy
\hyphenpenalty 10000


\colorlet{shadecolor}{orange!15}

\newtheorem{theorem}{Theorem}


\begin{document}


%\newboolean{prof}
%\setboolean{prof}{true}
% \makeatletter \let\ps@plain\ps@empty \makeatother
%% DEBUT DU DOCUMENT
%% =================




%------------- En tetes et Pieds de Pages ------------


\pagestyle{fancy}
\ifthenelse{\boolean{xp}}{%
\renewcommand{\headrulewidth}{0pt}}{%
\renewcommand{\headrulewidth}{0.2pt}} %pour mettre le trait en haut
%\renewcommand{\headrulewidth}{0.2pt}

\fancyhead{}
\fancyhead[L]{%
\noindent\begin{minipage}[c]{2.6cm}%
\includegraphics[width=2cm]{png/logo_ptsi.png}%
\end{minipage}}


\fancyhead[C]{\rule{12cm}{.5pt}}



\fancyhead[R]{%
\noindent\begin{minipage}[c]{3cm}
\begin{flushright}
\footnotesize{\textit{\textsf{Informatique}}}%
\end{flushright}
\end{minipage}
}



\fancyhead[C]{\rule{12cm}{.5pt}}

\renewcommand{\footrulewidth}{0.2pt}

\fancyfoot[C]{\footnotesize{\bfseries \thepage}}
\fancyfoot[L]{%
\begin{minipage}[c]{.2\linewidth}
\noindent\footnotesize{{\xxauteur}}
\end{minipage}
\ifthenelse{\boolean{xp}}{}{%
\begin{minipage}[c]{.15\linewidth}
\includegraphics[width=2cm]{png/logoCC.png}
\end{minipage}}
}

\ifthenelse{\boolean{prof}}{%
\fancyfoot[R]{\footnotesize{\xxpied}}}

\begin{center}
 \huge\textsc{\xxtitre}
\end{center}

\begin{center}
 \LARGE\textsc{\xxsoustitre}
\end{center}

\vspace{.5cm}
}{\input{style/enteteDI}}

\begin{minipage}[c]{.2\linewidth}
\begin{center}
%\includegraphics[width=.95\textwidth]{images/swing}
\end{center}
\end{minipage}\hfill
\begin{minipage}[c]{.33\linewidth}
\begin{center}
%\includegraphics[width=.9\textwidth]{images/situation}
\end{center}
\end{minipage}\hfill
\begin{minipage}[c]{.45\linewidth}
\begin{center}
%\includegraphics[width=.95\textwidth]{images/tir_alpha}
\end{center}
\end{minipage}
\vspace{.5cm}

\begin{savoir}
Problème discret multidimensionnel, linéaire, conduisant à la résolution d’un système linéaire inversible (ou de Cramer) par la méthode de Gauss avec recherche partielle du pivot.

% l’exécuter pour étudier sa mise en oeuvre et les problèmes que pose cette démarche. On souligne la complexité de l’algorithme en fonction de la taille desmatrices et son impact sur le temps de calcul
\end{savoir}


Le pivot de Gauss est une méthode permettant de résoudre les systèmes linéaires. Sur l'idée du pivot de Gauss, il est alors possible de calculer l'inverse d'une matrice (lorsqu'elle est inversible), de calculer le déterminant d'une matrice $A\in\mathcal{M}_n(\mathbb{R})$ ou de calculer le rang de 
$A\in\mathcal{M}_{n,p}(\mathbb{R})$.

\begin{prob}
Outre la compréhension et la mise en \oe{}uvre de l'algorithme, deux problèmes numérique peuvent se poser : 
\begin{itemize}
\item les erreurs d'arrondis peuvent provoquer des erreurs importantes selon la méthode choisie;
\item des comparaisons d'un réel à zéro peuvent aussi engendrer des erreurs numériques.
\end{itemize}
\end{prob}

\setlength{\parskip}{0ex plus 0.2ex minus 0ex}
 \renewcommand{\contentsname}{}
 \renewcommand{\baselinestretch}{1}

\tableofcontents

 \renewcommand{\baselinestretch}{1.2}
\setlength{\parskip}{2ex plus 0.5ex minus 0.2ex}





\section{Système linéaire}
\begin{rem}
Les résultats du cours sont écrits avec des coefficients réels mais restent vrais avec des coefficients complexes. 
\end{rem}

\subsection{Définitions}
\begin{defi}
On dit d'un système qu'il est linéaire s'il est de la forme :
$$
\left\{
\begin{array}{l}
a_{11}x_1 + a_{12}x_2 + ... + a_{1p}x_p = b_1 \\
... \\ 
a_{n1}x_1 + a_{n2}x_2 + ... + a_{np}x_p = b_n \\
\end{array}
\right.
$$

On note $n,p\in \mathbb{N}^*$ ($n$ équations et $p$ inconnues), $\forall i\in \{1,...,n\}$, $\forall j\in \{1,...,p\}$, $a_{ij} \in \mathbb{R}$. $i$ désigne l'indice de la ligne et $j$ l'indice de la colonne. 

$b_1, ..., b_n \in \mathbb{R}$ est appelé second membre. 
\end{defi}

\begin{defi}
On dit que le système est homogène su $b_1=b_2=...=b_n=0$.
\end{defi}


\begin{exemple}
Exemple de système linéaire $(S)$ et son système homogène associé $(S_0)$ :
$$
(S) \left\{
\begin{array}{l}
x+y-2z + 4t = 5 \\
2x+2y-3z+t=3 \\
3x+3y-4z-2t=1
\end{array}
\right.
\quad
(S_0) \left\{
\begin{array}{l}
x+y-2z + 4t = 0 \\
2x+2y-3z+t=0 \\
3x+3y-4z-2t=0
\end{array}
\right.
$$

Exemple de système non linéaire :
$$
\left\{
\begin{array}{l}
x+\mathbf{\cos y} + \mathbf{xy} =2 \\
x-\mathbf{y^2} = 4
\end{array}
\right.$$



\end{exemple}

\begin{defi}
Une solution d'un système linéaire est un p-uplet de réels, c'est-à-dire un élément de $(x_1,...,x_p)$ qui vérifie les $n$ équations. 
\end{defi}


\begin{defi}
Si un système a au moins une solution, il est dit compatible (incompatible sinon).
\end{defi}


\subsection{Opérations élémentaires}
\begin{defi}
Les opérations élémentaires sont les suivantes :
\begin{itemize}
\item opération de transvection (\textit{Gaussian Elimination}): 
\begin{itemize}
\item $(L_i)\leftarrow (L_i)+\lambda (L_j)$ où $i,j\in \{1,...,n\}$, $i\neq j$, $\lambda \in \mathbb{R}$;
\item $(L_i)\leftarrow \alpha (L_i)$ où $i\in \{1,...,n\}$, $\alpha \in \mathbb{R}^*$;
\end{itemize}
\item opération d'échange de lignes : 
\begin{itemize}
\item $(L_i)\leftrightarrow  (L_j)$ où $i,j\in \{1,...,n\}$. 
\end{itemize}
\end{itemize}
\end{defi}

\begin{rem}
Pour éviter les fractions on peut également utiliser $(L_i)\leftarrow \alpha (L_i) + \lambda L_j$ où $i,j\in \{1,...,n\}$, $\alpha \neq 0$, $\lambda \in \mathbb{R}$ qui est une combinaison des deux premiers items.
\end{rem}

\begin{prop}
L'utilisation des opérations élémentaires sur un système ne change pas l'ensemble de ses solutions. Autrement dit, elles donnent des systèmes équivalents au premier.
\end{prop}

\subsection{Notation matricielle}
\begin{rem}
Les opérations élémentaires sur les lignes du système n'opèrent que sur les coefficients. 
\end{rem}

\begin{defi}
Au système défini précédemment, on associe la matrice de ses coefficients :
$$
\begin{pmatrix}
a_{11} & a_{12} & \ldots & a_{1(p-1)} & a_{1p} \\
 & & a_{ij} & & \\
a_{n1} & a_{n2} & \ldots & a_{n(p-1)} & a_{np} \\
\end{pmatrix} \in \mathcal{M}_{n,p}(\mathbb{R})
$$
$\mathcal{M}_{n,p}(\mathbb{R})$ désigne l'ensemble des matrices à coefficients réels à $n$ lignes et $p$ colonnes. 
\end{defi}

\begin{defi}
On note $B=\begin{pmatrix} b_1 \\ \vdots \\ b_n \end{pmatrix}$ la matrice de second membre et $(A|B)$ la matrice $A$ augmentée de $B$. 
\end{defi}


\begin{defi}
Si deux matrices $M$ et $M'$ diffèrent d'un nombre fini d'opération sur les lignes, on dit qu'elles sont équivalentes en lignes et on note $M \underset{L}{\sim} M'$. 
\end{defi}

\begin{rem}
Sous forme matricielle le système s'écrit $AX=B$ avec $X$ le vecteur inconnu. 
\end{rem}


\section{Pivot de Gauss}
\subsection{Pivot d'une ligne}
\begin{defi}
On appelle pivot d'une ligne le premier nombre non nul de cette ligne. 
\end{defi}

\subsection{Algorithme du pivot}
Premier cas, chacun des coefficients de la première colonne est nul. En conséquence, on note 
$M = \begin{pmatrix} 0 & | &  \\ \vdots & | & M \\ 0 & | &  \end{pmatrix}$ 

Dans un second cas, la première colonne contient au moins un nombre non nul.

 Quitte à effectuer un changement de ligne, on se ramène au cas où $a_{11}\neq 0$ :
 $$
 \left(
 \begin{array}{c|ccc}
 a_{11} & \alpha & \alpha & \alpha \\
\hline
\alpha  & & & \\
\alpha  & & \alpha & \\
\alpha  & & & \\
\end{array}
 \right) \quad 
 \text{Les } \alpha \text{ représentent des nombres possiblement nuls.}
 $$ 

 On fait apparaître des zéros sous $a_{11}$ avec des opérations de la forme $(L_2)\leftarrow (L_2)-\dfrac{a_{21}}{a_{11}}(L_1)$ 
 $$
 \left(
 \begin{array}{c|ccc}
 a_{11} & \alpha & \alpha & \alpha \\
\hline
0  & & & \\
\vdots  & & M'& \\
0  & & & \\
\end{array}
 \right) 
 $$ 
 
 On effectue le pivot  à nouveau sur la matrice $M'$.
 
 En généralisant, lorsqu'on en est à la ligne $i$, les lignes $i+1$ à $n$ subissent la transvection suivante : 
 $$
 \forall k \in[i+1,n] \quad L_{k}\leftarrow L_k - \dfrac{a_{k,i}}{a_{i,i}}L_i
 $$
 
 \begin{rem}
 Le nombre de colonnes diminue à chaque pivot.
 
 L'algorithme se termine en un nombre fini d'étapes.
 \end{rem}


\begin{exemple}
Soit le système suivant à résoudre ainsi que la matrice augmentée qui lui est associée : 
$$
\left\{
\begin{array}{l}
x   + y  -2 z +4t = 5 \\
2x + 2y -3 z +2t = 3 \\
3x + 3y -4 z -2t = 1 \\
1x + 2y +3 z +3t = -1 \\
\end{array}
\right.
\quad\quad
\begin{pmatrix}
1 & 1 & -2 & 4 & | & 5 \\
2 & 2 & -3 & 2 & | &  3 \\
3 & 3 & -4 &  -2& | & 1 \\
1 & 2 & 3 & 3 & | & -1 \\
\end{pmatrix}
$$

$$
\left\{
\begin{array}{l}
(L_2) \leftarrow (L_2) - 2 (L_1) \\ 
(L_3) \leftarrow (L_3) - 3 (L_1) \\
(L_4) \leftarrow (L_4) -  (L_1) \\
\end{array}
\right.
\quad\Longrightarrow \quad
\begin{pmatrix}
1 & 1 & -2 & 4 & | & 5 \\
0 & 0 & 1 & -7 & | &  -6 \\
0 & 0 & 2 &  -14& | & -14 \\
0 & 1 & 5 & -1 & | & -6 \\
\end{pmatrix}
$$

$$
%\left\{
\begin{array}{l}
(L_2) \leftrightarrow (L_4)  \\ 
\end{array}
%\right.
\quad\Longrightarrow \quad
\begin{pmatrix}
1 & 1 & -2 & 4 & | & 5 \\
0 & 1 & 5 & -1 & | & -6 \\
0 & 0 & 2 &  -14& | & -14 \\
0 & 0 & 1 & -6 & | &  -7 \\
\end{pmatrix}
$$

$$
%\left\{
\begin{array}{l}
(L_4) \leftarrow (L_4) - \dfrac{1}{2} (L_3)\\ 
\end{array}
%\right.
\quad\Longrightarrow \quad
\begin{pmatrix}
1 & 1 & -2 & 4 & | & 5 \\
0 & 1 & -1 & -7 & | & -6 \\
0 & 0 & 2 &  -14& | & -14 \\
0 & 0 & 0 & 1 & | &  0 \\
\end{pmatrix}
$$
\end{exemple}


\begin{rem}
Numériquement, le codage du 0 posant des problèmes informatiques, il est difficile de s'assurer qu'un pivot est non nul. On utilisera donc la méthode du \textbf{pivot partiel}. Le pivot utilisé ne sera donc pas le premier nombre non nul d'une ligne, mais le plus grand élément d'une colonne (en valeur absolue). Si le pivot n'est pas sur la première ligne, on effectuera les échanges de lignes nécessaires. On s'affranchit ainsi de comparaisons à 0. 

\end{rem}
\subsection{Matrice échelonnée}
A la fin du pivot, on obtient une matrice de la \textbf{forme} :
$$
\begin{pmatrix}
\alpha &\beta & \beta & \beta &  | &\beta \\
0  & \alpha & \beta  &  \beta   &  | & \beta \\
0 & 0 &   \alpha& \beta &  | &\beta  \\
0 & 0 & 0 & \alpha &   | &\beta\\
\end{pmatrix}
\quad 
\text{ avec }\alpha \text{ non nul et }\beta \text{ réel quelconque.}
$$

\begin{defi}
On dit d'une matrice qu'elle est échelonnée en ligne si :
\begin{enumerate}
\item une ligne est nulle, les suivantes le sont aussi;
\item $L_1$, ..., $L_r$ désignent les lignes non nulles, $j(L_1)$, ..., $j(L_r)$ désignent la position des pivots dans ces lignes et $j(L_1)<...<j(L_r)$. 
\end{enumerate}
\end{defi}

\begin{exemple}
Les matrices suivantes ne sont pas échelonnées : 
$$
\begin{pmatrix}
1 & 0 & 2 & 3\\
0 & 0 & 1 & 0\\
0 & 1 & 0 & 2\\
0 & 0 & 0 & 0\\
\end{pmatrix}
\quad
\begin{pmatrix}
1 & 0 & 2 & 3\\
0 & 0 & 0 & 0\\
0 & 0 & 0 & 1\\
0 & 0 & 0 & 0\\
\end{pmatrix}
\quad
\begin{pmatrix}
1 & 0 & 2 & 3\\
0 & 0 & 1 & 2\\
0 & 0 & 1 & 5\\
0 & 0 & 0 & 0\\
\end{pmatrix}
$$
\end{exemple}

\begin{prop}
Le pivot de Gauss permet d'obtenir une matrice échelonnée en lignes. 
\end{prop}

\begin{defi}
Le rang d'une matrice échelonnée désigne le nombre de ses pivots (ce qui correspond aussi au nombre de lignes non nulles). 
\end{defi}

\subsection{Matrices échelonnées réduites}
On peut poursuivre le pivot en : 
\begin{enumerate}
\item divisant les lignes non nulles par leur pivot, les pivot deviennent alors 1;
\item faisant apparaître des zéros au dessus des pivots. 
\end{enumerate}

\begin{prop}
Toute matrice est équivalente en ligne à une et une seule matrice échelonnée réduite. 

En conséquence, on peut définir le rang d'une matrice quelconque comme le rang de sa matrice échelonnée réduite en ligne. 

On peut définir (le nombre) le rang d'un système comme le rang de la matrice de ses coefficients $(A)$. 
\end{prop}

\begin{rem}
Deux matrices équivalentes en ligne ont le même rang. 
\end{rem}

\section{Implémentation de l'algorithme}

\subsection{Choix d'un type de données}
On verra ultérieurement que la bibliothèque numpy permet de gérer plus aisément le calcul matriciel. Dans un premier temps, le système à résoudre sera mis sous forme matricielle afin que les valeurs puissent être stockées dans un tableau. 
On cherchera donc à résoudre le système suivant :
$$
A\cdot X = B
$$
La matrice $A\in\mathcal{M}_{n,n}(\mathbb{R})$ sera stockée dans un tableau de $n$ lignes et $n$ colonnes. Le vecteur $B$, second membre de l'équation, sera stocké dans un tableau de $n$ lignes et une colonne. 


\subsection{Fonctions élémentaires}
\subsubsection{Recherche du pivot}
Dans le cadre d'une résolution numérique, on utilise un pivot partiel. 

\begin{py}
\ifprof
\begin{python}
def recherche_pivot(A,i):
    n = len(A) # le nombre de lignes
    j = i # la ligne du maximum provisoire
    for k in range(i+1, n):
        if abs(A[k][i]) > abs(A[j][i]):
            j = k # un nouveau maximum provisoire
    return j
\end{python}
\else
\vspace{5cm}
\fi
\end{py}


On considère qu'on est à la ligne $i$. \`A la ligne $i$, les éléments compris entre les colonnes 0 et $i-1$ sont théoriquement nuls. Initialement, le plus grand élément de la ligne est a priori en position $i$. 

On cherche alors le plus grand élément, en valeur absolu de la ligne $i+1$à la ligne $n$ ($n$ exclus en Python). 

On renvoie enfin l'indice du pivot. 

\subsubsection{Échange de lignes}

L'échange de lignes peut être nécessaire pour réordonner les lignes de la matrice si le pivot de la ligne considéré est plus petit que le pivot d'une des lignes suivantes.  

\begin{py}
\ifprof
\begin{python}
def echange_lignes(A,i,j):
    # Li <-->Lj
    A[i][:],A[j][:]=A[j][:],A[i][:]
\end{python}
\else
\vspace{5cm}
\fi
\end{py}

\begin{rem}
En python, le passage des objets se faisant par référence, il n'est pas nécessaire de retourner une liste. 
\end{rem}

\subsubsection{Transvection de lignes}

\begin{py}
\ifprof
\begin{python}
def transvection_ligne(A, i, j, mu):
    # L_i <- L_i + mu.L_j 
    nc = len(A[0]) # le nombre de colonnes
    for k in range(nc):
        A[i][k] = A[i][k] + mu * A[j][k]
\end{python}
\else
\vspace{5cm}
\fi
\end{py}

\subsubsection{Phase de remontée : résolution du système}
Une fois toutes les transvections réalisées, on est en présence d'une matrice échelonnée $T$. Le système a donc été mis sous la forme $T\cdot X = Y$. On va donc résoudre le système en partant <<du bas>>.

\begin{py}
\ifprof
\begin{python}
def remontee(A,B):
    n=len(A)
    X = [0.] * n
    for i in range(n-1, -1, -1):
        somme=0
        for j in range (i+1,n):
            somme=somme+A[i][j]*X[j]
        X[i]=(B[i][0]-somme)/A[i][i]
        print(X[i])
    return X
\end{python}
\else
\vspace{8cm}
\fi


\end{py}

 

\subsection{Algorithme complet}

\begin{py}
\ifprof
\begin{python}
def resolution(AA, BB):
    """Résolution de AA.X=BB; AA doit etre inversible"""
    A, B = AA.copy(), BB.copy()
    n = len(A)
    assert len(A[0]) == n
    # Mise sous forme triangulaire
    for i in range(n):
        j = recherche_pivot(A, i)
        if j > i:
            echange_lignes(A, i, j)
            echange_lignes(B, i, j)
        for k in range(i+1, n):
            mu = - A[k][i] / A[i][i]
            transvection_ligne(A, k, i, mu)
            transvection_ligne(B, k, i, mu)
    # Phase de remontée
    return remontee(A,B)
\end{python}
\else
\vspace{10cm}
\fi

\end{py}
%\section{Résolution d'un système}
%\subsection{Inconnu principal, inconnu secondaire}
%\begin{defi}
%On dit d'une inconnue qu'elle est principale si dans une matrice échelonnée, sa colonne contient un pivot. Sinon on dit qu'elle est secondaire.
%\end{defi}
%
%\begin{exemple}
%
%\end{exemple}

%\subsection{Substitution inverse}

\subsection{Résolution de systèmes linéaires avec Numpy}
Il est possible d'utiliser le bibliothèque Numpy pour résoudre le système $A\cdot X = B$. 
Pour cela, les matrice $A$ et $B$ peuvent être mises sous la forme d'une liste de liste. 

\begin{py}
\begin{python}
A=[[1,2,3],[4,5,6],[7,8,9]]
B=[[1],[2],[3]]
numpy.linalg.solve
\end{python}
\end{py}

\begin{thebibliography}{2}
\bibitem{1}{Adrien Petri, \textit{Pivot de Gauss -- Systèmes linéaires}, Notes de cours de TSI 1, Lycée Rouvière, Toulon.}
\bibitem{2}{Wack et Al., \textit{Informatique pour tous en classes préparatoires aux grandes écoles}, Éditions Eyrolles.}
\end{thebibliography}
\end{document}


\section*{Exemple de système à résoudre}
$$
\left\{
\begin{array}{l}
3x + 3y + 2z = 1 \\
4x + 5y - z = 2 \\
5x + 2y - 2z = 0 \\
\end{array}
\right.
$$
Il faut d'abord mettre le système sous forme triangulaire. 
Pour cela, on commence à rechercher le premier pivot partiel. Il est ici sur la dernière ligne. On intervertit donc la première et la dernière ligne :
$$
\left\{
\begin{array}{l}
5x + 2y - 2z = 0 \\
4x + 5y - z = 2 \\
3x + 3y + 2z = 1 \\
\end{array}
\right.
$$
On réalise alors les opérations de transvection. Le coefficient est d'abord -A(1,0)/A(0,0) =-4/5
\end{document}
