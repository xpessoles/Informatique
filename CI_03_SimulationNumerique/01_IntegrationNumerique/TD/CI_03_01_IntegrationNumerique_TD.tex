\documentclass[10pt,oneside]{article}
\input{style/coursHeadings}
\usepackage{algorithm}
\usepackage{algorithmic}


% Python sources
\usepackage{listings}
\usepackage{textcomp}
\usepackage{setspace}
%\usepackage{palatino}

%\usepackage{color}
\definecolor{Bleu}{rgb}{0.1,0.1,1.0}
\definecolor{Noir}{rgb}{0,0,0}
\definecolor{Grau}{rgb}{0.5,0.5,0.5}
\definecolor{DunkelGrau}{rgb}{0.15,0.15,0.15}
\definecolor{Hellbraun}{rgb}{0.5,0.25,0.0}
\definecolor{Magenta}{rgb}{1.0,0.0,1.0}
\definecolor{Gris}{gray}{0.5}
\definecolor{Vert}{rgb}{0,0.5,0}
\definecolor{SourceHintergrund}{rgb}{1,1.0,0.95}

%
\renewcommand{\lstlistlistingname}{Listings}
\renewcommand{\lstlistingname}{Listing}

\lstnewenvironment{python}[1][]{
\lstset{
language=python,
basicstyle=\ttfamily\footnotesize\setstretch{1}, 	
stringstyle=\color{red}, 
showstringspaces=false, 
alsoletter={1234567890},
otherkeywords={\ , \}, \{},
keywordstyle=\color{blue},
emph={access,and,break,class,continue,def,del,elif ,else,
except,exec,finally,for,from,global,if,import,in,i s,
lambda,not,or,pass,print,raise,return,try,while},
emphstyle=\color{black}\bfseries,
emph={[2]True, False, None, self},
emphstyle=[2]\color{green},
emph={[3]from, import, as},
emphstyle=[3]\color{blue},
upquote=true,
morecomment=[s]{"""}{"""},
commentstyle=\color{Hellbraun}\slshape, 
%emph={[4]1, 2, 3, 4, 5, 6, 7, 8, 9, 0},
emphstyle=[4]\color{blue},
literate=*{:}{{\textcolor{blue}:}}{1}
{=}{{\textcolor{blue}=}}{1}
{-}{{\textcolor{blue}-}}{1}
{+}{{\textcolor{blue}+}}{1}
{*}{{\textcolor{blue}*}}{1}
{!}{{\textcolor{blue}!}}{1}
{(}{{\textcolor{blue}(}}{1}
{)}{{\textcolor{blue})}}{1}
{[}{{\textcolor{blue}[}}{1}
{]}{{\textcolor{blue}]}}{1}
{<}{{\textcolor{blue}<}}{1}
{>}{{\textcolor{blue}>}}{1},
%framexleftmargin=1mm, framextopmargin=1mm, frame=shadowbox, rulesepcolor=\color{blue},#1
backgroundcolor=\color{SourceHintergrund}, 
framexleftmargin=1mm, framexrightmargin=1mm, framextopmargin=1mm, frame=single, framerule=1pt, rulecolor=\color{black},#1
}}{}

%Si le boolen xp est vrai : compilation pour xabi
%Sinon compilation Damien
\newboolean{xp}
\setboolean{xp}{true}

\newboolean{prof}
\setboolean{prof}{false}

\def\xxtitre{\ifthenelse{\boolean{xp}}{
CI 3 : Ingénierie Numérique \& Simulation
}}

\def\xxsoustitre{\ifthenelse{\boolean{xp}}{
TD -- Étude du gel d'une canalisation d'eau}{
}}


\def\xxauteur{\ifthenelse{\boolean{xp}}{
\noindent \textsl{Cédric Lopez} \\
\textsl{Xavier Pessoles}
}{
}}


\def\xxpied{\ifthenelse{\boolean{xp}}{
CI 3 : Ingénierie Numérique \& Simulation -- TP \\
Intégration numérique -- \ifthenelse{\boolean{prof}}{P}{E}%
}{
}}

\usepackage[%
    pdftitle={Ingénierie Numérique et Simulation},
    pdfauthor={Xavier Pessoles},
    colorlinks=true,
    linkcolor=blue,
    citecolor=magenta]{hyperref}



\usepackage{pifont}
\sloppy
\hyphenpenalty 10000


\begin{document}


\usepackage[%
    pdftitle={\xxtitre},
    pdfauthor={\xxauteur},
    colorlinks=true,
    linkcolor=blue,
    citecolor=magenta]{hyperref}

\usepackage{pifont}


% \makeatletter \let\ps@plain\ps@empty \makeatother
%% DEBUT DU DOCUMENT
%% =================
\sloppy
\hyphenpenalty 10000

\newcommand{\Pointilles}[1][3]{%
\multido{}{#1}{\makebox[\linewidth]{\dotfill}\\[\parskip]
}}


\colorlet{shadecolor}{orange!15}

\newtheorem{theorem}{Theorem}


\begin{document}


\newboolean{prof}
\setboolean{prof}{true}
%------------- En tetes et Pieds de Pages ------------


\pagestyle{fancy}
\renewcommand{\headrulewidth}{0pt}
%\renewcommand{\headrulewidth}{0.2pt} %pour mettre le trait en haut

\fancyhead{}
\fancyhead[L]{%
%\footnotesize{\textit{\textsf{Lycée François Premier}}}%
\noindent\noindent\begin{minipage}[c]{2.6cm}
\includegraphics[width=2cm]{png/logo_ptsi.png}%
\end{minipage}
}

\fancyhead[C]{\rule{12cm}{.5pt}}  %pour mettre le petit trait en haut


\fancyhead[R]{%
\noindent\begin{minipage}[c]{3cm}
\begin{flushright}
\footnotesize{\textit{\textsf{Informatique}}}%
\end{flushright}
\end{minipage}
}

\renewcommand{\footrulewidth}{0.2pt}

\fancyfoot[C]{\footnotesize{\bfseries \thepage}}
\fancyfoot[L]{%
\begin{minipage}[c]{.2\linewidth}
%\noindent\footnotesize{{Damien Iceta}}
\noindent\footnotesize{\textsc{Xavier Pessoles}\\\textsc{Damien Iceta}}
\end{minipage}
%\begin{minipage}[c]{.15\linewidth}
%\includegraphics[width=2cm]{png/logoCC.png}
%\end{minipage}
}

\ifthenelse{\boolean{prof}}{%
\fancyfoot[R]{\footnotesize{\xxpiedd}}}



\begin{center}
 \Large\textsc{\xxtitre}
\end{center}

\begin{center}
 \large\textsc{\xxsoustitre}
\end{center}


%\begin{rem}
%\textbf{Utilisation de Spyder}

%Dans le cadre de ce TP, nous utiliserons l'environnement de programmation Spyder. Pour lancer cette application utiliser le raccourci sur le bureau.

%\end{rem}
%
%\begin{objectif}
%Les objectif de ce TP sont :
%\begin{itemize}
%\item d'acquérir les données provenant d'un fichier texte (au format \textsf{kml});
%\item de réaliser des fonctions permettant d'analyser les données pour avoir accès à différentes statistiques.
%\end{itemize}
%\end{objectif}



\subsection*{Exercice 1 : Mise hors gel des canalisations d'eau (temps : 45 min – difficulté : $\star\star $)}

La température dans le sol terrestre étant initialement constante, égale à 5\textdegree C , on cherche à déterminer
à quelle profondeur minimale il est nécessaire d’enterrer une canalisation d’eau pour qu’une brusque
chute de la température de sa surface à -15\textdegree C n’entraine pas le gel de cette canalisation après 10 jours.



\begin{minipage}[c]{.6\linewidth}
Les hypothèses sont les suivantes :
\begin{itemize}
\item la température en un point quelconque du sol et de sa surface à tout instant $t < 0$ est constante et égale à $T_0=278\;K$ ( $\theta_0=5^oC$ ) ;
\item la température à la surface du sol, confondue avec le plan d’équation $z = 0$, passe brutalement à l’instant $t = 0$ , de $T_0 = 278\;K$ à $T_1 =  258\; K$ ($\theta_1 = -15^o C$) et se maintient à cette valeur pendant
$t_f= $10 jours.
\end{itemize}
\end{minipage}\hfill
\begin{minipage}[c]{.35\linewidth}
\begin{center}
\includegraphics[width=.95\textwidth]{images/canalisation}
\end{center}
\end{minipage}

On peut montrer que la température $T(z, t)$ à la profondeur $z$ et à l’instant $t$ est donnée par la relation suivante :

$$
T(z,t)=T_1 + (T_0-T_1) \text{erf}\left( \dfrac{z}{2\sqrt{Dt}} \right)
$$
où $\text{erf}(x)$ désigne la fonction définie par :
$$
\text{erf}(x) = \dfrac{2}{\pi}\int^x_0 e^{-u^2} \mathrm{d}u
$$

Données numériques : $D=2,8\cdot 10^{-7} \; m^2\cdot s^{-1}$ (diffusivité thermique du sol terrestre).

\setcounter{subparagraph}{0}

\subparagraph{}
\textit{Écrire une fonction python, appelée \textsf{integrale}, permettant de réaliser d'intégrer une fonction sur un intervalle, en utilisant la méthode du point milieu.}

\subparagraph{}
\textit{Écrire une fonction Python, appelée \textsf{erf}, prenant en paramètre un nombre réel positif ou nul $x$ et retournant la valeur de \textsf{erf(x)}.}

\subparagraph{}
\textit{Écrire une fonction Python, appelée \textsf{Temperature}, prenant en paramètre la profondeur $z$ (exprimée en $m$) et le temps $t$ (exprimé en $s$) et retournant la valeur de la température $T(z, t)$.}

\subparagraph{}
\textit{Écrire un programme Python permettant de créer une liste, nommée \textsf{ListeErreur}, contenant les valeurs de la fonction \textsf{erf(x)} pour $x$ variant par pas de 0,05 dans l’intervalle $[0 ; 2]$.}

\subparagraph{}
\textit{En déduire, à 1 cm près, à quelle profondeur minimale $z_{min}$ il est nécessaire d'enterrer une canalisation d'eau pour qu'une brusque chute de la température de la surface du sol de 5\textdegree C à -15 \textdegree C n'entraine pas le gel de cette canalisation au bout de 10 jours.}

\end{document}

