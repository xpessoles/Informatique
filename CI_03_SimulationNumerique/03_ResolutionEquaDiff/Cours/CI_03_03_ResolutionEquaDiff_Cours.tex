\documentclass[10pt]{article}
\input{style/coursHeadings}
\usepackage{algorithm}
\usepackage{algorithmic}


% Python sources
\usepackage{listings}
\usepackage{textcomp}
\usepackage{setspace}
%\usepackage{palatino}

%\usepackage{color}
\definecolor{Bleu}{rgb}{0.1,0.1,1.0}
\definecolor{Noir}{rgb}{0,0,0}
\definecolor{Grau}{rgb}{0.5,0.5,0.5}
\definecolor{DunkelGrau}{rgb}{0.15,0.15,0.15}
\definecolor{Hellbraun}{rgb}{0.5,0.25,0.0}
\definecolor{Magenta}{rgb}{1.0,0.0,1.0}
\definecolor{Gris}{gray}{0.5}
\definecolor{Vert}{rgb}{0,0.5,0}
\definecolor{SourceHintergrund}{rgb}{1,1.0,0.95}

%
\renewcommand{\lstlistlistingname}{Listings}
\renewcommand{\lstlistingname}{Listing}

\lstnewenvironment{python}[1][]{
\lstset{
language=python,
basicstyle=\ttfamily\footnotesize\setstretch{1}, 	
stringstyle=\color{red}, 
showstringspaces=false, 
alsoletter={1234567890},
otherkeywords={\ , \}, \{},
keywordstyle=\color{blue},
emph={access,and,break,class,continue,def,del,elif ,else,
except,exec,finally,for,from,global,if,import,in,i s,
lambda,not,or,pass,print,raise,return,try,while},
emphstyle=\color{black}\bfseries,
emph={[2]True, False, None, self},
emphstyle=[2]\color{green},
emph={[3]from, import, as},
emphstyle=[3]\color{blue},
upquote=true,
morecomment=[s]{"""}{"""},
commentstyle=\color{Hellbraun}\slshape, 
%emph={[4]1, 2, 3, 4, 5, 6, 7, 8, 9, 0},
emphstyle=[4]\color{blue},
literate=*{:}{{\textcolor{blue}:}}{1}
{=}{{\textcolor{blue}=}}{1}
{-}{{\textcolor{blue}-}}{1}
{+}{{\textcolor{blue}+}}{1}
{*}{{\textcolor{blue}*}}{1}
{!}{{\textcolor{blue}!}}{1}
{(}{{\textcolor{blue}(}}{1}
{)}{{\textcolor{blue})}}{1}
{[}{{\textcolor{blue}[}}{1}
{]}{{\textcolor{blue}]}}{1}
{<}{{\textcolor{blue}<}}{1}
{>}{{\textcolor{blue}>}}{1},
%framexleftmargin=1mm, framextopmargin=1mm, frame=shadowbox, rulesepcolor=\color{blue},#1
backgroundcolor=\color{SourceHintergrund}, 
framexleftmargin=1mm, framexrightmargin=1mm, framextopmargin=1mm, frame=single, framerule=1pt, rulecolor=\color{black},#1
}}{}
%%%%%%%%%%%%
% Définition des vecteurs 
%%%%%%%%%%%%
\newcommand{\vect}[1]{\overrightarrow{#1}}
\newcommand{\axe}[2]{\left(#1,\vect{#2}\right)}
\newcommand{\couple}[2]{\left(#1,\vect{#2}\right)}
\newcommand{\angl}[2]{\left(\vect{#1},\vect{#2}\right)}

\newcommand{\rep}[1]{\mathcal{R}_{#1}}
\newcommand{\quadruplet}[4]{\left(#1;#2,#3,#4 \right)}
\newcommand{\repere}[4]{\left(#1;\vect{#2},\vect{#3},\vect{#4} \right)}
\newcommand{\base}[3]{\left(\vect{#1},\vect{#2},\vect{#3} \right)}


\newcommand{\vx}[1]{\vect{x_{#1}}}
\newcommand{\vy}[1]{\vect{y_{#1}}}
\newcommand{\vz}[1]{\vect{z_{#1}}}

\newcommand{\norm}[1]{\ensuremath{\left\Vert {#1}\right\Vert}}
\newcommand{\Ker}{\mathop{\mathrm{Ker}}\nolimits}

% d droit pour le calcul différentiel
\newcommand{\dd}{\text{d}}

\newcommand{\inertie}[2]{I_{#1}\left( #2\right)}
\newcommand{\matinertie}[7]{
\begin{pmatrix}
#1 & #6 & #5 \\
#6 & #2 & #4 \\
#5 & #4 & #3 \\
\end{pmatrix}_{#7}}
%%%%%%%%%%%%
% Définition des torseurs 
%%%%%%%%%%%%

\newcommand{\ec}[2]{%
\mathcal{E}_c\left(#1/#2\right)}

\newcommand{\pext}[3]{%
\mathcal{P}\left(#1\rightarrow#2/#3\right)}

\newcommand{\pint}[3]{%
\mathcal{P}\left(#1 \stackrel{\text{#3}}{\leftrightarrow} #2\right)}


 \newcommand{\torseur}[1]{%
\left\{{#1}\right\}
}

\newcommand{\torseurcin}[3]{%
\left\{\mathcal{#1} \left(#2/#3 \right) \right\}
}

\newcommand{\torseurci}[2]{%
\left\{\sigma \left(#1/#2 \right) \right\}
}
\newcommand{\torseurdyn}[2]{%
\left\{\mathcal{D} \left(#1/#2 \right) \right\}
}


\newcommand{\torseurstat}[3]{%
\left\{\mathcal{#1} \left(#2\rightarrow #3 \right) \right\}
}


 \newcommand{\torseurc}[8]{%
%\left\{#1 \right\}=
\left\{
{#1}
\right\}
 = 
\left\{%
\begin{array}{cc}%
{#2} & {#5}\\%
{#3} & {#6}\\%
{#4} & {#7}\\%
\end{array}%
\right\}_{#8}%
}

 \newcommand{\torseurcol}[7]{
\left\{%
\begin{array}{cc}%
{#1} & {#4}\\%
{#2} & {#5}\\%
{#3} & {#6}\\%
\end{array}%
\right\}_{#7}%
}

 \newcommand{\torseurl}[3]{%
%\left\{\mathcal{#1}\right\}_{#2}=%
\left\{%
\begin{array}{l}%
{#1} \\%
{#2} %
\end{array}%
\right\}_{#3}%
}

% Vecteur vitesse
 \newcommand{\vectv}[3]{%
\vect{V\left( {#1} \in {#2}/{#3}\right)}
}

% Vecteur force
\newcommand{\vectf}[2]{%
\vect{R\left( {#1} \rightarrow {#2}\right)}
}

% Vecteur moment stat
\newcommand{\vectm}[3]{%
\vect{\mathcal{M}\left( {#1}, {#2} \rightarrow {#3}\right)}
}




% Vecteur résultante cin
\newcommand{\vectrc}[2]{%
\vect{R_c \left( {#1}/ {#2}\right)}
}
% Vecteur moment cin
\newcommand{\vectmc}[3]{%
\vect{\sigma \left( {#1}, {#2} /{#3}\right)}
}


% Vecteur résultante dyn
\newcommand{\vectrd}[2]{%
\vect{R_d \left( {#1}/ {#2}\right)}
}
% Vecteur moment dyn
\newcommand{\vectmd}[3]{%
\vect{\delta \left( {#1}, {#2} /{#3}\right)}
}

% Vecteur accélération
 \newcommand{\vectg}[3]{%
\vect{\Gamma \left( {#1} \in {#2}/{#3}\right)}
}

% Vecteur omega
 \newcommand{\vecto}[2]{%
\vect{\Omega\left( {#1}/{#2}\right)}
}
% }$$\left\{\mathcal{#1} \right\}_{#2} =%
% \left\{%
% \begin{array}{c}%
%  #3 \\%
%  #4 %
% \end{array}%
% \right\}_{#5}}

\newcommand{\N}{\mathbb{N}}
\newcommand{\Z}{\mathbb{Z}}
\newcommand{\R}{\mathbb{R}}
\newcommand{\C}{\mathbb{C}}
\newcommand{\K}{\mathbb{K}}

\newcommand{\cA}{\mathscr{A}}
\newcommand{\cM}{\mathscr{M}}
\newcommand{\cL}{\mathscr{L}}
\newcommand{\cS}{\mathscr{S}}

\newcommand{\python}{\texttt{Python}}

\newcommand{\z}[1]{\Z_{#1}}
\newcommand{\ztimes}[1]{\Z_{#1}^{\times}}
\newcommand{\ii}[1]{[\![#1[\![}
\newcommand{\iif}[1]{[\![#1]\!]}
\newcommand{\llbr}{\ensuremath{\llbracket}}
\newcommand{\rrbr}{\ensuremath{\rrbracket}}
%\newcommand{\p}[1]{\left(#1\right)}
\newcommand{\ens}[1]{\left\{ #1 \right\}}
\newcommand{\croch}[1]{\left[ #1 \right]}
%\newcommand{\of}[1]{\lstinline{#1}}
% \newcommand{\py}[2]{%
%   \begin{tabular}{|l}
%     \lstinline+>>>+\textbf{\of{#1}}\\
%     \of{#2}
%   \end{tabular}\par{}
% }
\newcommand{\floor}[1]{\left\lfloor#1\right\rfloor}
\newcommand{\ceil}[1]{\left\lceil#1\right\rceil}
\newcommand{\abs}[1]{\left|#1\right|}


% Binaire, octal, hexa
\newcommand{\hex}[1]{\underline{\text{\texttt{#1}}}_{16}}
\newcommand{\oct}[1]{\underline{\text{\texttt{#1}}}_{8}}
\newcommand{\bin}[1]{\underline{\text{\texttt{#1}}}_{2}}
\DeclareMathOperator{\mmod}{\texttt{\%}}


% Fonctions et systèmes
\newcommand{\fct}[5][t]{%
  \begin{array}[#1]{rcl}
    #2 & \rightarrow & #3\\
    #4 & \mapsto     & #5\\
  \end{array}
}
\newcommand{\fonction}[5]{#1 : \left\{\begin{array}{rcl} #2& \longrightarrow &#3 \\ #4 &\longmapsto & #5\end{array}\right.}
\newenvironment{systeme}{\left\{ \begin{array}{rcl}}{\end{array}\right.}

% Matrices
\newcommand{\mat}[1]{
  \begin{pmatrix}
    #1
  \end{pmatrix}
}
\newcommand{\inv}{\ensuremath{^{-1}}}
\newcommand{\bpm}{\begin{pmatrix}}
\newcommand{\epm}{\end{pmatrix}}


% bases de données
\newcommand{\relat}[1]{\textsc{#1}}
\newcommand{\attr}[1]{\emph{#1}}
\newcommand{\prim}[1]{\uline{#1}}
\newcommand{\foreign}[1]{\#\textsl{#1}}


% Bases de données

\newcommand{\att}{\ensuremath{\mathbf{att}}}
\newcommand{\dom}{\ensuremath{\mathbf{dom}}}
\newcommand{\sort}{\ensuremath{\mathbf{sort}}}
\newcommand{\relname}{\ensuremath{\mathbf{relname}}}
\newcommand{\var}{\ensuremath{\mathbf{var}}}
\newcommand{\FILM}{\ensuremath{\mathtt{FILM}}}
\newcommand{\JOUE}{\ensuremath{\mathtt{JOUE}}}
\newcommand{\PERSONNE}{\ensuremath{\mathtt{PERSONNE}}}
\newcommand{\PERSONNAGE}{\ensuremath{\mathtt{PERSONNAGE}}}

\newcommand{\ttid}{\ensuremath{\mathtt{id}}}
\newcommand{\tttitre}{\ensuremath{\mathtt{titre}}}
\newcommand{\ttdate}{\ensuremath{\mathtt{date}}}
\newcommand{\ttidr}{\ensuremath{\mathtt{idrealisateur}}}
\newcommand{\ttdatenais}{\ensuremath{\mathtt{datenaissance}}}
\newcommand{\ttnom}{\ensuremath{\mathtt{nom}}}
\newcommand{\ttprenom}{\ensuremath{\mathtt{prenom}}}
\newcommand{\ttidacteur}{\ensuremath{\mathtt{idacteur}}}
\newcommand{\ttidfilm}{\ensuremath{\mathtt{idfilm}}}
\newcommand{\ttidpersonnage}{\ensuremath{\mathtt{idpersonnage}}}

\newcommand{\fv}{\mathrm{libre}}
\newcommand{\sem}[1]{[\![ #1 ]\!]}

\input{style/macros_Titres}
\input{style/macros_Frames}


%Si le boolen xp est vrai : compilation pour xabi
%Sinon compilation Damien
\newboolean{xp}
\setboolean{xp}{true}

\newboolean{prof}
\setboolean{prof}{true}

\usepackage[%
    pdftitle={Problèmes stationnaires},
    pdfauthor={Xavier Pessoles},
    colorlinks=true,
    linkcolor=blue,
    citecolor=magenta]{hyperref}


\def\discipline{Informatique}
\def\xxtitre{\ifthenelse{\boolean{xp}}{
CI 3 : Ingénierie Numérique \& Simulation}{
Chapitre  -- }}

\def\xxsoustitre{\ifthenelse{\boolean{xp}}{
Chapitre 3 -- Résolution des équation différentielles}{
Partie  -- }}

\def\xxauteur{\ifthenelse{\boolean{xp}}{
Xavier \textsc{Pessoles}}{
Damien \textsc{Iceta} \\ Xavier \textsc{Pessoles}}}

\def\xxpied{\ifthenelse{\boolean{xp}}{
CI 3 : Ingénierie Numérique \& Simulation\\
Ch. 3 : Résolution des équations différentielles -- Cours}{
\xxtitre}}

\def\xxcathegorie{\ifthenelse{\boolean{xp}}{
2013 -- 2014 \\
Xavier \textsc{Pessoles}}{
Informatique - Cours}}





%---------------------------------------------------------------------------


\begin{document}

\ifthenelse{\boolean{xp}}{\usepackage[%
    pdftitle={Représentation des nombres},
    pdfauthor={Xavier Pessoles},
    colorlinks=true,
    linkcolor=blue,
    citecolor=magenta]{hyperref}

\usepackage{pifont}
%\usepackage{lastpage}

% \makeatletter \let\ps@plain\ps@empty \makeatother
%% DEBUT DU DOCUMENT
%% =================
\sloppy
\hyphenpenalty 10000


\colorlet{shadecolor}{orange!15}

\newtheorem{theorem}{Theorem}


\begin{document}


%\newboolean{prof}
%\setboolean{prof}{true}
% \makeatletter \let\ps@plain\ps@empty \makeatother
%% DEBUT DU DOCUMENT
%% =================




%------------- En tetes et Pieds de Pages ------------


\pagestyle{fancy}
\ifthenelse{\boolean{xp}}{%
\renewcommand{\headrulewidth}{0pt}}{%
\renewcommand{\headrulewidth}{0.2pt}} %pour mettre le trait en haut
%\renewcommand{\headrulewidth}{0.2pt}

\fancyhead{}
\fancyhead[L]{%
\noindent\begin{minipage}[c]{2.6cm}%
\includegraphics[width=2cm]{png/logo_ptsi.png}%
\end{minipage}}


\fancyhead[C]{\rule{12cm}{.5pt}}



\fancyhead[R]{%
\noindent\begin{minipage}[c]{3cm}
\begin{flushright}
\footnotesize{\textit{\textsf{Informatique}}}%
\end{flushright}
\end{minipage}
}



\fancyhead[C]{\rule{12cm}{.5pt}}

\renewcommand{\footrulewidth}{0.2pt}

\fancyfoot[C]{\footnotesize{\bfseries \thepage}}
\fancyfoot[L]{%
\begin{minipage}[c]{.2\linewidth}
\noindent\footnotesize{{\xxauteur}}
\end{minipage}
\ifthenelse{\boolean{xp}}{}{%
\begin{minipage}[c]{.15\linewidth}
\includegraphics[width=2cm]{png/logoCC.png}
\end{minipage}}
}

\ifthenelse{\boolean{prof}}{%
\fancyfoot[R]{\footnotesize{\xxpied}}}

\begin{center}
 \huge\textsc{\xxtitre}
\end{center}

\begin{center}
 \LARGE\textsc{\xxsoustitre}
\end{center}

\vspace{.5cm}
}{\input{style/enteteDI}}

\begin{minipage}[c]{.2\linewidth}
\begin{center}
%\includegraphics[width=.95\textwidth]{images/swing}
\end{center}
\end{minipage}\hfill
\begin{minipage}[c]{.33\linewidth}
\begin{center}
%\includegraphics[width=.9\textwidth]{images/situation}
\end{center}
\end{minipage}\hfill
\begin{minipage}[c]{.45\linewidth}
\begin{center}
%\includegraphics[width=.95\textwidth]{images/tir_alpha}
\end{center}
\end{minipage}
\vspace{.5cm}

\begin{savoir}
Problème dynamique à une dimension,  linéaire ou non, conduisant à la résolution approchée d’une équation différentielle ordinaire par la méthode d’Euler.
\end{savoir}



\setlength{\parskip}{0ex plus 0.2ex minus 0ex}
 \renewcommand{\contentsname}{}
 \renewcommand{\baselinestretch}{1}

\tableofcontents

 \renewcommand{\baselinestretch}{1.2}
\setlength{\parskip}{2ex plus 0.5ex minus 0.2ex}





\section{Présentation}

\subsection{Contexte}

\begin{defi}
\textbf{Équation différentielle ordinaire -- EDO}

Soit $I$ un intervalle compact (\textit{ie.} fermé et borné) non vide. $I\in\mathbb{R}$. 

Soit $f$ une application continue telle que : 
$$
(t,x)\in I \times \mathbb{R}^m \rightarrow f(t,x)\in \mathbb{R}^m 
$$

On définit une équation différentielle ordinaire par :
$$
y'(t)=f\left( t,y(t)\right)
$$

Soit $t\in I \rightarrow y(t) \in \mathbb{R}^m$ une fonction continue. Cette fonction est solution globale de l'équation si elle vérifie l'EDO. 
\end{defi}

\begin{rem}
Dans notre cas, on se limitera aux équations à 1 dimension ($m=1$).
\end{rem}
\begin{exemple}
Équation différentielle d'ordre 1 à coefficients constants. 
$$
y'(t)=ay(t)\quad a\in \mathbb{R}
$$

Équation différentielle régissant le mouvement d'un pendule simple :
$$
\ddot{\theta} + \dfrac{g}{L}\sin\theta=0
$$

Équation différentielle régissant le mouvement d'un moteur à courant continu (lorsque l'inductance et les frottements visqueux sont négligés) :

$$
J \dfrac{d\omega(t)}{dt} + K_e\omega(t)=  K_t u(t)
$$

avec $R$, $K_e$ , $K_t$ constantes électriques du moteur (résistance de l’induit,
constante de force contre électromotrice et constante de couple).
\end{exemple}

Pour certains types d'équations différentielles, il est possible de déterminer une solution analytique. Pour d'autres, le problèmes peut s'avérer plus difficiles. 
\begin{exemple}
Soit l'équation différentielle suivante :
$$
\sum\limits_{i=0}^n a_i y^{(i)}(t) = \sum\limits_{i=0}^m b_i x^{(i)}(t)
$$
où $y^{(i)}$ désigne la ième dérivée de $y$ et les coefficients $a_i$ et $b_i$, sont des nombres réels. Pour cette <<famille>> d''équations différentielles, il existe une solution analytique (que l'on peut par exemple déterminer en passant par le domaine de Laplace). Le problème est alors de déterminer les pôles de la fonction de transfert.
\end{exemple}

La résolution numérique des équations différentielles a pour but d'approximer le solution d'une équation différentielle dont on a pas de solution analytique.

\subsection{Première approche -- Schéma d'Euler}

Supposons que l'on cherche à approximer la solution d'une équation différentielle suivante sur l'intervalle $[0;+\infty[$ :
$$
y'(t)=f(t,y(t))
$$ 

On va alors définir un pas de temps $h>0$ qui va permettre de discrétiser l'intervalle initial. On va alors chercher à résoudre l'équation différentielle à chaque instant $t$ tel que $t=nh$ pour $n=1,2,...$. \'A chaque instant on a donc : 
$$
y'(nh)=f(nh,y(nh))
$$ 

Dans une certaine mesure, $y'$ peut être approximé par :
$$
y'(nh)\simeq \dfrac{y((n+1)h)-y(nh)}{h}
$$

En introduisant la suite numérique $y_n$ définie par récurrence, on peut donc réécrire le problème ainsi :
$$
\dfrac{y_{n+1}-y_n}{h} = f(nh,y_n)
$$

\begin{exemple}
Reprenons l'équation différentielle suivante :
$$
J \dfrac{d\omega(t)}{dt} + K_e\omega(t)=  K_t u(t)
$$

Il est donc possible d'approximer $\omega(t)$ en recherchant $\omega_n$ défini par récurrence de la manière suivante :
$$
J \dfrac{\omega_{n+1}-\omega_n}{h} + K_e\omega_n=  K_t u_n \Longleftrightarrow 
\omega_{n+1} =  \dfrac{h K_t}{J} u_n +  \omega_n\left(1-\dfrac{h K_e}{J}\right)
$$

En fixant les conditions initiales du problème et en définissant $u_n$, on peut donc approximer une solution de l'équation différentielle. 
Par exemple, prenons $R=1$, $K_e=1$, $K_i=1$, $J=1$. On a alors $\omega_{n+1} =  h  u_n +  \omega_n\left(1-h\right)$

Par ailleurs, la solution exacte de l'équation différentielle est de la forme $\omega(t)=1-e^{-t}$.

\end{exemple}

\begin{center}
\includegraphics[width=.6\textwidth]{images/figure_1}
\end{center}

\begin{rem}
\begin{itemize}
\item Dans cet exemple, en diminuant le pas de calcul, il est possible d'avoir une meilleure approximation de la solution. 
\item Dans cet exemple, on pourrait montrer que $\omega_n$ tend vers $\omega(t)$ lorsque $h$ tend vers 0.
\end{itemize}
\end{rem}


\section{Contexte mathématique}
\subsection{Problème de Cauchy}

\begin{prob}
Le problème consiste à trouver les fonctions $y$ de $[0,T]\rightarrow \mathbb{R}^n$ telles que
$$
\left\{
\begin{array}{l}
y'(t)=f(t,y) \\
y(t_0)=y_0 \quad \text{avec } t_0\in [0,T] \text{ et } y_0\in \mathbb{R}^n \text{ donnés}
\end{array}
\right.
 $$ 
\end{prob}

 
On verra que la plupart des systèmes d'équations différentielles peuvent se mettre sous cette la forme d'un système d'équations différentielles du premier ordre. 



\subsection{Existence et unicité de la solution}

\begin{defi}
\textbf{Fonction lipschitzienne}

$f$ est lipschitzienne en $y$ s'il existe un réel $k>0$ tel que $\forall y\in\mathbb{R}^n$, $\forall z\in\mathbb{R}^n$, $\forall t\in[0,T]$, alors 
$$
||f(t,y)-f(t,z)||\leq k||y-z||
$$

\end{defi}

\begin{theo}
\textbf{Théorème de Cauchy -- Lipschitz}

Soit $f$ une fonction de $[0,T] \times \mathbb{R}^n \rightarrow \mathbb{R}^n$ continue et lipschitzienne en $y$. 

Alors, $\forall t_0 \in [0,T]$ et $\forall y_0 \in \mathbb{R}^n$, le problème de Cauchy admet une unique solution définie sur $[0,T]$.

\end{theo}

On considèrera dans ce cours que $f$ est toujours lipschitzienne et que les conditions du théorème de Cauchy -- Lipschitz sont remplies. 

En d'autres termes, on peut dire que les variations de $f$ restent bornées par un réel strictement positif $k$.

\subsection{Résolution numérique}
On pose $y'(t)=\dfrac{dy}{dt}$.

Pour résoudre l'équation sur $[0,T]$ on commence par discrétiser l'intervalle. Dans le cas d'une discrétisation en $n$ pas constants d'amplitude $h$ on a alors $t_0 = 0$, $t_1=h$, ... $t_n=n\cdot h$.

En intégrant l'équation différentielles sur un intervalle $[t_i, t_{i+1}]$, on a : 
$$
\int\limits_{y_i}^{y_{i+1}} dy = \int\limits_{t_i}^{t_{i+1}} f(t,y(t)) dt 
\Longleftrightarrow 
y_{i+1} - y_1 = \int\limits_{t_i}^{t_{i+1}} f(t,y(t)) dt 
$$

Cette relation de récurrence peut être écrite en connaissant un seul état du système. On parle de méthode à \textbf{pas séparé}.

On note $\varphi (y,t_i,h)= \dfrac{1}{h} \int\limits_{t_i}^{t_{i+1}} f(t,y(t)) dt $.

$y_n$ est une approximation de $y(t_n)$. Il faut donc s'assurer que l'approximation converge vers la solution. 

\subsection{Convergence de la méthode à pas séparés}

Pour s'assurer que $y_n$ converge vers $y(t_n)$ lorsque le pas d'intégration tend vers 0, il faut que la méthode numérique réunisse deux conditions :
\begin{itemize}
\item la condition de consistance;
\item la condition de stabilité. 
\end{itemize}

Ces conditions seront ici considérées comme étant réalisées.

\subsection{Notion d'erreur}

La notion d'erreur étant étroitement liée à la notion d'approximation, on définit 2 types d'erreurs lorsqu'on approxime la solution d'une équation différentielle.

\begin{defi}

\textbf{Erreur locale -- } $e^{\text{loc}}=y(t+h)-y(t)-h\varphi(y,t,h)$ : erreur commise en faisant un pas de la méthode partant de la solution exacte.

\textbf{Erreur cumulée -- } $e^{\text{cumu}} = y(t_n)-y_n$ : c'est l'erreur commise par accumulation depuis l'abscisse initiale.

\end{defi}

\begin{thebibliography}{2}
\bibitem{1}{Sylvie Delabriere, Équations différentielles, méthodes de résolution numérique -- Approximation numérique des fonctions, des intégrales, des solutions d'équations. Équations différentielles : approximation numérique des solutions.}
\bibitem{2}{Wack et Al., L’informatique pour tous en classes préparatoires aux grandes écoles, Editions Eyrolles.}
\bibitem{3}{O. Guindet, Résolution d'un problème dynamique par la méthode d'Euler, UPSTI.}
\bibitem{4}{Alain Caignot, Marc Derumaux, Résolution des équations différentielles, UPSTI.}
%\bibitem{1}{Adrien Petri, \textit{Analyse numérique : Intégration numérique}, Notes de cours de TSI 1, Lycée Rouvière, Toulon.}
\end{thebibliography}
\end{document}


