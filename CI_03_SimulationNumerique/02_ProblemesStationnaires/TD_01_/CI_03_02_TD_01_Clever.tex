\documentclass[10pt]{article}
\input{style/coursHeadings}
\usepackage{algorithm}
\usepackage{algorithmic}


% Python sources
\usepackage{listings}
\usepackage{textcomp}
\usepackage{setspace}
%\usepackage{palatino}

%\usepackage{color}
\definecolor{Bleu}{rgb}{0.1,0.1,1.0}
\definecolor{Noir}{rgb}{0,0,0}
\definecolor{Grau}{rgb}{0.5,0.5,0.5}
\definecolor{DunkelGrau}{rgb}{0.15,0.15,0.15}
\definecolor{Hellbraun}{rgb}{0.5,0.25,0.0}
\definecolor{Magenta}{rgb}{1.0,0.0,1.0}
\definecolor{Gris}{gray}{0.5}
\definecolor{Vert}{rgb}{0,0.5,0}
\definecolor{SourceHintergrund}{rgb}{1,1.0,0.95}

%
\renewcommand{\lstlistlistingname}{Listings}
\renewcommand{\lstlistingname}{Listing}

\lstnewenvironment{python}[1][]{
\lstset{
language=python,
basicstyle=\ttfamily\footnotesize\setstretch{1}, 	
stringstyle=\color{red}, 
showstringspaces=false, 
alsoletter={1234567890},
otherkeywords={\ , \}, \{},
keywordstyle=\color{blue},
emph={access,and,break,class,continue,def,del,elif ,else,
except,exec,finally,for,from,global,if,import,in,i s,
lambda,not,or,pass,print,raise,return,try,while},
emphstyle=\color{black}\bfseries,
emph={[2]True, False, None, self},
emphstyle=[2]\color{green},
emph={[3]from, import, as},
emphstyle=[3]\color{blue},
upquote=true,
morecomment=[s]{"""}{"""},
commentstyle=\color{Hellbraun}\slshape, 
%emph={[4]1, 2, 3, 4, 5, 6, 7, 8, 9, 0},
emphstyle=[4]\color{blue},
literate=*{:}{{\textcolor{blue}:}}{1}
{=}{{\textcolor{blue}=}}{1}
{-}{{\textcolor{blue}-}}{1}
{+}{{\textcolor{blue}+}}{1}
{*}{{\textcolor{blue}*}}{1}
{!}{{\textcolor{blue}!}}{1}
{(}{{\textcolor{blue}(}}{1}
{)}{{\textcolor{blue})}}{1}
{[}{{\textcolor{blue}[}}{1}
{]}{{\textcolor{blue}]}}{1}
{<}{{\textcolor{blue}<}}{1}
{>}{{\textcolor{blue}>}}{1},
%framexleftmargin=1mm, framextopmargin=1mm, frame=shadowbox, rulesepcolor=\color{blue},#1
backgroundcolor=\color{SourceHintergrund}, 
framexleftmargin=1mm, framexrightmargin=1mm, framextopmargin=1mm, frame=single, framerule=1pt, rulecolor=\color{black},#1
}}{}
%%%%%%%%%%%%
% Définition des vecteurs 
%%%%%%%%%%%%
\newcommand{\vect}[1]{\overrightarrow{#1}}
\newcommand{\axe}[2]{\left(#1,\vect{#2}\right)}
\newcommand{\couple}[2]{\left(#1,\vect{#2}\right)}
\newcommand{\angl}[2]{\left(\vect{#1},\vect{#2}\right)}

\newcommand{\rep}[1]{\mathcal{R}_{#1}}
\newcommand{\quadruplet}[4]{\left(#1;#2,#3,#4 \right)}
\newcommand{\repere}[4]{\left(#1;\vect{#2},\vect{#3},\vect{#4} \right)}
\newcommand{\base}[3]{\left(\vect{#1},\vect{#2},\vect{#3} \right)}


\newcommand{\vx}[1]{\vect{x_{#1}}}
\newcommand{\vy}[1]{\vect{y_{#1}}}
\newcommand{\vz}[1]{\vect{z_{#1}}}

\newcommand{\norm}[1]{\ensuremath{\left\Vert {#1}\right\Vert}}
\newcommand{\Ker}{\mathop{\mathrm{Ker}}\nolimits}

% d droit pour le calcul différentiel
\newcommand{\dd}{\text{d}}

\newcommand{\inertie}[2]{I_{#1}\left( #2\right)}
\newcommand{\matinertie}[7]{
\begin{pmatrix}
#1 & #6 & #5 \\
#6 & #2 & #4 \\
#5 & #4 & #3 \\
\end{pmatrix}_{#7}}
%%%%%%%%%%%%
% Définition des torseurs 
%%%%%%%%%%%%

\newcommand{\ec}[2]{%
\mathcal{E}_c\left(#1/#2\right)}

\newcommand{\pext}[3]{%
\mathcal{P}\left(#1\rightarrow#2/#3\right)}

\newcommand{\pint}[3]{%
\mathcal{P}\left(#1 \stackrel{\text{#3}}{\leftrightarrow} #2\right)}


 \newcommand{\torseur}[1]{%
\left\{{#1}\right\}
}

\newcommand{\torseurcin}[3]{%
\left\{\mathcal{#1} \left(#2/#3 \right) \right\}
}

\newcommand{\torseurci}[2]{%
\left\{\sigma \left(#1/#2 \right) \right\}
}
\newcommand{\torseurdyn}[2]{%
\left\{\mathcal{D} \left(#1/#2 \right) \right\}
}


\newcommand{\torseurstat}[3]{%
\left\{\mathcal{#1} \left(#2\rightarrow #3 \right) \right\}
}


 \newcommand{\torseurc}[8]{%
%\left\{#1 \right\}=
\left\{
{#1}
\right\}
 = 
\left\{%
\begin{array}{cc}%
{#2} & {#5}\\%
{#3} & {#6}\\%
{#4} & {#7}\\%
\end{array}%
\right\}_{#8}%
}

 \newcommand{\torseurcol}[7]{
\left\{%
\begin{array}{cc}%
{#1} & {#4}\\%
{#2} & {#5}\\%
{#3} & {#6}\\%
\end{array}%
\right\}_{#7}%
}

 \newcommand{\torseurl}[3]{%
%\left\{\mathcal{#1}\right\}_{#2}=%
\left\{%
\begin{array}{l}%
{#1} \\%
{#2} %
\end{array}%
\right\}_{#3}%
}

% Vecteur vitesse
 \newcommand{\vectv}[3]{%
\vect{V\left( {#1} \in {#2}/{#3}\right)}
}

% Vecteur force
\newcommand{\vectf}[2]{%
\vect{R\left( {#1} \rightarrow {#2}\right)}
}

% Vecteur moment stat
\newcommand{\vectm}[3]{%
\vect{\mathcal{M}\left( {#1}, {#2} \rightarrow {#3}\right)}
}




% Vecteur résultante cin
\newcommand{\vectrc}[2]{%
\vect{R_c \left( {#1}/ {#2}\right)}
}
% Vecteur moment cin
\newcommand{\vectmc}[3]{%
\vect{\sigma \left( {#1}, {#2} /{#3}\right)}
}


% Vecteur résultante dyn
\newcommand{\vectrd}[2]{%
\vect{R_d \left( {#1}/ {#2}\right)}
}
% Vecteur moment dyn
\newcommand{\vectmd}[3]{%
\vect{\delta \left( {#1}, {#2} /{#3}\right)}
}

% Vecteur accélération
 \newcommand{\vectg}[3]{%
\vect{\Gamma \left( {#1} \in {#2}/{#3}\right)}
}

% Vecteur omega
 \newcommand{\vecto}[2]{%
\vect{\Omega\left( {#1}/{#2}\right)}
}
% }$$\left\{\mathcal{#1} \right\}_{#2} =%
% \left\{%
% \begin{array}{c}%
%  #3 \\%
%  #4 %
% \end{array}%
% \right\}_{#5}}

\newcommand{\N}{\mathbb{N}}
\newcommand{\Z}{\mathbb{Z}}
\newcommand{\R}{\mathbb{R}}
\newcommand{\C}{\mathbb{C}}
\newcommand{\K}{\mathbb{K}}

\newcommand{\cA}{\mathscr{A}}
\newcommand{\cM}{\mathscr{M}}
\newcommand{\cL}{\mathscr{L}}
\newcommand{\cS}{\mathscr{S}}

\newcommand{\python}{\texttt{Python}}

\newcommand{\z}[1]{\Z_{#1}}
\newcommand{\ztimes}[1]{\Z_{#1}^{\times}}
\newcommand{\ii}[1]{[\![#1[\![}
\newcommand{\iif}[1]{[\![#1]\!]}
\newcommand{\llbr}{\ensuremath{\llbracket}}
\newcommand{\rrbr}{\ensuremath{\rrbracket}}
%\newcommand{\p}[1]{\left(#1\right)}
\newcommand{\ens}[1]{\left\{ #1 \right\}}
\newcommand{\croch}[1]{\left[ #1 \right]}
%\newcommand{\of}[1]{\lstinline{#1}}
% \newcommand{\py}[2]{%
%   \begin{tabular}{|l}
%     \lstinline+>>>+\textbf{\of{#1}}\\
%     \of{#2}
%   \end{tabular}\par{}
% }
\newcommand{\floor}[1]{\left\lfloor#1\right\rfloor}
\newcommand{\ceil}[1]{\left\lceil#1\right\rceil}
\newcommand{\abs}[1]{\left|#1\right|}


% Binaire, octal, hexa
\newcommand{\hex}[1]{\underline{\text{\texttt{#1}}}_{16}}
\newcommand{\oct}[1]{\underline{\text{\texttt{#1}}}_{8}}
\newcommand{\bin}[1]{\underline{\text{\texttt{#1}}}_{2}}
\DeclareMathOperator{\mmod}{\texttt{\%}}


% Fonctions et systèmes
\newcommand{\fct}[5][t]{%
  \begin{array}[#1]{rcl}
    #2 & \rightarrow & #3\\
    #4 & \mapsto     & #5\\
  \end{array}
}
\newcommand{\fonction}[5]{#1 : \left\{\begin{array}{rcl} #2& \longrightarrow &#3 \\ #4 &\longmapsto & #5\end{array}\right.}
\newenvironment{systeme}{\left\{ \begin{array}{rcl}}{\end{array}\right.}

% Matrices
\newcommand{\mat}[1]{
  \begin{pmatrix}
    #1
  \end{pmatrix}
}
\newcommand{\inv}{\ensuremath{^{-1}}}
\newcommand{\bpm}{\begin{pmatrix}}
\newcommand{\epm}{\end{pmatrix}}


% bases de données
\newcommand{\relat}[1]{\textsc{#1}}
\newcommand{\attr}[1]{\emph{#1}}
\newcommand{\prim}[1]{\uline{#1}}
\newcommand{\foreign}[1]{\#\textsl{#1}}


% Bases de données

\newcommand{\att}{\ensuremath{\mathbf{att}}}
\newcommand{\dom}{\ensuremath{\mathbf{dom}}}
\newcommand{\sort}{\ensuremath{\mathbf{sort}}}
\newcommand{\relname}{\ensuremath{\mathbf{relname}}}
\newcommand{\var}{\ensuremath{\mathbf{var}}}
\newcommand{\FILM}{\ensuremath{\mathtt{FILM}}}
\newcommand{\JOUE}{\ensuremath{\mathtt{JOUE}}}
\newcommand{\PERSONNE}{\ensuremath{\mathtt{PERSONNE}}}
\newcommand{\PERSONNAGE}{\ensuremath{\mathtt{PERSONNAGE}}}

\newcommand{\ttid}{\ensuremath{\mathtt{id}}}
\newcommand{\tttitre}{\ensuremath{\mathtt{titre}}}
\newcommand{\ttdate}{\ensuremath{\mathtt{date}}}
\newcommand{\ttidr}{\ensuremath{\mathtt{idrealisateur}}}
\newcommand{\ttdatenais}{\ensuremath{\mathtt{datenaissance}}}
\newcommand{\ttnom}{\ensuremath{\mathtt{nom}}}
\newcommand{\ttprenom}{\ensuremath{\mathtt{prenom}}}
\newcommand{\ttidacteur}{\ensuremath{\mathtt{idacteur}}}
\newcommand{\ttidfilm}{\ensuremath{\mathtt{idfilm}}}
\newcommand{\ttidpersonnage}{\ensuremath{\mathtt{idpersonnage}}}

\newcommand{\fv}{\mathrm{libre}}
\newcommand{\sem}[1]{[\![ #1 ]\!]}

\input{style/macros_Titres}
\input{style/macros_Frames}

%Si le boolen xp est vrai : compilation pour xabi
%Sinon compilation Damien
\newboolean{xp}
\setboolean{xp}{true}

\newboolean{prof}
\setboolean{prof}{false}

\usepackage[%
    pdftitle={TD f(x)=0},
    pdfauthor={Xavier Pessoles},
    colorlinks=true,
    linkcolor=blue,
    citecolor=magenta]{hyperref}



\def\discipline{Informatique}
\def\xxtitre{\ifthenelse{\boolean{xp}}{
CI 3 : Ingénierie Numérique \& Simulation}{
Chapitre  -- }}

\def\xxsoustitre{\ifthenelse{\boolean{xp}}{
Chapitre 2 -- Problèmes stationnaires\\ Résolution numérique de l'équation $f(x)=0$}{
Partie  -- }}

\def\xxauteur{\ifthenelse{\boolean{xp}}{
Xavier \textsc{Pessoles}}{
Damien \textsc{Iceta} \\ Xavier \textsc{Pessoles}}}

\def\xxpied{\ifthenelse{\boolean{xp}}{
CI 3 : Ingénierie Numérique \& Simulation\\
Ch. 2 : Problèmes stationnaires -- TD}{
\xxtitre}}

\def\xxcathegorie{\ifthenelse{\boolean{xp}}{
2013 -- 2014 \\
Xavier \textsc{Pessoles}}{
Informatique - TD}}






%---------------------------------------------------------------------------


\begin{document}

\ifthenelse{\boolean{xp}}{\usepackage[%
    pdftitle={Représentation des nombres},
    pdfauthor={Xavier Pessoles},
    colorlinks=true,
    linkcolor=blue,
    citecolor=magenta]{hyperref}

\usepackage{pifont}
%\usepackage{lastpage}

% \makeatletter \let\ps@plain\ps@empty \makeatother
%% DEBUT DU DOCUMENT
%% =================
\sloppy
\hyphenpenalty 10000


\colorlet{shadecolor}{orange!15}

\newtheorem{theorem}{Theorem}


\begin{document}


%\newboolean{prof}
%\setboolean{prof}{true}
% \makeatletter \let\ps@plain\ps@empty \makeatother
%% DEBUT DU DOCUMENT
%% =================




%------------- En tetes et Pieds de Pages ------------


\pagestyle{fancy}
\ifthenelse{\boolean{xp}}{%
\renewcommand{\headrulewidth}{0pt}}{%
\renewcommand{\headrulewidth}{0.2pt}} %pour mettre le trait en haut
%\renewcommand{\headrulewidth}{0.2pt}

\fancyhead{}
\fancyhead[L]{%
\noindent\begin{minipage}[c]{2.6cm}%
\includegraphics[width=2cm]{png/logo_ptsi.png}%
\end{minipage}}


\fancyhead[C]{\rule{12cm}{.5pt}}



\fancyhead[R]{%
\noindent\begin{minipage}[c]{3cm}
\begin{flushright}
\footnotesize{\textit{\textsf{Informatique}}}%
\end{flushright}
\end{minipage}
}



\fancyhead[C]{\rule{12cm}{.5pt}}

\renewcommand{\footrulewidth}{0.2pt}

\fancyfoot[C]{\footnotesize{\bfseries \thepage}}
\fancyfoot[L]{%
\begin{minipage}[c]{.2\linewidth}
\noindent\footnotesize{{\xxauteur}}
\end{minipage}
\ifthenelse{\boolean{xp}}{}{%
\begin{minipage}[c]{.15\linewidth}
\includegraphics[width=2cm]{png/logoCC.png}
\end{minipage}}
}

\ifthenelse{\boolean{prof}}{%
\fancyfoot[R]{\footnotesize{\xxpied}}}

\begin{center}
 \huge\textsc{\xxtitre}
\end{center}

\begin{center}
 \LARGE\textsc{\xxsoustitre}
\end{center}

\vspace{.5cm}
}{\input{style/enteteDI}}




\begin{flushright}
 \textit{D'après ressources de A. Caignot.}
\end{flushright}

 \renewcommand{\baselinestretch}{1}
\setlength{\parskip}{0ex plus 0.5ex minus 0.2ex}



\section{Loi entrée--sortie du Clever}

On s'intéresse au mouvement de la cabine du véhicule à trois roues Clever dont la cabine s'incline à l'image d'une moto pour prendre un virage.

\begin{figure}[htbp]\centering
	\includegraphics[width=7cm]{images/clever4}\hfill
	\includegraphics[width=10cm]{images/figure5}
	\caption{cinématique adopté pour l'étude analytique\label{schema_cine_param}}
\end{figure}

Pour piloter le mécanisme, il est nécessaire de connaître l'angle de la cabine en fonction de l'élongation des vérins. L'étude géométrique permet d'obtenir facilement l'élongation en fonction de l'angle :

$$\lambda_1(\alpha) = \sqrt{\left(L\cos(\alpha-130\degres)  +a\right)^2+\left(\ln(\alpha-{130}{\degres})-b\right)^2} $$ 

avec $\alpha\in[{-50}{\degres}, {50}{\degres}]$, $a = {0,14}{\; m}$, $b = {0,046}{\; m}$ et $L = {0,49}{\; m}$.

L'objectif du TP est de déterminer l'angle $\alpha$ pour une valeur d'élongation $\lambda_1$ donnée.

\subsection{Approche graphique}

\subparagraph{}\textit{Définir la fonction $\texttt{lambda1(alpha)}$.}

\subparagraph{}\textit{Tracer l'angle $\alpha$ en fonction de $\lambda_1$ sur le domaine d'étude considéré.}

\subparagraph{}\textit{Déterminer graphiquement, à l'aide de zoom sur la figure, l'angle $\alpha$ pour un allongement de $\lambda_1={0,4}{\; m}$.}

\subsection{Utilisation de bibliothèques Numpy/Scipy}
Un certain nombre d'algorithmes existent pour résoudre une équation, la plupart d'entre eux sont déjà implantés dans le module Scipy.

Nous allons utiliser la méthode de Newton pour déterminer une solution de référence. Cette méthode permet de résoudre une équation mise sous la forme $f(x) = 0$.

\subparagraph{}\textit{Définir la fonction \texttt{f(alpha)}.}

Pour utiliser la fonction de résolution avec l'algorithme de Newton, il faut taper les commandes :
\begin{py}
\begin{python}
import scipy.optimize as opt
opt.newton(f,5)  #la valeur initiale est 5
\end{python}
\end{py}

\subparagraph{}
\textit{Réaliser un schéma permettant d'expliquer la méthode de résolution utilisant l'algorithme de Newton.}

\subparagraph{}
\textit{Expliquer alors le but d'avoir une valeur initiale dans la fonction proposée par Scipy.}

\subparagraph{}
\textit{Déterminer numériquement l'angle $\alpha$ pour un allongement de $\lambda_1={0,4}\; m$. Comparer la valeur obtenue en fonction de la valeur initiale.}


\subsection{\'Etude de différents algorithmes de résolution}
Vous allez dans cette partie implanter différents algorithmes de résolution.

Ces algorithmes sont des algorithmes itératifs qu'il convient d'arrêter lorsque l'on atteint un critère.
Le critère d'arrêt retenu sera $|x_{n}-x_{n-1}| < 10^{-10}$ sauf indication contraire.

Pour chacun des algorithmes, on observera l'ordre de convergence $\textrm{ordre}(i) = \dfrac{\log(|x_{i}-x_{i-1}|)}{\log(|x_{i-1}-x_{i-2}|)}$. Plus l'ordre est élevé, plus l'algorithme converge rapidement vers la solution.

\subsubsection{Méthode par dichotomie}

\subparagraph{}
\textit{Décrire sous la forme d'un algorithme en pseudo code ou d'un algorigramme la méthode par dichotomie.}

\subparagraph{}
\textit{Sous quelles conditions est-il possible d'utiliser cet algorithme.}

\subparagraph{}\textit{Implanter cet algorithme dans une fonction \texttt{dichotomie(f, a, b, epsilon)} et vérifier que le résultat renvoyé correspond à celui attendu.}

%On estime que ce type d'algorithme itératif ne converge pas de manière satisfaisante si la convergence n'est pas atteinte au bout de 30 itérations.

\subparagraph{}\textit{Modifier votre algorithme pour afficher le nombre d'itérations ainsi que la solution quand la convergence est atteinte.}

\subparagraph{}\textit{Modifier votre algorithme pour que celui-ci renvoie la liste contenant les solutions successives obtenues à chaque itération. On supposera que la solution à chaque itération est $\dfrac{a+b}{2}$.}

\subparagraph{}\textit{Ecrire une fonction \texttt{ordre(liste\_x)} qui renvoie la liste contenant l'ordre de convergence en fonction des itérations. En déduire l'ordre de convergence de la méthode par dichotomie.}

\subsubsection{Méthode de Newton}


%Le principe de la méthode de Newton est de chercher le zéro d'une fonction en prenant comme nouvelle approximation l'abscisse du point
% d'intersection de la tangente à la fonction au point d'approximation précédente.

%\begin{center}
%\begin{tikzpicture}[yscale=1/6,xscale=1.5]
%\draw[thick,domain=0:5.2,prefix=./,] plot[id=temp] function {x*x-5};
%\draw[->,>=latex] (0,0) -- (6,0) node[right] {$x$};
%\draw[->,>=latex] (0,-5) -- (0,25) node[right] {$f(x)$};
%\draw[thin] (5,0) node[below] {$x_k$} -- (5,20) -- (3,0) node[below] {$x_{k+1}$}  -- (3,4) -- (2.333333333,0) node[below] {$x_{k+2}$} -- (2.333333333,0.4444444428888898) -- (2.238095238095238,0);
%\end{tikzpicture}
%\includegraphics[width=.45\textwidth]{images/fig3}
%\end{center}


\subparagraph{}
\textit{En utilisant un schéma, expliquer le déroulement de la méthode de Newton.}

\ifthenelse{\boolean{prof}}{
\begin{corrige}
\begin{center}
%\begin{tikzpicture}[yscale=1/6,xscale=1.5]
%\draw[thick,domain=0:5.2,prefix=./,] plot[id=temp] function {x*x-5};
%\draw[->,>=latex] (0,0) -- (6,0) node[right] {$x$};
%\draw[->,>=latex] (0,-5) -- (0,25) node[right] {$f(x)$};
%\draw[thin] (5,0) node[below] {$x_k$} -- (5,20) -- (3,0) node[below] {$x_{k+1}$}  -- (3,4) -- (2.333333333,0) node[below] {$x_{k+2}$} - (2.333333333,0.4444444428888898) -- (2.238095238095238,0);
%\end{tikzpicture}
\includegraphics[width=.45\textwidth]{images/fig3}
\end{center}

L'équation générale de la tangente à la fonction $f$ au point d'abscisse $x_k$, coupant l'axe des abscisses au point d'abscisse $x_{k+1}$ est : $y(x)=f'(x_k) (x-x_{k+1})$.

En particulier, en $x_k$, on a : $y(x_k)=f'(x_k)(x_k-x_{k+1})=f(x_k)$ (car la tangente et la courbe sont concourantes au point $(x_k,f(x_k))$).

On en déduit que \fbox{$x_{k+1} = x_k - \dfrac{f(x_k)}{f'(x_k)}$}.

Cette méthode a un ordre de convergence théorique de 2 (voir le cours) quand on est capable de déterminer précisément la dérivée. 

Quand cette dérivée n'est pas connue, on peut déterminer une approximation. Ici, on l'approchera par : $f'(x_k)=\dfrac{f(x_k+h)-f(x_k)}{h}$.

\end{corrige}
}{}

On donne la dérivée de $\lambda_1(x)$ :
$$\lambda_1'(x)=\dfrac{- a \ln(\alpha-{130}{\degres})-b L\cos(\alpha-{130}{\degres})}{\sqrt{\left(L\cos(\alpha-{130}{\degres})  +a\right)^2+\left(\ln(\alpha-{130}{\degres})-b\right)^2}}$$

\subparagraph{}\textit{Écrire une fonction \texttt{newton(f, xini)} qui affiche la solution, le nombre d'itérations et qui renvoie la liste des approximations $x_k$ successives en prenant la dérivée exacte.}

\subparagraph{}\textit{Dans le cas où la dérivée de $\lambda_1$ n'est pas connue, donner une méthode permettant de la calculer. }


\subparagraph{}\textit{Écrire une fonction \texttt{newton2(f, xini)} qui affiche la solution, le nombre d'itérations et qui renvoie la liste des approximations $x_k$ successives en prenant la dérivée approchée pour $h={0,1}$.}

\subparagraph{}\textit{Comparer la convergence de cet algorithme pour différentes valeurs de $h$ par pas de 10 $(10^{-2}, 10^{-3}, 10^{-4}...)$ et déterminer une valeur qui semble optimale.}

\subparagraph{}\textit{Déterminer l'évolution de l'ordre de convergence en fonction des itérations et en déduire l'ordre de convergence pour différentes valeurs de $h$. Comparer à la valeur théorique.}

L'algorithme est initialisé avec la valeur $-20$. le résultat est aberrant.

\subparagraph{}\textit{Tester votre algorithme avec cette valeur et expliquer ce qu'il se passe.}


\subsubsection{Méthode de la sécante}
Cette méthode est une variante de la méthode de Newton dans laquelle la dérivée est approximée par la pente de la droite passant par les deux points d'abscisses $x_{k-1}$ et $x_{k-2}$ calculés précédemment.

\fbox{$x_{k+2} = x_{k+1} - \dfrac{f(x_{k+1})}{f(x_{k+1})-f(x_{k})}(x_{k+1}-x_k)$}

\subparagraph{}\textit{Ecrire une fonction \texttt{secante(f, xini1, xini2)} qui affiche la solution, le nombre d'itérations et qui renvoie la liste des approximations $x_k$ successives.}

\subparagraph{}\textit{Déterminer l'évolution de l'ordre de convergence en fonction des itérations et en déduire l'ordre de convergence. Comparer à la valeur théorique qui est le nombre d'or $\dfrac{1+\sqrt{5}}{2}$.}

\subsubsection{Méthode du point fixe}
La méthode consiste cette fois à résoudre le problème sous la forme $g(x)=x$.

L'algorithme itératif consiste à déterminer successivement \fbox{$x_{k+1} = g(x_k)$}.

Toute la subtilité de cette méthode consiste à trouver une fonction $g$ suffisamment intelligente pour obtenir un ordre de convergence correct.

\subparagraph{}\textit{Déterminer une fonction $g$ qui permettrait d'appliquer la méthode du point fixe.}

\subparagraph{}\textit{Écrire une fonction \texttt{point\_fixe(g, xini)} qui affiche le résultat, le nombre d'itérations et qui renvoie la liste des approximations $x_k$ successives.}

\subparagraph{}\textit{Déterminer l'évolution de l'ordre de convergence en fonction des itérations et en déduire l'ordre de convergence.}

\subsection{Conclusion}
\subparagraph{}\textit{Comparer les différentes méthodes mises en \oe uvre en terme de nombres d'itérations pour converger, l'ordre de convergence. Tracer l'évolution de l'erreur en fonction du nombre d'itérations.}




\end{document}


