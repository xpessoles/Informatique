\documentclass[11pt,oneside]{article}
\input{style/coursHeadings}
\usepackage{algorithm}
\usepackage{algorithmic}


% Python sources
\usepackage{listings}
\usepackage{textcomp}
\usepackage{setspace}
%\usepackage{palatino}

%\usepackage{color}
\definecolor{Bleu}{rgb}{0.1,0.1,1.0}
\definecolor{Noir}{rgb}{0,0,0}
\definecolor{Grau}{rgb}{0.5,0.5,0.5}
\definecolor{DunkelGrau}{rgb}{0.15,0.15,0.15}
\definecolor{Hellbraun}{rgb}{0.5,0.25,0.0}
\definecolor{Magenta}{rgb}{1.0,0.0,1.0}
\definecolor{Gris}{gray}{0.5}
\definecolor{Vert}{rgb}{0,0.5,0}
\definecolor{SourceHintergrund}{rgb}{1,1.0,0.95}

%
\renewcommand{\lstlistlistingname}{Listings}
\renewcommand{\lstlistingname}{Listing}

\lstnewenvironment{python}[1][]{
\lstset{
language=python,
basicstyle=\ttfamily\footnotesize\setstretch{1}, 	
stringstyle=\color{red}, 
showstringspaces=false, 
alsoletter={1234567890},
otherkeywords={\ , \}, \{},
keywordstyle=\color{blue},
emph={access,and,break,class,continue,def,del,elif ,else,
except,exec,finally,for,from,global,if,import,in,i s,
lambda,not,or,pass,print,raise,return,try,while},
emphstyle=\color{black}\bfseries,
emph={[2]True, False, None, self},
emphstyle=[2]\color{green},
emph={[3]from, import, as},
emphstyle=[3]\color{blue},
upquote=true,
morecomment=[s]{"""}{"""},
commentstyle=\color{Hellbraun}\slshape, 
%emph={[4]1, 2, 3, 4, 5, 6, 7, 8, 9, 0},
emphstyle=[4]\color{blue},
literate=*{:}{{\textcolor{blue}:}}{1}
{=}{{\textcolor{blue}=}}{1}
{-}{{\textcolor{blue}-}}{1}
{+}{{\textcolor{blue}+}}{1}
{*}{{\textcolor{blue}*}}{1}
{!}{{\textcolor{blue}!}}{1}
{(}{{\textcolor{blue}(}}{1}
{)}{{\textcolor{blue})}}{1}
{[}{{\textcolor{blue}[}}{1}
{]}{{\textcolor{blue}]}}{1}
{<}{{\textcolor{blue}<}}{1}
{>}{{\textcolor{blue}>}}{1},
%framexleftmargin=1mm, framextopmargin=1mm, frame=shadowbox, rulesepcolor=\color{blue},#1
backgroundcolor=\color{SourceHintergrund}, 
framexleftmargin=1mm, framexrightmargin=1mm, framextopmargin=1mm, frame=single, framerule=1pt, rulecolor=\color{black},#1
}}{}


%Si le boolen xp est vrai : compilation pour xabi
%Sinon compilation Damien
\newboolean{xp}
\setboolean{xp}{false}

%\newboolean{prof}
%\setboolean{prof}{true}

\def\xxtitre{\ifthenelse{\boolean{xp}}{
CI 1 : Architecture matérielle et logicielle}{
Chapitre 2 -- Représentation des nombres}}

\def\xxsoustitre{\ifthenelse{\boolean{xp}}{
Chapitre 3 -- Représentation des nombres}{
Partie 1 -- Principe de la représentation des nombres entiers en mémoire}}

\def\xxauteur{\ifthenelse{\boolean{xp}}{
Xavier \textsc{Pessoles} \\ Damien \textsc{Iceta}}{
Damien \textsc{Iceta} \\ Xavier \textsc{Pessoles}}}

\def\xxpied{\ifthenelse{\boolean{xp}}{
Cours -- CI 1 : Architecture matérielle et logicielle\\
Représentation des Nombres}{
\xxtitre}}

\def\xxcathegorie{\ifthenelse{\boolean{xp}}{
2013 -- 2014 \\
Xavier \textsc{Pessoles}}{
Informatique - Cours}}

\ifthenelse{\boolean{xp}}{\usepackage[%
    pdftitle={Représentation des nombres},
    pdfauthor={Xavier Pessoles},
    colorlinks=true,
    linkcolor=blue,
    citecolor=magenta]{hyperref}

\usepackage{pifont}
%\usepackage{lastpage}

% \makeatletter \let\ps@plain\ps@empty \makeatother
%% DEBUT DU DOCUMENT
%% =================
\sloppy
\hyphenpenalty 10000


\colorlet{shadecolor}{orange!15}

\newtheorem{theorem}{Theorem}


\begin{document}


%\newboolean{prof}
%\setboolean{prof}{true}
% \makeatletter \let\ps@plain\ps@empty \makeatother
%% DEBUT DU DOCUMENT
%% =================




%------------- En tetes et Pieds de Pages ------------


\pagestyle{fancy}
\ifthenelse{\boolean{xp}}{%
\renewcommand{\headrulewidth}{0pt}}{%
\renewcommand{\headrulewidth}{0.2pt}} %pour mettre le trait en haut
%\renewcommand{\headrulewidth}{0.2pt}

\fancyhead{}
\fancyhead[L]{%
\noindent\begin{minipage}[c]{2.6cm}%
\includegraphics[width=2cm]{png/logo_ptsi.png}%
\end{minipage}}


\fancyhead[C]{\rule{12cm}{.5pt}}



\fancyhead[R]{%
\noindent\begin{minipage}[c]{3cm}
\begin{flushright}
\footnotesize{\textit{\textsf{Informatique}}}%
\end{flushright}
\end{minipage}
}



\fancyhead[C]{\rule{12cm}{.5pt}}

\renewcommand{\footrulewidth}{0.2pt}

\fancyfoot[C]{\footnotesize{\bfseries \thepage}}
\fancyfoot[L]{%
\begin{minipage}[c]{.2\linewidth}
\noindent\footnotesize{{\xxauteur}}
\end{minipage}
\ifthenelse{\boolean{xp}}{}{%
\begin{minipage}[c]{.15\linewidth}
\includegraphics[width=2cm]{png/logoCC.png}
\end{minipage}}
}

\ifthenelse{\boolean{prof}}{%
\fancyfoot[R]{\footnotesize{\xxpied}}}

\begin{center}
 \huge\textsc{\xxtitre}
\end{center}

\begin{center}
 \LARGE\textsc{\xxsoustitre}
\end{center}

\vspace{.5cm}
}{\input{style/enteteDI}}


%---------------------------------------------------------------------------



\begin{flushright}
%\textit{D'après ressources de Christophe François.}
\end{flushright}

\begin{minipage}[c]{.15\linewidth}
\begin{center}
%\includegraphics[height=.6cm]{png/w8}
\end{center}
\end{minipage}





\vspace{.5cm}

\ifthenelse{\boolean{xp}}{
\begin{savoir}

\textbf{Savoirs}

\begin{itemize}
\item Capacité Dec - C3 : Initier un sens critique au sujet de la qualité et de la précision des résultats de calculs numériques sur ordinateur
\begin{itemize}
\item Principe de la représentation des nombres entiers en mémoire
\item Principe de la représentation des nombres réels en mémoire
\end{itemize}
\end{itemize}
\end{savoir}
}{}



\setlength{\parskip}{0ex plus 0.2ex minus 0ex}
 \renewcommand{\contentsname}{}
 \renewcommand{\baselinestretch}{1}

\tableofcontents

 \renewcommand{\baselinestretch}{1.2}
\setlength{\parskip}{2ex plus 0.5ex minus 0.2ex}

% \vspace{1cm}

\begin{exercice}
Exercice avec corrigé: Trouver la représentation en base cinq de 58.

\begin{figure}[H]
\begin{center}
\includegraphics[width=.5\textwidth]{images/58.png}
\label{}
\end{center}
\end{figure}


Donc, 58 objets se regroupent en 11 paquets et 3 unités, puis les 11 paquets se regroupent en 2 paquets de paquets et 1 paquet.

$$58 = 11 × 5 + 3 = (2 × 5 + 1) × 5 + 3 = (2 × 52) + (1 × 51) + (3 × 50)$$

Donc $58 = 213_5.$
\end{exercice}

\begin{exercice}
Exercice avec corrigé Trouver la représentation en base seize du nombre 1207.

En base seize, on a besoin de 16 chiffres : 0, 1, 2, 3, 4, 5, 6, 7, 8, 9, puis a (dix), b (onze), c (douze), d (treize), e (quatorze) et f (quinze).


\begin{figure}[H]
\begin{center}
\includegraphics[width=.5\textwidth]{images/16.png}
\label{}
\end{center}
\end{figure}

Donc 1207 = 4b716.

\end{exercice}

\begin{exercice}
Exercice avec corrigé: Trouver les représentations binaires sur huit bits des entiers relatifs 0 et -128.
L’entier relatif 0 est représenté comme l’entier naturel 0 : 00000000. L’entier relatif -128 est représenté
comme l’entier naturel -128 + 256 = 128 : 10000000.
\end{exercice}

\begin{exercice}
Exercice avec corrigé Trouver les représentations décimales des entiers relatifs dont les représentations
binaires sur huit bits sont 00010111 et 10001100.
Le mot 00010111 représente l’entier naturel 23 et donc l’entier relatif 23. Le mot 10001100 représente
l’entier naturel 140 et donc l’entier relatif $140 - 256 = -116$.
\end{exercice}

\begin{thebibliography}{2}
\bibitem{cf}{Christophe François, Représentation de l'information, représentation des nombres.}
%\bibitem{zero}{Apprenez à programmer en Python \url{http://www.siteduzero.com/}.}
\bibitem{Manfred}{Manfred GILLI, METHODES NUMERIQUES, Département d’économétrie
Université de Genève, 2006.}
\end{thebibliography}
\end{document}
