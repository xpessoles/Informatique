%%%% Paramétrage du TD %%%%
\def\xxnumchapitre{Chapitre 1 \vspace{.2cm}}
\def\xxchapitre{\hspace{.12cm} Découverte de l'algorithmique et de la programmation}

\def\xxcompetences{%
\textsl{%
\textbf{Savoirs et compétences :}\\
\vspace{-.4cm}
\begin{itemize}[label=\ding{112},font=\color{bleuxp}] 
\item .
%\item \textit{Mod3.C2 :} pôles dominants et réduction de l’ordre du modèle : principe, justification
%\item \textit{Res2.C4 :} stabilité des SLCI : définition entrée bornée -- sortie bornée (EB -- SB)	
%\item \textit{Res2.C5 :} stabilité des SLCI : équation caractéristique	
%\item \textit{Res2.C6 :} stabilité des SLCI : position des pôles dans le plan complexe
%\item \textit{Res2.C7 :} stabilité des SLCI : marges de stabilité (de gain et de phase)
\end{itemize}
}}


\def\xxfigures{
%\includegraphics[width=3cm]{fig\_01}\\
%\textit{}
}%figues de la page de garde

\def\xxtitreexo{Structures algorithmiques}
\def\xxsourceexo{Lien Capytale \url{https://capytale2.ac-paris.fr/web/c/f807-628160/mcer}}
\def\xxactivite{TP 03 \ifprof  -- Corrigé \else \fi}

%\iflivret
\input{\repRel/Style/pagegarde\_TD}
%\else
%\input{../../style/new\_pagegarde}
%\fi

\setlength{\columnseprule}{.1pt}

\pagestyle{fancy}
\thispagestyle{plain}

\vspace{4.5cm}

\def\columnseprulecolor{\color{bleuxp}}
\setlength{\columnseprule}{0.4pt} 

%%%%%%%%%%%%%%%%%%%%%%%




\ifprof
\vspace{1.5cm}
\else
%\begin{multicols}{2}
\fi



Soit la liste suivante pour effectuer les tests : 
\begin{lstlisting}
les_notes=[13.5, 7.1, 14.0, 9.7, 5.9, 5.8, 6.5, 6.2, 9.1, 11.7, 8.6, 16.7, 12.0, 12.8, 9.8, 10.1, 8.3, 6.5, 11.4, 12.5, 7.0, 6.9, 7.9, 7.0, 10.1, 10.8, 9.1, 5.6, 8.2, 3.4, 10.8, 8.2, 13.3, 8.0, 14.9, 8.0, 8.2, 4.1, 6.5, 8.0, 8.2]
\end{lstlisting}

\subsection*{Moyenne et variance}

Soit $a=\verb![!a_0,a_1,\dots,a_{n-1}\verb!]!$ une liste de nombres. On rappelle les définitions de la moyenne $m$ et de la variance $v$ : 
$$m = \dfrac{1}{n} \sum_{i=0}^{n-1} a_i \qquad v = \dfrac{1}{n} \sum_{i=0}^{n-1} (a_i-m)^2.$$

\question{Ecrire une fonction \texttt{moyenne(a:list) -> float} qui prend pour argument une liste de nombres \texttt{a} et qui renvoie la moyenne de \texttt{a}.}

%%# votre fonction

\question{        Faire un test avec la liste \texttt{a=[1,2,3,4,5]}.}

%%# Tests de la fonction


\question{Ecrire une fonction \texttt{variance(a:list)} qui prend pour argument une liste de nombres \texttt{a}, et qui renvoie la variance de \texttt{a}.}

%%# votre fonction

\question{Faire un test avec la liste \texttt{a=[1,2,3,4,5]}.}

%%# Test

\question{Calculer la moyenne et l'écart-type (racine carrée de la variance) de la liste \texttt{les\_notes}.}

\begin{rem}On trouvera une moyenne d'environ $9.08$ et un écart-type d'environ $2.93$.
  \end{rem}

%%# Tests

On souhaite évaluer la moyenne par élément de deux listes, de même taille, de nombres comme présenté ci-dessous :

\begin{center}
\begin{tabular}{|l|c|c|c|c|c|c|c|c|}
\hline
Liste a & $a_0$ & $a_1$ & $a_2$ & $a_3$ & ...& $a_{n-1}$ \\ \hline
Liste b & $b_0$ & $b_1$ & $b_2$ & $b_3$ & ...& $b_{n-1}$ \\ \hline
Liste moyennes & $\dfrac{a_0+b_0}{2}$ & $\dfrac{a_1+b_1}{2}$ & $\dfrac{a_2+b_2}{2}$ & $\dfrac{a_3+b_3}{2}$ & ...& $\dfrac{a_{n-1}+b_{n-1}}{2}$ \\ \hline
\end{tabular}
\end{center}

\question{Ecrire une fonction \texttt{moyennes(a:list,b:list)->list} qui prend en entrée deux listes 
	\texttt{a} et \texttt{b} de même taille (condition qui ne doit \texttt{pas} être vérifiée)
	 et renvoie une liste de même taille contenant dans 
	la case d'indice \texttt{i} la valeur moyenne des valeurs des flottants 
	stockés dans les deux listes \texttt{a} et \texttt{b} à l'indice \texttt{i}.}

%%# votre fonction

\question{Tester le bon fonctionnement de votre fonction.}

%# test

%# test
%a=[i for i in range(1,11)]
%b=[i for i in range(10,0,-1)]
%test.moyennes_test(moyennes,a,b)

Pour aller plus loin, évaluation de la moyenne glissante sur n éléments consécutifs.

Exemple avec n=3 :

\begin{center}
\begin{tabular}{|l|c|c|c|c|c|c|c|c|}
\hline
Liste & 12 & 13.5 & 11.2 & 11.7 & 12.1 & ... & ... \\ \hline
Moyenne glissante & 12 & 13.5 & $\dfrac{12+13.5+11.2}{3}$ & $\dfrac{13.5+11.2+11.7}{3}$ & $\dfrac{11.2+11.7+12.1}{3}$ & ... & ... \\ \hline
\end{tabular}
\end{center}
%![image.png](attachment:image.png)

\question{Ecrire une fonction \texttt{moyenneGlissante(a:list, n:int)->list} d'argument une liste \texttt{a} d'entiers ou flottants et un entier \texttt{n} et renvoyant la liste des moyennes glissantes sur \texttt{n} éléments consécutifs.
        Vous pourrez avantageusement utiliser le slicing.<br/>
        Faire un test avec la listes \texttt{les\_notes}.}

%# votre fonction

%# test

%# test
%L=[12,13.5,11.2,11.7,12.1,13.05,11.6]
%test.moyenneGlissante_test(moyenneGlissante,L,3)
%test.moyenneGlissante_test(moyenneGlissante,L,4)

\subsection*{Le maximum}

\question{Ecrire une fonction \texttt{maximum(L:list)->float} qui à partir d'une liste de flottants ou entiers renvoie le max de cette liste. La fonction prédéfinie en Python \texttt{max} ne doit pas être utilisée.}
%# votre fonction

\subsection*{Seuil}

On souhaite maintenant, à partir d'une liste \texttt{a}
 d'entiers (ou flottants) et d'un entier (ou flottant) appelé \texttt{seuil} obtenir le nombre 
 d'éléments de \texttt{L} majorés, au sens strict, par \texttt{seuil}. %<br/>
% Voici un exemple d'utilisation de cette fonction :

%# majores\_par([12,-5,10,9],10)
%# 2

\question{Ecrire une fonction \texttt{majores\_par(L:list,x:float)->int} réalisant cette opération.}

%# votre fonction        

\question{Tester le bon fonctionnement de votre fonction.}

%# test
%majores\_par([12,-5,10,9],10)

\question{Modifier la fonction précédente pour obtenir une fonction \texttt{ elements\_majores\_par(L:list,x:float)->list}
  retournant la liste des éléments majorés par le seuil \texttt{x.}}

%# votre fonction

\question{Tester le bon fonctionnement de votre fonction.}

%# test

\subsection*{Recherche séquentielle dans une liste}

\question{Définir une fonction \texttt{sequentielle(a:list,x:float)->bool}, d'arguments
  une liste \texttt{a}  et un entier ou un flottant \texttt{x}, et qui renvoie le booléen \texttt{True} ou \texttt{False} selon que l'élément
  \texttt{x} est dans la liste  \texttt{x} ou non.<br/>
  <i> On fera attention à arrêter la recherche dès que l'élément est trouvé.</i><br/>
  Faire vos tests avec la liste \texttt{les\_notes} en prenant pour \texttt{x} la première valeur de \texttt{les\_notes}, puis la dernière et enfin une valeur au milieu. }

%# votre fonction

\question{Tester le bon fonctionnement de votre fonction.}

%# test

%# test
%a=[1,2,5,4,3,9]
%test.sequentielle_test(sequentielle,a,1)
%test.sequentielle_test(sequentielle,a,4)
%test.sequentielle_test(sequentielle,a,9)
%test.sequentielle_test(sequentielle,a,6)

\question{Définir une fonction \texttt{occurrenceElement(a:list,x:float)->int}, d'arguments
  une liste \texttt{a} et un flottant \texttt{x}, et qui renvoie le nombre d'occurrence de l'entier ou flottant \texttt{x}.}
%# votre fonction

\question{Tester le bon fonctionnement de votre fonction.}



\question{Définir une fonction \texttt{occurrenceListe(a:list)->list}, d'argument
  une liste \texttt{a} d'entiers compris dans l'intervalle [0,k] et de longueur \texttt{n} tel que $k<n$ et qui renvoie la liste des nombres d'occurrence des entiers de la liste \texttt{a}. Cette valeur est 0 si le nombre n'est pas dans la liste.}
%\begin{rem}Remarque</i> : Une seule boucle for est acceptée.}

%# votre fonction

\question{Tester le bon fonctionnement de votre fonction.}

%# votre test

\subsection*{Création de listes aléatoires}

Pour tester vos algorithmes, il peut être utile de créer des listes quelconques de nombres. On peut écrire une fonction qui crée des listes aléatoires à partir de la bibliothèque \texttt{random}.

\begin{lstlisting}
# import de la bibliothèque random
import random as r
# r est un alias
\end{lstlisting}
\question{Comprendre le fonctionnement de la fonction \texttt{randrange} de cette bibliothèque (en utilisant l'aide et avec des exemples).}

%# Tests
\begin{lstlisting}
help (r.randrange)
\end{lstlisting}

\question{Ecrire une fonction \texttt{hasard\_liste(n:int,k:int)->list} d'arguments deux entiers \texttt{n} et \texttt{k} permettant de générer une liste de \texttt{n} entiers aléatoires appartenant à \texttt{range(k)}.}

%# votre fonction

\question{Tester votre fonction avec $n=10$ et $k=7$.}


\question{Créer une liste à partir de la fonction \texttt{hasard\_liste(n:int,k:int)} et tester votre fonction \texttt{occurrenceListe(a:list)}.}



\ifprof
\else
%\end{multicols}
\fi

