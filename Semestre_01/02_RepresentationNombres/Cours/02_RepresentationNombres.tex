%%%% Paramétrage du cours %%%%
\def\xxactivite{Cours}

\fichefalse \proftrue \tdfalse \courstrue

\def\xxnumchapitre{Chapitre 1 \vspace{.2cm}}
\def\xxchapitre{\hspace{.12cm} Représentation des nombres en mémoire}

\def\xxcompetences{%
\textsl{%
\textbf{Savoirs et compétences :}\\
\begin{itemize}[label=\ding{112},font=\color{bleuxp}] 
\item Représentation des entiers positifs sur des mots de taille fixe.
\item Représentation des entiers signés sur des mots de taille fixe.
\item Entiers multi-précision de Python.
\item Distinction entre nombres réels, décimaux et flottants.
\item Représentation entre nombres réels, décimaux et flottants.
\item Représentation des flottants sur des mots de taille fixe. Notion de mantisse, d'exposant. 
\item Précision des calculs en flottants.
\end{itemize}
}}

\def\xxfigures{
%\includegraphics[width=\linewidth]{matlab}
%\\
%\textit{Modèle du pilote hydraulique avec pilotage interactif.}
}%figues de la page de garde

\input{\repRel/Style/pagegarde_cours_minitoc}
\setlength{\columnseprule}{.1pt}

\vspace{2cm}
\pagestyle{fancy}
\thispagestyle{plain}

%%%%%%%%%%%%%%%%%%%%%%%

\section{Représentation des nombres entiers}

On décompose un entier en dizaines, centaines, milliers, etc. 
L'essentiel est alors qu'il y ait strictement moins de dix éléments dans chaque type de paquet. Ce nombre d'éléments peut être représenté par un chiffre.
On écrit alors tous les chiffres à la suite. À gauche, on place les \emph{chiffres de poids fort} (gros paquets). À droite, les \emph{chiffres de poids faible}.

Ainsi $2735$ représente deux milliers plus sept centaines plus trois dizaines plus cinq unités.

\begin{defi}{Ecriture d'un nomre dans une base }
De manière générale :%, avec $B=10$ et $n\in\N$,
$$\displaystyle\underline{a_{n}a_{n-1}\ldots a_{1}a_{0}}_{~B} =
\sum_{k=0}^{n}a_{k}B^{k},~ \textrm{ et }~\forall k \in \iif{0;n},~ a_k\in\ii{0;B}.$$ 
On note $B$ la base, $a_k$ le chiffre de rang $k$.
\end{defi}

\begin{exemple}
Décomposition de 247 en base 10 : $247_{(10)} = 2\cdot 10^2 + 4\cdot 10^1 + 7\cdot 10^0$.

Décomposition de $1001_2$ en base 2 : $1001_2 = 1\cdot 2^{11_2} + 0\cdot 2^{10_2} + 0\cdot 2^{1_2} + 1\cdot 2^{0_2} 
= 1\cdot 2^{3_{10}} + 0\cdot 2^{2_{10}} + 0\cdot 2^{1_{10}} + 1\cdot 2^{0_{10}}
$.

\end{exemple}

\begin{rem}
On appelle : 
\begin{itemize}
\item $2$ est appelé digit de poids fort (\textit{most significant digit});
\item $7$ est appelé digit de poids faible (\textit{least significant digit}).
\end{itemize}
\end{rem}
