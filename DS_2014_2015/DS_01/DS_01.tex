\documentclass[10pt]{article}
\input{style/coursHeadings}
\usepackage{algorithm}
\usepackage{algorithmic}


% Python sources
\usepackage{listings}
\usepackage{textcomp}
\usepackage{setspace}
%\usepackage{palatino}

%\usepackage{color}
\definecolor{Bleu}{rgb}{0.1,0.1,1.0}
\definecolor{Noir}{rgb}{0,0,0}
\definecolor{Grau}{rgb}{0.5,0.5,0.5}
\definecolor{DunkelGrau}{rgb}{0.15,0.15,0.15}
\definecolor{Hellbraun}{rgb}{0.5,0.25,0.0}
\definecolor{Magenta}{rgb}{1.0,0.0,1.0}
\definecolor{Gris}{gray}{0.5}
\definecolor{Vert}{rgb}{0,0.5,0}
\definecolor{SourceHintergrund}{rgb}{1,1.0,0.95}

%
\renewcommand{\lstlistlistingname}{Listings}
\renewcommand{\lstlistingname}{Listing}

\lstnewenvironment{python}[1][]{
\lstset{
language=python,
basicstyle=\ttfamily\footnotesize\setstretch{1}, 	
stringstyle=\color{red}, 
showstringspaces=false, 
alsoletter={1234567890},
otherkeywords={\ , \}, \{},
keywordstyle=\color{blue},
emph={access,and,break,class,continue,def,del,elif ,else,
except,exec,finally,for,from,global,if,import,in,i s,
lambda,not,or,pass,print,raise,return,try,while},
emphstyle=\color{black}\bfseries,
emph={[2]True, False, None, self},
emphstyle=[2]\color{green},
emph={[3]from, import, as},
emphstyle=[3]\color{blue},
upquote=true,
morecomment=[s]{"""}{"""},
commentstyle=\color{Hellbraun}\slshape, 
%emph={[4]1, 2, 3, 4, 5, 6, 7, 8, 9, 0},
emphstyle=[4]\color{blue},
literate=*{:}{{\textcolor{blue}:}}{1}
{=}{{\textcolor{blue}=}}{1}
{-}{{\textcolor{blue}-}}{1}
{+}{{\textcolor{blue}+}}{1}
{*}{{\textcolor{blue}*}}{1}
{!}{{\textcolor{blue}!}}{1}
{(}{{\textcolor{blue}(}}{1}
{)}{{\textcolor{blue})}}{1}
{[}{{\textcolor{blue}[}}{1}
{]}{{\textcolor{blue}]}}{1}
{<}{{\textcolor{blue}<}}{1}
{>}{{\textcolor{blue}>}}{1},
%framexleftmargin=1mm, framextopmargin=1mm, frame=shadowbox, rulesepcolor=\color{blue},#1
backgroundcolor=\color{SourceHintergrund}, 
framexleftmargin=1mm, framexrightmargin=1mm, framextopmargin=1mm, frame=single, framerule=1pt, rulecolor=\color{black},#1
}}{}
%%%%%%%%%%%%
% Définition des vecteurs 
%%%%%%%%%%%%
\newcommand{\vect}[1]{\overrightarrow{#1}}
\newcommand{\axe}[2]{\left(#1,\vect{#2}\right)}
\newcommand{\couple}[2]{\left(#1,\vect{#2}\right)}
\newcommand{\angl}[2]{\left(\vect{#1},\vect{#2}\right)}

\newcommand{\rep}[1]{\mathcal{R}_{#1}}
\newcommand{\quadruplet}[4]{\left(#1;#2,#3,#4 \right)}
\newcommand{\repere}[4]{\left(#1;\vect{#2},\vect{#3},\vect{#4} \right)}
\newcommand{\base}[3]{\left(\vect{#1},\vect{#2},\vect{#3} \right)}


\newcommand{\vx}[1]{\vect{x_{#1}}}
\newcommand{\vy}[1]{\vect{y_{#1}}}
\newcommand{\vz}[1]{\vect{z_{#1}}}

\newcommand{\norm}[1]{\ensuremath{\left\Vert {#1}\right\Vert}}
\newcommand{\Ker}{\mathop{\mathrm{Ker}}\nolimits}

% d droit pour le calcul différentiel
\newcommand{\dd}{\text{d}}

\newcommand{\inertie}[2]{I_{#1}\left( #2\right)}
\newcommand{\matinertie}[7]{
\begin{pmatrix}
#1 & #6 & #5 \\
#6 & #2 & #4 \\
#5 & #4 & #3 \\
\end{pmatrix}_{#7}}
%%%%%%%%%%%%
% Définition des torseurs 
%%%%%%%%%%%%

\newcommand{\ec}[2]{%
\mathcal{E}_c\left(#1/#2\right)}

\newcommand{\pext}[3]{%
\mathcal{P}\left(#1\rightarrow#2/#3\right)}

\newcommand{\pint}[3]{%
\mathcal{P}\left(#1 \stackrel{\text{#3}}{\leftrightarrow} #2\right)}


 \newcommand{\torseur}[1]{%
\left\{{#1}\right\}
}

\newcommand{\torseurcin}[3]{%
\left\{\mathcal{#1} \left(#2/#3 \right) \right\}
}

\newcommand{\torseurci}[2]{%
\left\{\sigma \left(#1/#2 \right) \right\}
}
\newcommand{\torseurdyn}[2]{%
\left\{\mathcal{D} \left(#1/#2 \right) \right\}
}


\newcommand{\torseurstat}[3]{%
\left\{\mathcal{#1} \left(#2\rightarrow #3 \right) \right\}
}


 \newcommand{\torseurc}[8]{%
%\left\{#1 \right\}=
\left\{
{#1}
\right\}
 = 
\left\{%
\begin{array}{cc}%
{#2} & {#5}\\%
{#3} & {#6}\\%
{#4} & {#7}\\%
\end{array}%
\right\}_{#8}%
}

 \newcommand{\torseurcol}[7]{
\left\{%
\begin{array}{cc}%
{#1} & {#4}\\%
{#2} & {#5}\\%
{#3} & {#6}\\%
\end{array}%
\right\}_{#7}%
}

 \newcommand{\torseurl}[3]{%
%\left\{\mathcal{#1}\right\}_{#2}=%
\left\{%
\begin{array}{l}%
{#1} \\%
{#2} %
\end{array}%
\right\}_{#3}%
}

% Vecteur vitesse
 \newcommand{\vectv}[3]{%
\vect{V\left( {#1} \in {#2}/{#3}\right)}
}

% Vecteur force
\newcommand{\vectf}[2]{%
\vect{R\left( {#1} \rightarrow {#2}\right)}
}

% Vecteur moment stat
\newcommand{\vectm}[3]{%
\vect{\mathcal{M}\left( {#1}, {#2} \rightarrow {#3}\right)}
}




% Vecteur résultante cin
\newcommand{\vectrc}[2]{%
\vect{R_c \left( {#1}/ {#2}\right)}
}
% Vecteur moment cin
\newcommand{\vectmc}[3]{%
\vect{\sigma \left( {#1}, {#2} /{#3}\right)}
}


% Vecteur résultante dyn
\newcommand{\vectrd}[2]{%
\vect{R_d \left( {#1}/ {#2}\right)}
}
% Vecteur moment dyn
\newcommand{\vectmd}[3]{%
\vect{\delta \left( {#1}, {#2} /{#3}\right)}
}

% Vecteur accélération
 \newcommand{\vectg}[3]{%
\vect{\Gamma \left( {#1} \in {#2}/{#3}\right)}
}

% Vecteur omega
 \newcommand{\vecto}[2]{%
\vect{\Omega\left( {#1}/{#2}\right)}
}
% }$$\left\{\mathcal{#1} \right\}_{#2} =%
% \left\{%
% \begin{array}{c}%
%  #3 \\%
%  #4 %
% \end{array}%
% \right\}_{#5}}

\newcommand{\N}{\mathbb{N}}
\newcommand{\Z}{\mathbb{Z}}
\newcommand{\R}{\mathbb{R}}
\newcommand{\C}{\mathbb{C}}
\newcommand{\K}{\mathbb{K}}

\newcommand{\cA}{\mathscr{A}}
\newcommand{\cM}{\mathscr{M}}
\newcommand{\cL}{\mathscr{L}}
\newcommand{\cS}{\mathscr{S}}

\newcommand{\python}{\texttt{Python}}

\newcommand{\z}[1]{\Z_{#1}}
\newcommand{\ztimes}[1]{\Z_{#1}^{\times}}
\newcommand{\ii}[1]{[\![#1[\![}
\newcommand{\iif}[1]{[\![#1]\!]}
\newcommand{\llbr}{\ensuremath{\llbracket}}
\newcommand{\rrbr}{\ensuremath{\rrbracket}}
%\newcommand{\p}[1]{\left(#1\right)}
\newcommand{\ens}[1]{\left\{ #1 \right\}}
\newcommand{\croch}[1]{\left[ #1 \right]}
%\newcommand{\of}[1]{\lstinline{#1}}
% \newcommand{\py}[2]{%
%   \begin{tabular}{|l}
%     \lstinline+>>>+\textbf{\of{#1}}\\
%     \of{#2}
%   \end{tabular}\par{}
% }
\newcommand{\floor}[1]{\left\lfloor#1\right\rfloor}
\newcommand{\ceil}[1]{\left\lceil#1\right\rceil}
\newcommand{\abs}[1]{\left|#1\right|}


% Binaire, octal, hexa
\newcommand{\hex}[1]{\underline{\text{\texttt{#1}}}_{16}}
\newcommand{\oct}[1]{\underline{\text{\texttt{#1}}}_{8}}
\newcommand{\bin}[1]{\underline{\text{\texttt{#1}}}_{2}}
\DeclareMathOperator{\mmod}{\texttt{\%}}


% Fonctions et systèmes
\newcommand{\fct}[5][t]{%
  \begin{array}[#1]{rcl}
    #2 & \rightarrow & #3\\
    #4 & \mapsto     & #5\\
  \end{array}
}
\newcommand{\fonction}[5]{#1 : \left\{\begin{array}{rcl} #2& \longrightarrow &#3 \\ #4 &\longmapsto & #5\end{array}\right.}
\newenvironment{systeme}{\left\{ \begin{array}{rcl}}{\end{array}\right.}

% Matrices
\newcommand{\mat}[1]{
  \begin{pmatrix}
    #1
  \end{pmatrix}
}
\newcommand{\inv}{\ensuremath{^{-1}}}
\newcommand{\bpm}{\begin{pmatrix}}
\newcommand{\epm}{\end{pmatrix}}


% bases de données
\newcommand{\relat}[1]{\textsc{#1}}
\newcommand{\attr}[1]{\emph{#1}}
\newcommand{\prim}[1]{\uline{#1}}
\newcommand{\foreign}[1]{\#\textsl{#1}}


% Bases de données

\newcommand{\att}{\ensuremath{\mathbf{att}}}
\newcommand{\dom}{\ensuremath{\mathbf{dom}}}
\newcommand{\sort}{\ensuremath{\mathbf{sort}}}
\newcommand{\relname}{\ensuremath{\mathbf{relname}}}
\newcommand{\var}{\ensuremath{\mathbf{var}}}
\newcommand{\FILM}{\ensuremath{\mathtt{FILM}}}
\newcommand{\JOUE}{\ensuremath{\mathtt{JOUE}}}
\newcommand{\PERSONNE}{\ensuremath{\mathtt{PERSONNE}}}
\newcommand{\PERSONNAGE}{\ensuremath{\mathtt{PERSONNAGE}}}

\newcommand{\ttid}{\ensuremath{\mathtt{id}}}
\newcommand{\tttitre}{\ensuremath{\mathtt{titre}}}
\newcommand{\ttdate}{\ensuremath{\mathtt{date}}}
\newcommand{\ttidr}{\ensuremath{\mathtt{idrealisateur}}}
\newcommand{\ttdatenais}{\ensuremath{\mathtt{datenaissance}}}
\newcommand{\ttnom}{\ensuremath{\mathtt{nom}}}
\newcommand{\ttprenom}{\ensuremath{\mathtt{prenom}}}
\newcommand{\ttidacteur}{\ensuremath{\mathtt{idacteur}}}
\newcommand{\ttidfilm}{\ensuremath{\mathtt{idfilm}}}
\newcommand{\ttidpersonnage}{\ensuremath{\mathtt{idpersonnage}}}

\newcommand{\fv}{\mathrm{libre}}
\newcommand{\sem}[1]{[\![ #1 ]\!]}

\input{style/macros_Titres}
\input{style/macros_Frames}
\usepackage{fancybox}
\newsavebox{\codebox}
\newsavebox{\codeboxx}

%Si le boolen xp est vrai : compilation pour xabi
%Sinon compilation Damien
\newboolean{xp}
\setboolean{xp}{true}

\newboolean{prof}
\setboolean{prof}{false}

\newif\ifprof
%\proftrue
\proffalse

\usepackage[%
    pdftitle={Devoirs Surveillé 1},
    pdfauthor={Xavier Pessoles},
    colorlinks=true,
    linkcolor=blue,
    citecolor=magenta]{hyperref}


\def\discipline{Informatique}
\def\xxtitre{\ifthenelse{\boolean{xp}}{
Devoir surveillé d'informatique 1}{
Chapitre  -- }}

\def\xxsoustitre{\ifthenelse{\boolean{xp}}{
CI 1 : Architecture matérielle et logicielle \\
CI 2 : Algorithmique et programmation  }{
Partie  -- }}

\def\xxauteur{\ifthenelse{\boolean{xp}}{
Xavier \textsc{Pessoles}}{
}}

\def\xxpied{\ifthenelse{\boolean{xp}}{
DS Informatique\\
\ifthenelse{\boolean{prof}}{Sujet}{Corrige}}{
\xxtitre}}

\def\xxcathegorie{\ifthenelse{\boolean{xp}}{
2013 -- 2014 \\
Xavier \textsc{Pessoles}}{
Informatique - Cours}}





%---------------------------------------------------------------------------


\begin{document}

\ifthenelse{\boolean{xp}}{\usepackage[%
    pdftitle={Représentation des nombres},
    pdfauthor={Xavier Pessoles},
    colorlinks=true,
    linkcolor=blue,
    citecolor=magenta]{hyperref}

\usepackage{pifont}
%\usepackage{lastpage}

% \makeatletter \let\ps@plain\ps@empty \makeatother
%% DEBUT DU DOCUMENT
%% =================
\sloppy
\hyphenpenalty 10000


\colorlet{shadecolor}{orange!15}

\newtheorem{theorem}{Theorem}


\begin{document}


%\newboolean{prof}
%\setboolean{prof}{true}
% \makeatletter \let\ps@plain\ps@empty \makeatother
%% DEBUT DU DOCUMENT
%% =================




%------------- En tetes et Pieds de Pages ------------


\pagestyle{fancy}
\ifthenelse{\boolean{xp}}{%
\renewcommand{\headrulewidth}{0pt}}{%
\renewcommand{\headrulewidth}{0.2pt}} %pour mettre le trait en haut
%\renewcommand{\headrulewidth}{0.2pt}

\fancyhead{}
\fancyhead[L]{%
\noindent\begin{minipage}[c]{2.6cm}%
\includegraphics[width=2cm]{png/logo_ptsi.png}%
\end{minipage}}


\fancyhead[C]{\rule{12cm}{.5pt}}



\fancyhead[R]{%
\noindent\begin{minipage}[c]{3cm}
\begin{flushright}
\footnotesize{\textit{\textsf{Informatique}}}%
\end{flushright}
\end{minipage}
}



\fancyhead[C]{\rule{12cm}{.5pt}}

\renewcommand{\footrulewidth}{0.2pt}

\fancyfoot[C]{\footnotesize{\bfseries \thepage}}
\fancyfoot[L]{%
\begin{minipage}[c]{.2\linewidth}
\noindent\footnotesize{{\xxauteur}}
\end{minipage}
\ifthenelse{\boolean{xp}}{}{%
\begin{minipage}[c]{.15\linewidth}
\includegraphics[width=2cm]{png/logoCC.png}
\end{minipage}}
}

\ifthenelse{\boolean{prof}}{%
\fancyfoot[R]{\footnotesize{\xxpied}}}

\begin{center}
 \huge\textsc{\xxtitre}
\end{center}

\begin{center}
 \LARGE\textsc{\xxsoustitre}
\end{center}

\vspace{.5cm}
}{\input{style/enteteDI}}


\ifthenelse{\boolean{prof}}{
\begin{center}
\large{\textit{Éléments de corrigé}}
\end{center}
}{}


\begin{center}
\large{\textit{Nom : .......................................}}
\end{center}
\section{Codage des nombres}
Pour tout ce devoir, on dispose d'une machine dont le codage est limité à 8 bits. 

\subsection{Capacités de l'espace machine}
\subparagraph{} \textit{Quel est le nombre maximum d'entiers qu'il est possible de coder ? Donner le nombre maximal et le nombre minimal dans les systèmes décimal, binaire et hexadacimal.}

%%% Question 1
\vspace{.3cm}
\noindent\boxput*(-.85,1){
\colorbox{white}{\textbf{Question 1}}}{
\setlength{\fboxsep}{10pt}
\fbox{\begin{minipage}{.95\linewidth}
\usebox{\codebox}
\ifprof
\begin{corrige}
\begin{itemize}
\item Il est possible de coder $2^8=256$ entiers. 
\item Le plus petit est nombre est $0$.
\item Le plus grand est $(255)_{10}=(1111\; 1111)_2 = (FF)_{16}$
\end{itemize}
\end{corrige}
\else
\vspace{3cm}
\fi
\end{minipage}
}}



\subparagraph{} \textit{Quel est le nombre maximum d'entiers relatifs qu'il est possible de coder ? Donner le nombre minimal et le nombre maximal dans le système décimal.}

%%% Question 2
\vspace{.3cm}
\noindent\boxput*(-.85,1){
\colorbox{white}{\textbf{Question 2}}}{
\setlength{\fboxsep}{10pt}
\fbox{\begin{minipage}{.95\linewidth}
\ifprof
\begin{corrige}
\begin{itemize}
\item Il est possible de coder $2^8=256$ entiers relatifs. 
\item Le plus petit est nombre est $-128$.
\item Le plus grand est $127$.
\end{itemize}
\end{corrige}
\else
\usebox{\codebox}
\vspace{3cm}
\fi
\end{minipage}
}}

\ifprof
\else
\newpage
\fi

\subsection{Conversions}
Dans cette partie, les nombres sont tous des entiers relatifs codés en complément à 2.
 
\subparagraph{} \textit{Convertir le nombre 83 dans le système binaire et dans le système hexadécimal.}

%%% Question 3
\vspace{.3cm}
\noindent\boxput*(-.85,1){
\colorbox{white}{\textbf{Question 3}}}{
\setlength{\fboxsep}{10pt}
\fbox{\begin{minipage}{.95\linewidth}
\ifprof
\begin{corrige}
$$
(83)_{10} = (0101\;0011)_2 = (53)_{16}
$$
\end{corrige}
\else
\usebox{\codebox}
\vspace{5cm}
\fi
\end{minipage}
}}



\subparagraph{} \textit{Peut-on réaliser la somme 83 + 200 ? Justifier.}

%%% Question 4
\vspace{.3cm}
\noindent\boxput*(-.85,1){
\colorbox{white}{\textbf{Question 4}}}{
\setlength{\fboxsep}{10pt}
\fbox{\begin{minipage}{.95\linewidth}
\ifprof
\begin{corrige}
Il n'est pas possible de réaliser la somme $200+83$ car $283$ est en dehors des capacités du codage. 
\end{corrige}
\else
\usebox{\codebox}
\vspace{3cm}
\fi
\end{minipage}
}}

\ifprof
\else
\newpage
\fi

\subparagraph{} \textit{Réaliser l'opération 24 - 83. Donner le résultat en binaire.}

%%% Question 5
\vspace{.3cm}
\noindent\boxput*(-.85,1){
\colorbox{white}{\textbf{Question 5}}}{
\setlength{\fboxsep}{10pt}
\fbox{\begin{minipage}{.95\linewidth}
\ifprof
\begin{corrige}
\end{corrige}
\else
\usebox{\codebox}
\vspace{10cm}
\fi
\end{minipage}
}}


\setcounter{subparagraph}{6}
%\subparagraph{} \textit{Convertir le nombre $(A3)_{16,\mathbb{Z}}$. Coder ce nombre dans le système décimal.}
%
%%%% Question 6
%\vspace{.3cm}
%\noindent\boxput*(-.85,1){
%\colorbox{white}{\textbf{Question 6}}}{
%\setlength{\fboxsep}{10pt}
%\fbox{\begin{minipage}{.95\linewidth}
%\usebox{\codebox}
%\vspace{10cm}
%\end{minipage}
%}}



\subsection{Algorithmique et programmation}

Le but de cette partie est de réaliser un programme permettant de réaliser le codage d'un nombre entier relatif en utilisant le codage en complément à 2. 

\begin{py}
Une chaîne de caractère se comporte comme un liste. En effet prenons par exemple la chaîne de caractères \textsf{exemple} :

\begin{minipage}[c]{.95\linewidth}
\begin{python}
>>> chaine = ''exemple''
>>> print(chaine)
        exemple
>>> len(chaine)  # Retourne le nombre de caractères de la chaine : il y a 7 caractères dans le mot exemple
        7
>>> print(chaine[0]) # Affiche le premier e
        'e'
>>> print(chaine[6]) # Affiche le dernier e
        'e'
>>> for i in range(0,2,1) : # Pour i allant de 0 (inclus) à 2 (exclus) par pas de 1, faire : 
            print(str(i)+'' : ''+chaine[i])
            
        0 : e
        1 : x
>>> chaine = chaine+''s''
>>> print(chaine)
        exemples
>>> chaine = ''Les ''+chaine
>>> print(chaine)
        Les exemples
\end{python}
\end{minipage}
\end{py}

\subsubsection{Conversion d'un nombre décimal en binaire}
On donne l'extrait de programme suivant permettant de convertir un nombre entier positif \textsf{nb} en chaine de caractères \textsf{res} dont le contenu est le nombre en binaire. 


\begin{py}
\begin{minipage}[c]{.5\linewidth}
\begin{python}
nb = 10
dividende = nb
diviseur = 2
resultat = ""
quotient = -nb
    
while quotient != 0 :
    quotient = int(dividende/diviseur)
    reste = dividende - diviseur * quotient
    dividende = quotient
    resultat = resultat + str(reste)
\end{python}
\end{minipage}
\end{py}

\subparagraph{}\textit{Quel est le type des variables \textsf{dividende} et \textsf{resultat}.}

%%% Question 7
\vspace{.3cm}
\noindent\boxput*(-.85,1){
\colorbox{white}{\textbf{Question 7}}}{
\setlength{\fboxsep}{10pt}
\fbox{\begin{minipage}{.95\linewidth}
\ifprof
\begin{corrige}
\end{corrige}
\else
\usebox{\codebox}
\vspace{3cm}
\fi
\end{minipage}}}



\subparagraph{}\textit{Expliquer le ligne 7. Justifier ce choix.}

%%% Question 8
\vspace{.3cm}
\noindent\boxput*(-.85,1){
\colorbox{white}{\textbf{Question 8}}}{
\setlength{\fboxsep}{10pt}
\fbox{\begin{minipage}{.95\linewidth}
\ifprof
\begin{corrige}
\end{corrige}
\else
\usebox{\codebox}
\vspace{3cm}
\fi
\end{minipage}}}

\newpage

\subparagraph{}\textit{On cherche à analyser l'évolution des variables lors du parcours de la boucle \textsf{while}. Remplir les champs suivants.}

%%% Question 9
\vspace{.3cm}
\noindent\boxput*(-.85,1){
\colorbox{white}{\textbf{Question 9}}}{
\setlength{\fboxsep}{10pt}
\fbox{\begin{minipage}{.95\linewidth}
\textit{Remarque : le document réponse ne présume pas du nombre d'itérations de la boucle while.}
\begin{center}
\begin{tabular}{|p{6cm}|p{.3cm} c p{.3cm}|p{.3cm} c p{.3cm}|p{.3cm} c p{.3cm}|p{.3cm} c p{.3cm}|}
\hline
&&&&&&&&&&&& \\
&& \rotatebox{90}{Dividende}&& 
&\rotatebox{90}{Diviseur} &&
&\rotatebox{90}{Résultat} &&
&\rotatebox{90}{Quotient}&
\\
\hline \hline
&&&&&&&&&&&& \\
État des variables après la ligne 6 &&&&&&&&&&&& \\
&&&&&&&&&&&& \\
\hline
&&&&&&&&&&&& \\
État des variables après la ligne 11 - Première itération de la boucle while 
&&&&&&&&&&&& \\
\hline
&&&&&&&&&&&& \\
État des variables après la ligne 11 - Seconde itération de la boucle while 
&&&&&&&&&&&& \\
\hline
&&&&&&&&&&&& \\
État des variables après la ligne 11 - Troisième itération de la boucle while 
&&&&&&&&&&&& \\
\hline
&&&&&&&&&&&& \\
État des variables après la ligne 11 - Quatrième itération de la boucle while 
&&&&&&&&&&&& \\
\hline
&&&&&&&&&&&& \\
État des variables après la ligne 11 - Cinquième itération de la boucle while 
&&&&&&&&&&&& \\
\hline
&&&&&&&&&&&& \\
État des variables après la ligne 11 - Sixième itération de la boucle while 
&&&&&&&&&&&& \\
\hline

\end{tabular}
\end{center}
\usebox{\codebox}

\end{minipage}}}


\subparagraph{}\textit{Parmi les lignes 8, 9 et 10, réaliser des modifications qui permettent de mieux utiliser les opérations disponibles en Pyhon. }


%%% Question 10
\vspace{.3cm}
\noindent\boxput*(-.85,1){
\colorbox{white}{\textbf{Question 10}}}{
\setlength{\fboxsep}{10pt}
\fbox{\begin{minipage}{.95\linewidth}
\usebox{\codebox}
\vspace{3cm}
\end{minipage}}}


\newpage

\subparagraph{}\textit{Après exécution de la liste que contient la variable \textsf{resultat} ?
 Est-ce le résultat attendu ? Si ce n'est pas le résultat attendu, corriger l'algorithme en conséquence.} 
 
 
%%% Question 11
\vspace{.3cm}
\noindent\boxput*(-.85,1){
\colorbox{white}{\textbf{Question 11}}}{
\setlength{\fboxsep}{10pt}
\fbox{\begin{minipage}{.95\linewidth}
\usebox{\codebox}
\vspace{4cm}
\end{minipage}}}


\subsubsection{Programme mystère}
On cherche à convertir le nombre $(-10)_{10}$ en base 2. Le système utilisé utilise un codage sur 8 bits. La conversion du nombre $(10)_{10}$ en binaire est $(1010)_{2}$.

On donne cette partie de programme. 
\begin{py}
\begin{minipage}[c]{.75\linewidth}
\begin{python}
res_cv = ''1010''
nb_bits = 8
while(len(res_cv)!=nb_bits):
    res_cv = "0"+res_cv
\end{python}
\end{minipage}
\end{py}

\subparagraph{}
\textit{Quel est le but du programme précédent ? Que contient \textsf{res\_cv} après l'exécution du code ?}

%%% Question 12
\vspace{.3cm}
\noindent\boxput*(-.85,1){
\colorbox{white}{\textbf{Question 12}}}{
\setlength{\fboxsep}{10pt}
\fbox{\begin{minipage}{.95\linewidth}
\usebox{\codebox}
\vspace{4cm}
\end{minipage}}}


\subsubsection{Inversion des bits}
On cherche maintenant à inverser les bits d'une séquence.

\begin{py}
\begin{minipage}[c]{.75\linewidth}
\begin{python}
res_cv = "1010"
res_inv = ""
for i in range(len(res_cv)):
    if res_cv[i]=="0":
        res_inv=res_inv+"0"
    else :
        res_inv=res_inv+"1"
\end{python}
\end{minipage}
\end{py}

\newpage

\subparagraph{}
\textit{Que contient \textsf{res\_inv} après l'exécution de la boucle ?}

%%% Question 13
\vspace{.3cm}
\noindent\boxput*(-.85,1){
\colorbox{white}{\textbf{Question 13}}}{
\setlength{\fboxsep}{10pt}
\fbox{\begin{minipage}{.95\linewidth}
\ifprof
\else
\usebox{\codebox}
\vspace{2cm}
\fi
\end{minipage}}}


\subparagraph{}
\textit{Si le résultat obtenu n'est pas le résultat attendu, comment modifier la séquence précédente ?}

%%% Question 14
\vspace{.3cm}
\noindent\boxput*(-.85,1){
\colorbox{white}{\textbf{Question 14}}}{
\setlength{\fboxsep}{10pt}
\fbox{\begin{minipage}{.95\linewidth}
\usebox{\codebox}
\vspace{2cm}
\end{minipage}}}


%\begin{center}
%\textbf{Fin des questions}
%\end{center}
\subsubsection{Additionner 1}
Voici une séquence de programme permettant d'ajouter 1 à un nombre codé en binaire.


\begin{py}
\begin{minipage}[c]{.75\linewidth}
\begin{python}
# On ajoute +1
# Initialisation
retenue="1"
res=""
for i in range(len(res_inv)-1,-1,-1):
    if retenue=="0" and res_inv[i]=="0":
        retenue=="0"
        res = "0"+res
    elif retenue=="0" and res_inv[i]=="1":
        retenue ="0"
        res = "1"+res
    elif retenue=="1" and res_inv[i]=="0":
        retenue ="0"
        res = "1"+res
    elif retenue=="1" and res_inv[i]=="1":
        retenue ="1"
        res = "0"+res
\end{python}
\end{minipage}

\end{py}
  
  \begin{lrbox}{\codebox}
\begin{python}
\end{python}
\end{lrbox}

  
\subparagraph{}
\textit{Quels sont les structures algorithmiques utilisées dans ce programme ? Explique l'existence des lignes 6, 9, 12 et 15.}

%%% Question 15
\vspace{.3cm}
\noindent\boxput*(-.85,1){
\colorbox{white}{\textbf{Question 15}}}{
\setlength{\fboxsep}{10pt}
\fbox{\begin{minipage}{.95\linewidth}
\usebox{\codebox}
\vspace{4cm}
\end{minipage}}}
  

\end{document}
