\documentclass[10pt]{article}
\input{style/coursHeadings}
\usepackage{algorithm}
\usepackage{algorithmic}


% Python sources
\usepackage{listings}
\usepackage{textcomp}
\usepackage{setspace}
%\usepackage{palatino}

%\usepackage{color}
\definecolor{Bleu}{rgb}{0.1,0.1,1.0}
\definecolor{Noir}{rgb}{0,0,0}
\definecolor{Grau}{rgb}{0.5,0.5,0.5}
\definecolor{DunkelGrau}{rgb}{0.15,0.15,0.15}
\definecolor{Hellbraun}{rgb}{0.5,0.25,0.0}
\definecolor{Magenta}{rgb}{1.0,0.0,1.0}
\definecolor{Gris}{gray}{0.5}
\definecolor{Vert}{rgb}{0,0.5,0}
\definecolor{SourceHintergrund}{rgb}{1,1.0,0.95}

%
\renewcommand{\lstlistlistingname}{Listings}
\renewcommand{\lstlistingname}{Listing}

\lstnewenvironment{python}[1][]{
\lstset{
language=python,
basicstyle=\ttfamily\footnotesize\setstretch{1}, 	
stringstyle=\color{red}, 
showstringspaces=false, 
alsoletter={1234567890},
otherkeywords={\ , \}, \{},
keywordstyle=\color{blue},
emph={access,and,break,class,continue,def,del,elif ,else,
except,exec,finally,for,from,global,if,import,in,i s,
lambda,not,or,pass,print,raise,return,try,while},
emphstyle=\color{black}\bfseries,
emph={[2]True, False, None, self},
emphstyle=[2]\color{green},
emph={[3]from, import, as},
emphstyle=[3]\color{blue},
upquote=true,
morecomment=[s]{"""}{"""},
commentstyle=\color{Hellbraun}\slshape, 
%emph={[4]1, 2, 3, 4, 5, 6, 7, 8, 9, 0},
emphstyle=[4]\color{blue},
literate=*{:}{{\textcolor{blue}:}}{1}
{=}{{\textcolor{blue}=}}{1}
{-}{{\textcolor{blue}-}}{1}
{+}{{\textcolor{blue}+}}{1}
{*}{{\textcolor{blue}*}}{1}
{!}{{\textcolor{blue}!}}{1}
{(}{{\textcolor{blue}(}}{1}
{)}{{\textcolor{blue})}}{1}
{[}{{\textcolor{blue}[}}{1}
{]}{{\textcolor{blue}]}}{1}
{<}{{\textcolor{blue}<}}{1}
{>}{{\textcolor{blue}>}}{1},
%framexleftmargin=1mm, framextopmargin=1mm, frame=shadowbox, rulesepcolor=\color{blue},#1
backgroundcolor=\color{SourceHintergrund}, 
framexleftmargin=1mm, framexrightmargin=1mm, framextopmargin=1mm, frame=single, framerule=1pt, rulecolor=\color{black},#1
}}{}
%%%%%%%%%%%%
% Définition des vecteurs 
%%%%%%%%%%%%
\newcommand{\vect}[1]{\overrightarrow{#1}}
\newcommand{\axe}[2]{\left(#1,\vect{#2}\right)}
\newcommand{\couple}[2]{\left(#1,\vect{#2}\right)}
\newcommand{\angl}[2]{\left(\vect{#1},\vect{#2}\right)}

\newcommand{\rep}[1]{\mathcal{R}_{#1}}
\newcommand{\quadruplet}[4]{\left(#1;#2,#3,#4 \right)}
\newcommand{\repere}[4]{\left(#1;\vect{#2},\vect{#3},\vect{#4} \right)}
\newcommand{\base}[3]{\left(\vect{#1},\vect{#2},\vect{#3} \right)}


\newcommand{\vx}[1]{\vect{x_{#1}}}
\newcommand{\vy}[1]{\vect{y_{#1}}}
\newcommand{\vz}[1]{\vect{z_{#1}}}

\newcommand{\norm}[1]{\ensuremath{\left\Vert {#1}\right\Vert}}
\newcommand{\Ker}{\mathop{\mathrm{Ker}}\nolimits}

% d droit pour le calcul différentiel
\newcommand{\dd}{\text{d}}

\newcommand{\inertie}[2]{I_{#1}\left( #2\right)}
\newcommand{\matinertie}[7]{
\begin{pmatrix}
#1 & #6 & #5 \\
#6 & #2 & #4 \\
#5 & #4 & #3 \\
\end{pmatrix}_{#7}}
%%%%%%%%%%%%
% Définition des torseurs 
%%%%%%%%%%%%

\newcommand{\ec}[2]{%
\mathcal{E}_c\left(#1/#2\right)}

\newcommand{\pext}[3]{%
\mathcal{P}\left(#1\rightarrow#2/#3\right)}

\newcommand{\pint}[3]{%
\mathcal{P}\left(#1 \stackrel{\text{#3}}{\leftrightarrow} #2\right)}


 \newcommand{\torseur}[1]{%
\left\{{#1}\right\}
}

\newcommand{\torseurcin}[3]{%
\left\{\mathcal{#1} \left(#2/#3 \right) \right\}
}

\newcommand{\torseurci}[2]{%
\left\{\sigma \left(#1/#2 \right) \right\}
}
\newcommand{\torseurdyn}[2]{%
\left\{\mathcal{D} \left(#1/#2 \right) \right\}
}


\newcommand{\torseurstat}[3]{%
\left\{\mathcal{#1} \left(#2\rightarrow #3 \right) \right\}
}


 \newcommand{\torseurc}[8]{%
%\left\{#1 \right\}=
\left\{
{#1}
\right\}
 = 
\left\{%
\begin{array}{cc}%
{#2} & {#5}\\%
{#3} & {#6}\\%
{#4} & {#7}\\%
\end{array}%
\right\}_{#8}%
}

 \newcommand{\torseurcol}[7]{
\left\{%
\begin{array}{cc}%
{#1} & {#4}\\%
{#2} & {#5}\\%
{#3} & {#6}\\%
\end{array}%
\right\}_{#7}%
}

 \newcommand{\torseurl}[3]{%
%\left\{\mathcal{#1}\right\}_{#2}=%
\left\{%
\begin{array}{l}%
{#1} \\%
{#2} %
\end{array}%
\right\}_{#3}%
}

% Vecteur vitesse
 \newcommand{\vectv}[3]{%
\vect{V\left( {#1} \in {#2}/{#3}\right)}
}

% Vecteur force
\newcommand{\vectf}[2]{%
\vect{R\left( {#1} \rightarrow {#2}\right)}
}

% Vecteur moment stat
\newcommand{\vectm}[3]{%
\vect{\mathcal{M}\left( {#1}, {#2} \rightarrow {#3}\right)}
}




% Vecteur résultante cin
\newcommand{\vectrc}[2]{%
\vect{R_c \left( {#1}/ {#2}\right)}
}
% Vecteur moment cin
\newcommand{\vectmc}[3]{%
\vect{\sigma \left( {#1}, {#2} /{#3}\right)}
}


% Vecteur résultante dyn
\newcommand{\vectrd}[2]{%
\vect{R_d \left( {#1}/ {#2}\right)}
}
% Vecteur moment dyn
\newcommand{\vectmd}[3]{%
\vect{\delta \left( {#1}, {#2} /{#3}\right)}
}

% Vecteur accélération
 \newcommand{\vectg}[3]{%
\vect{\Gamma \left( {#1} \in {#2}/{#3}\right)}
}

% Vecteur omega
 \newcommand{\vecto}[2]{%
\vect{\Omega\left( {#1}/{#2}\right)}
}
% }$$\left\{\mathcal{#1} \right\}_{#2} =%
% \left\{%
% \begin{array}{c}%
%  #3 \\%
%  #4 %
% \end{array}%
% \right\}_{#5}}

\newcommand{\N}{\mathbb{N}}
\newcommand{\Z}{\mathbb{Z}}
\newcommand{\R}{\mathbb{R}}
\newcommand{\C}{\mathbb{C}}
\newcommand{\K}{\mathbb{K}}

\newcommand{\cA}{\mathscr{A}}
\newcommand{\cM}{\mathscr{M}}
\newcommand{\cL}{\mathscr{L}}
\newcommand{\cS}{\mathscr{S}}

\newcommand{\python}{\texttt{Python}}

\newcommand{\z}[1]{\Z_{#1}}
\newcommand{\ztimes}[1]{\Z_{#1}^{\times}}
\newcommand{\ii}[1]{[\![#1[\![}
\newcommand{\iif}[1]{[\![#1]\!]}
\newcommand{\llbr}{\ensuremath{\llbracket}}
\newcommand{\rrbr}{\ensuremath{\rrbracket}}
%\newcommand{\p}[1]{\left(#1\right)}
\newcommand{\ens}[1]{\left\{ #1 \right\}}
\newcommand{\croch}[1]{\left[ #1 \right]}
%\newcommand{\of}[1]{\lstinline{#1}}
% \newcommand{\py}[2]{%
%   \begin{tabular}{|l}
%     \lstinline+>>>+\textbf{\of{#1}}\\
%     \of{#2}
%   \end{tabular}\par{}
% }
\newcommand{\floor}[1]{\left\lfloor#1\right\rfloor}
\newcommand{\ceil}[1]{\left\lceil#1\right\rceil}
\newcommand{\abs}[1]{\left|#1\right|}


% Binaire, octal, hexa
\newcommand{\hex}[1]{\underline{\text{\texttt{#1}}}_{16}}
\newcommand{\oct}[1]{\underline{\text{\texttt{#1}}}_{8}}
\newcommand{\bin}[1]{\underline{\text{\texttt{#1}}}_{2}}
\DeclareMathOperator{\mmod}{\texttt{\%}}


% Fonctions et systèmes
\newcommand{\fct}[5][t]{%
  \begin{array}[#1]{rcl}
    #2 & \rightarrow & #3\\
    #4 & \mapsto     & #5\\
  \end{array}
}
\newcommand{\fonction}[5]{#1 : \left\{\begin{array}{rcl} #2& \longrightarrow &#3 \\ #4 &\longmapsto & #5\end{array}\right.}
\newenvironment{systeme}{\left\{ \begin{array}{rcl}}{\end{array}\right.}

% Matrices
\newcommand{\mat}[1]{
  \begin{pmatrix}
    #1
  \end{pmatrix}
}
\newcommand{\inv}{\ensuremath{^{-1}}}
\newcommand{\bpm}{\begin{pmatrix}}
\newcommand{\epm}{\end{pmatrix}}


% bases de données
\newcommand{\relat}[1]{\textsc{#1}}
\newcommand{\attr}[1]{\emph{#1}}
\newcommand{\prim}[1]{\uline{#1}}
\newcommand{\foreign}[1]{\#\textsl{#1}}


% Bases de données

\newcommand{\att}{\ensuremath{\mathbf{att}}}
\newcommand{\dom}{\ensuremath{\mathbf{dom}}}
\newcommand{\sort}{\ensuremath{\mathbf{sort}}}
\newcommand{\relname}{\ensuremath{\mathbf{relname}}}
\newcommand{\var}{\ensuremath{\mathbf{var}}}
\newcommand{\FILM}{\ensuremath{\mathtt{FILM}}}
\newcommand{\JOUE}{\ensuremath{\mathtt{JOUE}}}
\newcommand{\PERSONNE}{\ensuremath{\mathtt{PERSONNE}}}
\newcommand{\PERSONNAGE}{\ensuremath{\mathtt{PERSONNAGE}}}

\newcommand{\ttid}{\ensuremath{\mathtt{id}}}
\newcommand{\tttitre}{\ensuremath{\mathtt{titre}}}
\newcommand{\ttdate}{\ensuremath{\mathtt{date}}}
\newcommand{\ttidr}{\ensuremath{\mathtt{idrealisateur}}}
\newcommand{\ttdatenais}{\ensuremath{\mathtt{datenaissance}}}
\newcommand{\ttnom}{\ensuremath{\mathtt{nom}}}
\newcommand{\ttprenom}{\ensuremath{\mathtt{prenom}}}
\newcommand{\ttidacteur}{\ensuremath{\mathtt{idacteur}}}
\newcommand{\ttidfilm}{\ensuremath{\mathtt{idfilm}}}
\newcommand{\ttidpersonnage}{\ensuremath{\mathtt{idpersonnage}}}

\newcommand{\fv}{\mathrm{libre}}
\newcommand{\sem}[1]{[\![ #1 ]\!]}

\input{style/macros_Titres}
\input{style/macros_Frames}
\usepackage{fancybox}
\newsavebox{\codebox}
\newsavebox{\codeboxx}

%Si le boolen xp est vrai : compilation pour xabi
%Sinon compilation Damien
\newboolean{xp}
\setboolean{xp}{true}

\newboolean{prof}
\setboolean{prof}{false}


% Commenter \proffalse et décommenter \proftrue pour avoir le corrigé
\newif\ifprof
%\proftrue
\proffalse

\newif\iftd
\tdtrue
%\tdfalse

\usepackage[%
    pdftitle={Devoirs Surveillé 3},
    pdfauthor={Xavier Pessoles},
    colorlinks=true,
    linkcolor=blue,
    citecolor=magenta]{hyperref}


\def\discipline{Informatique}
\def\xxtitre{\ifthenelse{\boolean{xp}}{
Devoir surveillé d'informatique 3}{
Chapitre  -- }}

\def\xxsoustitre{\ifthenelse{\boolean{xp}}{
CI 2 : Algorithmique et programmation 

\vspace{.5cm}

Tracé de l'abaque du temps de réponse réduit}{
Partie  -- }}

\def\xxauteur{\ifthenelse{\boolean{xp}}{
Xavier \textsc{Pessoles}}{
}}

\def\xxpied{\ifthenelse{\boolean{xp}}{
DS Informatique\\
\ifprof Corrige \else Sujet \fi}{
\xxtitre}}

\def\xxcathegorie{\ifthenelse{\boolean{xp}}{
2013 -- 2014 \\
Xavier \textsc{Pessoles}}{
Informatique - Cours}}





%---------------------------------------------------------------------------


\begin{document}

\ifthenelse{\boolean{xp}}{\usepackage[%
    pdftitle={Représentation des nombres},
    pdfauthor={Xavier Pessoles},
    colorlinks=true,
    linkcolor=blue,
    citecolor=magenta]{hyperref}

\usepackage{pifont}
%\usepackage{lastpage}

% \makeatletter \let\ps@plain\ps@empty \makeatother
%% DEBUT DU DOCUMENT
%% =================
\sloppy
\hyphenpenalty 10000


\colorlet{shadecolor}{orange!15}

\newtheorem{theorem}{Theorem}


\begin{document}


%\newboolean{prof}
%\setboolean{prof}{true}
% \makeatletter \let\ps@plain\ps@empty \makeatother
%% DEBUT DU DOCUMENT
%% =================




%------------- En tetes et Pieds de Pages ------------


\pagestyle{fancy}
\ifthenelse{\boolean{xp}}{%
\renewcommand{\headrulewidth}{0pt}}{%
\renewcommand{\headrulewidth}{0.2pt}} %pour mettre le trait en haut
%\renewcommand{\headrulewidth}{0.2pt}

\fancyhead{}
\fancyhead[L]{%
\noindent\begin{minipage}[c]{2.6cm}%
\includegraphics[width=2cm]{png/logo_ptsi.png}%
\end{minipage}}


\fancyhead[C]{\rule{12cm}{.5pt}}



\fancyhead[R]{%
\noindent\begin{minipage}[c]{3cm}
\begin{flushright}
\footnotesize{\textit{\textsf{Informatique}}}%
\end{flushright}
\end{minipage}
}



\fancyhead[C]{\rule{12cm}{.5pt}}

\renewcommand{\footrulewidth}{0.2pt}

\fancyfoot[C]{\footnotesize{\bfseries \thepage}}
\fancyfoot[L]{%
\begin{minipage}[c]{.2\linewidth}
\noindent\footnotesize{{\xxauteur}}
\end{minipage}
\ifthenelse{\boolean{xp}}{}{%
\begin{minipage}[c]{.15\linewidth}
\includegraphics[width=2cm]{png/logoCC.png}
\end{minipage}}
}

\ifthenelse{\boolean{prof}}{%
\fancyfoot[R]{\footnotesize{\xxpied}}}

\begin{center}
 \huge\textsc{\xxtitre}
\end{center}

\begin{center}
 \LARGE\textsc{\xxsoustitre}
\end{center}

\vspace{.5cm}
}{\input{style/enteteDI}}


\ifprof
\begin{center}
\large{\textit{Éléments de corrigé}}
\end{center}
\else
\begin{flushright}
\large{\textsl{Nom : .......................................}}
\end{flushright}
\fi

\vspace{1cm}

\begin{obj}

L'objectif de ces travaux est de construire le programme permettant de tracer l'abaque du temps de réponse réduit utilisé en asservissement pour connaître le temps de réponse à 5\% des systèmes d'ordre 2. 
\end{obj}

\subsection*{Mise en situation}

L'équation différentielle d'un système du second ordre peut se mettre sous la forme :

\vspace{.25cm}

\begin{minipage}[c]{.48\linewidth}
$$
s(t)
+\dfrac{2\xi}{\omega_0}\cdot \dfrac{\text{d}s(t)}{\text{d}t}
+\dfrac{1}{\omega_0^2}\cdot \dfrac{\text{d}^2s(t)}{\text{d}t^2}
= K\cdot e(t)
$$
\end{minipage}\hfill
\begin{minipage}[c]{.48\linewidth}
en notant :
\begin{itemize}
\item $K$ : le gain statique;
\item $\xi$ : le coefficient d'amortissement;
\item $e(t)$ et $s(t)$ : l'entrée et la sortie du système. 
\end{itemize}
\end{minipage}

\vspace{.5cm}

On suppose que toutes les conditions initiales sont nulles. Pour une entrée unitaire de type échelon unitaire $e(t)=u(t)$, $K=1$ et $t\geq0$ on montre que : 
\begin{itemize}
\item si $\xi <1$, le régime est pseudo périodique et :
$$
s(t)=1-\dfrac{e^{-\xi\omega_0 t}}{\sqrt{1-\xi^2}}\sin\left(  \omega_0 t\sqrt{1-\xi^2}+\arcsin \sqrt{1-\xi^2} \right)
$$
\item si $\xi=1$, le régime est critique et : 
$$
s(t)=1-\left(1+\omega_0 t \right)e^{-\omega_0 t} 
$$
\item si $\xi>1$, le régime est apériodique et : 
$$
s(t)=1
+\dfrac{e^{- \omega_0 t\left( \xi + \sqrt{\xi^2-1}\right)}}{2\left(\xi\sqrt{\xi^2-1}+\xi^2-1 \right)}
-\dfrac{e^{- \omega_0 t \left( \xi - \sqrt{\xi^2-1}\right)}}{2\left(\xi\sqrt{\xi^2-1}-\xi^2+1 \right)}
$$
\end{itemize} 

\begin{center}
\textbf{Dans l'ensemble de ce sujet, on considèrera que $s$ est une fonction du temps réduit $t\cdot\omega_0$.}
\end{center}


\subsection*{Tracé de la réponse indicielle}

On dispose des fonctions Python \textbf{\textsf{f\_pseudo}} et \textbf{\textsf{f\_aperiodique}} permettant d'évaluer la fonction pour $s$ pour un couple $(t\omega_0,\xi)$.

\subparagraph{} \textit{Donner, en Python, le contenu de la fonction \textsf{f\_critique} permettant de définir la fonction $(t\omega_0) \rightarrow s(t\omega_0)$ dans le cas où $\xi=1$.}


\subparagraph{} \textit{Donner, en Python, le contenu de la fonction \textsf{f\_s} permettant de définir la fonction $(t\omega_0,\xi) \rightarrow s(t\omega_0\xi)$ dans le cas où $\xi\in \mathbb{R}_+^*$. On donne ci-dessous les spécifications de la fonction.}

\begin{py}
\begin{python}
def f_s(tom0,z):
    """
    Fonction permettant de calculer la réponse indicielle d'un système du second ordre. 
    Entrées : 
        * tom0, flt : temps de réponse réduit
        * z, flt : coefficient d'amortissement
    Sortie : 
        * s(tom0,z)
    """
\end{python}
\end{py}


La fonction \textbf{\textsf{trace\_s}} donnée ci-dessous permet de tracer $s(t\omega_0,\xi)$ pour $t\omega_0 \in [0,10]$ par pas de 1 et pour une valeur de $\xi$ déterminée. Les deux appels successifs de la fonction \textbf{\textsf{trace\_s}} permettent de réaliser le tracer les 2 courbes ci-dessous.


\begin{minipage}[c]{.48\linewidth}
\begin{py}
\begin{python}
# Définition de la fonction trace
def trace_s(z):
    x = []
    y = []
    for i in range(11):
        t = i
        x.append(t)
        y.append(f_s(t,z))
    plot(x,y)
# Appels de la fonction trace
trace_s(0.4)
trace_s(0.7)

\end{python}
\end{py}
\end{minipage} \hfill
\begin{minipage}[c]{.48\linewidth}
\begin{center}
\includegraphics[width=\textwidth]{images/courbe}
\end{center}
\end{minipage}

\subparagraph{}
\textit{Expliquer l'objectif des lignes 2 à 9.}

\vspace{.25cm}

On observe que la courbe tracée n'est pas lissée. Pour avoir un meilleur rendu, il est nécessaire d'évaluer la fonction en davantage de points. 

\subparagraph{}
\textit{Modifier les lignes 5 et 6 pour que la courbe tracée soit réalisée en 1000 points sur un intervalle de $t \omega_0$ variant de 0 à 10. }

\ifprof
\begin{corrige}
\begin{py}
\begin{python}
def trace_s(z):
    x = []
    y = []
    n = 1000
    for i in range(n+1):
        t = 10*i/n
        x.append(t)
        y.append(f_s(t,z))
    plot(x,y)
\end{python}
\end{py}
\end{corrige}
\else
\fi

\subsection*{Tracé de l'abaque}

On note $t_r$  le temps de réponse à 5\%. L'abaque du temps de réponse permet de tracer le produit $t_r\omega_0$ en fonction du coefficient d'amortissement $\xi$.

\subparagraph{}
\textit{Dans les conditions de la fonction $s$ définie dans la partie précédente, quelle est la valeur finale prise par $s(t)$ ?} 

\subparagraph{}
\textit{Écrire en Python la fonction \textsf{\textbf{is\_in\_strip}} ayant les spécifications suivantes : } 

\begin{py}
\begin{python}
def is_in_strip(x):
    """
    Fonction permettant de savoir si une valeur est dans la bande des + ou - 5% de la valeur finale.
    Entrée : 
        x, flt : réel
    Sortie : 
        True si la valeur est dans la bande à + ou - 5%
        False si la valeur n'est pas dans la bande à + ou - 5%
    """
\end{python}
\end{py}

On donne la fonction suivante permettant de connaître le temps de réponse réduit à partir duquel la réponse indicielle d'un système est dans la bande à plus ou moins 5\%.

\begin{py}
\begin{python}
def f(z):
    tom0  = 500
    pas_tom0  = 0.05
    x = f_s(tom0,z) 
    while is_in_strip(x) :
        x = f_s(tom0,z)
        tom0  = tom0 - pas_tom0
    return tom0
\end{python}
\end{py}

\end{document}
