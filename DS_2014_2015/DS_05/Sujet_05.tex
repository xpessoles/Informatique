\documentclass[10pt]{article}
\input{style/coursHeadings}
\usepackage{algorithm}
\usepackage{algorithmic}


% Python sources
\usepackage{listings}
\usepackage{textcomp}
\usepackage{setspace}
%\usepackage{palatino}

%\usepackage{color}
\definecolor{Bleu}{rgb}{0.1,0.1,1.0}
\definecolor{Noir}{rgb}{0,0,0}
\definecolor{Grau}{rgb}{0.5,0.5,0.5}
\definecolor{DunkelGrau}{rgb}{0.15,0.15,0.15}
\definecolor{Hellbraun}{rgb}{0.5,0.25,0.0}
\definecolor{Magenta}{rgb}{1.0,0.0,1.0}
\definecolor{Gris}{gray}{0.5}
\definecolor{Vert}{rgb}{0,0.5,0}
\definecolor{SourceHintergrund}{rgb}{1,1.0,0.95}

%
\renewcommand{\lstlistlistingname}{Listings}
\renewcommand{\lstlistingname}{Listing}

\lstnewenvironment{python}[1][]{
\lstset{
language=python,
basicstyle=\ttfamily\footnotesize\setstretch{1}, 	
stringstyle=\color{red}, 
showstringspaces=false, 
alsoletter={1234567890},
otherkeywords={\ , \}, \{},
keywordstyle=\color{blue},
emph={access,and,break,class,continue,def,del,elif ,else,
except,exec,finally,for,from,global,if,import,in,i s,
lambda,not,or,pass,print,raise,return,try,while},
emphstyle=\color{black}\bfseries,
emph={[2]True, False, None, self},
emphstyle=[2]\color{green},
emph={[3]from, import, as},
emphstyle=[3]\color{blue},
upquote=true,
morecomment=[s]{"""}{"""},
commentstyle=\color{Hellbraun}\slshape, 
%emph={[4]1, 2, 3, 4, 5, 6, 7, 8, 9, 0},
emphstyle=[4]\color{blue},
literate=*{:}{{\textcolor{blue}:}}{1}
{=}{{\textcolor{blue}=}}{1}
{-}{{\textcolor{blue}-}}{1}
{+}{{\textcolor{blue}+}}{1}
{*}{{\textcolor{blue}*}}{1}
{!}{{\textcolor{blue}!}}{1}
{(}{{\textcolor{blue}(}}{1}
{)}{{\textcolor{blue})}}{1}
{[}{{\textcolor{blue}[}}{1}
{]}{{\textcolor{blue}]}}{1}
{<}{{\textcolor{blue}<}}{1}
{>}{{\textcolor{blue}>}}{1},
%framexleftmargin=1mm, framextopmargin=1mm, frame=shadowbox, rulesepcolor=\color{blue},#1
backgroundcolor=\color{SourceHintergrund}, 
framexleftmargin=1mm, framexrightmargin=1mm, framextopmargin=1mm, frame=single, framerule=1pt, rulecolor=\color{black},#1
}}{}
%%%%%%%%%%%%
% Définition des vecteurs 
%%%%%%%%%%%%
\newcommand{\vect}[1]{\overrightarrow{#1}}
\newcommand{\axe}[2]{\left(#1,\vect{#2}\right)}
\newcommand{\couple}[2]{\left(#1,\vect{#2}\right)}
\newcommand{\angl}[2]{\left(\vect{#1},\vect{#2}\right)}

\newcommand{\rep}[1]{\mathcal{R}_{#1}}
\newcommand{\quadruplet}[4]{\left(#1;#2,#3,#4 \right)}
\newcommand{\repere}[4]{\left(#1;\vect{#2},\vect{#3},\vect{#4} \right)}
\newcommand{\base}[3]{\left(\vect{#1},\vect{#2},\vect{#3} \right)}


\newcommand{\vx}[1]{\vect{x_{#1}}}
\newcommand{\vy}[1]{\vect{y_{#1}}}
\newcommand{\vz}[1]{\vect{z_{#1}}}

\newcommand{\norm}[1]{\ensuremath{\left\Vert {#1}\right\Vert}}
\newcommand{\Ker}{\mathop{\mathrm{Ker}}\nolimits}

% d droit pour le calcul différentiel
\newcommand{\dd}{\text{d}}

\newcommand{\inertie}[2]{I_{#1}\left( #2\right)}
\newcommand{\matinertie}[7]{
\begin{pmatrix}
#1 & #6 & #5 \\
#6 & #2 & #4 \\
#5 & #4 & #3 \\
\end{pmatrix}_{#7}}
%%%%%%%%%%%%
% Définition des torseurs 
%%%%%%%%%%%%

\newcommand{\ec}[2]{%
\mathcal{E}_c\left(#1/#2\right)}

\newcommand{\pext}[3]{%
\mathcal{P}\left(#1\rightarrow#2/#3\right)}

\newcommand{\pint}[3]{%
\mathcal{P}\left(#1 \stackrel{\text{#3}}{\leftrightarrow} #2\right)}


 \newcommand{\torseur}[1]{%
\left\{{#1}\right\}
}

\newcommand{\torseurcin}[3]{%
\left\{\mathcal{#1} \left(#2/#3 \right) \right\}
}

\newcommand{\torseurci}[2]{%
\left\{\sigma \left(#1/#2 \right) \right\}
}
\newcommand{\torseurdyn}[2]{%
\left\{\mathcal{D} \left(#1/#2 \right) \right\}
}


\newcommand{\torseurstat}[3]{%
\left\{\mathcal{#1} \left(#2\rightarrow #3 \right) \right\}
}


 \newcommand{\torseurc}[8]{%
%\left\{#1 \right\}=
\left\{
{#1}
\right\}
 = 
\left\{%
\begin{array}{cc}%
{#2} & {#5}\\%
{#3} & {#6}\\%
{#4} & {#7}\\%
\end{array}%
\right\}_{#8}%
}

 \newcommand{\torseurcol}[7]{
\left\{%
\begin{array}{cc}%
{#1} & {#4}\\%
{#2} & {#5}\\%
{#3} & {#6}\\%
\end{array}%
\right\}_{#7}%
}

 \newcommand{\torseurl}[3]{%
%\left\{\mathcal{#1}\right\}_{#2}=%
\left\{%
\begin{array}{l}%
{#1} \\%
{#2} %
\end{array}%
\right\}_{#3}%
}

% Vecteur vitesse
 \newcommand{\vectv}[3]{%
\vect{V\left( {#1} \in {#2}/{#3}\right)}
}

% Vecteur force
\newcommand{\vectf}[2]{%
\vect{R\left( {#1} \rightarrow {#2}\right)}
}

% Vecteur moment stat
\newcommand{\vectm}[3]{%
\vect{\mathcal{M}\left( {#1}, {#2} \rightarrow {#3}\right)}
}




% Vecteur résultante cin
\newcommand{\vectrc}[2]{%
\vect{R_c \left( {#1}/ {#2}\right)}
}
% Vecteur moment cin
\newcommand{\vectmc}[3]{%
\vect{\sigma \left( {#1}, {#2} /{#3}\right)}
}


% Vecteur résultante dyn
\newcommand{\vectrd}[2]{%
\vect{R_d \left( {#1}/ {#2}\right)}
}
% Vecteur moment dyn
\newcommand{\vectmd}[3]{%
\vect{\delta \left( {#1}, {#2} /{#3}\right)}
}

% Vecteur accélération
 \newcommand{\vectg}[3]{%
\vect{\Gamma \left( {#1} \in {#2}/{#3}\right)}
}

% Vecteur omega
 \newcommand{\vecto}[2]{%
\vect{\Omega\left( {#1}/{#2}\right)}
}
% }$$\left\{\mathcal{#1} \right\}_{#2} =%
% \left\{%
% \begin{array}{c}%
%  #3 \\%
%  #4 %
% \end{array}%
% \right\}_{#5}}

\newcommand{\N}{\mathbb{N}}
\newcommand{\Z}{\mathbb{Z}}
\newcommand{\R}{\mathbb{R}}
\newcommand{\C}{\mathbb{C}}
\newcommand{\K}{\mathbb{K}}

\newcommand{\cA}{\mathscr{A}}
\newcommand{\cM}{\mathscr{M}}
\newcommand{\cL}{\mathscr{L}}
\newcommand{\cS}{\mathscr{S}}

\newcommand{\python}{\texttt{Python}}

\newcommand{\z}[1]{\Z_{#1}}
\newcommand{\ztimes}[1]{\Z_{#1}^{\times}}
\newcommand{\ii}[1]{[\![#1[\![}
\newcommand{\iif}[1]{[\![#1]\!]}
\newcommand{\llbr}{\ensuremath{\llbracket}}
\newcommand{\rrbr}{\ensuremath{\rrbracket}}
%\newcommand{\p}[1]{\left(#1\right)}
\newcommand{\ens}[1]{\left\{ #1 \right\}}
\newcommand{\croch}[1]{\left[ #1 \right]}
%\newcommand{\of}[1]{\lstinline{#1}}
% \newcommand{\py}[2]{%
%   \begin{tabular}{|l}
%     \lstinline+>>>+\textbf{\of{#1}}\\
%     \of{#2}
%   \end{tabular}\par{}
% }
\newcommand{\floor}[1]{\left\lfloor#1\right\rfloor}
\newcommand{\ceil}[1]{\left\lceil#1\right\rceil}
\newcommand{\abs}[1]{\left|#1\right|}


% Binaire, octal, hexa
\newcommand{\hex}[1]{\underline{\text{\texttt{#1}}}_{16}}
\newcommand{\oct}[1]{\underline{\text{\texttt{#1}}}_{8}}
\newcommand{\bin}[1]{\underline{\text{\texttt{#1}}}_{2}}
\DeclareMathOperator{\mmod}{\texttt{\%}}


% Fonctions et systèmes
\newcommand{\fct}[5][t]{%
  \begin{array}[#1]{rcl}
    #2 & \rightarrow & #3\\
    #4 & \mapsto     & #5\\
  \end{array}
}
\newcommand{\fonction}[5]{#1 : \left\{\begin{array}{rcl} #2& \longrightarrow &#3 \\ #4 &\longmapsto & #5\end{array}\right.}
\newenvironment{systeme}{\left\{ \begin{array}{rcl}}{\end{array}\right.}

% Matrices
\newcommand{\mat}[1]{
  \begin{pmatrix}
    #1
  \end{pmatrix}
}
\newcommand{\inv}{\ensuremath{^{-1}}}
\newcommand{\bpm}{\begin{pmatrix}}
\newcommand{\epm}{\end{pmatrix}}


% bases de données
\newcommand{\relat}[1]{\textsc{#1}}
\newcommand{\attr}[1]{\emph{#1}}
\newcommand{\prim}[1]{\uline{#1}}
\newcommand{\foreign}[1]{\#\textsl{#1}}


% Bases de données

\newcommand{\att}{\ensuremath{\mathbf{att}}}
\newcommand{\dom}{\ensuremath{\mathbf{dom}}}
\newcommand{\sort}{\ensuremath{\mathbf{sort}}}
\newcommand{\relname}{\ensuremath{\mathbf{relname}}}
\newcommand{\var}{\ensuremath{\mathbf{var}}}
\newcommand{\FILM}{\ensuremath{\mathtt{FILM}}}
\newcommand{\JOUE}{\ensuremath{\mathtt{JOUE}}}
\newcommand{\PERSONNE}{\ensuremath{\mathtt{PERSONNE}}}
\newcommand{\PERSONNAGE}{\ensuremath{\mathtt{PERSONNAGE}}}

\newcommand{\ttid}{\ensuremath{\mathtt{id}}}
\newcommand{\tttitre}{\ensuremath{\mathtt{titre}}}
\newcommand{\ttdate}{\ensuremath{\mathtt{date}}}
\newcommand{\ttidr}{\ensuremath{\mathtt{idrealisateur}}}
\newcommand{\ttdatenais}{\ensuremath{\mathtt{datenaissance}}}
\newcommand{\ttnom}{\ensuremath{\mathtt{nom}}}
\newcommand{\ttprenom}{\ensuremath{\mathtt{prenom}}}
\newcommand{\ttidacteur}{\ensuremath{\mathtt{idacteur}}}
\newcommand{\ttidfilm}{\ensuremath{\mathtt{idfilm}}}
\newcommand{\ttidpersonnage}{\ensuremath{\mathtt{idpersonnage}}}

\newcommand{\fv}{\mathrm{libre}}
\newcommand{\sem}[1]{[\![ #1 ]\!]}

\input{style/macros_Titres}
\input{style/macros_Frames}

%Si le boolen xp est vrai : compilation pour xabi
%Sinon compilation Damien
\newif\ifprof
\proftrue
%\proffalse

\newif\ifxp
\xptrue
%\xpfalse

\newif\iftd
\tdtrue
%\tdfalse

\usepackage[%
    pdftitle={DS Informatique - Concours Blanc},
    pdfauthor={Xavier Pessoles},
    colorlinks=true,
    linkcolor=blue,
    citecolor=magenta]{hyperref}

\def\discipline{Informatique}
\def\xxtitre{%
\ifxp
Concours Blanc : Informatique
\else
\fi
}

\def\xxsoustitre{%
\ifxp
Autour de données météorologiques
\else
\fi}

\def\xxauteur{%
\ifxp
Xavier \textsc{Pessoles} \\
Adapté de sujet 0 -- CCP PSI
\else
\fi}

\def\xxpied{%
\ifxp
Concours Blanc -- Informatique \\
Simulation des vibrations -- \ifprof P \else E \fi
\else
\fi}




%---------------------------------------------------------------------------


\begin{document}
\ifxp
\usepackage[%
    pdftitle={Représentation des nombres},
    pdfauthor={Xavier Pessoles},
    colorlinks=true,
    linkcolor=blue,
    citecolor=magenta]{hyperref}

\usepackage{pifont}
%\usepackage{lastpage}

% \makeatletter \let\ps@plain\ps@empty \makeatother
%% DEBUT DU DOCUMENT
%% =================
\sloppy
\hyphenpenalty 10000


\colorlet{shadecolor}{orange!15}

\newtheorem{theorem}{Theorem}


\begin{document}


%\newboolean{prof}
%\setboolean{prof}{true}
% \makeatletter \let\ps@plain\ps@empty \makeatother
%% DEBUT DU DOCUMENT
%% =================




%------------- En tetes et Pieds de Pages ------------


\pagestyle{fancy}
\ifthenelse{\boolean{xp}}{%
\renewcommand{\headrulewidth}{0pt}}{%
\renewcommand{\headrulewidth}{0.2pt}} %pour mettre le trait en haut
%\renewcommand{\headrulewidth}{0.2pt}

\fancyhead{}
\fancyhead[L]{%
\noindent\begin{minipage}[c]{2.6cm}%
\includegraphics[width=2cm]{png/logo_ptsi.png}%
\end{minipage}}


\fancyhead[C]{\rule{12cm}{.5pt}}



\fancyhead[R]{%
\noindent\begin{minipage}[c]{3cm}
\begin{flushright}
\footnotesize{\textit{\textsf{Informatique}}}%
\end{flushright}
\end{minipage}
}



\fancyhead[C]{\rule{12cm}{.5pt}}

\renewcommand{\footrulewidth}{0.2pt}

\fancyfoot[C]{\footnotesize{\bfseries \thepage}}
\fancyfoot[L]{%
\begin{minipage}[c]{.2\linewidth}
\noindent\footnotesize{{\xxauteur}}
\end{minipage}
\ifthenelse{\boolean{xp}}{}{%
\begin{minipage}[c]{.15\linewidth}
\includegraphics[width=2cm]{png/logoCC.png}
\end{minipage}}
}

\ifthenelse{\boolean{prof}}{%
\fancyfoot[R]{\footnotesize{\xxpied}}}

\begin{center}
 \huge\textsc{\xxtitre}
\end{center}

\begin{center}
 \LARGE\textsc{\xxsoustitre}
\end{center}

\vspace{.5cm}

\else
\input{style/enteteDI}
\fi



 \renewcommand{\baselinestretch}{1.2}
\setlength{\parskip}{2ex plus 0.5ex minus 0.2ex}



\section{Mise en situation}
La structure peut être modélisée par $n$ éléments de masse $m_i$ ($i$ variant de 1 à $n$) reliés  par des liaisons visco-élastiques eux-mêmes modélisés par des un ressort de raideur $k$ en parallèle d'un élément d'amortissement $c$. La structure est supposée unidimensionnelle de longueur $L$. Le nombre d'éléments peut être de l'ordre de plusieurs milliers. 

Le déplacement au cours du temps de l'élément $i$ autour de sa position d'équlibre est noté $u_i(t)$. Une force $f_n(t)$ est appliquée sur l'élément $n$ uniquement. L'extrémité gauche de
la structure est bloquée. Les effets de la pesanteur sont négligés.

Le théorème de la résultante dynamique appliqué à un élément $i$ ($i$ variant de 2 à $n-1$ inclus) s'écrit sous la forme : 
%$$
%m_i\dfrac{\text{d}u_i(t)}{\text{d}t} = 
%-k_i\left(u_i(t)-u_{i-1}(t) \right)   - k_{i+1}\left(u_i(t)-u_{i+1}(t) \right) 
%-c_i\left(u_i(t)-u_{i-1}(t) \right)   - c_{i+1}\left(u_i(t)-u_{i+1}(t) \right) 
%$$

%$$
%m\dfrac{\text{d}^2u_i(t)}{\text{d}t^2} = 
%- k\left(u_i(t)-u_{i-1}(t) \right)   
%- k\left(u_i(t)-u_{i+1}(t) \right) 
%- c\dfrac{\text{d}}{\text{d}t} \left[u_i(t)-u_{i-1}(t) \right]
%- c\dfrac{\text{d}}{\text{d}t} \left[u_i(t)-u_{i+1}(t) \right]
%$$


$$
m\dfrac{\text{d}^2u_i(t)}{\text{d}t^2} = 
- k\left(u_i(t)-u_{i-1}(t) \right)   
- k\left(u_i(t)-u_{i+1}(t) \right) 
- c\left(\dfrac{\text{d}u_i(t)}{\text{d}t}-\dfrac{\text{d}u_{i-1}}{\text{d}t} \right)   
- c\left(\dfrac{\text{d}u_i(t)}{\text{d}t}-\dfrac{\text{d}u_{i+1}}{\text{d}t} \right) 
$$

Le théorème de la résultante dynamique appliqué aux éléments 1 et $n$ donne les équations suivantes : 
%$$
%m_1\dfrac{\text{d}u_1(t)}{\text{d}t} = 
%-\left(k_1 + k_2\right) u_1(t) + k_2 u_2 (t) - \left(c_1 + c_2 \right)u_1(t) - c_2 u_2 (t)
%$$
%
%$$
%m_n\dfrac{\text{d}u_n(t)}{\text{d}t} = 
%-k_n\left(u_n(t)-u_{n-1}(t) \right)   - c_n\left(u_n(t)-u_{n-1}(t) \right) +f_n(t)
%$$


$$
m\dfrac{\text{d}^2u_i(t)}{\text{d}t^2} = 
- 2k u_1(t) + k u_2 (t) 
-2c \dfrac{\text{d}u_1(t)}{\text{d}t} 
- c \dfrac{\text{d}u_2(t)}{\text{d}t} 
$$

$$
m\dfrac{\text{d}^2u_n(t)}{\text{d}t^2} = 
-k\left(u_n(t)-u_{n-1}(t) \right)   - c\left(\dfrac{\text{d}u_{n}}{\text{d}t}-\dfrac{\text{d}u_{n-1}}{\text{d}t} \right) +f_n(t)
$$


Dans toute la suite, on imposera $f_n(t)=f_{max} \sin \omega \left(t\right)$.

\section{Résolution d'une équation différentielle}
On s'intéresse tout d'abord à un seul système masse ressort. L'application du théorème de la résultante dynamique en projection sur l'axe de déplacement s'écrit donc 
de la manière suivante : 
$$
m\cdot\ddot{u}(t)+c\cdot \dot{u}(t) + k\cdot u(t) = f(t)
$$

On précise que pour $t\leq 0$, $u(t)=0$, $\dot{u}(t)=0$ et $\ddot{u}(t)=0$.

On pose $\left\{ \begin{array}{l} x(t) = u(t) \\ v(t) = \dot{x}(t)\end{array} \right.$.

\subparagraph{}
\textit{Réécrire l'équation différentielle sous forme d'un système d'équation en fonction de $x(t)$,  $v(t)$, $\dot{x}(t)$ et $\dot{v}(t)$}
\ifprof
\begin{corrige}
On a donc : 
$$
\left\{ 
\begin{array}{l} 
v(t) = \dot{x}(t) \\ 
m\cdot\dot{v}(t)+c\cdot {v}(t) + k\cdot x(t) = f(t)
\end{array} \right.
$$
\end{corrige}
\else
\fi

\subparagraph{}
\textit{En utilisant un schéma d'Euler explicite et l'équation $v(t) = \dot{x}(t)$ 
exprimer la suite $x_{k+1}$ en fonction de $x_k$, $v_k$ et le pas de calcul noté noté $h$.}

\ifprof
\begin{corrige}
On a $\dfrac{dx(t)}{dt} \simeq \dfrac{x_{k+1}-x_k}{h}$. On a donc 
$v_k = \dfrac{x_{k+1}-x_k}{h} \Longleftrightarrow x_{k+1} = h\cdot v_k + x_k$.
\end{corrige}
\else
\fi

\subparagraph{}
\textit{En utilisant un schéma d'Euler explicite exprimer alors la suite $v_{k+1}$}

\ifprof
\begin{corrige}
On a $\dfrac{dv(t)}{dt} \simeq \dfrac{v_{k+1}-v_k}{h}$.  En conséquences, 
$m\cdot\dot{v}(t)+c\cdot {v}(t) + k\cdot x(t) = f(t)$
peut donc se mettre sous la forme suivante :

$$
m\cdot\dfrac{v_{k+1}-v_k}{h}+c\cdot {v_k} + k\cdot x_k = f_k
\Longleftrightarrow 
m\cdot {v_{k+1}}  = f_k h - kh\cdot x_k - c h\cdot {v_k}+v_k
\Longleftrightarrow
 {v_{k+1}}  = \dfrac{h}{m} f_k - \dfrac{kh}{m}x_k +\dfrac{1-ch}{m}v_k
$$
\end{corrige}
\else
\fi





\end{document}

\ifprof
\begin{corrige}
\end{corrige}
\else
\fi

