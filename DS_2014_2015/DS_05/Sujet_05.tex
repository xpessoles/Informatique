\documentclass[10pt]{article}
\input{style/coursHeadings}
\usepackage{algorithm}
\usepackage{algorithmic}


% Python sources
\usepackage{listings}
\usepackage{textcomp}
\usepackage{setspace}
%\usepackage{palatino}

%\usepackage{color}
\definecolor{Bleu}{rgb}{0.1,0.1,1.0}
\definecolor{Noir}{rgb}{0,0,0}
\definecolor{Grau}{rgb}{0.5,0.5,0.5}
\definecolor{DunkelGrau}{rgb}{0.15,0.15,0.15}
\definecolor{Hellbraun}{rgb}{0.5,0.25,0.0}
\definecolor{Magenta}{rgb}{1.0,0.0,1.0}
\definecolor{Gris}{gray}{0.5}
\definecolor{Vert}{rgb}{0,0.5,0}
\definecolor{SourceHintergrund}{rgb}{1,1.0,0.95}

%
\renewcommand{\lstlistlistingname}{Listings}
\renewcommand{\lstlistingname}{Listing}

\lstnewenvironment{python}[1][]{
\lstset{
language=python,
basicstyle=\ttfamily\footnotesize\setstretch{1}, 	
stringstyle=\color{red}, 
showstringspaces=false, 
alsoletter={1234567890},
otherkeywords={\ , \}, \{},
keywordstyle=\color{blue},
emph={access,and,break,class,continue,def,del,elif ,else,
except,exec,finally,for,from,global,if,import,in,i s,
lambda,not,or,pass,print,raise,return,try,while},
emphstyle=\color{black}\bfseries,
emph={[2]True, False, None, self},
emphstyle=[2]\color{green},
emph={[3]from, import, as},
emphstyle=[3]\color{blue},
upquote=true,
morecomment=[s]{"""}{"""},
commentstyle=\color{Hellbraun}\slshape, 
%emph={[4]1, 2, 3, 4, 5, 6, 7, 8, 9, 0},
emphstyle=[4]\color{blue},
literate=*{:}{{\textcolor{blue}:}}{1}
{=}{{\textcolor{blue}=}}{1}
{-}{{\textcolor{blue}-}}{1}
{+}{{\textcolor{blue}+}}{1}
{*}{{\textcolor{blue}*}}{1}
{!}{{\textcolor{blue}!}}{1}
{(}{{\textcolor{blue}(}}{1}
{)}{{\textcolor{blue})}}{1}
{[}{{\textcolor{blue}[}}{1}
{]}{{\textcolor{blue}]}}{1}
{<}{{\textcolor{blue}<}}{1}
{>}{{\textcolor{blue}>}}{1},
%framexleftmargin=1mm, framextopmargin=1mm, frame=shadowbox, rulesepcolor=\color{blue},#1
backgroundcolor=\color{SourceHintergrund}, 
framexleftmargin=1mm, framexrightmargin=1mm, framextopmargin=1mm, frame=single, framerule=1pt, rulecolor=\color{black},#1
}}{}
%%%%%%%%%%%%
% Définition des vecteurs 
%%%%%%%%%%%%
\newcommand{\vect}[1]{\overrightarrow{#1}}
\newcommand{\axe}[2]{\left(#1,\vect{#2}\right)}
\newcommand{\couple}[2]{\left(#1,\vect{#2}\right)}
\newcommand{\angl}[2]{\left(\vect{#1},\vect{#2}\right)}

\newcommand{\rep}[1]{\mathcal{R}_{#1}}
\newcommand{\quadruplet}[4]{\left(#1;#2,#3,#4 \right)}
\newcommand{\repere}[4]{\left(#1;\vect{#2},\vect{#3},\vect{#4} \right)}
\newcommand{\base}[3]{\left(\vect{#1},\vect{#2},\vect{#3} \right)}


\newcommand{\vx}[1]{\vect{x_{#1}}}
\newcommand{\vy}[1]{\vect{y_{#1}}}
\newcommand{\vz}[1]{\vect{z_{#1}}}

\newcommand{\norm}[1]{\ensuremath{\left\Vert {#1}\right\Vert}}
\newcommand{\Ker}{\mathop{\mathrm{Ker}}\nolimits}

% d droit pour le calcul différentiel
\newcommand{\dd}{\text{d}}

\newcommand{\inertie}[2]{I_{#1}\left( #2\right)}
\newcommand{\matinertie}[7]{
\begin{pmatrix}
#1 & #6 & #5 \\
#6 & #2 & #4 \\
#5 & #4 & #3 \\
\end{pmatrix}_{#7}}
%%%%%%%%%%%%
% Définition des torseurs 
%%%%%%%%%%%%

\newcommand{\ec}[2]{%
\mathcal{E}_c\left(#1/#2\right)}

\newcommand{\pext}[3]{%
\mathcal{P}\left(#1\rightarrow#2/#3\right)}

\newcommand{\pint}[3]{%
\mathcal{P}\left(#1 \stackrel{\text{#3}}{\leftrightarrow} #2\right)}


 \newcommand{\torseur}[1]{%
\left\{{#1}\right\}
}

\newcommand{\torseurcin}[3]{%
\left\{\mathcal{#1} \left(#2/#3 \right) \right\}
}

\newcommand{\torseurci}[2]{%
\left\{\sigma \left(#1/#2 \right) \right\}
}
\newcommand{\torseurdyn}[2]{%
\left\{\mathcal{D} \left(#1/#2 \right) \right\}
}


\newcommand{\torseurstat}[3]{%
\left\{\mathcal{#1} \left(#2\rightarrow #3 \right) \right\}
}


 \newcommand{\torseurc}[8]{%
%\left\{#1 \right\}=
\left\{
{#1}
\right\}
 = 
\left\{%
\begin{array}{cc}%
{#2} & {#5}\\%
{#3} & {#6}\\%
{#4} & {#7}\\%
\end{array}%
\right\}_{#8}%
}

 \newcommand{\torseurcol}[7]{
\left\{%
\begin{array}{cc}%
{#1} & {#4}\\%
{#2} & {#5}\\%
{#3} & {#6}\\%
\end{array}%
\right\}_{#7}%
}

 \newcommand{\torseurl}[3]{%
%\left\{\mathcal{#1}\right\}_{#2}=%
\left\{%
\begin{array}{l}%
{#1} \\%
{#2} %
\end{array}%
\right\}_{#3}%
}

% Vecteur vitesse
 \newcommand{\vectv}[3]{%
\vect{V\left( {#1} \in {#2}/{#3}\right)}
}

% Vecteur force
\newcommand{\vectf}[2]{%
\vect{R\left( {#1} \rightarrow {#2}\right)}
}

% Vecteur moment stat
\newcommand{\vectm}[3]{%
\vect{\mathcal{M}\left( {#1}, {#2} \rightarrow {#3}\right)}
}




% Vecteur résultante cin
\newcommand{\vectrc}[2]{%
\vect{R_c \left( {#1}/ {#2}\right)}
}
% Vecteur moment cin
\newcommand{\vectmc}[3]{%
\vect{\sigma \left( {#1}, {#2} /{#3}\right)}
}


% Vecteur résultante dyn
\newcommand{\vectrd}[2]{%
\vect{R_d \left( {#1}/ {#2}\right)}
}
% Vecteur moment dyn
\newcommand{\vectmd}[3]{%
\vect{\delta \left( {#1}, {#2} /{#3}\right)}
}

% Vecteur accélération
 \newcommand{\vectg}[3]{%
\vect{\Gamma \left( {#1} \in {#2}/{#3}\right)}
}

% Vecteur omega
 \newcommand{\vecto}[2]{%
\vect{\Omega\left( {#1}/{#2}\right)}
}
% }$$\left\{\mathcal{#1} \right\}_{#2} =%
% \left\{%
% \begin{array}{c}%
%  #3 \\%
%  #4 %
% \end{array}%
% \right\}_{#5}}

\newcommand{\N}{\mathbb{N}}
\newcommand{\Z}{\mathbb{Z}}
\newcommand{\R}{\mathbb{R}}
\newcommand{\C}{\mathbb{C}}
\newcommand{\K}{\mathbb{K}}

\newcommand{\cA}{\mathscr{A}}
\newcommand{\cM}{\mathscr{M}}
\newcommand{\cL}{\mathscr{L}}
\newcommand{\cS}{\mathscr{S}}

\newcommand{\python}{\texttt{Python}}

\newcommand{\z}[1]{\Z_{#1}}
\newcommand{\ztimes}[1]{\Z_{#1}^{\times}}
\newcommand{\ii}[1]{[\![#1[\![}
\newcommand{\iif}[1]{[\![#1]\!]}
\newcommand{\llbr}{\ensuremath{\llbracket}}
\newcommand{\rrbr}{\ensuremath{\rrbracket}}
%\newcommand{\p}[1]{\left(#1\right)}
\newcommand{\ens}[1]{\left\{ #1 \right\}}
\newcommand{\croch}[1]{\left[ #1 \right]}
%\newcommand{\of}[1]{\lstinline{#1}}
% \newcommand{\py}[2]{%
%   \begin{tabular}{|l}
%     \lstinline+>>>+\textbf{\of{#1}}\\
%     \of{#2}
%   \end{tabular}\par{}
% }
\newcommand{\floor}[1]{\left\lfloor#1\right\rfloor}
\newcommand{\ceil}[1]{\left\lceil#1\right\rceil}
\newcommand{\abs}[1]{\left|#1\right|}


% Binaire, octal, hexa
\newcommand{\hex}[1]{\underline{\text{\texttt{#1}}}_{16}}
\newcommand{\oct}[1]{\underline{\text{\texttt{#1}}}_{8}}
\newcommand{\bin}[1]{\underline{\text{\texttt{#1}}}_{2}}
\DeclareMathOperator{\mmod}{\texttt{\%}}


% Fonctions et systèmes
\newcommand{\fct}[5][t]{%
  \begin{array}[#1]{rcl}
    #2 & \rightarrow & #3\\
    #4 & \mapsto     & #5\\
  \end{array}
}
\newcommand{\fonction}[5]{#1 : \left\{\begin{array}{rcl} #2& \longrightarrow &#3 \\ #4 &\longmapsto & #5\end{array}\right.}
\newenvironment{systeme}{\left\{ \begin{array}{rcl}}{\end{array}\right.}

% Matrices
\newcommand{\mat}[1]{
  \begin{pmatrix}
    #1
  \end{pmatrix}
}
\newcommand{\inv}{\ensuremath{^{-1}}}
\newcommand{\bpm}{\begin{pmatrix}}
\newcommand{\epm}{\end{pmatrix}}


% bases de données
\newcommand{\relat}[1]{\textsc{#1}}
\newcommand{\attr}[1]{\emph{#1}}
\newcommand{\prim}[1]{\uline{#1}}
\newcommand{\foreign}[1]{\#\textsl{#1}}


% Bases de données

\newcommand{\att}{\ensuremath{\mathbf{att}}}
\newcommand{\dom}{\ensuremath{\mathbf{dom}}}
\newcommand{\sort}{\ensuremath{\mathbf{sort}}}
\newcommand{\relname}{\ensuremath{\mathbf{relname}}}
\newcommand{\var}{\ensuremath{\mathbf{var}}}
\newcommand{\FILM}{\ensuremath{\mathtt{FILM}}}
\newcommand{\JOUE}{\ensuremath{\mathtt{JOUE}}}
\newcommand{\PERSONNE}{\ensuremath{\mathtt{PERSONNE}}}
\newcommand{\PERSONNAGE}{\ensuremath{\mathtt{PERSONNAGE}}}

\newcommand{\ttid}{\ensuremath{\mathtt{id}}}
\newcommand{\tttitre}{\ensuremath{\mathtt{titre}}}
\newcommand{\ttdate}{\ensuremath{\mathtt{date}}}
\newcommand{\ttidr}{\ensuremath{\mathtt{idrealisateur}}}
\newcommand{\ttdatenais}{\ensuremath{\mathtt{datenaissance}}}
\newcommand{\ttnom}{\ensuremath{\mathtt{nom}}}
\newcommand{\ttprenom}{\ensuremath{\mathtt{prenom}}}
\newcommand{\ttidacteur}{\ensuremath{\mathtt{idacteur}}}
\newcommand{\ttidfilm}{\ensuremath{\mathtt{idfilm}}}
\newcommand{\ttidpersonnage}{\ensuremath{\mathtt{idpersonnage}}}

\newcommand{\fv}{\mathrm{libre}}
\newcommand{\sem}[1]{[\![ #1 ]\!]}

\input{style/macros_Titres}
\input{style/macros_Frames}

%Si le boolen xp est vrai : compilation pour xabi
%Sinon compilation Damien
\newif\ifprof
\proftrue
%\proffalse

\newif\ifxp
\xptrue
%\xpfalse

\newif\iftd
\tdtrue
%\tdfalse

\usepackage[%
    pdftitle={DS Informatique - Concours Blanc},
    pdfauthor={Xavier Pessoles},
    colorlinks=true,
    linkcolor=blue,
    citecolor=magenta]{hyperref}

\def\discipline{Informatique}
\def\xxtitre{%
\ifxp
Devoir Surveillé d'informatique 5 -- 1 Heure
\else
\fi
}

\def\xxsoustitre{%
\ifxp
Simulation de vibrations -- CCP PSI -- Adapté du sujet 0
\else
\fi}

\def\xxauteur{%
\ifxp
Xavier \textsc{Pessoles} \\
Adapté de sujet 0 -- CCP PSI
\else
\fi}

\def\xxpied{%
\ifxp
DS 5 -- Informatique \\
Simulation des vibrations -- \ifprof P \else E \fi
\else
\fi}




%---------------------------------------------------------------------------


\begin{document}
\ifxp
\usepackage[%
    pdftitle={Représentation des nombres},
    pdfauthor={Xavier Pessoles},
    colorlinks=true,
    linkcolor=blue,
    citecolor=magenta]{hyperref}

\usepackage{pifont}
%\usepackage{lastpage}

% \makeatletter \let\ps@plain\ps@empty \makeatother
%% DEBUT DU DOCUMENT
%% =================
\sloppy
\hyphenpenalty 10000


\colorlet{shadecolor}{orange!15}

\newtheorem{theorem}{Theorem}


\begin{document}


%\newboolean{prof}
%\setboolean{prof}{true}
% \makeatletter \let\ps@plain\ps@empty \makeatother
%% DEBUT DU DOCUMENT
%% =================




%------------- En tetes et Pieds de Pages ------------


\pagestyle{fancy}
\ifthenelse{\boolean{xp}}{%
\renewcommand{\headrulewidth}{0pt}}{%
\renewcommand{\headrulewidth}{0.2pt}} %pour mettre le trait en haut
%\renewcommand{\headrulewidth}{0.2pt}

\fancyhead{}
\fancyhead[L]{%
\noindent\begin{minipage}[c]{2.6cm}%
\includegraphics[width=2cm]{png/logo_ptsi.png}%
\end{minipage}}


\fancyhead[C]{\rule{12cm}{.5pt}}



\fancyhead[R]{%
\noindent\begin{minipage}[c]{3cm}
\begin{flushright}
\footnotesize{\textit{\textsf{Informatique}}}%
\end{flushright}
\end{minipage}
}



\fancyhead[C]{\rule{12cm}{.5pt}}

\renewcommand{\footrulewidth}{0.2pt}

\fancyfoot[C]{\footnotesize{\bfseries \thepage}}
\fancyfoot[L]{%
\begin{minipage}[c]{.2\linewidth}
\noindent\footnotesize{{\xxauteur}}
\end{minipage}
\ifthenelse{\boolean{xp}}{}{%
\begin{minipage}[c]{.15\linewidth}
\includegraphics[width=2cm]{png/logoCC.png}
\end{minipage}}
}

\ifthenelse{\boolean{prof}}{%
\fancyfoot[R]{\footnotesize{\xxpied}}}

\begin{center}
 \huge\textsc{\xxtitre}
\end{center}

\begin{center}
 \LARGE\textsc{\xxsoustitre}
\end{center}

\vspace{.5cm}

\else
\input{style/enteteDI}
\fi



 \renewcommand{\baselinestretch}{1.2}
\setlength{\parskip}{2ex plus 0.5ex minus 0.2ex}



\section{Mise en situation}
\ifprof
\else
\begin{minipage}[c]{.7\linewidth}
Pour étudier une poutre soumises à des vibrations en traction compression, il est possible de la modéliser par $n$ éléments de masse $m_i$ ($i$ variant de 1 à $n$). Ces éleménts sont reliés par des liaisons visco-élastiques elles-mêmes modélisées par des un ressort de raideur $k$ ($k$ dépendant du module de Young du matériau et des dimensions de la poutre) en parallèle d'un élément d'amortissement $c$ (dépendant d'un coefficient d'amortissement visqueux entre les éléments, du nombre d'éléments et de la masse de la structure). La structure est supposée unidimensionnelle de longueur $L$. Le nombre d'éléments peut être de l'ordre de plusieurs milliers. 

Le déplacement au cours du temps de l'élément $i$ autour de sa position d'équilibre est noté $u_i(t)$. Une force $f_n(t)$ est appliquée sur l'élément $n$ uniquement. L'extrémité gauche de
la structure est bloquée. Les effets de la pesanteur sont négligés.
\end{minipage}
\begin{minipage}[c]{.27\linewidth}
\begin{center}
\includegraphics[width=.95\textwidth]{images/vibrations}
\end{center}
\end{minipage}

\begin{center}
\includegraphics[width=.9\textwidth]{images/structure}
\end{center}


Le théorème de la résultante dynamique appliqué à un élément $i$ ($i$ variant de 2 à $n-1$ inclus) s'écrit sous la forme : 
%$$
%m_i\dfrac{\text{d}u_i(t)}{\text{d}t} = 
%-k_i\left(u_i(t)-u_{i-1}(t) \right)   - k_{i+1}\left(u_i(t)-u_{i+1}(t) \right) 
%-c_i\left(u_i(t)-u_{i-1}(t) \right)   - c_{i+1}\left(u_i(t)-u_{i+1}(t) \right) 
%$$

%$$
%m\dfrac{\text{d}^2u_i(t)}{\text{d}t^2} = 
%- k\left(u_i(t)-u_{i-1}(t) \right)   
%- k\left(u_i(t)-u_{i+1}(t) \right) 
%- c\dfrac{\text{d}}{\text{d}t} \left[u_i(t)-u_{i-1}(t) \right]
%- c\dfrac{\text{d}}{\text{d}t} \left[u_i(t)-u_{i+1}(t) \right]
%$$


\begin{equation}
m\dfrac{\text{d}^2u_i(t)}{\text{d}t^2} = 
- 2 k u_i(t)  + k u_{i-1}(t) + k u_{i+1}(t)
- 2c \dfrac{\text{d}u_i(t)}{\text{d}t}   +c \dfrac{\text{d}u_{i-1}(t)}{\text{d}t} +c \dfrac{\text{d}u_{i+1}(t)}{\text{d}t}
\end{equation}

Le théorème de la résultante dynamique appliqué aux éléments 1 et $n$ donne les équations suivantes : 
%$$
%m_1\dfrac{\text{d}u_1(t)}{\text{d}t} = 
%-\left(k_1 + k_2\right) u_1(t) + k_2 u_2 (t) - \left(c_1 + c_2 \right)u_1(t) - c_2 u_2 (t)
%$$
%
%$$
%m_n\dfrac{\text{d}u_n(t)}{\text{d}t} = 
%-k_n\left(u_n(t)-u_{n-1}(t) \right)   - c_n\left(u_n(t)-u_{n-1}(t) \right) +f_n(t)
%$$


\begin{equation}
m\dfrac{\text{d}^2u_1(t)}{\text{d}t^2} = 
- 2k u_1(t) + k u_2 (t) 
-2c \dfrac{\text{d}u_1(t)}{\text{d}t} 
+ c \dfrac{\text{d}u_2(t)}{\text{d}t} 
\end{equation}

\begin{equation}
m\dfrac{\text{d}^2u_n(t)}{\text{d}t^2} = 
-k\left(u_n(t)-u_{n-1}(t) \right)   - c\left(\dfrac{\text{d}u_{n}(t)}{\text{d}t}-\dfrac{\text{d}u_{n-1}(t)}{\text{d}t} \right) +f_n(t)
\end{equation}


Dans toute la suite, on imposera $f_n(t)=f_{max} \sin \left(\omega t\right)$.
\fi

\section{Résolution d'une équation différentielle}
\ifprof
\else

\begin{minipage}[c]{.6\linewidth}
On s'intéresse tout d'abord à une seule cellule masse ressort amortisseur. L'application du théorème de la résultante dynamique en projection sur l'axe de déplacement s'écrit donc 
de la manière suivante : 
$$
m\cdot\ddot{u}(t)+c\cdot \dot{u}(t) + k\cdot u(t) = f(t)
$$
\end{minipage}\hfill
\begin{minipage}[c]{.37\linewidth}
\begin{center}
\includegraphics[width=.95\textwidth]{images/cellule}
\end{center}
\end{minipage}

On précise que pour $t\leq 0$, $u(t)=0$, $\dot{u}(t)=0$. On pose  $\left\{ \begin{array}{l} x(t) = u(t) \\ v(t) = \dot{x}(t)\end{array} \right.$.


\fi
\subparagraph{}
\textit{Réécrire l'équation différentielle sous forme d'un système d'équations en fonction de $x(t)$,  $v(t)$, $\dot{x}(t)$ et $\dot{v}(t)$.}
\ifprof
\begin{corrige}
On a donc : 
$$
\left\{ 
\begin{array}{l} 
v(t) = \dot{x}(t) \\ 
m\cdot\dot{v}(t)+c\cdot {v}(t) + k\cdot x(t) = f(t)
\end{array} \right.
$$
\end{corrige}
\else
\fi

%\subparagraph{}
%\textit{En utilisant un schéma d'Euler explicite et l'équation $v(t) = \dot{x}(t)$ 
%exprimer la suite $x_{k+1}$ en fonction de $x_k$, $v_k$ et le pas de calcul noté noté $h$.}
%
%\ifprof
%\begin{corrige}
%On a $\dfrac{dx(t)}{dt} \simeq \dfrac{x_{k+1}-x_k}{h}$. On a donc 
%$v_k = \dfrac{x_{k+1}-x_k}{h} \Longleftrightarrow x_{k+1} = h\cdot v_k + x_k$.
%\end{corrige}
%\else
%\fi

\subparagraph{}
\textit{En utilisant un schéma d'Euler implicite et l'équation $v(t) = \dot{x}(t)$ 
exprimer la suite $x_{n}$ en fonction de $x_{n-1}$, $v_n$ et du pas de calcul noté noté $h$.}

\ifprof
\begin{corrige}
On a $\dfrac{dx(t)}{dt} \simeq \dfrac{x_{n}-x_ {n-1}}{h}$. On a donc 
$v_n = \dfrac{x_{n}-x_{n-1}}{h} \Longleftrightarrow x_{n} = h\cdot v_n + x_{n-1}$.

\end{corrige}
\else
\fi


\subparagraph{}
\textit{En utilisant un schéma d'Euler implicite exprimer $v_n$ en fonction de $v_{n-1}$, $x_{n}$, $x_{n-1}$, $f_n$, $h$, $k$, $c$ et $m$.}

\ifprof
\begin{corrige}

On a $\dfrac{dv(t)}{dt} \simeq \dfrac{v_{n}-v_ {n-1}}{h}$. On a donc 
$\dot{v}_n = \dfrac{v_{n}-v_{n-1}}{h} \Longleftrightarrow v_{n} = h\cdot \dot{v}_n + v_{n-1}$.


$$
m\cdot \dfrac{v_{n}-v_{n-1}}{h} +c\cdot \dfrac{x_{n}-x_{n-1}}{h} + k\cdot x_n = f_n
\Longleftrightarrow
m\cdot (v_{n}-v_{n-1}) +c\cdot (x_{n}-x_{n-1}) + kh\cdot x_n = hf_n
$$


On a donc :
$$
mv_{n}  = hf_n -  (kh+c) \cdot x_n + mv_{n-1}+cx_{n-1}
$$

\end{corrige}
\else
\fi

\subparagraph{}
\textit{En déduire que la suite $x_n$ peut se mettre sous la forme suivante :}
$x_{n}\left(m +ch+ kh^2\right)-x_{n-1}(2m+ch) +mx_{n-2}= h^2f_n$.

\ifprof
\begin{corrige}

D'après la question 2 on a :
$$v_n = \dfrac{x_{n}-x_{n-1}}{h} \quad \text{et} \quad v_{n-1} = \dfrac{x_{n-1}-x_{n-2}}{h}$$ 

En utilisant le résultat de la question 3, on a :
$$
m\dfrac{x_{n}-x_{n-1}}{h}  = hf_n -  (kh+c) \cdot x_n + m\dfrac{x_{n-1}-x_{n-2}}{h}+cx_{n-1}
$$

$$
\Longleftrightarrow 
m(x_{n}-x_{n-1}) = h^2f_n -  h(kh+c) \cdot x_n + m(x_{n-1}-x_{n-2})+chx_{n-1}
$$

$$
\Longleftrightarrow 
m(x_{n}-x_{n-1}) +  h(kh+c) \cdot x_n - m(x_{n-1}-x_{n-2})-chx_{n-1} = h^2f_n 
$$


$$
\Longleftrightarrow 
x_{n}\left(m +kh^2+ch\right)-x_{n-1}(2m+ch) +mx_{n-2} = h^2f_n 
$$

$$
\Longleftrightarrow 
x_{n}= \dfrac{1}{m +kh^2+ch} \left(h^2f_n +x_{n-1}(2m+ch) -mx_{n-2} \right)
$$

\end{corrige}
\else
\fi

\ifprof
\else

\vspace{.5cm}

On rappelle que $f(t)=f_{max} \sin \left(\omega t\right)$. On note \textsf{Tsimu} le temps de simulation et \textsf{h} le pas de temps. 

\fi
\subparagraph{}
\textit{Implémenter en Python le fonction \textsf{f\_omega} permettant de créer une liste contenant l'ensemble des valeurs prises par la fonction $f_n$. On utilisera une boucle \textsf{while}. Les spécifications de la fonction sont les suivantes : }
\ifprof
\else


\begin{py}
\begin{python}
def f_omega(Tsimu,h,fmax,fsign):
   """
   Entrées :
       * Tsimu (flt) : temps de la simulation en seconde
       * h (flt) : pas de temps de la simulation
       * fmax (flt) : amplitude du signal (en Newton)
       * fsign (flt) : fréquence du signal (en Hertz)
   Sortie : 
       * F (list) : liste des valeurs de la fonction 
          f_n (t)= fmax sin (omega *t)
   """
\end{python}
\end{py}

\fi

\ifprof
\begin{corrige}
\begin{py}
\begin{python}
def f_omega(Tsimu,h,fmax,fsign):
   """
   Entrées :
       * Tsimu (flt) : temps de la simulation en seconde
       * h (flt) : pas de temps de a simulation
       * fmax (flt) : amplitude du signal (en Newton)
       * fsign (flt) : fréquence du signal (en Hertz)
   Sortie : 
       * F (list) : liste des valeurs de la fonction 
          f_n (t)= fmax sin (omega *t)
   """
    omega  = 2*math.pi*fsign
    t=0 
    F = []
    while t<Tsimu :
        F.append(fmax*math.sin(omega*t))
        t=t+h
    return F
    
\end{python}
\end{py}

\end{corrige}
\else
\fi

\ifprof
\else

Il est possible de mettre la suite déterminée à la question 4 sous la forme $x_n = \alpha f_n +\beta x_{n-1} + \gamma x_{n-2}$.

\fi

\subparagraph{}
\textit{En utilisant une boucle \textsf{while}, générer, en Python, les listes \textsf{T} et \textsf{X} contenant respectivement le temps de simulation et le déplacement de la masse mobile.}
\ifprof
\begin{corrige}
\begin{python}

T=[0,h]
X=[0,0]
t=2*h
i=2
while t<Tsimu:
    T.append(t)
    X.append(alpha*F[i]+beta*X[i-1] +gamma*X(i-2))
    i=i+1
    t = t+h
\end{python}
\end{corrige}
\else
\fi


\section{Résolution du problème général}

\ifprof
\else

Le système d'équations différentielles défini dans la première partie peut s'écrire sous forme matricielle :
$$
M \ddot{X}(t) +C \dot{X}(t) +K {X}(t) = F(t) \quad \text{avec} \quad 
X(t)=\left[
\begin{array}{c} 
u_1(t) \\ 
... \\ 
u_n(t) \\
\end{array}\right] \quad \text{et $M$, $C$ et $K$ des matrices carrées de taille $n\times n$.}
$$

\fi

\subparagraph{}
\textit{En reprenant les équations (1), (2) et (3) déterminer les matrices $M$, $C$, $K$ et $F$.}
\ifprof
\begin{corrige}
\footnotesize{
$$
M =
\begin{pmatrix}
m  &  0 & 0 \\
0  & \ddots &0 \\
0 & 0 &  m \\
 \end{pmatrix}
 \quad
C =
\begin{pmatrix}
2c  & -c &  0 & \cdots & \cdots & \cdots  & 0 \\
-c  & 2c & -c & \ddots &  & 0& \vdots\\
0 & -c  & 2c & \ddots & \ddots &  & \vdots\\
\vdots & \ddots &\ddots & \ddots & \ddots & \ddots & \vdots\\
\vdots &  & \ddots &\ddots& 2c & -c & 0 \\
\vdots & 0 &  & \ddots & -c & 2c & -c\\
0 & \cdots & \cdots & \cdots & 0 & -c & 2c\\
 \end{pmatrix}
 \quad
K =
\begin{pmatrix}
2k  & -k &  0 & \cdots & \cdots & \cdots  & 0 \\
-k & 2k & -k & \ddots &  & 0& \vdots\\
0 & -k  & 2k & \ddots & \ddots &  & \vdots\\
\vdots & \ddots &\ddots & \ddots & \ddots & \ddots & \vdots\\
\vdots &  & \ddots &\ddots& 2k & -k & 0 \\
\vdots & 0 &  & \ddots & -k & 2k & -k\\
0 & \cdots & \cdots & \cdots & 0 & -k & 2k\\
 \end{pmatrix}
 \quad
 F =
 \begin{pmatrix}
 0 \\
 \vdots \\
 0 \\
 f_n(t) 
 \end{pmatrix}
$$}
\end{corrige}
\else
\fi

\ifprof
\else


En appliquant le schéma d'Euler implicite à l'équation différentielle matricielle, la solution reste identique à celle déterminée dans la partie précédente : $\left(M +h^2K+hX\right)X_{n}= \left(h^2F_n +X_{n-1}(2M+hC) -MX_{n-2} \right)$. On montre qu'à l'instant $k$, le système peut se mettre sous la forme $H\cdot X_k = G_k$, la matrice $H$ étant de la forme suivante (dite tridiagonale) :
\begin{minipage}[c]{.4\linewidth}
$$
H =
\begin{pmatrix}
H_{00}  & H_{01} &  0 & 0 & 0 \\
H_{10}  & H_{11} &  \ddots & 0 & 0 \\
0 & \ddots & \ddots & \ddots & 0 \\
0 & 0 & \ddots & H_{n-1,n-1} & H_{n-1,n} \\
0 & 0 & 0 & H_{n-1,n} & H_{n,n} \\
\end{pmatrix}
 $$
 \end{minipage} \hfill
\begin{minipage}[c]{.57\linewidth}
\begin{rem}
Dans ces conditions on a donc  $G_k = \left(\dfrac{2M}{h^2}+\dfrac{C}{h} \right)X_{k-1} - \dfrac{M}{h^2} X_{k-2} + F_k $ et $H = \dfrac{M}{h^2} + \dfrac{C}{h} + K$.
\end{rem}
 \end{minipage} 

\fi

\subparagraph{}
\textit{L'équation  $H\cdot X_k = G_k$ peut être résolue grâce à la méthode du pivot de Gauss. Donner les étapes de cette méthode ainsi que l'objectif ce chacune d'entre elles. Quelle est la complexité algorithmique de la résolution d'une équation en utilisant le pivot de Gauss ?}
\ifprof
\begin{corrige}
\end{corrige}
\else
\fi

\ifprof
\else

Plutôt que d'utiliser la méthode du pivot de Gauss classique, un programmeur utilise la méthode suivante pour résoudre le système d'équation : 
\begin{py}
\begin{python}
def resoud(H,X,G):
    """
    Fonction permettant de résoudre H.X = G
    Entrées : 
        * H (list) : matrice de tridiagonale de taille nxn
        * X (list) : vecteur de n lignes, 1 colonne dont toutes les valeurs sont nulles
        * G (list) : second membre du système d'équation de taille nx1
    Sortie : 
        * X : solution du système
    """
    H2=copy(H)  # Copie de la matrice H
    G2=copy(G)  # Copie du vecteur G
    n=len(G2)
    for i in range(1,n):
        a=H2[i-1][i-1]
        b=H2[i][i-1]
        for k in range(n) :
            H2[i][k]=H2[i][k]*a-H2[i-1][k]*b
        G2[i]=G2[i]*a-G2[i-1]*b
    X[n-1]=G2[n-1]/H2[n-1][n-1]
    for i in range(n-2,-1,-1):
        X[i]=(G2[i]-H2[i][i+1]*X[i+1])/H2[i][i]
    return X

\end{python}
\end{py}

\fi

\subparagraph{}
\textit{Expliquer en quoi cet algorithme permet de résoudre le système $H\cdot X_n = G_n$ ? Quelle est la complexité de cet algorithme ?}
\ifprof
\begin{corrige}
\end{corrige}
\else
\fi


\section{Détermination de l'énergie dissipée}
\ifprof
\else

Pour une poutre subdivisée en $n$ éléments sur laquelle on a réalisé une simulation de $p$ pas de temps, la résolution totale nous permet d'obtenir une matrice $U$ de $p$ lignes $n$ colonnes. Ainsi l'élément $U[i][j]$ nous permet d'avoir la valeur du déplacement du jième point à l'ième temps de calcul.

La puissance et l'énergie dissipées par la poutre sont données par les relations suivantes :
$$
P_{\text{diss}}(t) = c(\dot{u}_i(t))^2 + \sum\limits_{i=2}^n c\left(\dot{u}_i(t) - \dot{u}_{i-1}(t)\right)^2
\quad \text{et} \quad 
E_{\text{diss}}(t) = \int\limits_{0}^t P_{\text{diss}}(\tau) d\tau
$$


Un programmeur propose la fonction suivante pour traduire la fonction puissance donnée ci-dessus :
\begin{py}
\begin{python}
def calculPuissance(U,c,h):
    p,n=len(X),len(X[0])
    P,V=[],[]
    for i in range(1,p):
        T=[]
        for j in range(n):
            T.append((U[i][j]-U[i-1][j])/h)
        V.append(T)
    for i in range(1,p):
        Ps=0
        for j in range(n):
                Ps = Ps + c*(V[i][j]-V[i][j-1])**2
        P.append(Ps)
    return P
\end{python}
\end{py}

\fi

\subparagraph{}
\textit{Expliquez le but des blocs constitués des lignes 4 à 8 puis des lignes 9 à 14. La fonction proposée permet-elle de calculer la puissance ? Si non, que faudrait-il modifier ? }
\ifprof
\begin{corrige}
\end{corrige}
\else
\fi


\subparagraph{}\textit{La programme précédent retourne la liste des puissances dissipées à chaque temps de calcul. Calculer l'énergie dissipée en utilisant la méthode des trapèzes. Vous réaliserez un schéma pour illustrer la méthdode. On rappelle que le pas de calcul est noté $h$.}
\ifprof

\begin{corrige}

\begin{python}
P = calculPuissance(U,c,h)
somme = 0
for i in range(1,len(P)-1):
    somme = somme + P[i]

Energie = (somme + 0.5*(P[0]+P[len(P)-1]))*h
\end{python}
\end{corrige}
\else
\fi
\end{document}

\subparagraph{}
\textit{}
\ifprof
\begin{corrige}
\end{corrige}
\else
\fi





 
$$
m\ddot{x}_i -c \dot{x}_{i-1} + 2c \dot{x}_i   -c \dot{x}_{i+1}- k x_{i-1} + 2 k x_i   - k x_{i+1}=0
$$


$$
\begin{array}{l}
m\ddot{x}_1 -c \dot{x}_{1-1} + 2c \dot{x}_1   -c \dot{x}_{1+1}- k x_{1-1} + 2 k x_1   - k x_{1+1}=0 \\
... \\
m\ddot{x}_2 -c \dot{x}_{2-1} + 2c \dot{x}_2   -c \dot{x}_{2+1}- k x_{2-1} + 2 k x_2   - k x_{2+1}=0 \\
m\ddot{x}_3 -c \dot{x}_{3-1} + 2c \dot{x}_3   -c \dot{x}_{3+1}- k x_{3-1} + 2 k x_3   - k x_{3+1}=0 \\
m\ddot{x}_4 -c \dot{x}_{4-1} + 2c \dot{x}_4   -c \dot{x}_{4+1}- k x_{4-1} + 2 k x_4   - k x_{4+1}=0 \\
... \\
m\ddot{x}_n -c \dot{x}_{n-1} + 2c \dot{x}_n   -c \dot{x}_{n+1}- k x_{n-1} + 2 k x_n   - k x_{n+1}=f_n \\
\end{array}
$$


