\documentclass[10pt]{article}
\input{style/coursHeadings}
\usepackage{algorithm}
\usepackage{algorithmic}


% Python sources
\usepackage{listings}
\usepackage{textcomp}
\usepackage{setspace}
%\usepackage{palatino}

%\usepackage{color}
\definecolor{Bleu}{rgb}{0.1,0.1,1.0}
\definecolor{Noir}{rgb}{0,0,0}
\definecolor{Grau}{rgb}{0.5,0.5,0.5}
\definecolor{DunkelGrau}{rgb}{0.15,0.15,0.15}
\definecolor{Hellbraun}{rgb}{0.5,0.25,0.0}
\definecolor{Magenta}{rgb}{1.0,0.0,1.0}
\definecolor{Gris}{gray}{0.5}
\definecolor{Vert}{rgb}{0,0.5,0}
\definecolor{SourceHintergrund}{rgb}{1,1.0,0.95}

%
\renewcommand{\lstlistlistingname}{Listings}
\renewcommand{\lstlistingname}{Listing}

\lstnewenvironment{python}[1][]{
\lstset{
language=python,
basicstyle=\ttfamily\footnotesize\setstretch{1}, 	
stringstyle=\color{red}, 
showstringspaces=false, 
alsoletter={1234567890},
otherkeywords={\ , \}, \{},
keywordstyle=\color{blue},
emph={access,and,break,class,continue,def,del,elif ,else,
except,exec,finally,for,from,global,if,import,in,i s,
lambda,not,or,pass,print,raise,return,try,while},
emphstyle=\color{black}\bfseries,
emph={[2]True, False, None, self},
emphstyle=[2]\color{green},
emph={[3]from, import, as},
emphstyle=[3]\color{blue},
upquote=true,
morecomment=[s]{"""}{"""},
commentstyle=\color{Hellbraun}\slshape, 
%emph={[4]1, 2, 3, 4, 5, 6, 7, 8, 9, 0},
emphstyle=[4]\color{blue},
literate=*{:}{{\textcolor{blue}:}}{1}
{=}{{\textcolor{blue}=}}{1}
{-}{{\textcolor{blue}-}}{1}
{+}{{\textcolor{blue}+}}{1}
{*}{{\textcolor{blue}*}}{1}
{!}{{\textcolor{blue}!}}{1}
{(}{{\textcolor{blue}(}}{1}
{)}{{\textcolor{blue})}}{1}
{[}{{\textcolor{blue}[}}{1}
{]}{{\textcolor{blue}]}}{1}
{<}{{\textcolor{blue}<}}{1}
{>}{{\textcolor{blue}>}}{1},
%framexleftmargin=1mm, framextopmargin=1mm, frame=shadowbox, rulesepcolor=\color{blue},#1
backgroundcolor=\color{SourceHintergrund}, 
framexleftmargin=1mm, framexrightmargin=1mm, framextopmargin=1mm, frame=single, framerule=1pt, rulecolor=\color{black},#1
}}{}
%%%%%%%%%%%%
% Définition des vecteurs 
%%%%%%%%%%%%
\newcommand{\vect}[1]{\overrightarrow{#1}}
\newcommand{\axe}[2]{\left(#1,\vect{#2}\right)}
\newcommand{\couple}[2]{\left(#1,\vect{#2}\right)}
\newcommand{\angl}[2]{\left(\vect{#1},\vect{#2}\right)}

\newcommand{\rep}[1]{\mathcal{R}_{#1}}
\newcommand{\quadruplet}[4]{\left(#1;#2,#3,#4 \right)}
\newcommand{\repere}[4]{\left(#1;\vect{#2},\vect{#3},\vect{#4} \right)}
\newcommand{\base}[3]{\left(\vect{#1},\vect{#2},\vect{#3} \right)}


\newcommand{\vx}[1]{\vect{x_{#1}}}
\newcommand{\vy}[1]{\vect{y_{#1}}}
\newcommand{\vz}[1]{\vect{z_{#1}}}

\newcommand{\norm}[1]{\ensuremath{\left\Vert {#1}\right\Vert}}
\newcommand{\Ker}{\mathop{\mathrm{Ker}}\nolimits}

% d droit pour le calcul différentiel
\newcommand{\dd}{\text{d}}

\newcommand{\inertie}[2]{I_{#1}\left( #2\right)}
\newcommand{\matinertie}[7]{
\begin{pmatrix}
#1 & #6 & #5 \\
#6 & #2 & #4 \\
#5 & #4 & #3 \\
\end{pmatrix}_{#7}}
%%%%%%%%%%%%
% Définition des torseurs 
%%%%%%%%%%%%

\newcommand{\ec}[2]{%
\mathcal{E}_c\left(#1/#2\right)}

\newcommand{\pext}[3]{%
\mathcal{P}\left(#1\rightarrow#2/#3\right)}

\newcommand{\pint}[3]{%
\mathcal{P}\left(#1 \stackrel{\text{#3}}{\leftrightarrow} #2\right)}


 \newcommand{\torseur}[1]{%
\left\{{#1}\right\}
}

\newcommand{\torseurcin}[3]{%
\left\{\mathcal{#1} \left(#2/#3 \right) \right\}
}

\newcommand{\torseurci}[2]{%
\left\{\sigma \left(#1/#2 \right) \right\}
}
\newcommand{\torseurdyn}[2]{%
\left\{\mathcal{D} \left(#1/#2 \right) \right\}
}


\newcommand{\torseurstat}[3]{%
\left\{\mathcal{#1} \left(#2\rightarrow #3 \right) \right\}
}


 \newcommand{\torseurc}[8]{%
%\left\{#1 \right\}=
\left\{
{#1}
\right\}
 = 
\left\{%
\begin{array}{cc}%
{#2} & {#5}\\%
{#3} & {#6}\\%
{#4} & {#7}\\%
\end{array}%
\right\}_{#8}%
}

 \newcommand{\torseurcol}[7]{
\left\{%
\begin{array}{cc}%
{#1} & {#4}\\%
{#2} & {#5}\\%
{#3} & {#6}\\%
\end{array}%
\right\}_{#7}%
}

 \newcommand{\torseurl}[3]{%
%\left\{\mathcal{#1}\right\}_{#2}=%
\left\{%
\begin{array}{l}%
{#1} \\%
{#2} %
\end{array}%
\right\}_{#3}%
}

% Vecteur vitesse
 \newcommand{\vectv}[3]{%
\vect{V\left( {#1} \in {#2}/{#3}\right)}
}

% Vecteur force
\newcommand{\vectf}[2]{%
\vect{R\left( {#1} \rightarrow {#2}\right)}
}

% Vecteur moment stat
\newcommand{\vectm}[3]{%
\vect{\mathcal{M}\left( {#1}, {#2} \rightarrow {#3}\right)}
}




% Vecteur résultante cin
\newcommand{\vectrc}[2]{%
\vect{R_c \left( {#1}/ {#2}\right)}
}
% Vecteur moment cin
\newcommand{\vectmc}[3]{%
\vect{\sigma \left( {#1}, {#2} /{#3}\right)}
}


% Vecteur résultante dyn
\newcommand{\vectrd}[2]{%
\vect{R_d \left( {#1}/ {#2}\right)}
}
% Vecteur moment dyn
\newcommand{\vectmd}[3]{%
\vect{\delta \left( {#1}, {#2} /{#3}\right)}
}

% Vecteur accélération
 \newcommand{\vectg}[3]{%
\vect{\Gamma \left( {#1} \in {#2}/{#3}\right)}
}

% Vecteur omega
 \newcommand{\vecto}[2]{%
\vect{\Omega\left( {#1}/{#2}\right)}
}
% }$$\left\{\mathcal{#1} \right\}_{#2} =%
% \left\{%
% \begin{array}{c}%
%  #3 \\%
%  #4 %
% \end{array}%
% \right\}_{#5}}

\newcommand{\N}{\mathbb{N}}
\newcommand{\Z}{\mathbb{Z}}
\newcommand{\R}{\mathbb{R}}
\newcommand{\C}{\mathbb{C}}
\newcommand{\K}{\mathbb{K}}

\newcommand{\cA}{\mathscr{A}}
\newcommand{\cM}{\mathscr{M}}
\newcommand{\cL}{\mathscr{L}}
\newcommand{\cS}{\mathscr{S}}

\newcommand{\python}{\texttt{Python}}

\newcommand{\z}[1]{\Z_{#1}}
\newcommand{\ztimes}[1]{\Z_{#1}^{\times}}
\newcommand{\ii}[1]{[\![#1[\![}
\newcommand{\iif}[1]{[\![#1]\!]}
\newcommand{\llbr}{\ensuremath{\llbracket}}
\newcommand{\rrbr}{\ensuremath{\rrbracket}}
%\newcommand{\p}[1]{\left(#1\right)}
\newcommand{\ens}[1]{\left\{ #1 \right\}}
\newcommand{\croch}[1]{\left[ #1 \right]}
%\newcommand{\of}[1]{\lstinline{#1}}
% \newcommand{\py}[2]{%
%   \begin{tabular}{|l}
%     \lstinline+>>>+\textbf{\of{#1}}\\
%     \of{#2}
%   \end{tabular}\par{}
% }
\newcommand{\floor}[1]{\left\lfloor#1\right\rfloor}
\newcommand{\ceil}[1]{\left\lceil#1\right\rceil}
\newcommand{\abs}[1]{\left|#1\right|}


% Binaire, octal, hexa
\newcommand{\hex}[1]{\underline{\text{\texttt{#1}}}_{16}}
\newcommand{\oct}[1]{\underline{\text{\texttt{#1}}}_{8}}
\newcommand{\bin}[1]{\underline{\text{\texttt{#1}}}_{2}}
\DeclareMathOperator{\mmod}{\texttt{\%}}


% Fonctions et systèmes
\newcommand{\fct}[5][t]{%
  \begin{array}[#1]{rcl}
    #2 & \rightarrow & #3\\
    #4 & \mapsto     & #5\\
  \end{array}
}
\newcommand{\fonction}[5]{#1 : \left\{\begin{array}{rcl} #2& \longrightarrow &#3 \\ #4 &\longmapsto & #5\end{array}\right.}
\newenvironment{systeme}{\left\{ \begin{array}{rcl}}{\end{array}\right.}

% Matrices
\newcommand{\mat}[1]{
  \begin{pmatrix}
    #1
  \end{pmatrix}
}
\newcommand{\inv}{\ensuremath{^{-1}}}
\newcommand{\bpm}{\begin{pmatrix}}
\newcommand{\epm}{\end{pmatrix}}


% bases de données
\newcommand{\relat}[1]{\textsc{#1}}
\newcommand{\attr}[1]{\emph{#1}}
\newcommand{\prim}[1]{\uline{#1}}
\newcommand{\foreign}[1]{\#\textsl{#1}}


% Bases de données

\newcommand{\att}{\ensuremath{\mathbf{att}}}
\newcommand{\dom}{\ensuremath{\mathbf{dom}}}
\newcommand{\sort}{\ensuremath{\mathbf{sort}}}
\newcommand{\relname}{\ensuremath{\mathbf{relname}}}
\newcommand{\var}{\ensuremath{\mathbf{var}}}
\newcommand{\FILM}{\ensuremath{\mathtt{FILM}}}
\newcommand{\JOUE}{\ensuremath{\mathtt{JOUE}}}
\newcommand{\PERSONNE}{\ensuremath{\mathtt{PERSONNE}}}
\newcommand{\PERSONNAGE}{\ensuremath{\mathtt{PERSONNAGE}}}

\newcommand{\ttid}{\ensuremath{\mathtt{id}}}
\newcommand{\tttitre}{\ensuremath{\mathtt{titre}}}
\newcommand{\ttdate}{\ensuremath{\mathtt{date}}}
\newcommand{\ttidr}{\ensuremath{\mathtt{idrealisateur}}}
\newcommand{\ttdatenais}{\ensuremath{\mathtt{datenaissance}}}
\newcommand{\ttnom}{\ensuremath{\mathtt{nom}}}
\newcommand{\ttprenom}{\ensuremath{\mathtt{prenom}}}
\newcommand{\ttidacteur}{\ensuremath{\mathtt{idacteur}}}
\newcommand{\ttidfilm}{\ensuremath{\mathtt{idfilm}}}
\newcommand{\ttidpersonnage}{\ensuremath{\mathtt{idpersonnage}}}

\newcommand{\fv}{\mathrm{libre}}
\newcommand{\sem}[1]{[\![ #1 ]\!]}

\input{style/macros_Titres}
\input{style/macros_Frames}

%Si le boolen xp est vrai : compilation pour xabi
%Sinon compilation Damien
\newboolean{xp}
\setboolean{xp}{true}

\newboolean{prof}
\setboolean{prof}{false}

\usepackage[%
    pdftitle={Devoir Surveillé 3},
    pdfauthor={Xavier Pessoles},
    colorlinks=true,
    linkcolor=blue,
    citecolor=magenta]{hyperref}


\def\discipline{Informatique}
\def\xxtitre{\ifthenelse{\boolean{xp}}{
Devoir Surveillé 3 -- 1 heure
}{
Chapitre  -- }}

\def\xxsoustitre{\ifthenelse{\boolean{xp}}{
Algorithmique et programmation}{
Partie  -- }}

\def\xxauteur{\ifthenelse{\boolean{xp}}{
Xavier \textsc{Pessoles}}{
Damien \textsc{Iceta} \\ Xavier \textsc{Pessoles}}}

\def\xxpied{\ifthenelse{\boolean{xp}}{
DS 03 -- Sujet}{
\xxtitre}}

\def\xxcathegorie{\ifthenelse{\boolean{xp}}{
2013 -- 2014 \\
Xavier \textsc{Pessoles}}{
Informatique - Cours}}





%---------------------------------------------------------------------------


\begin{document}

\ifthenelse{\boolean{xp}}{\usepackage[%
    pdftitle={Représentation des nombres},
    pdfauthor={Xavier Pessoles},
    colorlinks=true,
    linkcolor=blue,
    citecolor=magenta]{hyperref}

\usepackage{pifont}
%\usepackage{lastpage}

% \makeatletter \let\ps@plain\ps@empty \makeatother
%% DEBUT DU DOCUMENT
%% =================
\sloppy
\hyphenpenalty 10000


\colorlet{shadecolor}{orange!15}

\newtheorem{theorem}{Theorem}


\begin{document}


%\newboolean{prof}
%\setboolean{prof}{true}
% \makeatletter \let\ps@plain\ps@empty \makeatother
%% DEBUT DU DOCUMENT
%% =================




%------------- En tetes et Pieds de Pages ------------


\pagestyle{fancy}
\ifthenelse{\boolean{xp}}{%
\renewcommand{\headrulewidth}{0pt}}{%
\renewcommand{\headrulewidth}{0.2pt}} %pour mettre le trait en haut
%\renewcommand{\headrulewidth}{0.2pt}

\fancyhead{}
\fancyhead[L]{%
\noindent\begin{minipage}[c]{2.6cm}%
\includegraphics[width=2cm]{png/logo_ptsi.png}%
\end{minipage}}


\fancyhead[C]{\rule{12cm}{.5pt}}



\fancyhead[R]{%
\noindent\begin{minipage}[c]{3cm}
\begin{flushright}
\footnotesize{\textit{\textsf{Informatique}}}%
\end{flushright}
\end{minipage}
}



\fancyhead[C]{\rule{12cm}{.5pt}}

\renewcommand{\footrulewidth}{0.2pt}

\fancyfoot[C]{\footnotesize{\bfseries \thepage}}
\fancyfoot[L]{%
\begin{minipage}[c]{.2\linewidth}
\noindent\footnotesize{{\xxauteur}}
\end{minipage}
\ifthenelse{\boolean{xp}}{}{%
\begin{minipage}[c]{.15\linewidth}
\includegraphics[width=2cm]{png/logoCC.png}
\end{minipage}}
}

\ifthenelse{\boolean{prof}}{%
\fancyfoot[R]{\footnotesize{\xxpied}}}

\begin{center}
 \huge\textsc{\xxtitre}
\end{center}

\begin{center}
 \LARGE\textsc{\xxsoustitre}
\end{center}

\vspace{.5cm}
}{\input{style/enteteDI}}



\subsection*{Avant-propos -- Calcul d'une puissance}
On souhaite calculer la puissance $b$ d'un nombre $x$ : $x^b$ avec $x\in\mathbb{R}$ et $b\in\mathbb{N}$. On utilise pour cela la fonction $\textsf{expo}$ basée sur un algorithme naïf prenant comme argument un entier naturel $b$ et un nombre réel $x$ :

\begin{py}
\begin{python}
def expo(x,b):
    res = 1
    j=b
    inv = x
    while j>=1:
        res = res * x
        j=j-1
    return res
\end{python}
\end{py}


\subparagraph{}
\textit{Proposer une autre formulation de l'algorithme de la fonction \textsf{expo} en utilisant une boucle $\textsf{for}$.}

%\begin{corrige}
%\begin{python}
%def expo(x,n):
%    res = 1
%    for i in range(n):
%        res =res*x
%    return res
%\end{python}
%\end{corrige}
\subparagraph{}
\textit{On conserve la fonction \textsf{expo} utilisant la boucle \textsf{while}. Montrer que $j$ est un \textbf{variant} de boucle.}

\ifthenelse{\boolean{prof}}{
\begin{corrige}
La boucle \textsf{while} est conditionnée par $j>=1$. Par ailleurs, $j$ est toujours positif est décroit à chaque boucle. $j$ est donc un variant de boucle. Il nous assure que l'algorithme se terminera.
\end{corrige}}{}

\subparagraph{}
\textit{On conserve toujours la fonction \textsf{expo} utilisant la boucle \textsf{while}. Montrer que la propriété $\mathcal{P}(n)$ $x^b = inv_n^{j_n}\cdot res_n$ est un \textbf{invariant} de boucle.}


\ifthenelse{\boolean{prof}}{
\begin{corrige}
\begin{enumerate}
\item Initialement, $res=1$, $j=n$.
\item L'invariant de boucle suggéré est $x^b = inv_n^{j_n}\cdot res_n$.
\item Montrons la validité de notre invariant :
\begin{itemize}
\item au rang $0$ : $j_0=b$, $inv_0=x$, $res_0=1$. On a donc $ inv_0^{j_0}\cdot res_0 = x^b\cdot 1 = x^b$. La propriété est donc vraie. 
\item au rang $n$ : supposons que la propriété $inv_n^{j_n}\cdot res_n$ vraie.
\item au rang $n+1$ : $j=j_n-1$, $res_{n+1}=res_n\cdot x$, $inv_n=x$. On a donc :
$inv_{n+1}^{j_{n+1}}\cdot res_{n+1} = x ^{j_n-1}\cdot res_n\cdot x =  x ^{j_n}\cdot x^{-1}\cdot res_n\cdot x = x ^{j_n}\cdot res_n = x^b $. La propriété est donc vérifiée au rang $n+1$. 
\end{itemize}
\item La terminaison du programme est vérifiée par l'existence du variant de boucle $j$. 
\item En sortie de boucle, $j=0$, et $res_n = x^b$. En conséquence, l'invariant de boucle est encore vrai.
\end{enumerate}
\end{corrige}
}{}


\subparagraph{}
\textit{On note $C_e$ le coût d'une opération élémentaire (affectation, opération mathématique simple, incrémentation de boucle, comparaison). Évaluer la complexité temporelle de l'algorithme proposé dans la fonction \textsf{expo}.}

\ifthenelse{\boolean{prof}}{
\begin{corrige}
La fonction $\textsf{exo}$  est constituée :
\begin{itemize}
\item trois affectations de coût $C_e$ (coût total $3_Ce$);
\item une boucle \textsf{while} qui doit s'exécuter $b$ fois et qui est constituée :
\begin{itemize}
\item de deux instructions composées de de 2 affectations et de deux opérations élémentaires (coût total $4_Ce$);
\end{itemize}
\item du coût du \textsf{return} de coût $C_e$. 
\end{itemize}
Au final, le coût temporel est de :
$$
C_T(b)=3\cdot C_e + b\cdot 4 C_e + C_e
$$

Ainsi, $C_T(b)\underset{+\infty}{\sim}4C_e b$. La complexité algorithmique est donc linéaire (en $\mathcal{O}(n)$).
\end{corrige}
}{}


\subparagraph{}
\textit{Citer une méthode plus efficace permettant de calculer $x^b$. Détailler brièvement son fonctionnement et préciser sa complexité temporelle.}
\ifthenelse{\boolean{prof}}{
\begin{corrige}
La méthode d'exponentiation rapide permet de calculer plus rapidement $x^b$. Sa complexité est en $\mathcal{O}(log(n))$. Pour rappel, $x^b$ se calcule ainsi : 
$$
x^b\left\{
\begin{array}{l}
\text{si }b=0 \quad x^b=1 \\
\text{si } $b$ \text{ est pair, } x^b=x^{\dfrac{b}{2}}\cdot x^{\dfrac{b}{2}}\\
\text{si } $b$ \text{ est impair, } x^b=x^{b-1}\cdot x \\
\end{array}
\right.
$$

\end{corrige}}{}



\subsection*{Calcul de polynômes}
On cherche à évaluer un polynôme en différentes valeurs. On note :
$$
\forall x \in \mathbb{R} \quad P(x) = \sum\limits_{i=0}^n a_i x^i
$$

Les coefficients $a_i$ du polynôme sont des entiers positifs stockés dans un tableau \textsf{a} tels que $a=[a_0,a_1,a_2,...,a_n]$. La fonction suivante appelée \textsf{evaluer} prend comme argument un nombre flottant $x$ et un tableau $a$ contenant les coefficients $a_i$ du polynôme.
Ainsi, si $a=[0,1,2,3]$, alors $a[0]=a_0$, $a[1]=a_1$, etc. alors $P(x)=x+2x^2+3x^3$. La fonction  \textsf{evaluer} retourne $P(x)$.

\begin{py}
\begin{python}
def evaluer(a,x):
    for i in range(len(a)):
        res = res+a[i]*expo(x,i) 
    return res
\end{python}
\end{py}


\subparagraph{}
\textit{La fonction \textsf{evaluer} a-t-elle l'effet désiré ? Si non, modifier le programme.}
\ifthenelse{\boolean{prof}}{
\begin{corrige}
Il est nécessaire d'initialiser la variable \textsf{res} à 0. 
\end{corrige}}{}

\subparagraph{}
\textit{Estimer la complexité algorithmique de la fonction \textsf{evaluer}.}
\ifthenelse{\boolean{prof}}{
\begin{corrige}
Pour un polynôme de degré $n$, la boucle \textsf{for} s'exécutera $n+1$ fois. 

Au rang $i$, le coût de la fonction \textsf{expo} est $3\cdot C_e + i\cdot 4 C_e + C_e$.

Le coût d'un incrément de boucle est donc $C(i)=3\cdot C_e + i\cdot 4 C_e + C_e + 4 C_e$

On a donc un coût total $C(i)=\sum\limits_{0}^{n+1}C(i)$.

On peut donc en conclure que la complexité sera en $\mathcal{O}(n^2)$.
\end{corrige}}{}
\subsection*{Méthode de Horner}

Afin de diminuer le coût temporel de l'évaluation d'un polynôme, il est possible d'utiliser la méthode de Horner. Elle consiste en une réécriture du polynôme $P(x)$ : 
$$
P(x) = a_0 + x\left(a_1 + x \left(a_2 + x \left(a_3 + .... \right) \right) \right)
$$

Ainsi le polynôme $P(x)=x+2x^2+3x^3$ est réécrit ainsi :  
$P(x)=0+x\left( 1 + x\left(2+ 3x \right) \right)$.

\begin{py}
\begin{python}
def horner(a,x):
    res=0
    n = len(a)-1
    while n>=0:
        res = a[n]+x*res
        n=n-1 
    return res
\end{python}
\end{py}

\subparagraph{}
\textit{On prend $a=[0,1,2,3]$ et $x=2$. En remplissant un tableau, donner l'évolution des variables $res$ et $n$ à chaque incrément de boucle.}
\ifthenelse{\boolean{prof}}{
\begin{corrige}
\begin{center}
\begin{tabular}{|c|c|c|}
\hline
$n$ & res(x) & res \\
\hline 
3 & $res=3 + x\cdot 0 = 3$ & 3 \\
\hline 
2 & $res=2 + x\cdot 3 $ & 8  \\
\hline 
1 & $res=1 + x\left(2 + x\cdot 3\right) =1+ 2x + 3x^2 $ & 17 \\
\hline 
0 & $res=0 + x\left( 1+ 2x + 3x^2 \right)= x+2x^2 + 3x^3$  & 34\\
\hline 
\end{tabular}
\end{center}
\end{corrige}}{}

\subparagraph{}
\textit{Expliquer en quoi l'algorithme proposé répond à la réécriture du polynôme $P(x)$ suivant la méthode de Horner ?}
\ifthenelse{\boolean{prof}}{
\begin{corrige}
Cf question précédente. 
\end{corrige}}{}

\subparagraph{}
\textit{Estimer la complexité algorithmique de la fonction \textsf{horner}. Conclure sur l'intérêt de cet algorithme.}
\ifthenelse{\boolean{prof}}{
\begin{corrige}
On constate directement que la complexité de l'algorithme est linéaire ce qui lui confère une plus grande rapidité que la méthode naïve. 
\end{corrige}}{}

\subsection*{Intégration numérique}
On cherche maintenant à intégrer numériquement $P(x)$ sur l'intervalle $[u,v]$ par la méthode des rectangles à gauche :

$$
I = \int\limits_{u}^{v} P(x) \; \mathrm{d}x
$$

\subparagraph{}
\textit{Écrire la fonction \textsf{integrale\_rectangle} prenant comme argument le nombre d'échantillons $n$, le tableau $a$ des coefficients du polynôme ainsi que $u$ et $v$ les bornes de l'intégrale et retournant la valeur $I$ de l'intégrale. }
\ifthenelse{\boolean{prof}}{
\begin{corrige}

\end{corrige}}{}
%\begin{corrige}
%\begin{py}
%\begin{python}
%def integrale_rectangle(n,u,v,a):
%    res = 0
%    pas = (v-u)/n)
%    for i in range(0,n): 
%        val = a+pas*i
%        res = res + pas*horner(a,val)
%    return res
%\end{python}    
%\end{py}    
%\end{corrige}
\subparagraph{}
\textit{Quel est l'ordre de grandeur de l'erreur effectuée sur le calcul de l'intégrale.}
\ifthenelse{\boolean{prof}}{
\begin{corrige}
Pour $n$ échantillons, l'erreur peut être majorée par $\dfrac{M}{2n}$ avec $M$ le sup de $P'(x)$ sur l'intervalle $[u,v]$.

\vspace{1cm}
Il est à noter qu'utiliser la méthode des rectangles pour calculer l'intégrale d'un polynôme n'est pas forcément judicieux. En effet, il est aisé de trouver une primitive de $P(x)$. 

%Ainsi, si $\forall x \in \mathbb{R}$,  $ P(x) = \sum\limits_{i=0}^n a_i x^i$, alors, 
%$$
%I = \int\limits_u^{v}\sum\limits_{i=0}^n a_i x^i \;\mathrm{d}x = 
%\sum\limits_{i=0}^n\int\limits_u^{v} a_i x^i \;\mathrm{d}x = 
%\sum\limits_{i=0}^n \left( \dfrac{a_i}{i+1}x^{i+1}\right)
%$$
%
% Par la méthode des rectangles à gauche, en notant $p$ le nombre d'échantillons, on peut donc approximer $I$ par $I_m$ avec  
% $$
% I_p=\sum\limits_{i=0}^{p-1} \dfrac{v-u}{p} \cdot P\left(u+i \dfrac{v-u}{p}\right) 
% $$


\end{corrige}}{}

\end{document}


