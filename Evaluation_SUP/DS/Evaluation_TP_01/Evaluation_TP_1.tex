\documentclass[10pt,oneside]{article}
\input{style/coursHeadings}
\usepackage{algorithm}
\usepackage{algorithmic}


% Python sources
\usepackage{listings}
\usepackage{textcomp}
\usepackage{setspace}
%\usepackage{palatino}

%\usepackage{color}
\definecolor{Bleu}{rgb}{0.1,0.1,1.0}
\definecolor{Noir}{rgb}{0,0,0}
\definecolor{Grau}{rgb}{0.5,0.5,0.5}
\definecolor{DunkelGrau}{rgb}{0.15,0.15,0.15}
\definecolor{Hellbraun}{rgb}{0.5,0.25,0.0}
\definecolor{Magenta}{rgb}{1.0,0.0,1.0}
\definecolor{Gris}{gray}{0.5}
\definecolor{Vert}{rgb}{0,0.5,0}
\definecolor{SourceHintergrund}{rgb}{1,1.0,0.95}

%
\renewcommand{\lstlistlistingname}{Listings}
\renewcommand{\lstlistingname}{Listing}

\lstnewenvironment{python}[1][]{
\lstset{
language=python,
basicstyle=\ttfamily\footnotesize\setstretch{1}, 	
stringstyle=\color{red}, 
showstringspaces=false, 
alsoletter={1234567890},
otherkeywords={\ , \}, \{},
keywordstyle=\color{blue},
emph={access,and,break,class,continue,def,del,elif ,else,
except,exec,finally,for,from,global,if,import,in,i s,
lambda,not,or,pass,print,raise,return,try,while},
emphstyle=\color{black}\bfseries,
emph={[2]True, False, None, self},
emphstyle=[2]\color{green},
emph={[3]from, import, as},
emphstyle=[3]\color{blue},
upquote=true,
morecomment=[s]{"""}{"""},
commentstyle=\color{Hellbraun}\slshape, 
%emph={[4]1, 2, 3, 4, 5, 6, 7, 8, 9, 0},
emphstyle=[4]\color{blue},
literate=*{:}{{\textcolor{blue}:}}{1}
{=}{{\textcolor{blue}=}}{1}
{-}{{\textcolor{blue}-}}{1}
{+}{{\textcolor{blue}+}}{1}
{*}{{\textcolor{blue}*}}{1}
{!}{{\textcolor{blue}!}}{1}
{(}{{\textcolor{blue}(}}{1}
{)}{{\textcolor{blue})}}{1}
{[}{{\textcolor{blue}[}}{1}
{]}{{\textcolor{blue}]}}{1}
{<}{{\textcolor{blue}<}}{1}
{>}{{\textcolor{blue}>}}{1},
%framexleftmargin=1mm, framextopmargin=1mm, frame=shadowbox, rulesepcolor=\color{blue},#1
backgroundcolor=\color{SourceHintergrund}, 
framexleftmargin=1mm, framexrightmargin=1mm, framextopmargin=1mm, frame=single, framerule=1pt, rulecolor=\color{black},#1
}}{}


%Si le boolen xp est vrai : compilation pour xabi
%Sinon compilation Damien
\newboolean{xp}
\setboolean{xp}{true}

\newboolean{prof}
\setboolean{prof}{false}

\def\xxtitre{\ifthenelse{\boolean{xp}}{
Évaluation de TP }{
}}


\def\xxsoustitre{\ifthenelse{\boolean{xp}}{
CI 2 : Algorithmique et programmation}{
}}

\def\xxauteur{\ifthenelse{\boolean{xp}}{
Cédric \textsc{Lopez} \\ Xavier \textsc{Pessoles} }{% \\ Damien \textsc{Iceta}}{
Damien \textsc{Iceta} \\ Xavier \textsc{Pessoles}}}

\def\xxpied{\ifthenelse{\boolean{xp}}{
Évaluation TP  -- CI 2}{
\xxtitre}}

\def\xxcathegorie{\ifthenelse{\boolean{xp}}{
2013 -- 2014 \\
Xavier \textsc{Pessoles}\\Informatique -- DS 2}{
Informatique -- DS 2}}

\ifthenelse{\boolean{xp}}{\usepackage[%
    pdftitle={Représentation des nombres},
    pdfauthor={Xavier Pessoles},
    colorlinks=true,
    linkcolor=blue,
    citecolor=magenta]{hyperref}

\usepackage{pifont}
%\usepackage{lastpage}

% \makeatletter \let\ps@plain\ps@empty \makeatother
%% DEBUT DU DOCUMENT
%% =================
\sloppy
\hyphenpenalty 10000


\colorlet{shadecolor}{orange!15}

\newtheorem{theorem}{Theorem}


\begin{document}


%\newboolean{prof}
%\setboolean{prof}{true}
% \makeatletter \let\ps@plain\ps@empty \makeatother
%% DEBUT DU DOCUMENT
%% =================




%------------- En tetes et Pieds de Pages ------------


\pagestyle{fancy}
\ifthenelse{\boolean{xp}}{%
\renewcommand{\headrulewidth}{0pt}}{%
\renewcommand{\headrulewidth}{0.2pt}} %pour mettre le trait en haut
%\renewcommand{\headrulewidth}{0.2pt}

\fancyhead{}
\fancyhead[L]{%
\noindent\begin{minipage}[c]{2.6cm}%
\includegraphics[width=2cm]{png/logo_ptsi.png}%
\end{minipage}}


\fancyhead[C]{\rule{12cm}{.5pt}}



\fancyhead[R]{%
\noindent\begin{minipage}[c]{3cm}
\begin{flushright}
\footnotesize{\textit{\textsf{Informatique}}}%
\end{flushright}
\end{minipage}
}



\fancyhead[C]{\rule{12cm}{.5pt}}

\renewcommand{\footrulewidth}{0.2pt}

\fancyfoot[C]{\footnotesize{\bfseries \thepage}}
\fancyfoot[L]{%
\begin{minipage}[c]{.2\linewidth}
\noindent\footnotesize{{\xxauteur}}
\end{minipage}
\ifthenelse{\boolean{xp}}{}{%
\begin{minipage}[c]{.15\linewidth}
\includegraphics[width=2cm]{png/logoCC.png}
\end{minipage}}
}

\ifthenelse{\boolean{prof}}{%
\fancyfoot[R]{\footnotesize{\xxpied}}}

\begin{center}
 \huge\textsc{\xxtitre}
\end{center}

\begin{center}
 \LARGE\textsc{\xxsoustitre}
\end{center}

\vspace{.5cm}
}{\input{style/enteteDI}}


%---------------------------------------------------------------------------


\ifthenelse{\boolean{prof}}{
\begin{center}
 \large\textsc{Éléments de corrigés}
\end{center}
}{
%\begin{center}
% \large\textsc{CI 2 : Algorithmique et programmation}
%\end{center}
}
\vspace{.5cm}


\begin{obj}
\textbf{Consignes : 
\begin{itemize}
\item tous les programmes réalisés seront enregistrés sous la forme \textsf{Nom.Prenom.py};
\item ces programmes seront envoyés par mail lors des 5 dernières minutes de la séance;
\item les sujets seront restitués à la fin de la séance.
\end{itemize}}
\end{obj}




\begin{obj}
\textbf{Objectifs des exercices 1 et 2 : }
\begin{itemize}
\item Alg -- C1 : comprendre un algorithme et expliquer ce qu’il fait;
\item Alg -- C2 : modifier un algorithme existant pour obtenir un résultat différent (ici modifier un algorithme pour obtenir un résultat similaire;
\item Alg -- C4 : expliquer le fonctionnement d’un algorithme.
\end{itemize}
\end{obj}

\subsection*{Exercice 1 -- Déchiffrer un programme Python -- Sur feuille}

On donne le programme suivant en Python :
\begin{py}
\begin{minipage}[c]{.05\linewidth}
$\quad$
\end{minipage} \hfill
\begin{minipage}[c]{.75\linewidth}
\begin{python}
tab = [17, 38, 10, 25, 72, 4, 98, 32, 11]
N = len (tab)
tampon = tab [N - 1]
for i in range (0, N - 1, 1) :
    tab [N - 1 - i] = tab [N - 2 - i]
tab [0] = tampon
print (tab)
\end{python}
\end{minipage}
\end{py}


\subparagraph{}
\textit{Expliquer ce que fait le programme précédent. Pour cela : 
\begin{itemize}
\item décrire les instructions de chacune des lignes;
\item en utilisant un exemple simple, vous expliquerez comment évoluent chacune des variables;
\item vous donnerez l'objectif du programme.
\end{itemize}}

\subparagraph{}
\textit{Sur feuille, proposer un programme Python réalisant la même tâche avec une boucle \textsf{while} à la place de la boucle \textsl{for}.}


\setcounter{subparagraph}{0}
\subsection*{Exercice 2 -- Déchiffrer un programme Python -- Sur feuille}

On donne le programme suivant en Python :
\begin{py}
\begin{minipage}[c]{.05\linewidth}
$\quad$
\end{minipage} \hfill
\begin{minipage}[c]{.75\linewidth}
\begin{python}
tab = [17, 38, 10, 25, 72, 4, 98, 32, 11]
N = len (tab)
tampon = tab [0]
for i in range (1, N) :
    if tab [i] > tampon :
        tampon=tab[i]
print (tampon)
\end{python}
\end{minipage}
\end{py}


\subparagraph{}
\textit{Expliquer ce que fait le programme précédent. Pour cela : 
\begin{itemize}
\item décrire les instructions de chacune des lignes;
\item en utilisant un exemple simple, vous expliquerez comment évoluent chacune des variables;
\item vous donnerez l'objectif du programme.
\end{itemize}}


\setcounter{subparagraph}{0}
\subsection*{Exercice 3 -- Suite de Fibonacci -- Sur PC}

\begin{obj}
Objectifs : 
\begin{itemize}
\item Alg -- C3 : concevoir un algorithme répondant à un problème précisément posé;
\item Alg -- C5 : écrire des instructions conditionnelles avec alternatives, éventuellement imbriquées;
\item Alg -- C9 : choisir un type de données en fonction d’un problème à résoudre;
\item Alg -- C10 : concevoir l’en-tête (ou la spécification) d’une fonction, puis la fonction elle-même;
\item Alg -- C14 : documenter une fonction, un programme plus complexe.
\end{itemize}
\end{obj}

On considère la suite $(U_n)$ définie par : 
$$
U_0 = 0 \quad U_1 =1 \quad U_n = U_{n-1} + U_{n-2} \quad (n\geq 2)
$$
appelée suite de Fibonacci.

\subparagraph{}
\textit{Écrire une fonction Python appelée \textbf{fibonacci}, prenant en paramètre un entier $n$ et retournant le $n^{\text{ième}}$ terme de la suite de Fibonacci.}

\subparagraph{}
\textit{Déterminer, en utilisant la fonction fibonacci, les termes $U_5$, $U_{11}$, $U_{17}$ de la suite.}

\subparagraph{}
\textit{Écrire une fonction Python appelée \textbf{fibonacci\_liste}, prenant en paramètre un entier $n$ et retournant une liste contenant les $n$ premiers termes de la suite de Fibonacci.}

\subparagraph{}
\textit{Vérifier que vos fonctions répondent aux objectifs Alg -- C14.}


\setcounter{subparagraph}{0}
\subsection*{Exercice 4 -- Sur PC}
\subparagraph{}
\textit{Écrire deux fonctions Python, appelées \textsf{somme\_for} et \textsf{somme\_while}, prenant en paramètre un
entier $n$ et retournant la somme des n premiers entiers compris entre 1 (inclus) et $n$ (inclus). La
fonction \textsf{somme\_for} utilisera une boucle for tandis que la fonction \textsf{somme\_while} utilisera une boucle \textsf{while}.}

\subparagraph{}
\textit{Déterminer, en utilisant les fonctions \textsf{somme\_for} et \textsf{somme\_while}, la somme des entiers compris entre 1 et 10, entre 1 et 67 et entre 1 et 128.}

\setcounter{subparagraph}{0}
\subsection*{Exercice 5 - Recherche d'un mot dans une chaîne de caractère  -- Sur PC}

\begin{obj}
Objectifs : 
\begin{itemize}
\item Alg -- C11 : traduire un algorithme dans un langage de programmation.
\end{itemize}
\end{obj}

Le but de la fonction suivante est de savoir combien de fois un mot apparaît dans une chaîne :
\begin{pseudo}
\begin{algorithm}[H]
\Donnees{texte (String), mot(String)}

\Fonction{
Recherche (mot,texte):\\
nb\_mot $\gets$ 0\\
\Pour{i de 0 à longueur(texte)}{
\If{texte[i]=mot[0]}{
j$\gets$0\\
\While{j$\neq$longueur(mot)}{
\If{(i+j)$\geq$ longueur(texte)}{
\Retour{nb\_mot}}
\ElseIf{texte[i+j]!=mot[j]}{break}
j$\gets$ j+1
}
\If{j=longueur(mot)}{
nb\_mot $\gets$ nb\_mot+1}}}
\Retour{nb\_mot}}
\end{algorithm}
\end{pseudo}

\subparagraph{}
\textit{Retranscrire l'algorithme dans Python.}

\subparagraph{}
\textit{Tester son bon fonctionnement.}

\end{document}
