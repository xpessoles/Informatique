\usepackage[%
    pdftitle={\xxtitre},
    pdfauthor={\xxauteur},
    colorlinks=true,
    linkcolor=blue,
    citecolor=magenta]{hyperref}

\usepackage{pifont}


% \makeatletter \let\ps@plain\ps@empty \makeatother
%% DEBUT DU DOCUMENT
%% =================
\sloppy
\hyphenpenalty 10000

\newcommand{\Pointilles}[1][3]{%
\multido{}{#1}{\makebox[\linewidth]{\dotfill}\\[\parskip]
}}


\colorlet{shadecolor}{orange!15}

\newtheorem{theorem}{Theorem}


\begin{document}


\newboolean{prof}
\setboolean{prof}{true}
%------------- En tetes et Pieds de Pages ------------


\pagestyle{fancy}
\renewcommand{\headrulewidth}{0pt}
%\renewcommand{\headrulewidth}{0.2pt} %pour mettre le trait en haut

\fancyhead{}
\fancyhead[L]{%
%\footnotesize{\textit{\textsf{Lycée François Premier}}}%
\noindent\noindent\begin{minipage}[c]{2.6cm}
\includegraphics[width=2cm]{png/logo_ptsi.png}%
\end{minipage}
}

\fancyhead[C]{\rule{12cm}{.5pt}}  %pour mettre le petit trait en haut


\fancyhead[R]{%
\noindent\begin{minipage}[c]{3cm}
\begin{flushright}
\footnotesize{\textit{\textsf{Informatique}}}%
\end{flushright}
\end{minipage}
}

\renewcommand{\footrulewidth}{0.2pt}

\fancyfoot[C]{\footnotesize{\bfseries \thepage}}
\fancyfoot[L]{%
\begin{minipage}[c]{.2\linewidth}
%\noindent\footnotesize{{Damien Iceta}}
\noindent\footnotesize{\textsc{Xavier Pessoles}\\\textsc{Damien Iceta}}
\end{minipage}
%\begin{minipage}[c]{.15\linewidth}
%\includegraphics[width=2cm]{png/logoCC.png}
%\end{minipage}
}

\ifthenelse{\boolean{prof}}{%
\fancyfoot[R]{\footnotesize{\xxpiedd}}}

