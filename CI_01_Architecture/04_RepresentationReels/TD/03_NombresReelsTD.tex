\documentclass[11pt,oneside]{article}
\input{style/coursHeadings}
\usepackage{algorithm}
\usepackage{algorithmic}


% Python sources
\usepackage{listings}
\usepackage{textcomp}
\usepackage{setspace}
%\usepackage{palatino}

%\usepackage{color}
\definecolor{Bleu}{rgb}{0.1,0.1,1.0}
\definecolor{Noir}{rgb}{0,0,0}
\definecolor{Grau}{rgb}{0.5,0.5,0.5}
\definecolor{DunkelGrau}{rgb}{0.15,0.15,0.15}
\definecolor{Hellbraun}{rgb}{0.5,0.25,0.0}
\definecolor{Magenta}{rgb}{1.0,0.0,1.0}
\definecolor{Gris}{gray}{0.5}
\definecolor{Vert}{rgb}{0,0.5,0}
\definecolor{SourceHintergrund}{rgb}{1,1.0,0.95}

%
\renewcommand{\lstlistlistingname}{Listings}
\renewcommand{\lstlistingname}{Listing}

\lstnewenvironment{python}[1][]{
\lstset{
language=python,
basicstyle=\ttfamily\footnotesize\setstretch{1}, 	
stringstyle=\color{red}, 
showstringspaces=false, 
alsoletter={1234567890},
otherkeywords={\ , \}, \{},
keywordstyle=\color{blue},
emph={access,and,break,class,continue,def,del,elif ,else,
except,exec,finally,for,from,global,if,import,in,i s,
lambda,not,or,pass,print,raise,return,try,while},
emphstyle=\color{black}\bfseries,
emph={[2]True, False, None, self},
emphstyle=[2]\color{green},
emph={[3]from, import, as},
emphstyle=[3]\color{blue},
upquote=true,
morecomment=[s]{"""}{"""},
commentstyle=\color{Hellbraun}\slshape, 
%emph={[4]1, 2, 3, 4, 5, 6, 7, 8, 9, 0},
emphstyle=[4]\color{blue},
literate=*{:}{{\textcolor{blue}:}}{1}
{=}{{\textcolor{blue}=}}{1}
{-}{{\textcolor{blue}-}}{1}
{+}{{\textcolor{blue}+}}{1}
{*}{{\textcolor{blue}*}}{1}
{!}{{\textcolor{blue}!}}{1}
{(}{{\textcolor{blue}(}}{1}
{)}{{\textcolor{blue})}}{1}
{[}{{\textcolor{blue}[}}{1}
{]}{{\textcolor{blue}]}}{1}
{<}{{\textcolor{blue}<}}{1}
{>}{{\textcolor{blue}>}}{1},
%framexleftmargin=1mm, framextopmargin=1mm, frame=shadowbox, rulesepcolor=\color{blue},#1
backgroundcolor=\color{SourceHintergrund}, 
framexleftmargin=1mm, framexrightmargin=1mm, framextopmargin=1mm, frame=single, framerule=1pt, rulecolor=\color{black},#1
}}{}


%Si le boolen xp est vrai : compilation pour xabi
%Sinon compilation Damien
\newboolean{xp}
\setboolean{xp}{true}

%\newboolean{prof}
%\setboolean{prof}{true}

\def\xxtitre{\ifthenelse{\boolean{xp}}{
CI 1 : Architecture matérielle et logicielle}{
Chapitre 2 -- Représentation des nombres}}

\def\xxsoustitre{\ifthenelse{\boolean{xp}}{
Chapitres 3 \& 4 -- Principe de la représentation des nombres en mémoire}{
Partie 2 -- Principe de la représentation des nombres réels en mémoire}}

\def\xxauteur{\ifthenelse{\boolean{xp}}{
Xavier \textsc{Pessoles} \\ Damien \textsc{Iceta}}{
Damien \textsc{Iceta} \\ Xavier \textsc{Pessoles}}}

\def\xxpied{\ifthenelse{\boolean{xp}}{
Cours -- CI 1 : Architecture matérielle et logicielle\\
Représentation des nombres}{
\xxtitre}}

\def\xxcathegorie{\ifthenelse{\boolean{xp}}{
2013 -- 2014 \\
Xavier \textsc{Pessoles}}{
Informatique - Cours}}

\ifthenelse{\boolean{xp}}{\usepackage[%
    pdftitle={Représentation des nombres},
    pdfauthor={Xavier Pessoles},
    colorlinks=true,
    linkcolor=blue,
    citecolor=magenta]{hyperref}

\usepackage{pifont}
%\usepackage{lastpage}

% \makeatletter \let\ps@plain\ps@empty \makeatother
%% DEBUT DU DOCUMENT
%% =================
\sloppy
\hyphenpenalty 10000


\colorlet{shadecolor}{orange!15}

\newtheorem{theorem}{Theorem}


\begin{document}


%\newboolean{prof}
%\setboolean{prof}{true}
% \makeatletter \let\ps@plain\ps@empty \makeatother
%% DEBUT DU DOCUMENT
%% =================




%------------- En tetes et Pieds de Pages ------------


\pagestyle{fancy}
\ifthenelse{\boolean{xp}}{%
\renewcommand{\headrulewidth}{0pt}}{%
\renewcommand{\headrulewidth}{0.2pt}} %pour mettre le trait en haut
%\renewcommand{\headrulewidth}{0.2pt}

\fancyhead{}
\fancyhead[L]{%
\noindent\begin{minipage}[c]{2.6cm}%
\includegraphics[width=2cm]{png/logo_ptsi.png}%
\end{minipage}}


\fancyhead[C]{\rule{12cm}{.5pt}}



\fancyhead[R]{%
\noindent\begin{minipage}[c]{3cm}
\begin{flushright}
\footnotesize{\textit{\textsf{Informatique}}}%
\end{flushright}
\end{minipage}
}



\fancyhead[C]{\rule{12cm}{.5pt}}

\renewcommand{\footrulewidth}{0.2pt}

\fancyfoot[C]{\footnotesize{\bfseries \thepage}}
\fancyfoot[L]{%
\begin{minipage}[c]{.2\linewidth}
\noindent\footnotesize{{\xxauteur}}
\end{minipage}
\ifthenelse{\boolean{xp}}{}{%
\begin{minipage}[c]{.15\linewidth}
\includegraphics[width=2cm]{png/logoCC.png}
\end{minipage}}
}

\ifthenelse{\boolean{prof}}{%
\fancyfoot[R]{\footnotesize{\xxpied}}}

\begin{center}
 \huge\textsc{\xxtitre}
\end{center}

\begin{center}
 \LARGE\textsc{\xxsoustitre}
\end{center}

\vspace{.5cm}
}{\input{style/enteteDI}}


%---------------------------------------------------------------------------



\begin{flushright}
\textit{D'après ressources de Laurent Deschamps.}
\end{flushright}

\subsection*{Exercice 1}
\textit{Réalisez la conversion des nombres suivants dans les autres systèmes de numération :}
\begin{itemize}
\item $(10050)_{10}$;
\item $(343,56)_{10}$;
\item $(1001 0001)_{2}$;
\item $(A3F)_{16}$;
\item $(1C2A)_{16}$.
\end{itemize}

\subsection*{Exercice 2}
On désire utiliser 12 bits pour comptabiliser des objets.
\textit{
\begin{enumerate}
\item Quel est le nombre maximum d’objets qu’il est possible de compter ?
\item Indiquer le numéro du premier et du dernier (dans les systèmes de numération décimale, binaire et hexadécimale).
\end{enumerate}}

\subsection*{Exercice 3}
On désire compter 65 000 objets.
\textit{
\begin{enumerate}
\item Sur combien de bit peut-on réaliser cette opération ? 
\item Quel est le numéro du premier et du dernier (dans les systèmes de numération binaire et hexadécimale) ?
\end{enumerate}}

\subsection*{Exercice 4}
\textit{Effectuez les opérations arithmétiques suivantes dans les systèmes de numération binaire (codé sur 8 bits) :}
\begin{itemize}
\item $71+35$
\item $121-75$
\item $15-25$
\item $- 51 - 77$
\end{itemize}

\subsection*{Exercice 5}
Soit une machine où les nombres sont codés sur 8 bits.
\textit{
\begin{enumerate}
\item Donner le nombre le plus grand et le plus petit nombre représentable selon que le codage utilisé est non signé ou signé. 
\item Écrire dans le format signé les nombres décimaux 1, -1, 111 et 55.
\item Quelles sont  les valeurs décimales codées par $4C_{16}$ et $B4_{16}$ si le codage est signé ou non ?
\item Un périphérique de la machine lui délivre des données sur 8 bits dans le format valeur absolue + signe, la valeur absolue sur 7 bits est précédée d’un bit de signe valant 1 si positif ou nul, 0 sinon. Il transmet successivement $9A_{16}$ puis $3C_{16}$, quelles sont les valeurs signifiées ? 
\end{enumerate}}


\subsection*{Exercice 6}
\textit{Écrire dans le format flottant simple précision (IEEE 754) les nombres 
1,0 ; 	–1,0 ; 	15,25  et –3,26. Les résultats seront donnés en hexadécimal.}

\end{document}

\subsection*{Exercice 7}
\subsection*{Exercice 8}
\subsection*{Exercice 9}
\subsection*{Exercice 10}
\subsection*{Exercice 11}
\subsection*{Exercice 12}

\end{document}
