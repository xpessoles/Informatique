
%

\exer{Surfing Porquerolles} 

\textit{D'après Concours Mines Ponts 2018}
\begin{obj} ~\\
\begin{itemize}
\item Lire un fichier texte.
\item Analyser les données d'un fichiers. 
\end{itemize}
\end{obj}

Le fichier \texttt{vagues.txt} contient un relevé des niveaux d'eau mesurés par une bouée au large de Porquerolles. (Pour ne pas se mentir, on a plutôt généré un profil qui pourrait vaguement ressembler à un tel relevé.)

Il est constitué de deux colonnes, séparées par une virgule, la première colonne correspondant à une mesure de temps (en secondes), la seconde colonne correspondant à une mesure de niveau de hauteur d'eau (en mètres). 

\question{Écrire une fonction 
\texttt{lire\_fichier(file: str) -> list,list :} 
prenant comme argument le nom d'un fichier et renvoyant la liste des temps que l'on notera \texttt{les\_t} et la liste des niveaux de vagues que l'on notera \texttt{liste\_niveaux}.}

\question{Écrire une fonction 
\texttt{trace\_vagues(file: str) -> None :} 
prenant comme argument le nom d'un fichier affichant le profil des vagues en fonction du temps.}

Le résultat attendu est le suivant : 
......

\question{Écrire une fonction 
\texttt{moyenne(liste\_niveaux: list) -> float :} 
prenant comme argument une liste non vide \texttt{liste\_niveaux}, et retournant sa valeur moyenne.}


\question{Écrire une fonction
\texttt{ind\_premier\_pzd(liste\_niveaux: list) -> int : } 
retournant, s’il existe, l’indice
du premier élément de la liste tel que cet élément soit supérieur à la moyenne et l’élément suivant
soit inférieur à la moyenne. Cette fonction devra retourner -1 si aucun élément vérifiant cette condition n’existe.}
