%
\subsection*{Recherche séquentielle}

\exer{Exercices d'échauffement}

\begin{obj}
Rechercher séquentiellement un élément dans un tableau unidimensionnel ou dans un dictionnaire.
\end{obj}

Nous allons commencer par rechercher si un nombre est dans un tableau. 

Commençons par définir la liste des entiers pairs compris entre 0 et \texttt{nb} exclus.

\begin{lstlisting}
def generate_pair_01(nb: int) -> list :
    """
	Génération d'une liste de nb entiers compris entre 0 (exclus) et nb (exclus).
    """
    res = []
    for i in range(1,nb//2):
        res.append(2*i)
    return res
\end{lstlisting}

Recopier la fonction dans un terminal.

\question{Vérifier que la fonction  \texttt{generate\_pair\_01} fonctionne pour \texttt{nb=0}, \texttt{nb=9}, \texttt{nb=10}.}

\question{Écrire une fonction de signature  \texttt{generate\_pair\_02(nb: int) -> list} en utilisant une boucle \texttt{while}.}



Il est possible de définir la fonction \texttt{generate\_tab\_alea} différemment.
\begin{lstlisting}
def generate_tab_alea_02(deb: int,fin: int,nb: int) -> list :
    """
	Génération d'une liste de nb entiers compris entre deb (inclus) et fin (exclus).
    """
    return [rd.randrange(deb,fin) for i in range(nb)]

\end{lstlisting}



Pour cela commençons par générer une liste d'entiers aléatoires.




\begin{lstlisting}
import random as rd # Permet de charger une bibliothèque permettant de générer des nombres aléatoires.

def generate_tab_alea_01(deb: int,fin: int,nb: int) -> list :
    """
	Génération d'une liste de nb entiers compris entre deb (inclus) et fin (exclus).
    """
    res = []
    for i in range(nb):
        res.append(rd.randrange(deb,fin))
    return res
\end{lstlisting}


Il est possible de définir la fonction \texttt{generate\_tab\_alea} différemment.
\begin{lstlisting}
def generate_tab_alea_02(deb: int,fin: int,nb: int) -> list :
    """
	Génération d'une liste de nb entiers compris entre deb (inclus) et fin (exclus).
    """
    return [rd.randrange(deb,fin) for i in range(nb)]

\end{lstlisting}


Recherche d'un élément 
Recherche du maximum
Recherche du second maximum
