\input entete
	
	\large MPSI \hfill DICTIONNAIRE.
	
	\newcommand{\indente}{\hspace*{1cm}}
	
	\smallskip
	\hline
	
	\bigskip
	
\newcommand{\indente}{\hspace*{1cm}}
	
	%%%%%%%%%%%%%%%%%%%%%%%%%%%%%%%%%%%%%%%%%%%%%%%%%%%%%%%%%%%%%%%%%%%%%%%%%%%%%%%%%%%%%%%%%%%%%%%%%%%%%%%%%%%%%%%%%%%%%%%%%%%
	\vskip1cm
	
\bi\q Un dictionnaire est une suite de couple (clé, valeur) non ordonnée. Chaque élément est repéré par sa clé qui est donc unique.

\q Le dictionnaire vide est \texttt{dico=\{\}}

\q On peut définir un dictionnaire globalement:

\texttt{dico=\{”MPSI”:46,”PCSI”:47,”PTSI”,45\}}	ou “MPSI” est une clé, 46 sa valeur.

\q On peut ajouter des éléments à un dictionnaire:

\texttt{dico[”MP”]=45}

et afficher:

\texttt{print(dico)} renvoie: \texttt{dico=\{”MPSI”:46,”PCSI”:47,”PTSI”,45,”MP”:45\}}}

\texttt{print(dico[”MPSI”])} affiche 46

\q Ecrire \texttt{dico[”MP”]=46} enverra un message d'erreur car la clé est unique.
	
\q On peut supprimer un élément:

\texttt{del dico[”MP”]}	

\texttt{print(dico}} renvoie \texttt{dico=\{”MPSI”:46,”PCSI”:47,”PTSI”,45\}}

\q Test d'appartenance d'une clé:

\texttt{print(“PCSI” in dico)} qui renvoie ici \texttt{True} (ou \texttt{False} dans le cas général)

\q Nombre d'élémnts d'un dictionnaire:

\texttt{len(dico)}

\q Parcourir les éléments d'un dictionnaire:

\texttt{for elt in dico.items():}

\texttt{\indente a=elt[0]}  \# clé

\texttt{\indente b=elt[1]}   \# valeur
	
OU

\texttt{for clé, valeur in dico.items():}	

\texttt{\indente ...}
	
\q Copie d'un dictionnaire:

\texttt{import copy}

\texttt{dico1=copy.deepcopy(dico)}	

(\textit{et non \texttt{dico1=dico} qui va créer un alias seulement.})
	
	
	
	
	
	
	
	
	
	
	
	




	
\end{document}