\input entete
	
	\large MPSI \hfill TP05: Algorithmes dichotomiques.
	
	\newcommand{\indente}{\hspace*{1cm}}
	
	\smallskip
	\hline
	
	\bigskip
	

	
	%%%%%%%%%%%%%%%%%%%%%%%%%%%%%%%%%%%%%%%%%%%%%%%%%%%%%%%%%%%%%%%%%%%%%%%%%%%%%%%%%%%%%%%%%%%%%%%%%%%%%%%%%%%%%%%%%%%%%%%%%%%
	\vskip1cm
	
\section{Recherche dichotomique}

\subsection{Recherche d'un élément dans une liste triée}

On se donne une liste \texttt{L} de nombres de longueur \texttt{n}, {triée dans l'ordre croissant}, et un nombre \texttt{x0}. 

Pour chercher \texttt{x0}, on va couper la liste en deux moitiés et chercher dans la moitié qui encadre \texttt{x0} et ainsi de suite...

On appelle \texttt{g} l'indice de l'élément du début de la sous-liste dans laquelle on travaille et \texttt{d} l'indice de l'élément de fin.

Au début, \texttt{g =} \cache{0} et \texttt{d = } \cache{n - 1}

%On souhaite construire un algorithme admettant l'invariant suivant:
%\bigskip
%
%\centerline{\fbox{si \t{x0} est dans \t{L} alors \t{x0} est dans la sous-liste \t{L[g:d]} (\t{g} inclus et \t{d} exclu).}}

%\bigskip
%\newpage
\medskip 
On utilise la méthode suivante :\li
\q On compare \texttt{x0} à "l'élément du milieu"  \texttt{L[m]} avec \texttt{m = (g + d) // 2}}
%son indice est \t{m} $=$\cache{\t{ n//2} (division euclidienne)}

\q Si \texttt{x0 = L[m]}, on a trouvé \texttt{x0}, on peut alors s'arrêter.
\q Si \texttt{x0} $<$ \texttt{L[m]}, c'est qu'il faut chercher entre \texttt{L[g]} et  \texttt{L[m-1]}% (\texttt{L[m]} exclu).
}%dans \cache{la première moitié de la liste \t{L[g:m]}}\\
\q Si \texttt{x0} $>$ \texttt{L[m]}, c'est qu'il faut chercher  entre \texttt{L[m+1]} et \texttt{L[d]}% (\texttt{L[m]} exclu).
}\\
\finli

On poursuit jusqu'à ce qu'on a trouvé \texttt{x0} ou lorsque l'on a épuisé la liste \texttt{L}.

%\emph{Remarque} : On verra plus tard que l'algorithme est tellement rapide que cela ne change pas grand chose de s'arrêter avant et cela simplifie son écriture.

%En notant $g$ et $d$  les indices de gauche et de droite du morceau de la liste $l$ où l'on est en train de faire la %recherche, 

\be\q Illustrer la méthode avec les deux exemples suivants:


-   \texttt{x0 = 5} et \texttt{L} $= \begin{array}{|c|c|c|c|c|c|c|c|c|} 
\hline -3 & 5 & 7 & 10 & 11 & 14 & 17 & 21 & 30 \\ \hline
\end{array}$

% g=0\\d=8\\m=4,L[m]>x0
% g=0\\d=3\\m=1,L[m]=x0

\medskip 

- \texttt{x0 = 11} et \texttt{L} $= \begin{array}{|c|c|c|c|c|c|c|c|c|} 
\hline -2 & 1 & 2 & 7 & 8 & 10 & 13 & 16 & 17  \\ \hline
\end{array}$


%g=0\\d=8\\m=4,L[m]<x0
%g=5\\d=8\\m=6,L[m]>x0
%g=5\\d=5\\m=5,L[m]<x0
%g=6\\d=5


\q \'Ecrire une fonction \texttt{dichotomie(x0,L)} qui renvoie \texttt{True} ou \texttt{False} selon que \texttt{x0} figure ou non dans \texttt{L} par cette méthode.

\q Dans le pire des cas où \texttt{x0} ne figure pas dans \texttt{L}, donner un équivalent du nombre de comparaisons qui sont effectuées dans cette fonction. Comparer avec le nombre de comparaisons à faire dans le cas d'un algorithme de recherche séquentielle.

\ee

\subsection{Recherche d'un indice satisfaisant une condition donnée}

On se donne dans cette question une liste \texttt{L} strictement décroissante de réels contenant des termes strictement positifs et des termes strictement négatifs. 

\'Ecrire une fonction \texttt{recherche\_dicho(L)}, d'argument la liste \texttt{L}, qui renvoie l'indice ainsi que la valeur du dernier élément positif ou nul par une méthode dichotomique adaptée de la précédente.



\subsection{Recherche d'un zéro d'une fonction}


Soit une fonction $f:[a,b]\to\Rr$ ($a<b$) vérifiant : 
$$\cond{$f$ continue sur $[a,b]$\\$f(a).f(b)\<0$ ie $f(a)$ et $f(b)$ de signes opposés.}.$$

Le théorème des valeurs intermédiaires s'applique et assure que $f$ possède au moins un zéro $\ell$ entre $a$ et $b$. 

\medskip 

\begin{center}
	\begin{tikzpicture}[scale=2]
		\shorthandoff{:};
		\draw[->] (-2.5,0)--(2.5,0);
		\draw[->] (-2.25,-1.5)--(-2.25,1.5);
		\fenetre
		\draw[domain=-2:2, samples=200, very thick]  plot ({\x},{((\x)^5+3*(\x)-7)/34});
		\draw (-2,0)node{$\cdot$};
		\draw (-1,0)node{$\cdot$};
		\draw (1,0)node{$\cdot$};
		\draw (0,0)node{$\cdot$};
		\draw (2,0)node{$\cdot$};
		\draw (-2 , 0) node[below] {$g_0=a$};
		\draw (2 , 0) node[below] {$d_0=b$};
		\draw (1 , 0.15) node[allow] {$g_2=m_1$};
		\draw (2 , 0.15) node[allow] {$d_2=b$};
		\draw (2 , -0.2) node[below] {$d_1=b$};
		\draw (0 , 0) node[below] {$g_1=m_0$};
		\draw (1.26 , 0) node[below] {$\ell$};
	\end{tikzpicture}
\end{center}

L'idée consiste à créer une suite d'intervelles $[g_n,d_n]$ tels que pour tout entier naturel $n$, $$g_n\<\ell\<\d_n \textrm{ et } 0\< g_n-d_n=\Frac{g_{n-1}-d_{n-1}}2.$$

\medskip 

On considère $m_0 = \dfrac{g_0+d_0}{2}$ et on évalue $f(m_0)$ : 

\bigskip 

\begin{itemize}
	\item Si $f(m_0)\times f(d_0)\> 0$, on va poursuivre la recherche d'un zéro dans l'intervalle $[g_1,d_1]=[g_0,m_0]$

	\item Sinon,  on poursuit la recherche dans l'intervalle $[g_1,d_1]=[m_0,d_0]$. 


	\item On recommence alors en considérant $m_1 = \dfrac{g_1+d_1}{2}$ ...\\
\end{itemize}

\bigskip

\begin{enumerate}
	
	\item Si l'on souhaite que $g_n$ et $d_n$ soient des solutions approchées de $\ell$ à une précision $\varepsilon$, quelle est la condition d'arrêt de l'algorithme ? Préciser alors la valeur approchée de $\ell$ qui sera renvoyée par la fonction.
	
	\item \'Ecrire une fonction \texttt{recherche\_zero(f,a,b,epsilon)} qui renvoie une valeur approchée du zéro de \texttt{f} sur \texttt{[a,b]} a epsilon près.
	
	\item Tester la fonction avec $f:x\mapsto x^2-2$ sur $[0,2]$ et $\varepsilon=0.001$.
	
	\item Avec une erreur de $\varepsilon=\Frac 1{2^p}$, combien y a-t-il de comparaisons au final en fonction de $p$ ?
	
	
\end{enumerate}

\ee

\section{Valeur d'un polynôme par plusieurs méthodes}

\begin{enumerate} 
	
	\item On se propose de calculer une valeur $x^n$ par un algorithme rapide et par la méthode de l'exponentiation rapide.

	\begin{enumerate} 
	
	\item  Ecrire une fonction \texttt{exponaif(x,n)} d'arguments un réel $x$ et un entier naturel $n$, qui renvoie la valeur de $x^n$ par la méthode $x^n=x\times x \times ... \times x$ ($n$ termes).

	\medskip  Donner un équivalent du nombre d'opérations effectuées.

	\item La méthode de l'exponentiation rapide consiste à remarquer que $$x^n=\left\{\begin{tabular}{ccc} ${(x^2)}^{n/2}$ &si &$n$ pair \\ ${x\times (x^2)}^{(n-1)/2}$& si& $n$ impair\end{tabular}\right.$$

	\medskip 

	\'Ecrire une fonction \texttt{exporapide(x,n)} d'arguments un réel $x$ et un entier naturel $n$,  qui renvoie la valeur de $x^n$ par la méthode de l'exponentiation rapide.

	\medskip  Donner un équivalent du nombre d'opérations effectuées.

	\end{enumerate}


	\item On considère un polynôme $$P(x)=\displaystyle\sum_{k=0}^n a_k.x^k$$ que l'on modélisera en Python par la liste $P=[a_0,a_1,...,a_n]$. Dans la suite, on prendra pour tout $k\in\Nn$, $a_k=k$.

	\begin{enumerate}\item Ecrire une fonction \texttt{Pnaif(x,n)} qui renvoie $P(x)$ à l'aide de la fonction \texttt{exponaif}.

	\medskip Donner un équivalent du nombre d'opérations faites pour ce calcul et vérifier que la complexité est quadratique.

	\item Faire de même pour une fonction 
	\texttt{Prapide(x,n)} qui renvoie $P(x)$ à l'aide de la fonction \texttt{exporapide}. On peut montrer dans ce cas que la complexité est dominée par $n.\ln(n)$ (on admettra ce résultat).

	\item Une dernière méthode consiste à utiliser le schéma de Hörner:
	\[P(x)= (\ldots((a_nx+a_{n-1})x+a_{n-2})x+...+a_1)x+a_0}\]

	\'Ecrire une fonction \texttt{horner(x,L)} de paramètres un réel $x$ et une liste $L$ représentant un polynôme $P$, renvoie la valeur de $P(x)$ par la méthode de Horner.

	\medskip Compter le nombre d'opérations au total pour calculer $P(x)$ et en donner un équivalent lorsque $n\to +\infty$.

	\end{enumerate}


	\item On désire maintenant visualiser les temps d'éxécution des trois fonctions précédentes pour des grandes valeurs de $n$.

	\begin{enumerate}\item Définir la liste $N$ des entiers naturels compris entre 0 et 100.

	\item Grâce à la fonction \texttt{perf\_counter} de la bibliothèque \texttt{time},  écrire une fonction 
	
	\texttt{Temps\_calcul(x)}qui:

	-  définit 3 listes \texttt{Tn}, \texttt{Tr} et \texttt{Th} contenant les temps de calcul de $P(x)$ pour $P=\displaystyle\sum_{k=0}^n k.x^k$ lorsque $n$ décrit $N$ avec respectivement la méthode naïve, la méthode rapide puis la méthode de Horner.

	- trace les trois courbes  \texttt{Tn}, \texttt{Tr} et \texttt{Th} en fonction de $N$ (on prendra $x=2$). Interpréter le résultat.

	\end{enumerate}

\end{enumerate}



















	
\end{document}