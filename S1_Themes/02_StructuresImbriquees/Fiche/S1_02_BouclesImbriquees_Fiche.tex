\fichetrue
\proftrue
\tdfalse
\coursfalse

\def\xxnumchapitre{Révisions 1 \vspace{.2cm}}
\def\xxchapitre{\hspace{.12cm} Définitions préliminaires et détermination des performances}
\def\xxYCartouche{-2.25cm}
\def\xxposongletx{2}
\def\xxposonglettext{1.45}
\def\xxposonglety{19}%16

\def\xxonglet{\textsl{Cy 01 -- Rév 1}}

\def\xxactivite{Fiche}


\def\xxpied{%
Cycle 01 -- Modéliser le comportement des systèmes multiphysiques\\
Révision 1 -- \xxactivite%
}

\setcounter{secnumdepth}{5}
%---------------------------------------------------------------------------

\pagestyle{empty}


%%%%%%%% PAGE DE GARDE COURS
\ifcours
% ==== BANDEAU DES TITRES ==== 
\begin{tikzpicture}[remember picture,overlay]
\node at (current page.north west)
{\begin{tikzpicture}[remember picture,overlay]
\node[anchor=north west,inner sep=0pt] at (0,0) {\includegraphics[width=\paperwidth]{\thechapterimage}};
\draw[anchor=west] (-2cm,-8cm) node [line width=2pt,rounded corners=15pt,draw=ocre,fill=white,fill opacity=0.6,inner sep=40pt]{\strut\makebox[22cm]{}};
\draw[anchor=west] (1cm,-8cm) node {\huge\sffamily\bfseries\color{black} %
\begin{minipage}{1cm}
\rotatebox{90}{\LARGE\sffamily\textsc{\color{ocre}\textbf{\xxnumpartie}}}
\end{minipage} \hfill
\begin{minipage}[c]{14cm}
\begin{titrepartie}
\begin{flushright}
\renewcommand{\baselinestretch}{1.1} 
\Large\sffamily\textsc{\textbf{\xxpartie}}
\renewcommand{\baselinestretch}{1} 
\end{flushright}
\end{titrepartie}
\end{minipage} \hfill
\begin{minipage}[c]{3.5cm}
{\large\sffamily\textsc{\textbf{\color{ocre} \discipline}}}
\end{minipage} 
 };
\end{tikzpicture}};
\end{tikzpicture}
% ==== FIN BANDEAU DES TITRES ==== 


% ==== ONGLET 
\begin{tikzpicture}[overlay]
\node[shape=rectangle, 
      rounded corners = .25 cm,
	  draw= ocre,
	  line width=2pt, 
	  fill = ocre!10,
	  minimum width  = 2.5cm,
	  minimum height = 3cm,] at (18.3cm,0) {};
\node at (17.7cm,0) {\rotatebox{90}{\textbf{\Large\color{ocre}{\classe}}}};
%{};
\end{tikzpicture}
% ==== FIN ONGLET 


\vspace{3.5cm}

\begin{tikzpicture}[remember picture,overlay]
\draw[anchor=west] (-2cm,-6cm) node {\huge\sffamily\bfseries\color{black} %
\begin{minipage}{2cm}
\begin{center}
\LARGE\sffamily\textsc{\color{ocre}\textbf{\xxactivite}}
\end{center}
\end{minipage} \hfill
\begin{minipage}[c]{15cm}
\begin{titrechapitre}
\renewcommand{\baselinestretch}{1.1} 
\Large\sffamily\textsc{\textbf{\xxnumchapitre}}

\Large\sffamily\textsc{\textbf{\xxchapitre}}
\vspace{.5cm}

\renewcommand{\baselinestretch}{1} 
\normalsize\normalfont
\xxcompetences
\end{titrechapitre}
\end{minipage}  };
\end{tikzpicture}
\vfill

\begin{flushright}
\begin{minipage}[c]{.3\linewidth}
\begin{center}
\xxfigures
\end{center}
\end{minipage}\hfill
\begin{minipage}[c]{.6\linewidth}
\startcontents
%\printcontents{}{1}{}
\printcontents{}{1}{}
\end{minipage}
\end{flushright}

\begin{tikzpicture}[remember picture,overlay]
\draw[anchor=west] (4.5cm,-.7cm) node {
\begin{minipage}[c]{.2\linewidth}
\begin{flushright}
\includegraphics[width=2cm]{logoCC}
\end{flushright}
\end{minipage}
\begin{minipage}[c]{.2\linewidth}
\textsl{\xxauteur} \\
\textsl{\classe}
\end{minipage}
 };
\end{tikzpicture}

\newpage
\pagestyle{fancy}

%\newpage
%\pagestyle{fancy}

\else
\fi
%% FIN PAGE DE GARDE DES COURS

%%%%%%%% PAGE DE GARDE TD
\iftd
%\begin{tikzpicture}[remember picture,overlay]
%\node at (current page.north west)
%{\begin{tikzpicture}[remember picture,overlay]
%\draw[anchor=west] (-2cm,-3.25cm) node [line width=2pt,rounded corners=15pt,draw=ocre,fill=white,fill opacity=0.6,inner sep=40pt]{\strut\makebox[22cm]{}};
%\draw[anchor=west] (1cm,-3.25cm) node {\huge\sffamily\bfseries\color{black} %
%\begin{minipage}{1cm}
%\rotatebox{90}{\LARGE\sffamily\textsc{\color{ocre}\textbf{\xxnumpartie}}}
%\end{minipage} \hfill
%\begin{minipage}[c]{13.5cm}
%\begin{titrepartie}
%\begin{flushright}
%\renewcommand{\baselinestretch}{1.1} 
%\Large\sffamily\textsc{\textbf{\xxpartie}}
%\renewcommand{\baselinestretch}{1} 
%\end{flushright}
%\end{titrepartie}
%\end{minipage} \hfill
%\begin{minipage}[c]{3.5cm}
%{\large\sffamily\textsc{\textbf{\color{ocre} \discipline}}}
%\end{minipage} 
% };
%\end{tikzpicture}};
%\end{tikzpicture}

%%%%%%%%%% PAGE DE GARDE TD %%%%%%%%%%%%%%%
%\begin{tikzpicture}[overlay]
%\node[shape=rectangle, 
%      rounded corners = .25 cm,
%	  draw= ocre,
%	  line width=2pt, 
%	  fill = ocre!10,
%	  minimum width  = 2.5cm,
%	  minimum height = 2.5cm,] at (18.5cm,0) {};
%\node at (17.7cm,0) {\rotatebox{90}{\textbf{\Large\color{ocre}{\classe}}}};
%%{};
%\end{tikzpicture}

% PARTIE ET CHAPITRE
%\begin{tikzpicture}[remember picture,overlay]
%\draw[anchor=west] (-1cm,-2.1cm) node {\large\sffamily\bfseries\color{black} %
%\begin{minipage}[c]{15cm}
%\begin{flushleft}
%\xxnumchapitre \\
%\xxchapitre
%\end{flushleft}
%\end{minipage}  };
%\end{tikzpicture}

% BANDEAU EXO
\iflivret % SI LIVRET
\begin{tikzpicture}[remember picture,overlay]
\draw[anchor=west] (-2cm,-3.3cm) node {\huge\sffamily\bfseries\color{black} %
\begin{minipage}{5cm}
\begin{center}
\LARGE\sffamily\color{ocre}\textbf{\textsc{\xxactivite}}

\begin{center}
\xxfigures
\end{center}

\end{center}
\end{minipage} \hfill
\begin{minipage}[c]{12cm}
\begin{titrechapitre}
\renewcommand{\baselinestretch}{1.1} 
\large\sffamily\textbf{\textsc{\xxtitreexo}}

\small\sffamily{\textbf{\textit{\color{black!70}\xxsourceexo}}}
\vspace{.5cm}

\renewcommand{\baselinestretch}{1} 
\normalsize\normalfont
\xxcompetences
\end{titrechapitre}
\end{minipage}};
\end{tikzpicture}
\else % ELSE NOT LIVRET
\begin{tikzpicture}[remember picture,overlay]
\draw[anchor=west] (-2cm,-4.5cm) node {\huge\sffamily\bfseries\color{black} %
\begin{minipage}{5cm}
\begin{center}
\LARGE\sffamily\color{ocre}\textbf{\textsc{\xxactivite}}

\begin{center}
\xxfigures
\end{center}

\end{center}
\end{minipage} \hfill
\begin{minipage}[c]{12cm}
\begin{titrechapitre}
\renewcommand{\baselinestretch}{1.1} 
\large\sffamily\textbf{\textsc{\xxtitreexo}}

\small\sffamily{\textbf{\textit{\color{black!70}\xxsourceexo}}}
\vspace{.5cm}

\renewcommand{\baselinestretch}{1} 
\normalsize\normalfont
\xxcompetences
\end{titrechapitre}
\end{minipage}};
\end{tikzpicture}

\fi

\else   % FIN IF TD
\fi


%%%%%%%% PAGE DE GARDE FICHE
\iffiche
\begin{tikzpicture}[remember picture,overlay]
\node at (current page.north west)
{\begin{tikzpicture}[remember picture,overlay]
\draw[anchor=west] (-2cm,-2.25cm) node [line width=2pt,rounded corners=15pt,draw=ocre,fill=white,fill opacity=0.6,inner sep=40pt]{\strut\makebox[22cm]{}};
\draw[anchor=west] (1cm,-2.25cm) node {\huge\sffamily\bfseries\color{black} %
\begin{minipage}{1cm}
\rotatebox{90}{\LARGE\sffamily\textsc{\color{ocre}\textbf{\xxnumpartie}}}
\end{minipage} \hfill
\begin{minipage}[c]{14cm}
\begin{titrepartie}
\begin{flushright}
\renewcommand{\baselinestretch}{1.1} 
\large\sffamily\textsc{\textbf{\xxpartie} \\} 

\vspace{.2cm}

\normalsize\sffamily\textsc{\textbf{\xxnumchapitre -- \xxchapitre}}
\renewcommand{\baselinestretch}{1} 
\end{flushright}
\end{titrepartie}
\end{minipage} \hfill
\begin{minipage}[c]{3.5cm}
{\large\sffamily\textsc{\textbf{\color{ocre} \discipline}}}
\end{minipage} 
 };
\end{tikzpicture}};
\end{tikzpicture}

\iflivret
\begin{tikzpicture}[overlay]
\node[shape=rectangle, 
      rounded corners = .25 cm,
	  draw= ocre,
	  line width=2pt, 
	  fill = ocre!10,
	  minimum width  = 2.5cm,
	  minimum height = 2.5cm,] at (18.5cm,1.1cm) {};
\node at (17.9cm,1.1cm) {\rotatebox{90}{\textsf{\textbf{\large\color{ocre}{\classe}}}}};
%{};
\end{tikzpicture}
\else
\begin{tikzpicture}[overlay]
\node[shape=rectangle, 
      rounded corners = .25 cm,
	  draw= ocre,
	  line width=2pt, 
	  fill = ocre!10,
	  minimum width  = 2.5cm,
%	  minimum height = 2.5cm,] at (18.5cm,1.1cm) {};
	  minimum height = 2.5cm,] at (18.6cm,1cm) {};
\node at (18cm,1cm) {\rotatebox{90}{\textsf{\textbf{\large\color{ocre}{\classe}}}}};
%{};
\end{tikzpicture}

\fi

\else
\fi



%\iflivret
%\pagestyle{empty}


%%%%%%%% PAGE DE GARDE COURS
\ifcours
% ==== BANDEAU DES TITRES ==== 
\begin{tikzpicture}[remember picture,overlay]
\node at (current page.north west)
{\begin{tikzpicture}[remember picture,overlay]
\node[anchor=north west,inner sep=0pt] at (0,0) {\includegraphics[width=\paperwidth]{\thechapterimage}};
\draw[anchor=west] (-2cm,-8cm) node [line width=2pt,rounded corners=15pt,draw=ocre,fill=white,fill opacity=0.6,inner sep=40pt]{\strut\makebox[22cm]{}};
\draw[anchor=west] (1cm,-8cm) node {\huge\sffamily\bfseries\color{black} %
\begin{minipage}{1cm}
\rotatebox{90}{\LARGE\sffamily\textsc{\color{ocre}\textbf{\xxnumpartie}}}
\end{minipage} \hfill
\begin{minipage}[c]{14cm}
\begin{titrepartie}
\begin{flushright}
\renewcommand{\baselinestretch}{1.1} 
\Large\sffamily\textsc{\textbf{\xxpartie}}
\renewcommand{\baselinestretch}{1} 
\end{flushright}
\end{titrepartie}
\end{minipage} \hfill
\begin{minipage}[c]{3.5cm}
{\large\sffamily\textsc{\textbf{\color{ocre} \discipline}}}
\end{minipage} 
 };
\end{tikzpicture}};
\end{tikzpicture}
% ==== FIN BANDEAU DES TITRES ==== 


% ==== ONGLET 
\begin{tikzpicture}[overlay]
\node[shape=rectangle, 
      rounded corners = .25 cm,
	  draw= ocre,
	  line width=2pt, 
	  fill = ocre!10,
	  minimum width  = 2.5cm,
	  minimum height = 3cm,] at (18.3cm,0) {};
\node at (17.7cm,0) {\rotatebox{90}{\textbf{\Large\color{ocre}{\classe}}}};
%{};
\end{tikzpicture}
% ==== FIN ONGLET 


\vspace{3.5cm}

\begin{tikzpicture}[remember picture,overlay]
\draw[anchor=west] (-2cm,-6cm) node {\huge\sffamily\bfseries\color{black} %
\begin{minipage}{2cm}
\begin{center}
\LARGE\sffamily\textsc{\color{ocre}\textbf{\xxactivite}}
\end{center}
\end{minipage} \hfill
\begin{minipage}[c]{15cm}
\begin{titrechapitre}
\renewcommand{\baselinestretch}{1.1} 
\Large\sffamily\textsc{\textbf{\xxnumchapitre}}

\Large\sffamily\textsc{\textbf{\xxchapitre}}
\vspace{.5cm}

\renewcommand{\baselinestretch}{1} 
\normalsize\normalfont
\xxcompetences
\end{titrechapitre}
\end{minipage}  };
\end{tikzpicture}
\vfill

\begin{flushright}
\begin{minipage}[c]{.3\linewidth}
\begin{center}
\xxfigures
\end{center}
\end{minipage}\hfill
\begin{minipage}[c]{.6\linewidth}
\startcontents
%\printcontents{}{1}{}
\printcontents{}{1}{}
\end{minipage}
\end{flushright}

\begin{tikzpicture}[remember picture,overlay]
\draw[anchor=west] (4.5cm,-.7cm) node {
\begin{minipage}[c]{.2\linewidth}
\begin{flushright}
\includegraphics[width=2cm]{logoCC}
\end{flushright}
\end{minipage}
\begin{minipage}[c]{.2\linewidth}
\textsl{\xxauteur} \\
\textsl{\classe}
\end{minipage}
 };
\end{tikzpicture}

\newpage
\pagestyle{fancy}

%\newpage
%\pagestyle{fancy}

\else
\fi
%% FIN PAGE DE GARDE DES COURS

%%%%%%%% PAGE DE GARDE TD
\iftd
%\begin{tikzpicture}[remember picture,overlay]
%\node at (current page.north west)
%{\begin{tikzpicture}[remember picture,overlay]
%\draw[anchor=west] (-2cm,-3.25cm) node [line width=2pt,rounded corners=15pt,draw=ocre,fill=white,fill opacity=0.6,inner sep=40pt]{\strut\makebox[22cm]{}};
%\draw[anchor=west] (1cm,-3.25cm) node {\huge\sffamily\bfseries\color{black} %
%\begin{minipage}{1cm}
%\rotatebox{90}{\LARGE\sffamily\textsc{\color{ocre}\textbf{\xxnumpartie}}}
%\end{minipage} \hfill
%\begin{minipage}[c]{13.5cm}
%\begin{titrepartie}
%\begin{flushright}
%\renewcommand{\baselinestretch}{1.1} 
%\Large\sffamily\textsc{\textbf{\xxpartie}}
%\renewcommand{\baselinestretch}{1} 
%\end{flushright}
%\end{titrepartie}
%\end{minipage} \hfill
%\begin{minipage}[c]{3.5cm}
%{\large\sffamily\textsc{\textbf{\color{ocre} \discipline}}}
%\end{minipage} 
% };
%\end{tikzpicture}};
%\end{tikzpicture}

%%%%%%%%%% PAGE DE GARDE TD %%%%%%%%%%%%%%%
%\begin{tikzpicture}[overlay]
%\node[shape=rectangle, 
%      rounded corners = .25 cm,
%	  draw= ocre,
%	  line width=2pt, 
%	  fill = ocre!10,
%	  minimum width  = 2.5cm,
%	  minimum height = 2.5cm,] at (18.5cm,0) {};
%\node at (17.7cm,0) {\rotatebox{90}{\textbf{\Large\color{ocre}{\classe}}}};
%%{};
%\end{tikzpicture}

% PARTIE ET CHAPITRE
%\begin{tikzpicture}[remember picture,overlay]
%\draw[anchor=west] (-1cm,-2.1cm) node {\large\sffamily\bfseries\color{black} %
%\begin{minipage}[c]{15cm}
%\begin{flushleft}
%\xxnumchapitre \\
%\xxchapitre
%\end{flushleft}
%\end{minipage}  };
%\end{tikzpicture}

% BANDEAU EXO
\iflivret % SI LIVRET
\begin{tikzpicture}[remember picture,overlay]
\draw[anchor=west] (-2cm,-3.3cm) node {\huge\sffamily\bfseries\color{black} %
\begin{minipage}{5cm}
\begin{center}
\LARGE\sffamily\color{ocre}\textbf{\textsc{\xxactivite}}

\begin{center}
\xxfigures
\end{center}

\end{center}
\end{minipage} \hfill
\begin{minipage}[c]{12cm}
\begin{titrechapitre}
\renewcommand{\baselinestretch}{1.1} 
\large\sffamily\textbf{\textsc{\xxtitreexo}}

\small\sffamily{\textbf{\textit{\color{black!70}\xxsourceexo}}}
\vspace{.5cm}

\renewcommand{\baselinestretch}{1} 
\normalsize\normalfont
\xxcompetences
\end{titrechapitre}
\end{minipage}};
\end{tikzpicture}
\else % ELSE NOT LIVRET
\begin{tikzpicture}[remember picture,overlay]
\draw[anchor=west] (-2cm,-4.5cm) node {\huge\sffamily\bfseries\color{black} %
\begin{minipage}{5cm}
\begin{center}
\LARGE\sffamily\color{ocre}\textbf{\textsc{\xxactivite}}

\begin{center}
\xxfigures
\end{center}

\end{center}
\end{minipage} \hfill
\begin{minipage}[c]{12cm}
\begin{titrechapitre}
\renewcommand{\baselinestretch}{1.1} 
\large\sffamily\textbf{\textsc{\xxtitreexo}}

\small\sffamily{\textbf{\textit{\color{black!70}\xxsourceexo}}}
\vspace{.5cm}

\renewcommand{\baselinestretch}{1} 
\normalsize\normalfont
\xxcompetences
\end{titrechapitre}
\end{minipage}};
\end{tikzpicture}

\fi

\else   % FIN IF TD
\fi


%%%%%%%% PAGE DE GARDE FICHE
\iffiche
\begin{tikzpicture}[remember picture,overlay]
\node at (current page.north west)
{\begin{tikzpicture}[remember picture,overlay]
\draw[anchor=west] (-2cm,-2.25cm) node [line width=2pt,rounded corners=15pt,draw=ocre,fill=white,fill opacity=0.6,inner sep=40pt]{\strut\makebox[22cm]{}};
\draw[anchor=west] (1cm,-2.25cm) node {\huge\sffamily\bfseries\color{black} %
\begin{minipage}{1cm}
\rotatebox{90}{\LARGE\sffamily\textsc{\color{ocre}\textbf{\xxnumpartie}}}
\end{minipage} \hfill
\begin{minipage}[c]{14cm}
\begin{titrepartie}
\begin{flushright}
\renewcommand{\baselinestretch}{1.1} 
\large\sffamily\textsc{\textbf{\xxpartie} \\} 

\vspace{.2cm}

\normalsize\sffamily\textsc{\textbf{\xxnumchapitre -- \xxchapitre}}
\renewcommand{\baselinestretch}{1} 
\end{flushright}
\end{titrepartie}
\end{minipage} \hfill
\begin{minipage}[c]{3.5cm}
{\large\sffamily\textsc{\textbf{\color{ocre} \discipline}}}
\end{minipage} 
 };
\end{tikzpicture}};
\end{tikzpicture}

\iflivret
\begin{tikzpicture}[overlay]
\node[shape=rectangle, 
      rounded corners = .25 cm,
	  draw= ocre,
	  line width=2pt, 
	  fill = ocre!10,
	  minimum width  = 2.5cm,
	  minimum height = 2.5cm,] at (18.5cm,1.1cm) {};
\node at (17.9cm,1.1cm) {\rotatebox{90}{\textsf{\textbf{\large\color{ocre}{\classe}}}}};
%{};
\end{tikzpicture}
\else
\begin{tikzpicture}[overlay]
\node[shape=rectangle, 
      rounded corners = .25 cm,
	  draw= ocre,
	  line width=2pt, 
	  fill = ocre!10,
	  minimum width  = 2.5cm,
%	  minimum height = 2.5cm,] at (18.5cm,1.1cm) {};
	  minimum height = 2.5cm,] at (18.6cm,1cm) {};
\node at (18cm,1cm) {\rotatebox{90}{\textsf{\textbf{\large\color{ocre}{\classe}}}}};
%{};
\end{tikzpicture}

\fi

\else
\fi



%\else
%\pagestyle{empty}


%%%%%%%% PAGE DE GARDE COURS
\ifcours
% ==== BANDEAU DES TITRES ==== 
\begin{tikzpicture}[remember picture,overlay]
\node at (current page.north west)
{\begin{tikzpicture}[remember picture,overlay]
\node[anchor=north west,inner sep=0pt] at (0,0) {\includegraphics[width=\paperwidth]{\thechapterimage}};
\draw[anchor=west] (-2cm,-8cm) node [line width=2pt,rounded corners=15pt,draw=ocre,fill=white,fill opacity=0.6,inner sep=40pt]{\strut\makebox[22cm]{}};
\draw[anchor=west] (1cm,-8cm) node {\huge\sffamily\bfseries\color{black} %
\begin{minipage}{1cm}
\rotatebox{90}{\LARGE\sffamily\textsc{\color{ocre}\textbf{\xxnumpartie}}}
\end{minipage} \hfill
\begin{minipage}[c]{14cm}
\begin{titrepartie}
\begin{flushright}
\renewcommand{\baselinestretch}{1.1} 
\Large\sffamily\textsc{\textbf{\xxpartie}}
\renewcommand{\baselinestretch}{1} 
\end{flushright}
\end{titrepartie}
\end{minipage} \hfill
\begin{minipage}[c]{3.5cm}
{\large\sffamily\textsc{\textbf{\color{ocre} \discipline}}}
\end{minipage} 
 };
\end{tikzpicture}};
\end{tikzpicture}
% ==== FIN BANDEAU DES TITRES ==== 


% ==== ONGLET 
\begin{tikzpicture}[overlay]
\node[shape=rectangle, 
      rounded corners = .25 cm,
	  draw= ocre,
	  line width=2pt, 
	  fill = ocre!10,
	  minimum width  = 2.5cm,
	  minimum height = 3cm,] at (18.3cm,0) {};
\node at (17.7cm,0) {\rotatebox{90}{\textbf{\Large\color{ocre}{\classe}}}};
%{};
\end{tikzpicture}
% ==== FIN ONGLET 


\vspace{3.5cm}

\begin{tikzpicture}[remember picture,overlay]
\draw[anchor=west] (-2cm,-6cm) node {\huge\sffamily\bfseries\color{black} %
\begin{minipage}{2cm}
\begin{center}
\LARGE\sffamily\textsc{\color{ocre}\textbf{\xxactivite}}
\end{center}
\end{minipage} \hfill
\begin{minipage}[c]{15cm}
\begin{titrechapitre}
\renewcommand{\baselinestretch}{1.1} 
\Large\sffamily\textsc{\textbf{\xxnumchapitre}}

\Large\sffamily\textsc{\textbf{\xxchapitre}}
\vspace{.5cm}

\renewcommand{\baselinestretch}{1} 
\normalsize\normalfont
\xxcompetences
\end{titrechapitre}
\end{minipage}  };
\end{tikzpicture}
\vfill

\begin{flushright}
\begin{minipage}[c]{.3\linewidth}
\begin{center}
\xxfigures
\end{center}
\end{minipage}\hfill
\begin{minipage}[c]{.6\linewidth}
\startcontents
%\printcontents{}{1}{}
\printcontents{}{1}{}
\end{minipage}
\end{flushright}

\begin{tikzpicture}[remember picture,overlay]
\draw[anchor=west] (4.5cm,-.7cm) node {
\begin{minipage}[c]{.2\linewidth}
\begin{flushright}
\includegraphics[width=2cm]{logoCC}
\end{flushright}
\end{minipage}
\begin{minipage}[c]{.2\linewidth}
\textsl{\xxauteur} \\
\textsl{\classe}
\end{minipage}
 };
\end{tikzpicture}

\newpage
\pagestyle{fancy}

%\newpage
%\pagestyle{fancy}

\else
\fi
%% FIN PAGE DE GARDE DES COURS

%%%%%%%% PAGE DE GARDE TD
\iftd
%\begin{tikzpicture}[remember picture,overlay]
%\node at (current page.north west)
%{\begin{tikzpicture}[remember picture,overlay]
%\draw[anchor=west] (-2cm,-3.25cm) node [line width=2pt,rounded corners=15pt,draw=ocre,fill=white,fill opacity=0.6,inner sep=40pt]{\strut\makebox[22cm]{}};
%\draw[anchor=west] (1cm,-3.25cm) node {\huge\sffamily\bfseries\color{black} %
%\begin{minipage}{1cm}
%\rotatebox{90}{\LARGE\sffamily\textsc{\color{ocre}\textbf{\xxnumpartie}}}
%\end{minipage} \hfill
%\begin{minipage}[c]{13.5cm}
%\begin{titrepartie}
%\begin{flushright}
%\renewcommand{\baselinestretch}{1.1} 
%\Large\sffamily\textsc{\textbf{\xxpartie}}
%\renewcommand{\baselinestretch}{1} 
%\end{flushright}
%\end{titrepartie}
%\end{minipage} \hfill
%\begin{minipage}[c]{3.5cm}
%{\large\sffamily\textsc{\textbf{\color{ocre} \discipline}}}
%\end{minipage} 
% };
%\end{tikzpicture}};
%\end{tikzpicture}

%%%%%%%%%% PAGE DE GARDE TD %%%%%%%%%%%%%%%
%\begin{tikzpicture}[overlay]
%\node[shape=rectangle, 
%      rounded corners = .25 cm,
%	  draw= ocre,
%	  line width=2pt, 
%	  fill = ocre!10,
%	  minimum width  = 2.5cm,
%	  minimum height = 2.5cm,] at (18.5cm,0) {};
%\node at (17.7cm,0) {\rotatebox{90}{\textbf{\Large\color{ocre}{\classe}}}};
%%{};
%\end{tikzpicture}

% PARTIE ET CHAPITRE
%\begin{tikzpicture}[remember picture,overlay]
%\draw[anchor=west] (-1cm,-2.1cm) node {\large\sffamily\bfseries\color{black} %
%\begin{minipage}[c]{15cm}
%\begin{flushleft}
%\xxnumchapitre \\
%\xxchapitre
%\end{flushleft}
%\end{minipage}  };
%\end{tikzpicture}

% BANDEAU EXO
\iflivret % SI LIVRET
\begin{tikzpicture}[remember picture,overlay]
\draw[anchor=west] (-2cm,-3.3cm) node {\huge\sffamily\bfseries\color{black} %
\begin{minipage}{5cm}
\begin{center}
\LARGE\sffamily\color{ocre}\textbf{\textsc{\xxactivite}}

\begin{center}
\xxfigures
\end{center}

\end{center}
\end{minipage} \hfill
\begin{minipage}[c]{12cm}
\begin{titrechapitre}
\renewcommand{\baselinestretch}{1.1} 
\large\sffamily\textbf{\textsc{\xxtitreexo}}

\small\sffamily{\textbf{\textit{\color{black!70}\xxsourceexo}}}
\vspace{.5cm}

\renewcommand{\baselinestretch}{1} 
\normalsize\normalfont
\xxcompetences
\end{titrechapitre}
\end{minipage}};
\end{tikzpicture}
\else % ELSE NOT LIVRET
\begin{tikzpicture}[remember picture,overlay]
\draw[anchor=west] (-2cm,-4.5cm) node {\huge\sffamily\bfseries\color{black} %
\begin{minipage}{5cm}
\begin{center}
\LARGE\sffamily\color{ocre}\textbf{\textsc{\xxactivite}}

\begin{center}
\xxfigures
\end{center}

\end{center}
\end{minipage} \hfill
\begin{minipage}[c]{12cm}
\begin{titrechapitre}
\renewcommand{\baselinestretch}{1.1} 
\large\sffamily\textbf{\textsc{\xxtitreexo}}

\small\sffamily{\textbf{\textit{\color{black!70}\xxsourceexo}}}
\vspace{.5cm}

\renewcommand{\baselinestretch}{1} 
\normalsize\normalfont
\xxcompetences
\end{titrechapitre}
\end{minipage}};
\end{tikzpicture}

\fi

\else   % FIN IF TD
\fi


%%%%%%%% PAGE DE GARDE FICHE
\iffiche
\begin{tikzpicture}[remember picture,overlay]
\node at (current page.north west)
{\begin{tikzpicture}[remember picture,overlay]
\draw[anchor=west] (-2cm,-2.25cm) node [line width=2pt,rounded corners=15pt,draw=ocre,fill=white,fill opacity=0.6,inner sep=40pt]{\strut\makebox[22cm]{}};
\draw[anchor=west] (1cm,-2.25cm) node {\huge\sffamily\bfseries\color{black} %
\begin{minipage}{1cm}
\rotatebox{90}{\LARGE\sffamily\textsc{\color{ocre}\textbf{\xxnumpartie}}}
\end{minipage} \hfill
\begin{minipage}[c]{14cm}
\begin{titrepartie}
\begin{flushright}
\renewcommand{\baselinestretch}{1.1} 
\large\sffamily\textsc{\textbf{\xxpartie} \\} 

\vspace{.2cm}

\normalsize\sffamily\textsc{\textbf{\xxnumchapitre -- \xxchapitre}}
\renewcommand{\baselinestretch}{1} 
\end{flushright}
\end{titrepartie}
\end{minipage} \hfill
\begin{minipage}[c]{3.5cm}
{\large\sffamily\textsc{\textbf{\color{ocre} \discipline}}}
\end{minipage} 
 };
\end{tikzpicture}};
\end{tikzpicture}

\iflivret
\begin{tikzpicture}[overlay]
\node[shape=rectangle, 
      rounded corners = .25 cm,
	  draw= ocre,
	  line width=2pt, 
	  fill = ocre!10,
	  minimum width  = 2.5cm,
	  minimum height = 2.5cm,] at (18.5cm,1.1cm) {};
\node at (17.9cm,1.1cm) {\rotatebox{90}{\textsf{\textbf{\large\color{ocre}{\classe}}}}};
%{};
\end{tikzpicture}
\else
\begin{tikzpicture}[overlay]
\node[shape=rectangle, 
      rounded corners = .25 cm,
	  draw= ocre,
	  line width=2pt, 
	  fill = ocre!10,
	  minimum width  = 2.5cm,
%	  minimum height = 2.5cm,] at (18.5cm,1.1cm) {};
	  minimum height = 2.5cm,] at (18.6cm,1cm) {};
\node at (18cm,1cm) {\rotatebox{90}{\textsf{\textbf{\large\color{ocre}{\classe}}}}};
%{};
\end{tikzpicture}

\fi

\else
\fi



%\fi
\vspace{1.5cm}
\pagestyle{fancy}
\thispagestyle{plain}
\setcounter{section}{0}



Thème : Algorithmes opérant sur une structure séquentielle par boucles imbriquées. 
Commentaires :
\begin{itemize}
\item recherche d'un facteur dans un texte;
\item recherche des deux valeurs les plus proches dans un tableau;
\item tri à bulles;
\item notion de complexité quadratique
\item outils pour valider la correction de l'algorithme
\end{itemize}


\section{Parcours d'une liste de listes}
Les listes de listes permettent de mettre les données en deux dimensions. 
\begin{exemple}
~\\
\begin{minipage}[t]{.3\linewidth}
Grille de mots mêlés.
\begin{center}
\begin{tabular}{|c|c|c|}
\hline
L & E & S \\ \hline
E & T & E \\ \hline
S & E & C \\ \hline
\end{tabular}
\end{center}

\begin{lstlisting}
grille = [['L','E','S'],['E','T','E'],['S','E','C']]
\end{lstlisting}
\end{minipage}
\hfill
\begin{minipage}[t]{.3\linewidth}

Table de multiplication
\begin{center}
\begin{tabular}{|c||c|c|c|}
\hline
$\times $ & 1 & 2 & 3 \\
\hline
\hline
1 & 1 & 2 & 3 \\
2 & 2 & 4 &  6 \\
3 & 3 & 6 &  9 \\
\hline
\end{tabular}
\end{center}

\begin{lstlisting}
table = [[1,2,3],[2,4,6],[3,6,9]]
\end{lstlisting}
\end{minipage}
\hfill
\begin{minipage}[t]{.3\linewidth}

Température en fonction du temps.

\begin{center}
\begin{tabular}{|c|c|c|c|c|}
\hline
T (s)& 1 & 2 & 3 & 4 \\
 \hline
 T°C & 18 & 19 & 21 & 24 \\ 
\hline
\end{tabular}
\end{center}

\begin{lstlisting}
data = [[1,18], [2,19],[3,21],[4,24]]
\end{lstlisting}

\end{minipage}
\end{exemple}

Pour parcourir les éléments d'un tableau on procède de la même façon que pour une recherche séquentielle. Prenons l'exemple d'un tableau \texttt{tab} de \texttt{n} lignes et {p} colonnes.

\textbf{Utilisation de boucles \texttt{while}}
\begin{lstlisting}
n = len(tab)
p = len(tab[0])
i,j = 0,0
while i<n :
    while j<p :
        print(tab[i][j])
        j = j+1
    j=0
    i=i+1
\end{lstlisting}

\textbf{Utilisation de boucles \texttt{for}}
\begin{lstlisting}
n = len(tab)
p = len(tab[0])
for i in range(n) :
    for j in range(p) :
        print(tab[i][j])
\end{lstlisting}

\textbf{Écriture de boucles \texttt{for} en \texttt{Python}}
\begin{lstlisting}
for t in tab :
    for e in t :
        print(e)
\end{lstlisting}

\begin{rem}
Il est possible de dénombrer le nombre d'itérations réalisées par les algorithmes ci-dessus. Dans chaque cas, la première boucle est réalisée $n$ fois. La seconde boucle, imbriquée dans la première est parcourue $p$ fois. 
On peut donc dénombrer le nombre de fois que la fonction \texttt{print} est appelée : $n\times p$. 

Une estimation grossière du nombre d'opérations réalisées en tout est donc  $n\times p$. On dit que la complexité dans ces algorithmes, dans le pire des cas est $\mathcal{O}\left(np\right)$. Si $n=p$, la complexité est de $\mathcal{O}\left(n^2\right)$. On parle de complexité quadratique. 

\end{rem}
 

\section{Recherche de facteur dans un mot}
Rechercher un facteur dans un mot signifie rechercher une (sous-)chaîne de caractères dans une chaîne de caractères (ou encore un mot dans une chaîne).
\begin{lstlisting}
def recherche_01(m:str, s:str) -> bool:
    """Recherche le mot m dans la chaine s
       Préconditions : m et s sont des chaines de caractères"""
    long_s = len(s) # Longueur de s
    long_m = len(m) # Longueur de m
    for i in range(long_s-long_m+1): 
        # Invariant : m n'a pas été trouvé dans s[0:i+long_m-1]
        j = 0
        while j < long_m and m[j] == s[i+j]:
            # Invariant : m[:j] == s[i:i+j]
            j = j+1
            # Invariant : m[:j] == s[i:i+j]
        if j == long_m:
            # Invariant précédent : m == s[i:i+long_m]
            return True
    return False
\end{lstlisting}

Cet algorithme est simplifiable en utilisant le slicing.
\begin{lstlisting}
def recherche_02(m:str, s:str) -> bool:
    """Recherche le mot m dans la chaine s
       Préconditions : m et s sont des chaines de caractères"""
    long_s = len(s) # Longueur de s
    long_m = len(m) # Longueur de m
    for i in range(long_s-long_m+1): 
        # Invariant : m n'a pas été trouvé dans s[0:i+long_m-1]
        if s[i:i+long_m] == m: # On a trouvé m
            return True
    return False
\end{lstlisting}

En utilisant les possibilités de \texttt{Python}, il est possible de simplifier encore l'algorithme. 

\begin{lstlisting}
def recherche_03(m:str, s:str) -> bool:
    """Recherche le mot m dans la chaine s
       Préconditions : m et s sont des chaines de caractères"""
    return m in s
\end{lstlisting}




\textbf{Pour réaliser l'activité associée à ce cours, suivre le lien suivant : }
\url{https://bit.ly/3AmRgdH}

\newpage
\section*{QCM}

\question{Quelle est la valeur de la variable \texttt{image} après exécution du programme \texttt{Python} suivant ?}

\begin{lstlisting}
image = [[0,0,0,0],[0,0,0,0],[0,0,0,0],[0,0,0,0]]
for i in range(4) :
     for j in range(4) :
          if (i+j) == 3 :
               image[i][j] = 1
\end{lstlisting}
\begin{enumerate}
\item \texttt{[[0,0,0,1],[0,0,1,0],[0,1,0,0],[1,0,0,0]]}. % +
\item \texttt{[[0,0,0,1],[0,0,0,1],[0,0,0,1],[0,0,0,1]]}.
\item \texttt{[[0,0,0,1],[0,0,1,1],[0,1,1,1],[1,1,1,1]]}.
\item \texttt{[[0,0,0,0],[0,0,0,0],[0,0,0,0],[1,1,1,1]]}.
\end{enumerate}


\question{Quelle est la valeur de la variable \texttt{table} après exécution du programme \texttt{Python} suivant ?}
\begin{lstlisting}
table = [12, 43, 6, 22, 37]
for i in range(len(table) - 1):
    if table [i] > table [i+1] :
        table [i] ,table [i+1] = table [i+1] ,table [i]
\end{lstlisting}
\begin{enumerate}
\item \texttt{[12, 6, 22, 37, 43]}. % +
\item \texttt{[6, 12, 22, 37, 43]}.
\item \texttt{[43, 12, 22, 37, 6]}.
\item \texttt{[43, 37, 22, 12, 6]}.
\end{enumerate}


\question{On considère le programme suivant. Quelle est la valeur de maxi(L) ?}
\begin{lstlisting}
def maxi(tab):
   """
   tab est une liste de couples (nom, note)
     * nom est de type str
     * note est un entier entre 0 et 20.
   """
   m = tab[0]
   for x in tab:
      if x[1] >= m[1]:
         m = x
   return m
L =  [('Adrien', 17), ('Barnabe', 17), ('Casimir', 17), ('Dorian', 17), ('Emilien', 16), ('Fabien', 16)]
\end{lstlisting}
\begin{enumerate}
\item \texttt{('Adrien',17)}.
\item \texttt{('Dorian',17)}. %+
\item \texttt{('Fabien',16)}.
\item \texttt{('Emilien',16)}.
\end{enumerate}


\question{Que contient la variable compteur à la fin de l’exécution de ce script ?}
\begin{lstlisting}
liste = [0, 1, 2, 3]
compteur = 0
for i in range(len(liste)-1) :
    for j in range(i,len(liste)) :
        compteur += 1
\end{lstlisting}
\begin{enumerate}
\item 4.
\item 8.
\item 9. %+
\item 10.
\end{enumerate}

\question{}
On considère la liste de p-uplets suivante :
\begin{lstlisting}
table = [ ('Grace', 'Hopper', 'F', 1906), ('Tim', 'Berners-Lee', 'H', 1955), ('Ada', 'Lovelace', 'F', 1815), ('Alan', 'Turing', 'H', 1912) ]
\end{lstlisting}
où chaque p-uplet représente un informaticien ou une informaticienne célèbre ; le premier élément est son prénom, le deuxième élément son nom, le troisième élément son sexe (‘H’ pour un homme, ‘F’ pour une femme) et le quatrième élément son année de naissance (un nombre entier entre 1000 et 2000).

On définit une fonction :
\begin{lstlisting}
def fonctionMystere(table):
    mystere = []
    for ligne in table:
        if ligne[2] == 'F':
            mystere.append(ligne[1])
    return mystere
\end{lstlisting}
Que vaut fonctionMystere(table)?
\begin{enumerate}
\item \texttt{[‘Grace’, ‘Ada’]}.
\item \texttt{[('Grace', 'Hopper', 'F', 1906), ('Ada', 'Lovelace', 'F', 1815)]}.
\item \texttt{['Hopper', 'Lovelace']}. %+
\item \texttt{[]}.
\end{enumerate}

\question{Quelle est la valeur de la variable table à la fin de l'exécution du script suivant ?}
\begin{lstlisting}
table = [[1, 2, 3], [1, 2, 3], [1, 2, 3], [1, 2, 3]]
table [1][2] = 5
\end{lstlisting}
\begin{enumerate}
\item \texttt{[[1, 5, 3], [1, 2, 3], [1, 2, 3], [1, 2, 3]]}.
\item \texttt{[[1, 2, 3], [5, 2, 3], [1, 2, 3], [1, 2, 3]]}.
\item \texttt{[[1, 2, 3], [1, 2, 5], [1, 2, 3], [1, 2, 3]]}. %+
\item \texttt{[[1, 2, 3], [1, 2, 3], [1, 2, 3], [1, 5, 3]]}.
\end{enumerate}

\question{Soit le tableau défini de la manière suivante : \texttt{tableau = [[1,3,4],[2,7,8],[9,10,6],[12,11,5]]}
On souhaite accéder à la valeur 12.}
\begin{enumerate}
\item \texttt{tableau[4][1]}.
\item \texttt{tableau[1][4]}.
\item \texttt{tableau[3][0]}. %+
\item \texttt{tableau[0][3]}.
\end{enumerate}

\question{Une erreur s'est glissée dans le tableau, car le symbole du Fluor est F et non Fl. Quelle instruction permet de rectifier ce tableau ?}
\begin{lstlisting}
mendeleiev = [ ['H','.', '.','.','.','.','.','He'],
               ['Li','Be','B','C','N','O','Fl','Ne'],
               ['Na','Mg','Al','Si','P','S','Cl','Ar'],
                 ...... ]
\end{lstlisting}
\begin{enumerate}
\item \texttt{mendeleiev.append('F')}.
\item \texttt{mendeleiev[1][6] = 'F'}. %+
\item \texttt{mendeleiev[6][1] = 'F'}.
\item \texttt{mendeleiev[-1][-1] = 'F'}.
\end{enumerate}

\question{Quelle est la valeur de la variable t1 à la fin de l'exécution du script suivant ?}
\begin{lstlisting}
t1 = [['Valenciennes', 24],['Lille', 23],['Laon', 31],['Arras', 18]]
t2 = [['Lille', 62],['Arras', 53],['Valenciennes', 67],['Laon', 48]]
for i in range(len(t1)):
    for v in t2:
        if v[0] == t1[i][0]:
           t1[i].append(v[1])
\end{lstlisting}
\begin{enumerate}
\item \texttt{[['Valenciennes', 67], ['Lille', 62], ['Laon', 48], ['Arras', 53]]}.
\item \texttt{[['Valenciennes', 24, 67], ['Lille', 23, 62], ['Laon', 31, 48], ['Arras', 18, 53]]}. %+
\item \texttt{[['Arras', 18, 53],['Laon', 31, 48], ['Lille', 23, 62], ['Valenciennes', 24, 67]]}.
\item \texttt{[['Valenciennes', 67, 24], ['Lille', 62,23], ['Laon', 48, 31], ['Arras', 53, 18]]}.
\end{enumerate}

\question{Que vaut asso à la fin de l'exécution ?}
\begin{lstlisting}
asso = []
L =  [ ['marc','marie'], ['marie','jean'], ['paul','marie'], ['marie','marie'], ['marc','anne'] ]
for c in L :
    if c[1]=='marie':
        asso.append(c[0])
\end{lstlisting}
\begin{enumerate}
\item \texttt{['marc', 'jean', 'paul']}.
\item \texttt{[['marc','marie'], ['paul','marie'], ['marie','marie']]}.
\item \texttt{['marc', 'paul', 'marie'] }.%+
\item \texttt{['marie', 'anne']}.
\end{enumerate}

\question{Quelle est la valeur de x après exécution du programme ci-dessous ?}
\begin{lstlisting}
t = [[3,4,5,1],[33,6,1,2]]
x = t[0][0]
for i in range(len(t)):
    for j in range(len(t[i])):
        if x < t[i][j]:
            x = t[i][j]
\end{lstlisting}
\begin{enumerate}
\item 3.
\item 5.
\item 6.
\item 33. %+
\end{enumerate}
