\exer{Trouver les deux valeurs les plus proches dans un tableau}
\setcounter{numques}{0}


\question{On se donne un tableau \texttt{T} unidimensionnel. \'Ecrire une fonction \texttt{distance\_min1(T)} qui renvoie les deux éléments qui sont les plus proches ie dont la valeur absolue de la différence est minimale. On indiquera les valeurs obtenues ainsi que les indices correspondants.}

\question{Pour un tableau à $n$ cases, montrer que le nombre de comparaisons $C(n)$ faites dans cette fonction est tel que la suite $\left(\dfrac{C(n)}{n^2}\right)$ est bornée: on dit que la  \textbf{complexité est quadratique}.}

\question{Faire la même question pour un tableau bidimensionnel en écrivant une fonction \texttt{distance\_min2(T)}. Ici, \texttt{T} sera donc une matrice pas nécessairement carrée par exemple de la forme $$\texttt{T=[[1,2,3],[6,4,3],[3,8,9],[3,-2,0]]}:$$ chaque élément de \texttt{T} désignera une ligne du tableau. Le nombre de ligne est le nombre d'éléments de \texttt{T}, le nombre de colonnes est le nombre d'éléments de \texttt{T[0]}.}



%\end{multicols}
