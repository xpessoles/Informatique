\section*{Introduction au programme}

\paragraph{Les objectifs du programme} Le programme d’informatique de MPSI, PCSI, PTSI, MP, PC, PSI et PT s’inscrit en continuité en amont avec les programmes rénovés du lycée, et en aval avec les enseignements dispensés dans les grandes écoles, et plus généralement les poursuites d’études universitaires. Il a pour objectif la formation de futurs ingénieures et ingénieurs, enseignantes et enseignants, chercheuses et chercheurs et avant tout des personnes informées, capables 
de gouverner leur vie professionnelle et citoyenne nourrie par les pratiques de la démarche scientifique, en pleine connaissance et maîtrise des techniques et des enjeux de l'informatique.

Le présent programme a pour ambition de poser les bases d'un enseignement cohérent et mesuré d'une science informatique encore jeune et dont les manifestations technologiques connaissent des cycles d'obsolescence rapide. On garde donc à l'esprit~: %
\begin{itemize}
\item de privilégier la présentation de concepts fondamentaux pérennes sans s’attacher outre mesure à la description de technologies, protocoles ou normes actuels~;
\item de donner aux futurs diplômées et diplômés les moyens de réussir dans un domaine en mutation rapide et dont les technologies qui en sont issues peuvent sauter brutalement d'un paradigme à un autre très différent~;
\item de préparer les étudiantes et étudiants à tout un panel de professions et de situations 
de la vie professionnelle qui les amène à remplir tour à tour une mission d'expertise, de création ou d'invention, de prescription de méthodes ou de techniques, de contrôle critique des choix opérés ou encore de décision en interaction avec des spécialistes~;
\item que les concepts à enseigner sont les mêmes dans toutes les filières mais que le professeur ou la professeure d'informatique de chaque classe peut adapter la façon de les transmettre et les exemples concrets sur lesquels il ou elle s'appuie au profil de ses étudiantes et étudiants et aux autres enseignements qu'ils suivent.
\end{itemize}



\paragraph{Compétences visées} Ce programme vise à développer les six grandes compétences suivantes~:
\begin{description}
\item[analyser et modéliser] un problème ou une situation, notamment en utilisant les objets conceptuels de l'informatique pertinents (table relationnelle, graphe, dictionnaire, etc.)~;
\item[imaginer et concevoir une solution,] décomposer en blocs, se ramener à des sous-problèmes simples et indépendants, adopter une stratégie appropriée, décrire une démarche, un algorithme ou une structure de données permettant de résoudre le problème~;
\item[décrire et spécifier] les caractéristiques d'un processus, les données d'un problème, ou celles manipulées par un algorithme ou une fonction~;
\item[mettre en \oe uvre une solution,] par la traduction d'un algorithme ou d'une structure de données dans un langage de programmation ou un langage de requête~;
\item[justifier et critiquer une solution,] que ce soit en démontrant un algorithme par une preuve mathématique ou en développant des processus d'évaluation, de contrôle, de validation d'un code que l'on a produit~;
\item[communiquer à l'écrit ou à l'oral,] présenter des travaux informatiques, une problématique et sa solution~; défendre ses choix~; documenter sa production et son implémentation. 
\end{description}



La pratique régulière de la résolution de problèmes par une approche algorithmique et des activités de programmation qui en résultent constitue un aspect essentiel de l’apprentissage de l’informatique.
Les exemples ou les exercices d'application peuvent être choisis au sein de l'informatique elle-même ou en lien avec 
d'autres champs disciplinaires.

\paragraph{Sur les partis pris par le programme} Ce programme impose aussi souvent que possible des choix de vocabulaire ou de notation de certaines notions. Les choix opérés ne présument pas la supériorité de l’option retenue. Ils ont été précisés dans l’unique but d’aligner les pratiques d’une classe à une autre et d’éviter l’introduction de longues définitions récapitulatives préliminaires à un exercice ou un problème. De même, ce programme nomme aussi souvent que possible l’un des algorithmes possibles parmi les classiques qui répondent à un problème donné. Là encore, le programme ne défend pas la prééminence d’un algorithme ou d’une méthode par rapport à un autre mais il invite à faire bien plutôt que beaucoup.

\paragraph{Sur les langages et la programmation} L’enseignement du présent programme repose sur un langage de manipulation de données (SQL) ainsi que le langage de programmation Python, pour lequel une annexe liste de façon limitative les éléments qui sont exigibles des étudiants ainsi que ceux auxquels les étudiants sont familiarisés et qui peuvent être attendus à condition qu’ils soient accompagnés d’une documentation. La poursuite de l’apprentissage du langage Python est vue en particulier par les étudiants pour adopter immédiatement une bonne discipline de programmation tout en se concentrant sur le noyau du langage plutôt que sur une API pléthorique. 

\paragraph{Mode d’emploi} Ce programme a été rédigé par semestre pour assurer une certaine homogénéité de la formation. Le premier semestre permet d’asseoir les bases de programmation vues au lycée et les concepts associés. L’organisation de la progression au sein des semestres relève de la responsabilité pédagogique de la professeure ou du professeur et le tissage de liens entre les thèmes contribue à la valeur de son enseignement. Les notions étudiées lors d’un semestre précédent sont régulièrement revisitées tout au long des deux années d’enseignement. 

