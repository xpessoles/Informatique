\section*{Introduction au programme}


\paragraph{Les objectifs du programme} L'enseignment d'informatique de classe préparatoire MPI a pour objectif la formation de futurs ingénieures et ingénieurs, enseignantes et enseignants, chercheuses et chercheurs et avant tout des personnes informées, capables 
de gouverner leur vie professionnelle et citoyenne nourrie par les pratiques de la démarche scientifique, en pleine connaissance et maîtrise des techniques et des enjeux de l'informatique.

Le présent programme a pour ambition de poser les bases d'un enseignement cohérent et mesuré d'une science informatique encore jeune et dont les manifestations technologiques connaissent des cycles d'obsolescence rapide. On garde donc à l'esprit~: 
\begin{itemize}
\item de privilégier la présentation de concepts fondamentaux pérennes sans s’attacher outre mesure à la description de technologies, protocoles ou normes actuels~;
\item de donner aux futurs diplômées et diplômés les moyens de réussir dans un domaine en mutation rapide et dont les technologies qui en sont issues peuvent sauter brutalement d'un paradigme à un autre très différent~;
\item de préparer les étudiantes et étudiants à tout un panel de professions et de situations 
de la vie professionnelle qui les amène à remplir tour à tour une mission d'expertise, de création ou d'invention, de prescription de méthodes ou de techniques, de contrôle critique des choix opérés ou encore de décision en interaction avec des spécialistes~;
\item d'enseigner de manière à donner aux étudiantes et étudiants la flexibilité de travailler dans de nombreuses disciplines, l'informatique étant un domaine vaste qui se connecte à et tire parti de nombreuses autres disciplines.
\end{itemize}


\paragraph{Compétences visées} Au delà de l'acquisition d'un bagage substantiel de connaissances et de méthodes de l'informatique, ce programme vise à développer les six grandes compétences suivantes~:
\begin{description}
\item[analyser et modéliser] un problème ou une situation, notamment en utilisant les objets conceptuels de l'informatique pertinents (table relationnelle, graphe, arbre, automate, modèle abstrait d'ordonnancement, etc.)~;
\item[imaginer et concevoir une solution,] décomposer en blocs, se ramener à des sous-problèmes simples et indépendants, adopter une stratégie appropriée, décrire une démarche, un algorithme ou une structure de données permettant de résoudre le problème~;
\item[décrire et spécifier] une syntaxe, les caractéristiques d'un processus, les données d'un problème, ou celles manipulées par un algorithme ou une fonction en utilisant le formalisme approprié (notamment langue française, formule logique, grammaire formelle)~;
\item[mettre en \oe uvre une solution,] par le choix d'un langage, par la traduction d'un algorithme ou d'une structure de données dans un langage de programmation ou un langage de requête~;
\item[justifier et critiquer une solution,] que ce soit en démontrant un algorithme par une preuve mathématique, en développant des processus d'évaluation, de contrôle, de validation d'un code que l'on a produit ou en écrivant une preuve au sein d'un système formel~;
\item[communiquer à l'écrit ou à l'oral,] présenter des travaux informatiques, une problématique et sa solution~; défendre ses choix~; documenter sa production et son implémentation. 
\end{description}

%\paragraph{Recommandations particulières} 
L'enseignement de ce programme ne saurait rester aveugle aux questions sociales, juridiques, éthiques et culturelles inhérentes à la discipline de l'informatique. Ces enjeux deviennent particulièrement prégnants eu égard au rôle croissant que jouent l'intelligence artificielle et les techniques d'analyse de données dans la technologie contemporaine. La professeure ou le professeur expose ses étudiants et étudiantes à l'interaction des questions éthiques et des problèmes techniques qui jouent un rôle important dans le développement des algorithmes et des systèmes informatiques.


\paragraph{Sur les partis pris par le programme} 
Ce programme impose aussi souvent que possible des choix de vocabulaire ou de notation de certaines notions. Les choix opérés ne présument pas la supériorité de l'option retenue. Ils ont été précisés dans l'unique but d'aligner les pratiques d'une classe à une autre et d'éviter l'introduction de longues définitions récapitulatives préliminaires à un exercice ou un problème. 
Quand des termes peu usités ont été clarifiés par leur traduction en anglais, seul le libellé en langue française est au programme.

De même, ce programme nomme aussi souvent que possible l'un des algorithmes parmi les classiques qui répondent à un problème donné. Là encore, le programme ne défend pas la prééminence d'un algorithme ou d'une méthode par rapport à un autre mais il invite à faire bien plutôt que beaucoup.

\paragraph{Sur les langages et la programmation} L'enseignement du présent programme repose sur un langage de manipulation de données (SQL) ainsi que deux langages de programmation, C et OCaml. Des annexes listent de façon limitative les éléments de ces langages qui sont exigibles des étudiants ainsi que ceux auxquels les étudiants sont familiarisés et qui peuvent être attendus à condition qu'ils soient accompagnés d'une documentation. Après des enseignements centrés sur les langages enseignés dans les classes du secondaire (au jour de l'écriture de ce programme~: Scratch et Python), ces trois nouveaux langages de natures très différentes permettent d'approfondir le multilinguisme des étudiants tout en illustrant la diversité des paradigmes de programmation ou la diversité des moyens de contrôler les ressources de la machine physique et de les abstraire. 

L'apprentissage du langage C conduit en particulier les étudiants à adopter immédiatement une bonne discipline de programmation tout en se concentrant sur le noyau du langage plutôt que sur une API pléthorique. En tant que langage dit de bas niveau d'abstraction utilisé entre autres pour écrire tous les systèmes d'exploitation, il permet une gestion explicite de la mémoire et des ressources de la machine, indispensable dans le cas où celles-ci sont limitées (systèmes embarqués, mobiles). 

L'apprentissage du langage OCaml permet en particulier aux étudiants de recourir rapidement à un niveau d'abstraction supérieur et de manipuler facilement des structures de données récursives. Pour autant, son utilisation peut également simplifier certaines manipulations sur les fils d'exécution \textit{(threads)}, par exemple.

La plupart des algorithmes qui figurent au programme se prêtent indifféremment à une programmation en C ou en OCaml. On veille à développer de façon parallèle les compétences de programmation dans ces deux langages. 

Il convient de ne pas axer uniquement l'enseignement de ce programme sur le développement de compétences en programmation~: si la capacité à écrire des programmes courts, précis, agréables à lire et documentés fait partie d'une formation exhaustive en informatique, un accent trop important sur l'écriture de code peut donner une vision étroite et trompeuse de la place de la programmation dans la discipline informatique.

Les défauts, les bogues et les failles de logique constituent systématiquement la cause première des vulnérabilités des logiciels exploitées de façon malveillante. La vigilance vis-à-vis de pratiques de programmation sûres est apprise dès les premiers stades de l'apprentissage de la programmation. On s'attache à sensibiliser les étudiants à ces techniques, à la prévention des vulnérabilités et à une validation formelle ou expérimentale rigoureuse des résultats obtenus. Les étudiants sont incités à analyser les sources possibles d'invalidité des données manipulées par leurs programmes, y compris en cas d'exécution concurrente, et à savoir appliquer des principes de programmation défensive.

\paragraph{Mode d'emploi} Pour une meilleure lisibilité de l'ensemble, les acquis d'apprentissage finaux ont été structurés par chapitres thématiques, sans chercher à éviter une redondance qui ne fait que témoigner des liens que ces thèmes entretiennent. Des repères temporels peuvent être proposés mais l'organisation de la progression au sein de ces acquis relève de la responsabilité pédagogique de la professeure ou du professeur et le tissage de liens entre les thèmes contribue à la valeur de son enseignement. Les symboles~\semUn{}, \semDeux{} et~\semTroisQuatre{} indiquent que les notions associées sont étudiées avant la fin du premier ou du second semestre de la première année, ou durant la deuxième année, respectivement~; ces notions sont régulièrement revisitées tout au long des deux années d'enseignement. Lorsqu'une telle spécification s'applique uniformément à une section ou sous-section, elle n'est pas répétée aux niveaux inférieurs.
