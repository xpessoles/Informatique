\section{Logique \semDeux \semTroisQuatre}

\subsection{Syntaxe des formules logiques \semDeux}

Le but de cette partie est de familiariser progressivement les étudiants avec
la différence entre syntaxe et sémantique d'une part et de donner le
vocabulaire permettant de modéliser une grande variété de situations (par
exemple, satisfaction de contraintes, planification, diagnostic, vérification
de modèles, etc.).

L'étude des quantificateurs est l'occasion de formaliser les notions de
variables libres et liées, et de portée, notions que l'on retrouve dans 
la pratique de la programmation.

\noindent
\begin{longtable}{|p{\lnotion}|p{\comment}|}
    \hline
    \textbf{Notions} & \textbf{Commentaires} \\
    \hline \hline
    Variables propositionnelles, connecteurs logiques, arité. & 
    Notations~: $\neg, \vee, \wedge, \rightarrow, \leftrightarrow$.
    \\
    Formules propositionnelles, définition par induction, représentation comme un arbre. Sous-formule.
    & Les formules sont des données informatiques. On fait le lien entre les écritures d'une formule comme mot et les parcours d'arbres.
    \\
    Taille et hauteur d'une formule. &
    \\
    \hline 
    Quantificateurs universel et existentiel. Variables liées, variables
    libres, portée. Substitution d'une variable. 
    & On ne soulève aucune difficulté technique sur la substitution.
    L'unification est hors programme.
    \\
    \hline \hline
    \multicolumn{2}{|p{\lmoe}|}{\textbf{Mise en \oe uvre}} \\
    \hline
    \multicolumn{2}{|p{\lmoe}|}{
        On implémente uniquement les formules propositionnelles 
        sous forme d'arbres.
    } \\
    \hline
\end{longtable}


\subsection{Sémantique de vérité du calcul propositionnel \semDeux}

Par souci d'éviter trop de technicité, on ne présente la notion de valeur
de vérité que pour des formules sans quantificateurs.

\begin{longtable}{|p{\lnotion}|p{\comment}|}
    \hline
    \textbf{Notions} & \textbf{Commentaires} \\
    \hline \hline
    Valuations, valeurs de vérité d'une formule 
    propositionnelle.
    & Notations $V$ pour la valeur vraie, $F$ pour la valeur fausse. \\
Satisfiabilité, modèle, ensemble de modèles, tautologie, antilogie. & Une formule est satisfiable si elle admet un modèle, tautologique si toute valuation en est un modèle. On peut être amené à ajouter à la syntaxe une formule tautologique et une formule antilogique~; elles sont en ce cas notées $\top$ et $\bot$. 
    \\
    \'Equivalence sur les formules. & On présente les lois de De Morgan, le tiers exclu
    et la décomposition de l'implication. 
    \\
    Conséquence logique entre deux formules. & On étend la notion à celle de conséquence $\phi$ d'un ensemble de formules $\Gamma$~: on note $\Gamma\models\phi$. La compacité est hors programme. 
    \\
    \hline
    Forme normale conjonctive, forme normale disjonctive. 
    & Lien entre forme normale disjonctive complète et table de vérité. 
    \\
    Mise sous forme normale. & On peut représenter les formes normales
    comme des listes de listes de littéraux. Exemple de formule dont la taille des formes
    normales est exponentiellement plus grande. 
    \\
    \hline
    Problème SAT, $n$-SAT, algorithme de Quine. & On incarne SAT par la modélisation d'un problème (par exemple la coloration des sommets d'un graphe). %On résout 2-SAT par le calcul de composantes fortement connexes d'un certain graphe orienté. 
    \\
    \hline 
\end{longtable}

\clearpage

\subsection{Déduction naturelle \semTroisQuatre}

Il s'agit de présenter les preuves comme permettant de pallier 
deux problèmes de la présentation précédente du calcul propositionnel~: nature
exponentielle de la vérification d'une tautologie, faible lien avec les preuves
mathématiques.

Il ne s'agit, en revanche, que d'introduire la notion d'arbre de preuve. La
déduction naturelle est présentée comme un jeu de règles d'inférence simple
permettant de faire un calcul plus efficace que l'étude de la table de vérité. Toute technicité dans
les preuves dans ce système est à proscrire.

\noindent
\begin{longtable}{|p{\lnotion}|p{\comment}|}
    \hline
    \textbf{Notions} & \textbf{Commentaires} \\
    \hline \hline
    Règle d'inférence, dérivation. &
    Notation $\vdash$. Séquent $H_1,\dots,H_n
    \vdash C$. On présente des exemples tels que le \textit{modus ponens} ($p, p \rightarrow q \vdash q$) ou le syllogisme \textit{barbara} ($p \rightarrow q, q \rightarrow r \vdash p \rightarrow r$). 
\\
    Définition inductive d'un arbre de preuve. & On présente des exemples utilisant les règles précédentes. 
\\ 
    \hline
    Règles d'introduction et d'élimination de la déduction naturelle pour les
    formules propositionnelles. &
    On présente les règles pour $\wedge, \vee, \neg$ et
    $\rightarrow$. On écrit de petits exemples d'arbre de preuves (par exemple $\vdash (p \rightarrow q) \rightarrow \neg (p \wedge \neg q) $, etc.). \\
    Correction de la déduction naturelle pour les formules propositionnelles. & 
\\
    \hline
    Règles d'introduction et d'élimination pour les quantificateurs universels
    et existentiels.
    & On motive ces règles par une approche sémantique intuitive.
    \\
    \hline \hline 
    \multicolumn{2}{|p{\lmoe}|}{\textbf{Mise en \oe uvre}} \\
    \hline
    \multicolumn{2}{|p{\lmoe}|}{
        Il ne s'agit pas d'implémenter ces règles mais plutôt d'être
        capable d'écrire de petites preuves dans ce système. On peut également présenter
         d'autres utilisations de règles d'inférences pour raisonner.
    } \\
    \hline
\end{longtable}

