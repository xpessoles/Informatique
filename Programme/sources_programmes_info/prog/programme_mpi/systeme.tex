\section{Gestion des ressources de la machine \semUn \semTroisQuatre}

Le programme vise à donner un premier aperçu des liens qu'assurent les systèmes
d'exploitation (plus largement les plates-formes d'exécution) entre les
programmes et les ressources offertes par les machines qui les exécutent. Le
fonctionnement du matériel, l'architecture des ordinateurs, la conception des
systèmes, la gestion des interfaces, les protocoles de communication, la
virtualisation (de la mémoire, des processeurs, etc.) sont hors programme. Ce
programme se focalise sur trois aspects de la gestion de la machine~:
\begin{itemize}
\item la mémoire au sein d'un processus qui exécute un programme~;
\item les systèmes de fichiers qui permettent d'interagir avec un processus, en
  entrée et sortie~;
\item la concurrence au sein des processus par des fils
  d'exécution, exploitant les possibilités d'exécution concurrente des
  processeurs actuels.
\end{itemize}
Bien que ces notions soient indépendantes du système d'exploitation, le système
Linux est le plus propice pour introduire les éléments de ce programme.

\subsection{Gestion de la mémoire d'un programme \semUn}
\noindent
\begin{longtable}{|p{\lnotion}|p{\comment}|}
  \hline
  \textbf{Notions} & \textbf{Commentaires}\\
  \hline \hline
  Utilisation de la pile et du tas par un programme compilé.
  &
  On présente l'allocation des variables globales, le bloc d'activation d'un
      appel. 
  \\
  \hline
  Notion de portée syntaxique et durée de vie d'une variable. Allocation des
  variables locales et paramètres sur la pile.
  &
  On indique la répartition selon la nature des variables~: globales, locales,
    paramètres.
  \\
  \hline
  Allocation dynamique.
  &
  On présente les notions en lien avec le langage C~: \texttt{malloc} et
    \texttt{free}, pointeur nul, type \texttt{void*}, transtypage, relation avec
    les tableaux, protection mémoire (\textit{segmentation violation}).
  \\
  \hline
\end{longtable}


\subsection{Gestion des fichiers et entrées-sorties \semUn}


\noindent
\begin{longtable}{|p{\lnotion}|p{\comment}|}
  \hline
  \textbf{Notions} & \textbf{Commentaires}\\
  \hline \hline
  Interface de fichiers~: taille, accès séquentiel.
  &
  \\
  \hline
  Implémentation interne~: blocs et n\oe uds d'index (\textit{inode}).
  &
  On présente le partage de blocs (avec liens physiques ou symboliques)
    et l'organisation hiérarchique de l'espace de nommage.
  \\
  \hline
  Accès, droits et attributs.
  &
On utilise sur des exemples les fonctions d'accès et d'écriture dans les différents modes.
  \\
  \hline
  Fichiers spéciaux : flux standard (entrée standard \texttt{stdin}, sortie standard \texttt{stdout}, sortie d'erreur standard \texttt{stderr}) et
  redirections dans l'interface système (\textit{shell}).
  &
On présente la notion de tube (\textit{pipe}).
  \\

  \hline\hline
  \multicolumn{2}{|p{\lmoe}|}{\textbf{Mise en \oe uvre}} \\
  \hline
  \multicolumn{2}{|p{\lmoe}|}{
Les seules notions exigibles sont celles permettant à un programme de gérer
l'ouverture, la fermeture et l'accès à un ou plusieurs fichiers, selon les modalités précisées en annexes. 
On attend toutefois d'un étudiant une expérience
du montage d'un support de fichiers amovible, 
de la gestion des droits d'accès à des parties de l'arborescence, 
de la création et du déplacement des parties de l'aborescence 
et de la gestion des liens physiques et symboliques. 
Le professeur expose également ses étudiants à la réalisation d'enchaînements de programmes via des tubes (\textit{pipes}).
  } \\
  \hline
\end{longtable}

\clearpage

\subsection{Gestion de la concurrence et synchronisation \semTroisQuatre}

L'apprentissage des notions liées au parallélisme d'exécution se limite au cas
de fils d'exécutions (\textit{threads}) internes à un processus, sur une machine. Les
problèmes d'algorithmes répartis et les notions liées aux réseaux et à la
communication asynchrone sont hors programme.

\noindent
\begin{longtable}{|p{\lnotion}|p{\comment}|}
  \hline
  \textbf{Notions} & \textbf{Commentaires}\\
  \hline \hline
  Notion de fils d'exécution. Non-déterminisme de l'exécution. 
  &
  Les notions sont présentées au tableau en privilégiant le pseudo-code~; elles sont mises en \oe uvre au cours de travaux pratiques en utilisant les bibliothèques POSIX
   \£pthread£ (en langage C) ou \°Thread° (en langage OCaml), au choix du professeur, selon les modalités précisées en annexe. On s'en tient aux notions de base~: création, attente de
    terminaison.
  \\
  \hline
  Synchronisation de fils d'exécution. Algorithme de Peterson pour deux fils d'exécution. Algorithme de la boulangerie de Lamport pour plusieurs fils d'exécution.
  &
  On illustre l'importance de l'atomicité par quelques exemples et les dangers d’accès à une variable en l’absence de synchronisation. On présente les notions de mutex et sémaphores.
  \\
  \hline \hline
  \multicolumn{2}{|p{\lmoe}|}{\textbf{Mise en \oe uvre}} \\
  \hline
  \multicolumn{2}{|p{\lmoe}|}{
  Les concepts sont illustrés sur des schémas de synchronisation classiques~:
  rendez-vous, producteur-consommateur. Les étudiants sont également
  sensibilisés au non-déterminisme et aux problèmes d'interblocage et d'équité
  d'accès, illustrables sur le problème classique du dîner des philosophes.
  } \\
  \hline
\end{longtable}











