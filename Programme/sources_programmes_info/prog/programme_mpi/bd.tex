\section{Bases de données \semDeux}

On se limite volontairement à une description applicative des bases
de données en langage SQL. Il s'agit de permettre d'interroger une base présentant des données
à travers plusieurs relations. On ne présente ni l'algèbre relationnelle ni le
calcul relationnel.

\noindent
\begin{longtable}{|p{\lnotion}|p{\comment}|}
    \hline
    \textbf{Notions} & \textbf{Commentaires} \\ \hline \hline
    Vocabulaire des bases de données : tables ou relations, attributs ou colonnes, domaine, schéma de tables, enregistrements ou lignes, types de données.&
    On présente ces concepts à travers de nombreux exemples. On s'en tient à une notion sommaire de domaine~: entier, flottant, chaîne~; aucune considération quant aux types des moteurs SQL n'est au programme. Aucune notion relative à la représentation des dates n'est au programme~; en tant que de besoin on s'appuie sur des types numériques ou chaîne pour lesquels la relation d'ordre coïncide avec l'écoulement du temps. Toute notion relative aux collations est hors programme~; en tant que de besoin on se place dans l'hypothèse que la relation d'ordre correspond à l'ordre lexicographique usuel.
 \\ \hline

    Clé primaire. & Une clé primaire n'est pas forcément associée à un unique
    attribut même si c'est le cas le plus fréquent. La notion d'index est hors
    programme.
 \\ \hline
    Entités et associations, clé étrangère. &
    On s'intéresse au modèle entité--association au travers de cas
    concrets d'associations $1-1, 1-*, *-*$.
    Séparation d'une association $*-*$ en deux associations $1-*$. L'utilisation
    de clés primaires et de clés étrangères permet de traduire en SQL les
    associations $1-1$ et $1-*$.
\\ \hline

    Requêtes \§SELECT§ avec simple clause \§WHERE§ (sélection), projection, renommage \§AS§.

    Utilisation des mots-clés \§DISTINCT§, \§LIMIT§, \§OFFSET§, \§ORDER BY§. & Les opérateurs au programme sont \§+§, \§-§, \§*§, \§/§ (on passe outre les subtilités liées à la division entière ou flottante), \§=§, \§<>§, \§<§, \§<=§, \§>§, \§>=§, \§AND§, \§OR§, \§NOT§, \§IS NULL§, \§IS NOT NULL§.
\\
    Opérateurs ensemblistes \§UNION§, \§INTERSECT§ et \§EXCEPT§, produit cartésien. &
\\ \hline

    Jointures internes \§$T_1$ JOIN $T_2$ $\dots$ JOIN $T_n$ ON $\phi$§, externes à gauche \§$T_1$ LEFT JOIN $T_2$ ON $\phi$§. &
      On présente les jointures (internes) en lien avec la notion d'associations
    entre entités.
 \\ \hline

    Agrégation avec les fonctions \§MIN§, \§MAX§, \§SUM§,
        \§AVG§ et \§COUNT§, y compris avec \§GROUP BY§. & Pour la mise en \oe uvre des agrégats, on s'en tient à la norme SQL99.
            On présente quelques exemples de requêtes imbriquées.
\\
\hline
    Filtrage des agrégats avec \§HAVING§. & On marque la différence entre \§WHERE§ et \§HAVING§ sur des exemples.
\\

    \hline \hline
    \multicolumn{2}{|p{\lmoe}|}{\textbf{Mise en \oe uvre}} \\
    \hline
    \multicolumn{2}{|p{\lmoe}|}{
        La création, la suppression et la modification de tables au travers du langage
        SQL sont hors programme. La mise en \oe uvre effective se fait au
        travers d'un logiciel permettant d'interroger une base de données à
        l'aide de requêtes SQL. Récupérer le résultat d'une requête à partir
        d'un programme n'est pas un objectif.

        Même si aucun formalisme graphique précis n'est au programme,
        on peut décrire
        les entités et les associations qui les lient au travers de
        diagrammes sagittaux informels.

        Sont hors programme~: la notion de modèle logique \textit{vs} physique, les bases de données non relationnelles, les méthodes de modélisation de base,
        les fragments DDL, TCL et ACL du langage SQL, l'optimisation de requêtes par l'algèbre relationnelle.
        } \\
        \hline

\end{longtable}
