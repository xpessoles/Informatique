\section*{Introduction au programme}

\paragraph{Les objectifs du programme} L'enseignement optionnel en informatique est conçu comme un complément à la formation de tronc commun dispensée en classe de MPSI et de MP.
Omniprésentes dans les révolutions industrielles et sociales du XXI\ieme{} siècle,
les techniques rattachées à l'informatique connaissent aussi des cycles d'obsolescence rapide et sautent brutalement d'un paradigme à un autre très différent.
Voilà pourquoi ce programme privilégie d'opérer par un approfondissement de certains fondamentaux scientifiques de la discipline.
Il prépare ainsi de futurs ingénieures et ingénieurs, professeures et professeurs, chercheuses et chercheurs à affronter avec perspective, recul et adaptabilité les défis qu'elles et ils rencontreront dans leur carrière.


\paragraph{Articulation entre le programme de tronc commun et le programme d'option} 
Tantôt le programme d'option complète des chapitres du tronc commun en apportant une profondeur, notamment théorique, sur certains points traités plus sommairement dans le programme initial~; tantôt le programme d'option aborde des thématiques nouvelles 
et s'attarde à poser les jalons de connaissances et de techniques, plutôt fondamentales,
qu'un complément de temps autorise à aborder dans le cadre d'une formation au long cours.
La professeure ou le professeur d'informatique de la classe de 
MPSI, de MP ou de MP* 
veille donc à la fois à assurer l'intelligibilité de son cours de tronc commun pour les étudiants qui ont choisi de ne pas suivre l'option et à organiser une articulation harmonieuse de son cours d'option avec son cours de tronc commun.



\paragraph{Compétences visées} Ce programme amplifie le développement des six grandes compétences identifiées dans le programme de tronc commun~:
\begin{description}
\item[analyser et modéliser] un problème ou une situation, notamment en utilisant les objets conceptuels de l'informatique pertinents (graphe, arbre, automate, etc.)~;
\item[imaginer et concevoir une solution,] décomposer en blocs, se ramener à des sous-problèmes simples et indépendants, adopter une stratégie appropriée, décrire une démarche, un algorithme ou une structure de données permettant de résoudre le problème~;
\item[décrire et spécifier] un motif textuel, les données d'un problème, ou celles manipulées par un algorithme ou une fonction en utilisant le formalisme approprié (notamment langue française, formule logique, expression régulière)~;
\item[mettre en \oe uvre une solution,] par la traduction d'un algorithme ou d'une structure de données dans le langage de programmation du programme~;
\item[justifier et critiquer une solution,] que ce soit en démontrant un algorithme par une preuve mathématique, en développant des processus d'évaluation, de contrôle, de validation d'un code que l'on a produit ou en écrivant une preuve au sein d'un système formel~;
\item[communiquer à l'écrit ou à l'oral,] présenter des travaux informatiques, une problématique et sa solution~; défendre ses choix~; documenter sa production et son implémentation. 
\end{description}


\paragraph{Sur les partis pris par le programme} 
Ce programme impose aussi souvent que possible des choix de vocabulaire ou de notation de certaines notions. Les choix opérés ne présument pas la supériorité de l'option retenue. Ils ont été précisés dans l'unique but d'aligner les pratiques d'une classe à une autre et d'éviter l'introduction de longues définitions récapitulatives préliminaires à un exercice ou un problème.
Quand des termes peu usités ont été clarifiés par leur traduction en anglais, seul le libellé en langue française est au programme.

De même, ce programme nomme aussi souvent que possible l'un des algorithmes possibles parmi les classiques qui répondent à un problème donné. Là encore, le programme ne défend pas la prééminence d'un algorithme ou d'une méthode par rapport à un autre mais il invite à faire bien plutôt que beaucoup. %Chaque sujet est volontairement circonscrit pour être exploré avec plus de profondeur. 

\paragraph{Sur le langage et la programmation}
L'enseignement du présent programme repose sur le langage de programmation OCaml dans les perspectives et les limites qui suivent.
Après des enseignements centrés sur les langages enseignés dans les classes du secondaire et poursuivis pour partie en tronc commun, ce nouveau langage de nature très différente permet d'approfondir le multilinguisme des étudiants tout en illustrant la diversité des paradigmes de programmation. 
Le langage OCaml est utilisé en raison de sa capacité à s'ouvrir rapidement à un niveau d'abstraction supérieur et pour sa pertinence dans la manipulation de fonctions ou de structures de données récursives. 
Les traits impératifs de OCaml sont également présentés~: ils permettent en particulier de découvrir des notions centrales, telles que les références, sans excessives difficultés liées au langage employé. 

La discipline de programmation mise en place dans le cours de tronc commun reste observée, tout en étant le cas échéant adaptée au cadre de la programmation récursive ou des structure de données immuables.
On ne se cantonne pas à écrire des programmes sur papier~; on veille à mettre régulièrement les étudiants en situation de programmer sur machine.
Toutefois, la virtuosité dans l'écriture de programmes ou une connaissance exhaustive des
bibliothèques de programmation ne sont pas des objectifs de la formation. Au contraire, l'annexe~\ref{annexe-ocaml} liste de façon limitative les éléments du langage OCaml qui sont exigibles des étudiants ainsi que ceux auxquels les étudiants sont familiarisés et qui peuvent être attendus à condition que ceux-ci soient accompagnés d'une documentation. 
