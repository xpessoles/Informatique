\documentclass[10pt,oneside]{article}
\input{style/coursHeadings}
\usepackage{algorithm}
\usepackage{algorithmic}


% Python sources
\usepackage{listings}
\usepackage{textcomp}
\usepackage{setspace}
%\usepackage{palatino}

%\usepackage{color}
\definecolor{Bleu}{rgb}{0.1,0.1,1.0}
\definecolor{Noir}{rgb}{0,0,0}
\definecolor{Grau}{rgb}{0.5,0.5,0.5}
\definecolor{DunkelGrau}{rgb}{0.15,0.15,0.15}
\definecolor{Hellbraun}{rgb}{0.5,0.25,0.0}
\definecolor{Magenta}{rgb}{1.0,0.0,1.0}
\definecolor{Gris}{gray}{0.5}
\definecolor{Vert}{rgb}{0,0.5,0}
\definecolor{SourceHintergrund}{rgb}{1,1.0,0.95}

%
\renewcommand{\lstlistlistingname}{Listings}
\renewcommand{\lstlistingname}{Listing}

\lstnewenvironment{python}[1][]{
\lstset{
language=python,
basicstyle=\ttfamily\footnotesize\setstretch{1}, 	
stringstyle=\color{red}, 
showstringspaces=false, 
alsoletter={1234567890},
otherkeywords={\ , \}, \{},
keywordstyle=\color{blue},
emph={access,and,break,class,continue,def,del,elif ,else,
except,exec,finally,for,from,global,if,import,in,i s,
lambda,not,or,pass,print,raise,return,try,while},
emphstyle=\color{black}\bfseries,
emph={[2]True, False, None, self},
emphstyle=[2]\color{green},
emph={[3]from, import, as},
emphstyle=[3]\color{blue},
upquote=true,
morecomment=[s]{"""}{"""},
commentstyle=\color{Hellbraun}\slshape, 
%emph={[4]1, 2, 3, 4, 5, 6, 7, 8, 9, 0},
emphstyle=[4]\color{blue},
literate=*{:}{{\textcolor{blue}:}}{1}
{=}{{\textcolor{blue}=}}{1}
{-}{{\textcolor{blue}-}}{1}
{+}{{\textcolor{blue}+}}{1}
{*}{{\textcolor{blue}*}}{1}
{!}{{\textcolor{blue}!}}{1}
{(}{{\textcolor{blue}(}}{1}
{)}{{\textcolor{blue})}}{1}
{[}{{\textcolor{blue}[}}{1}
{]}{{\textcolor{blue}]}}{1}
{<}{{\textcolor{blue}<}}{1}
{>}{{\textcolor{blue}>}}{1},
%framexleftmargin=1mm, framextopmargin=1mm, frame=shadowbox, rulesepcolor=\color{blue},#1
backgroundcolor=\color{SourceHintergrund}, 
framexleftmargin=1mm, framexrightmargin=1mm, framextopmargin=1mm, frame=single, framerule=1pt, rulecolor=\color{black},#1
}}{}


%Si le boolen xp est vrai : compilation pour xabi
%Sinon compilation Damien
\newboolean{xp}
\setboolean{xp}{true}

\newboolean{prof}
\setboolean{prof}{true}

\def\xxtitre{\ifthenelse{\boolean{xp}}{
Devoir Surveillé 2 -- 1 heure}{
}}


\def\xxsoustitre{\ifthenelse{\boolean{xp}}{
CI 2 : Algorithmique et programmation}{
}}

\def\xxauteur{\ifthenelse{\boolean{xp}}{
Xavier \textsc{Pessoles}}{% \\ Damien \textsc{Iceta}}{
Damien \textsc{Iceta} \\ Xavier \textsc{Pessoles}}}

\def\xxpied{\ifthenelse{\boolean{xp}}{
DS 02 -- CI 2}{
\xxtitre}}

\def\xxcathegorie{\ifthenelse{\boolean{xp}}{
2013 -- 2014 \\
Xavier \textsc{Pessoles}\\Informatique -- DS 2}{
Informatique -- DS 2}}

\ifthenelse{\boolean{xp}}{\usepackage[%
    pdftitle={Représentation des nombres},
    pdfauthor={Xavier Pessoles},
    colorlinks=true,
    linkcolor=blue,
    citecolor=magenta]{hyperref}

\usepackage{pifont}
%\usepackage{lastpage}

% \makeatletter \let\ps@plain\ps@empty \makeatother
%% DEBUT DU DOCUMENT
%% =================
\sloppy
\hyphenpenalty 10000


\colorlet{shadecolor}{orange!15}

\newtheorem{theorem}{Theorem}


\begin{document}


%\newboolean{prof}
%\setboolean{prof}{true}
% \makeatletter \let\ps@plain\ps@empty \makeatother
%% DEBUT DU DOCUMENT
%% =================




%------------- En tetes et Pieds de Pages ------------


\pagestyle{fancy}
\ifthenelse{\boolean{xp}}{%
\renewcommand{\headrulewidth}{0pt}}{%
\renewcommand{\headrulewidth}{0.2pt}} %pour mettre le trait en haut
%\renewcommand{\headrulewidth}{0.2pt}

\fancyhead{}
\fancyhead[L]{%
\noindent\begin{minipage}[c]{2.6cm}%
\includegraphics[width=2cm]{png/logo_ptsi.png}%
\end{minipage}}


\fancyhead[C]{\rule{12cm}{.5pt}}



\fancyhead[R]{%
\noindent\begin{minipage}[c]{3cm}
\begin{flushright}
\footnotesize{\textit{\textsf{Informatique}}}%
\end{flushright}
\end{minipage}
}



\fancyhead[C]{\rule{12cm}{.5pt}}

\renewcommand{\footrulewidth}{0.2pt}

\fancyfoot[C]{\footnotesize{\bfseries \thepage}}
\fancyfoot[L]{%
\begin{minipage}[c]{.2\linewidth}
\noindent\footnotesize{{\xxauteur}}
\end{minipage}
\ifthenelse{\boolean{xp}}{}{%
\begin{minipage}[c]{.15\linewidth}
\includegraphics[width=2cm]{png/logoCC.png}
\end{minipage}}
}

\ifthenelse{\boolean{prof}}{%
\fancyfoot[R]{\footnotesize{\xxpied}}}

\begin{center}
 \huge\textsc{\xxtitre}
\end{center}

\begin{center}
 \LARGE\textsc{\xxsoustitre}
\end{center}

\vspace{.5cm}
}{\input{style/enteteDI}}


%---------------------------------------------------------------------------


\ifthenelse{\boolean{prof}}{
\begin{center}
 \large\textsc{Éléments de corrigés}
\end{center}
}{
%\begin{center}
% \large\textsc{CI 2 : Algorithmique et programmation}
%\end{center}
}
\vspace{.5cm}


\ifthenelse{\boolean{prof}}{}{
\begin{obj}
Objectifs : 
\begin{itemize}
\item comprendre un algorithme et expliquer ce qu’il fait;
\item modifier un algorithme existant pour obtenir un résultat différent;
\item concevoir un algorithme répondant à un problème précisément posé;
\item expliquer le fonctionnement d’un algorithme.
\end{itemize}
\end{obj}

\textbf{Vous pourrez répondre aux questions en utilisant du pseudo code ou du code Python.}
}


\subsection*{Exercice}
\ifthenelse{\boolean{prof}}{}{
L'objectif est de trier un tableau dans le but de diminuer le temps nécessaire à la recherche d'un ou de plusieurs éléments. Une étape préalable à la recherche d'un élément est le tri du tableau.


Avant tout on se propose d'écrire la fonction \textsf{permute} qui permet de permuter deux éléments d'un tableau. Si on veut permuter le premier élément et le second élément d'un tableau, la fonction doit avoir le comportement suivant :
}

%\begin{py}
%\begin{minipage}[c]{.05\linewidth}
%$\quad$
%\end{minipage} \hfill
%\begin{minipage}[c]{.75\linewidth}
%\begin{python}
%>>> tab=[10,20,30]
%>>> permute(tab,0,1)
%>>> print(tab)
%                [20,10,30]
%\end{python}
%\end{minipage}
%\end{py}
%



\subparagraph{}
\textit{Écrire la fonction permettant de permuter les valeurs du tableau. Cette fonction devra correspondre aux spécifications suivantes : 
\begin{itemize}
\item nom de la fonction : \textsf{permute};
\item arguments de la fonction : un tableau, les deux indices à permuter. 
\end{itemize}}


%\ifthenelse{\boolean{prof}}{
\begin{corrige}
\begin{py}
\begin{python2}
def permute(tab, i,j):
    tab[i],tab[j]=tab[j],tab[i]
    return tab
\end{python2}
\end{py}
\end{corrige}
%}{}



%\ifthenelse{\boolean{prof}}{
%\vspace{.25cm}
%On propose maintenant un algorithme permettant de trier le tableau à proprement parlé. Ce tri est appelé tri par sélection. Le voici :
%\begin{py}
%\begin{minipage}[c]{.05\linewidth}
%$\quad$
%\end{minipage} \hfill
%\begin{minipage}[c]{.75\linewidth}
%\begin{python}
%def tri(tab):
%    for i in range(0,len(tab)):
%        indice = i
%        for j in range(i+1,len(tab)):
%            if tab[j]<tab[indice]:
%               indice = j
%        permute(tab,i,indice)
%    return tab
%\end{python}
%\end{minipage}
%\end{py}

\subparagraph{}
\textit{Soit le tableau $\text{tab}=[2,3,1,4]$. Pour chaque valeur de $i$ et pour chaque valeur de $j$, indiquer le contenu du tableau tab. Pour répondre à la question on utilisera le tableau donné en fin de sujet. On le remplira à partir de la double barre. Les colonnes $i$, $j$, indice et tab seront à remplir intégralement. Les * sont à remplacer par les valeurs correspondantes.}

%\begin{itemize}
%\item Pour $i$ variant de 0 à 3 :
%\begin{itemize}
%\item Pour $i=0$:
%\begin{itemize}
%\item indice = 0
%\item Pour $j$ variant de 1 à 3 :
%\begin{itemize}
%\item Pour j = 1 :
%\begin{itemize}
%\item La condition $tab[1]<tab[0] \Rightarrow 3<2$ est fausse. 
%\end{itemize}
%\item Pour j = 2 :
%\begin{itemize}
%\item La condition $tab[2]<tab[0] \Rightarrow 1<2$ est vraie. 
%\begin{itemize}
%\item On a donc 
%\end{itemize}
%\end{itemize}
%\end{itemize}
%%\end{itemize}
%\end{itemize}
%\end{itemize}



\subparagraph{}
\textit{Écrire la fonction qui vérifie si un tableau est trié. Cette fonction devra correspondre aux spécifications suivantes : 
\begin{itemize}
\item nom de la fonction : \textsf{is\_sorted};
\item argument de la fonction : un tableau;
\item retour de la fonction : la fonction doit retourner la valeur booléenne \textsf{True} si le tableau est trié. Elle devra retourner la valeur booléenne\textsf{False} si le tableau n'est pas trié.
\end{itemize}
On utilisera une boucle \textbf{\textsf{Pour}.}}
%\ifthenelse{\boolean{prof}}{}


\subparagraph{}
\textit{Réécrire la fonction précédente en utilisant exclusivement une boucle \textbf{\textsf{Tant que}}.}

%\ifthenelse{\boolean{prof}}{}

\subparagraph{}
\textit{Des deux structures proposées, estimez laquelle peut être le plus efficace ? (c'est-à-dire, laquelle nécessite, le cas échéant, le moins d'opérations).}
%\ifthenelse{\boolean{prof}}{}
%\vspace{.25cm}

%Dans le but de diminuer le temps d'exécution lors de la recherche d'éléments dans le tableau, on propose un algorithme permettant de vérifier qu'une valeur se trouve dans un tableau. 
%
%La méthode consiste en découper le tableau en 2. On regarde ensuite dans quelle partie du tableau est susceptible de se trouver la valeur recherchée. On redivise en 2 la partie de tableau considérée et on regarde dans quelle partie du tableau est susceptible de se trouver la valeur recherchée \textit{etc}. La fonction doit renvoyer à l'utilisateur l'index de l'élément recherché s'il existe et qui renvoie \textsf{None} sinon. 
%On donne le programme suivant : 


%\begin{py}
%\begin{minipage}[c]{.05\linewidth}
%$\quad$
%\end{minipage} \hfill
%\begin{minipage}[c]{.75\linewidth}
%\begin{python}
%def recherche_dichotomique(x, a):
%    g, d = 0, len(a)-1
%    while g <= d:
%        m = (g + d) // 2
%        if a[m] == x:
%        if a[m] < x:
%            g = m+1
%        else:
%            d = m-1
%\end{python}
%\end{minipage}
%\end{py}

%\ifthenelse{\boolean{prof}}{}



\subparagraph{}
\textit{De quel type sont les variables $x$ et $a$ ?}
%\ifthenelse{\boolean{prof}}{}

\subparagraph{}
\textit{Quelle opération est effectuée ligne 4 ? Expliquer ce choix.}
%\ifthenelse{\boolean{prof}}{}

\subparagraph{}
\textit{Pour chaque itération de la boucle \textsl{while}, quelle est l'étendue de la zone de recherche ?}
%\ifthenelse{\boolean{prof}}{}

\subparagraph{}
\textit{Compléter l'algorithme (en recopiant les lignes qui vous semblent nécessaires) pour qu'il renvoie l'index de l'élément recherché.}

\subparagraph{}
\textit{L'algorithme a-t-il le comportement souhaité ? Si ce n'est pas le cas, compléter l'algorithme.}

\subparagraph{}
\textit{A partir des fonctions définies précédemment, écrire une fonction permettant, à partir d'un tableau d'entier quelconque (trié ou non trié), de dire si un élément appartient au tableau ou non :
\begin{itemize}
\item données d'entrées de la fonction : un tableau, un nombre;
\item données de sortie de la fonction : un booléen.
\end{itemize}}
%\ifthenelse{\boolean{prof}}{}
\subparagraph{}
\textit{Quel est l'intérêt de trier un tableau dans le cas où on cherche une valeur dans le tableau ? Quel est l'intérêt de le trier lorsqu'on cherche plusieurs valeurs ? Commenter.}

%\ifthenelse{\boolean{prof}}{}
\newpage

NOM :
\vspace{.25cm}

\begin{center}
\begin{tabular}{|p{.25\textwidth}p{.25\textwidth}l|c|c|c|c|}
\hline
Commentaires & & &$i$ & $j$ & indice & tab\\
\hline
\hline
Instant initial &&& -- & -- &  -- & [2,3,1,4] \\  
\hline
Pour $i$ allant de 0 à 3 :&& & -- & -- & -- & [2,3,1,4]\\
\hline
Pour $i=0$ & && 0 & -- & --&  [2,3,1,4]\\ \hline
Affectation & && 0 & -- &0 &  [2,3,1,4]\\ \hline
Pour $j$ allant de 1 à 3 & & & 0 & -- & 0 & [2,3,1,4] \\ \hline
& Pour $j = 1$ & &0 &1 & 0&  [2,3,1,4]\\ \hline
& tab[1]<tab[0] & $ \Rightarrow$  3<2 est faux &0 &1 & 0&  [2,3,1,4]\\ \hline
& Pas d'affectation de l'indice  &&0 &1 & 0 &  [2,3,1,4]\\ \hline
& Pour $j = 2$ & & 0 & 2 & 0 & [2,3,1,4]\\ \hline
& tab[2]<tab[0] &$ \Rightarrow$  1<2 est vrai &0 &2 & 0&  [2,3,1,4]\\ \hline
& Affectation de indice  &&0 &2 & 2&  [2,3,1,4]\\ \hline\hline %%%%%
& Pour $j=  *$ & & & & & \\ \hline
& tab[*]<tab[*] & &&& & \\ \hline
& Affectation ?  && & & & \\ \hline%%%%%
Permutation ? & && & & & \\ \hline
Pour $i=*$ & && & & & \\ \hline
Affectation & &&  &  & & \\ \hline
Pour $j$ allant de ** à ** & & & &  & & \\ \hline
& Pour $j=  *$ & & & & & \\ \hline
& tab[*]<tab[*] & &&& & \\ \hline
& Affectation ?  && & & & \\ \hline
& Pour $j=  *$ & & & & & \\ \hline
& tab[*]<tab[*] & &&& & \\ \hline
& Affectation ?  && & & & \\ \hline
& Pour $j=  *$ & & & & & \\ \hline
& tab[*]<tab[*] & &&& & \\ \hline
& Affectation ?  && & & & \\ \hline
Permutation ? & && & & & \\ \hline
\end{tabular}
\end{center}

\end{document}
