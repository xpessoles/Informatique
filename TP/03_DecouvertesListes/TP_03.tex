\documentclass[11pt,oneside]{article}
\input{coursHeadings}
\usepackage{algorithm}
\usepackage{algorithmic}


% Python sources
\usepackage{listings}
\usepackage{textcomp}
\usepackage{setspace}
%\usepackage{palatino}

%\usepackage{color}
\definecolor{Bleu}{rgb}{0.1,0.1,1.0}
\definecolor{Noir}{rgb}{0,0,0}
\definecolor{Grau}{rgb}{0.5,0.5,0.5}
\definecolor{DunkelGrau}{rgb}{0.15,0.15,0.15}
\definecolor{Hellbraun}{rgb}{0.5,0.25,0.0}
\definecolor{Magenta}{rgb}{1.0,0.0,1.0}
\definecolor{Gris}{gray}{0.5}
\definecolor{Vert}{rgb}{0,0.5,0}
\definecolor{SourceHintergrund}{rgb}{1,1.0,0.95}

%
\renewcommand{\lstlistlistingname}{Listings}
\renewcommand{\lstlistingname}{Listing}

\lstnewenvironment{python}[1][]{
\lstset{
language=python,
basicstyle=\ttfamily\footnotesize\setstretch{1}, 	
stringstyle=\color{red}, 
showstringspaces=false, 
alsoletter={1234567890},
otherkeywords={\ , \}, \{},
keywordstyle=\color{blue},
emph={access,and,break,class,continue,def,del,elif ,else,
except,exec,finally,for,from,global,if,import,in,i s,
lambda,not,or,pass,print,raise,return,try,while},
emphstyle=\color{black}\bfseries,
emph={[2]True, False, None, self},
emphstyle=[2]\color{green},
emph={[3]from, import, as},
emphstyle=[3]\color{blue},
upquote=true,
morecomment=[s]{"""}{"""},
commentstyle=\color{Hellbraun}\slshape, 
%emph={[4]1, 2, 3, 4, 5, 6, 7, 8, 9, 0},
emphstyle=[4]\color{blue},
literate=*{:}{{\textcolor{blue}:}}{1}
{=}{{\textcolor{blue}=}}{1}
{-}{{\textcolor{blue}-}}{1}
{+}{{\textcolor{blue}+}}{1}
{*}{{\textcolor{blue}*}}{1}
{!}{{\textcolor{blue}!}}{1}
{(}{{\textcolor{blue}(}}{1}
{)}{{\textcolor{blue})}}{1}
{[}{{\textcolor{blue}[}}{1}
{]}{{\textcolor{blue}]}}{1}
{<}{{\textcolor{blue}<}}{1}
{>}{{\textcolor{blue}>}}{1},
%framexleftmargin=1mm, framextopmargin=1mm, frame=shadowbox, rulesepcolor=\color{blue},#1
backgroundcolor=\color{SourceHintergrund}, 
framexleftmargin=1mm, framexrightmargin=1mm, framextopmargin=1mm, frame=single, framerule=1pt, rulecolor=\color{black},#1
}}{}
\usepackage[%
    pdftitle={TP - Algorithmes de recherche},
    pdfauthor={Xavier Pessoles},
    colorlinks=true,
    linkcolor=blue,
    citecolor=magenta]{hyperref}

\usepackage{pifont}


% \makeatletter \let\ps@plain\ps@empty \makeatother
%% DEBUT DU DOCUMENT
%% =================
\sloppy
\hyphenpenalty 10000

\newcommand{\Pointilles}[1][3]{%
\multido{}{#1}{\makebox[\linewidth]{\dotfill}\\[\parskip]
}}


\colorlet{shadecolor}{orange!15}

\newtheorem{theorem}{Theorem}


\begin{document}


\newboolean{prof}
\setboolean{prof}{true}
%------------- En tetes et Pieds de Pages ------------
\pagestyle{fancy}
\renewcommand{\headrulewidth}{0pt}

\fancyhead{}
\fancyhead[L]{%
\noindent\noindent\begin{minipage}[c]{2.6cm}
%Lycée Rouvière PTSI
\includegraphics[width=2cm]{png/logo_ptsi.png}%
\end{minipage}
}

\fancyhead[C]{\rule{12cm}{.5pt}}

\fancyhead[R]{%
\noindent\begin{minipage}[c]{3cm}
\begin{flushright}
\footnotesize{\textit{\textsf{Informatique}}}%
\end{flushright}
\end{minipage}
}

\renewcommand{\footrulewidth}{0.2pt}

\fancyfoot[C]{\footnotesize{\bfseries \thepage}}
\fancyfoot[L]{\footnotesize{\textit{X. Pessoles}} \\ \footnotesize{\textit{C. Lopez}}}
\ifthenelse{\boolean{prof}}{%
\fancyfoot[R]{\footnotesize{TP 3 -- CI 2 : Algorithmique \& Programmation}}
}{%
\fancyfoot[R]{\footnotesize{TP 3 -- CI 2 : Algorithmique \& Programmation}}
}



\begin{center}
 \Large\textsc{CI 2 -- Algorithmique et Programmation}

% \large\textsc{Introduction à la programmation}
\end{center}

\begin{center}
 \large\textsc{TP 3 -- Découverte des listes}
\end{center}
%
\begin{savoir}
\textbf{Objectif -- Découvrir les listes :}
\begin{itemize}
\item créer une liste;
\item ajouter et supprimer un élément;
\item accéder à un élément;
\item extraire une partie de liste.
\end{itemize}
\end{savoir}
%\item Recherche par dichotomie dans un tableau trié
%\item Recherche par dichotomie du zéro d'une fonction monotone
%\item Recherche d'un mot dans une chaîne de caractères
%\end{itemize}
%\end{savoir}
% 
%\begin{warn}
%\textbf{TODO : Trouver une contextualisation aux exercices}
%\end{warn}

%\setlength{\parskip}{0ex plus 0.2ex minus 0ex}
% \renewcommand{\contentsname}{}
% \renewcommand{\baselinestretch}{1}
%
%\tableofcontents
%
% \renewcommand{\baselinestretch}{1.2}
%\setlength{\parskip}{2ex plus 0.5ex minus 0.2ex}

% \vspace{1cm}
%\textit{Ce document évolue. Merci de signaler toutes erreurs ou coquilles.}


\setcounter{paragraph}{0}
\subsection*{Exercice 1 *}

\begin{py}
\textbf{Manipulation de listes}

\begin{python}
>>> tableau = [1,2,3] # Creation d'un tableau
>>> tableau[0] # Acces a l'element 0 du tableau - ici 1
>>> len(tableau) # Renvoie la taille d'un tableau
\end{python}

\textbf{Import de la fonction randrange}

\begin{python}
>>> from random import randrange
>>> randrange(0,100) # renvoie aleatoirement un nombre compris entre 0 et 99
\end{python}
\end{py}

\paragraph{}
\textit{Créer une liste composée des nombres pairs allant de 0 (inclus) à 20 (inclus).}

\paragraph{}
\textit{Tester si le nombre choisi au hasard par le programme dans l'intervalle [0,20] appartient ou non à la liste.}

\setcounter{paragraph}{0}
\subsection*{Exercice 2 *}

\paragraph{}
\textit{Créer une liste composée des nombres pairs allant de 0 (inclus) à 20 (inclus).}

\paragraph{}
\textit{Implémenter une fonction permettant d'ajouter 3 à chaque entier pair et 4 à chaque entier impair et d'afficher la nouvelle liste.}


\setcounter{paragraph}{0}
\subsection*{Exercice 3 * }

\paragraph{}
\textit{Créer une liste composée de toutes les chaînes de deux caractères construites à partir de "abcdef".}

\paragraph{}
\textit{Afficher la liste.}

\begin{exemple}
Si la chaîne de départ est "abc" alors la liste sera composée de "ab", "bc" et "ac".
\end{exemple}


\setcounter{paragraph}{0}
\subsection*{Exercice 4 *}

\paragraph{}
\textit{Créer une liste composée des nombres 17, 38, 10, 25 et 72.}

\paragraph{}
\textit{Implémenter la fonction permettant d'ajouter l'élément 12 à la liste entre 38 et 10 et d'afficher la nouvelle liste.}


\paragraph{}
\textit{Implémenter la fonction permettant d'inverser les éléments de la liste (le premier élément devenant le dernier, le second élément devenant l'avant dernier ...) et d'afficher la nouvelle liste.}


\paragraph{}
\textit{Implémenter la fonction permettant de supprimer l'élément 38 de la liste.}

\paragraph{}
\textit{Implémenter la fonction permettant d'afficher la sous-liste allant du deuxième au troisième élément.}

\paragraph{}
\textit{Implémenter la fonction permettant d'afficher la sous-liste allant du premier au deuxième élément.}

\paragraph{}
\textit{Implémenter la fonction permettant de trier la liste par ordre croissant et d'afficher la nouvelle liste.}



\end{document}