\documentclass[10pt,oneside]{article}
\input{style/coursHeadings}
\usepackage{algorithm}
\usepackage{algorithmic}


% Python sources
\usepackage{listings}
\usepackage{textcomp}
\usepackage{setspace}
%\usepackage{palatino}

%\usepackage{color}
\definecolor{Bleu}{rgb}{0.1,0.1,1.0}
\definecolor{Noir}{rgb}{0,0,0}
\definecolor{Grau}{rgb}{0.5,0.5,0.5}
\definecolor{DunkelGrau}{rgb}{0.15,0.15,0.15}
\definecolor{Hellbraun}{rgb}{0.5,0.25,0.0}
\definecolor{Magenta}{rgb}{1.0,0.0,1.0}
\definecolor{Gris}{gray}{0.5}
\definecolor{Vert}{rgb}{0,0.5,0}
\definecolor{SourceHintergrund}{rgb}{1,1.0,0.95}

%
\renewcommand{\lstlistlistingname}{Listings}
\renewcommand{\lstlistingname}{Listing}

\lstnewenvironment{python}[1][]{
\lstset{
language=python,
basicstyle=\ttfamily\footnotesize\setstretch{1}, 	
stringstyle=\color{red}, 
showstringspaces=false, 
alsoletter={1234567890},
otherkeywords={\ , \}, \{},
keywordstyle=\color{blue},
emph={access,and,break,class,continue,def,del,elif ,else,
except,exec,finally,for,from,global,if,import,in,i s,
lambda,not,or,pass,print,raise,return,try,while},
emphstyle=\color{black}\bfseries,
emph={[2]True, False, None, self},
emphstyle=[2]\color{green},
emph={[3]from, import, as},
emphstyle=[3]\color{blue},
upquote=true,
morecomment=[s]{"""}{"""},
commentstyle=\color{Hellbraun}\slshape, 
%emph={[4]1, 2, 3, 4, 5, 6, 7, 8, 9, 0},
emphstyle=[4]\color{blue},
literate=*{:}{{\textcolor{blue}:}}{1}
{=}{{\textcolor{blue}=}}{1}
{-}{{\textcolor{blue}-}}{1}
{+}{{\textcolor{blue}+}}{1}
{*}{{\textcolor{blue}*}}{1}
{!}{{\textcolor{blue}!}}{1}
{(}{{\textcolor{blue}(}}{1}
{)}{{\textcolor{blue})}}{1}
{[}{{\textcolor{blue}[}}{1}
{]}{{\textcolor{blue}]}}{1}
{<}{{\textcolor{blue}<}}{1}
{>}{{\textcolor{blue}>}}{1},
%framexleftmargin=1mm, framextopmargin=1mm, frame=shadowbox, rulesepcolor=\color{blue},#1
backgroundcolor=\color{SourceHintergrund}, 
framexleftmargin=1mm, framexrightmargin=1mm, framextopmargin=1mm, frame=single, framerule=1pt, rulecolor=\color{black},#1
}}{}

%Si le boolen xp est vrai : compilation pour xabi
%Sinon compilation Damien
\newboolean{xp}
\setboolean{xp}{true}

\newboolean{prof}
\setboolean{prof}{false}

\def\xxtitre{\ifthenelse{\boolean{xp}}{
CI 3 : Ingénierie Numérique \& Simulation
}}

\def\xxsoustitre{\ifthenelse{\boolean{xp}}{
TP -- Résolution de l'équation $f(x)=0$.}{
}}


\def\xxauteur{\ifthenelse{\boolean{xp}}{
\noindent \textsl{Eric Olivi} \\
\textsl{Xavier Pessoles}
}{
}}


\def\xxpied{\ifthenelse{\boolean{xp}}{
CI 3 : Ingénierie Numérique \& Simulation -- TP \\
Résolution de $f(x)=0$ -- \ifthenelse{\boolean{prof}}{P}{E}%
}{
}}

\usepackage[%
    pdftitle={Ingénierie Numérique et Simulation},
    pdfauthor={Xavier Pessoles},
    colorlinks=true,
    linkcolor=blue,
    citecolor=magenta]{hyperref}



\usepackage{pifont}
\sloppy
\hyphenpenalty 10000


\begin{document}


\usepackage[%
    pdftitle={\xxtitre},
    pdfauthor={\xxauteur},
    colorlinks=true,
    linkcolor=blue,
    citecolor=magenta]{hyperref}

\usepackage{pifont}


% \makeatletter \let\ps@plain\ps@empty \makeatother
%% DEBUT DU DOCUMENT
%% =================
\sloppy
\hyphenpenalty 10000

\newcommand{\Pointilles}[1][3]{%
\multido{}{#1}{\makebox[\linewidth]{\dotfill}\\[\parskip]
}}


\colorlet{shadecolor}{orange!15}

\newtheorem{theorem}{Theorem}


\begin{document}


\newboolean{prof}
\setboolean{prof}{true}
%------------- En tetes et Pieds de Pages ------------


\pagestyle{fancy}
\renewcommand{\headrulewidth}{0pt}
%\renewcommand{\headrulewidth}{0.2pt} %pour mettre le trait en haut

\fancyhead{}
\fancyhead[L]{%
%\footnotesize{\textit{\textsf{Lycée François Premier}}}%
\noindent\noindent\begin{minipage}[c]{2.6cm}
\includegraphics[width=2cm]{png/logo_ptsi.png}%
\end{minipage}
}

\fancyhead[C]{\rule{12cm}{.5pt}}  %pour mettre le petit trait en haut


\fancyhead[R]{%
\noindent\begin{minipage}[c]{3cm}
\begin{flushright}
\footnotesize{\textit{\textsf{Informatique}}}%
\end{flushright}
\end{minipage}
}

\renewcommand{\footrulewidth}{0.2pt}

\fancyfoot[C]{\footnotesize{\bfseries \thepage}}
\fancyfoot[L]{%
\begin{minipage}[c]{.2\linewidth}
%\noindent\footnotesize{{Damien Iceta}}
\noindent\footnotesize{\textsc{Xavier Pessoles}\\\textsc{Damien Iceta}}
\end{minipage}
%\begin{minipage}[c]{.15\linewidth}
%\includegraphics[width=2cm]{png/logoCC.png}
%\end{minipage}
}

\ifthenelse{\boolean{prof}}{%
\fancyfoot[R]{\footnotesize{\xxpiedd}}}



\begin{center}
 \Large\textsc{\xxtitre}
\end{center}

\begin{center}
 \large\textsc{\xxsoustitre}
\end{center}


%\begin{rem}
%\textbf{Utilisation de Spyder}

%Dans le cadre de ce TP, nous utiliserons l'environnement de programmation Spyder. Pour lancer cette application utiliser le raccourci sur le bureau.

%\end{rem}
%
%\begin{objectif}
%Les objectif de ce TP sont :
%\begin{itemize}
%\item d'acquérir les données provenant d'un fichier texte (au format \textsf{kml});
%\item de réaliser des fonctions permettant d'analyser les données pour avoir accès à différentes statistiques.
%\end{itemize}
%\end{objectif}



\subsection*{Test préliminaire -- Calcul intégral}

Écrire une fonction \textbf{Intgrl}(f,a,b,n) où :
\begin{itemize}
\item a et b sont les bornes (ordonnées) d'un intervalle,
\item n est le nombre de subdivisions de [a;b]
\item f désigne une fonction définie sur [a;b]
\end{itemize}
qui renvoie la somme de aires des trapèzes.



\subsection*{Comparaison des méthodes de résolution}


Soit l'équation \emph{E} : \quad $x\ln(x)=1$ .

\medskip

On admet (si un doute vérifiez-le rapidement) que \emph{E} admet une unique solution $\alpha$ sur $\mathbb R$ et que
$\alpha\in[1;\text{e}]$. On prendra e$\approx2.8$. On peut aussi noter qu'en python \texttt{e} est un attribut de la bibliothèque \texttt{math}. On y a donc accès par \texttt{math.e}.

\medskip

\paragraph*{Méthode de dichotomie}


\begin{minipage}[c]{.47\linewidth}

\subparagraph{} \textit{Complétez la fonction ci-contre renvoyant une valeur approchée à \texttt{epsilon} près de cette solution par dichotomie.}
\subparagraph{} \textit{Vérifier qu'à 10$^{-6}$ près : $\alpha\approx 1.763223$ .}
\subparagraph{} \textit{Modifier cette fonction afin qu'elle affiche le nombre de répétition de la boucle.}

\end{minipage} \hfill
\begin{minipage}[c]{.47\linewidth}
\begin{py}

\begin{minipage}{\linewidth}
\ttfamily

def dichotomie(f,epsilon):

\rule{4ex}{0pt} a=\ldots

\rule{4ex}{0pt} b=\ldots

\rule{4ex}{0pt} while abs(\ldots

\rule{8ex}{0pt} c=(a+b)/2

\rule{8ex}{0pt} if f(a)*f(c) < 0 :

\rule{12ex}{0pt} \ldots =c

\rule{8ex}{0pt} else :

\rule{12ex}{0pt} \ldots =c

\rule{4ex}{0pt} return \ldots
\end{minipage}

\medskip\normalfont

\end{py}
\end{minipage} 




\paragraph*{Méthode des cordes (Lagrange)}


\subparagraph{} \textit{A partir de la fonction précédente, créer une nouvelle fonction nommée \texttt{lagrange} renvoyant une valeur approchée à \texttt{epsilon} près de cette solution par la méthode des cordes (dite de Lagrange).}

\subparagraph{} \textit{Affichez le nombre de répétition de la boucle. Comparer avec la méthode précédente.}




\paragraph*{Méthode des Newton}
\vspace{.25cm}

\begin{minipage}[c]{.47\linewidth}

On souhaite comparer avec la méthode Newton, mais le test d'arrêt ne permet pas d'obtenir une valeur approchée avec un précision donnée. Nous allons donc nous servir des résultats précédents.

\texttt{f} et \texttt{fp} étant des fonctions saisies en python, et correspondant à $f$ et $f'$ alors l'algorithme de Newton est donné ci-contre.


\subparagraph{} \textit{Compléter et implémenter la méthode de Newton dans une fonction nommée \texttt{newton}. }

\subparagraph{} \textit{Comparer les méthodes. }

\end{minipage} \hfill
\begin{minipage}[c]{.47\linewidth}
\begin{py}
\ttfamily
x=2.8

compteur=0 

while abs(x-1.763223)>10**(-6):

\rule{4ex}{0pt} x=\ldots

\rule{4ex}{0pt} compteur=\ldots

print(compteur)

\end{py}
\end{minipage} 


\setcounter{subparagraph}{0}



\subsection*{Suite définie implicitement}

Soit $f_n$ une suite de fonction définie sur $\mathbb R$ pour tout $n\in\mathbb N^*$ tel que :

\hfil $f_n(x)=x^n+x^{n-1}+x-1$

\subparagraph{} \textit{ Montrer que, pour tout $n\in\mathbb N^*$, l'équation $f_n(x)=0$ possède une unique solution sur $[0;1]$.}

\medskip

Notons $u_n$ cette solution.

\subparagraph{} \textit{\'Ecrire un programme Python qui :
\begin{itemize}
\item définit une fonction \texttt{f(x,n)} renvoyant la valeur $f_n(x)$;
\item définit une fonction \texttt{dichotomie(f,n)} renvoyant un valeur à $10^{-3}$ près de $u_n$;
\item calcule et affiche les 100 premières valeurs approchées de $(u_n)$.
\end{itemize}}

\subparagraph{} \textit{Conjecturer du comportement de cette suite.}


\setcounter{subparagraph}{0}

\subsection*{Les délices empoisonnées de la méthode de Newton}

\paragraph*{Partie A}

\begin{minipage}[c]{.57\linewidth}
L'interprétation graphique de la méthode de Newton est donnée ci-contre.

Supposons que les fonctions $f$ et $f'$ soient définies en entête du programme.

\subparagraph{}\textit{Compléter l'algorithme afin de calculer les 20 premières valeurs de la suite générée par la méthode de Newton.}

\begin{pseudo}

\ttfamily

\rule{0pt}{1.5em}
x = x$_0$\\
pour k allant de 1 à 20 faire :\\
\rule{1cm}{0pt} x =
\end{pseudo}


\end{minipage} \hfill
\begin{minipage}[c]{.4\linewidth}
\begin{center}
\includegraphics[width=\textwidth]{images/fig1}
\end{center}
\end{minipage}

%\psset{xunit=1.5cm,yunit=1.0cm,algebraic=true,dimen=middle,dotstyle=o,dotsize=3pt 0,linewidth=0.2pt,arrowsize=3pt 2,arrowinset=0.25}
%\begin{pspicture*}(-1.,-1.)(4.,5.5)
%%\psaxes[labelFontSize=\scriptstyle,xAxis=true,yAxis=true,Dx=1.,Dy=1.,ticksize=-2pt 0,subticks=2]{->}(0,0)(-1.,-1.)(4.,5.5)
%\psline{->}(-1.,0)(4.,0)
%\psplot[linewidth=1pt]{-1.}{4.}{-1+2.71828^(0.5*x)}
%\psplot{-1.}{4.}{-3.2408445351690327+2.2408445351690323*x}
%\psplot{.1}{2}{(-0.4294061482447453+1.0304369953425587*x)}
%\psset{linestyle=dashed}
%\psline(1.4462603202968598,0.)(1.4462603202968598,1.0608739906851175)
%\psline(3.,0.)(3.,3.4816890703380645)
%%\begin{scriptsize}
%\psdots[dotstyle=*](0,0)(3.,0.)(3.,3.4816890703380645)(1.4462603202968598,0.)(1.4462603202968598,1.0608739906851175)(0.4167223713682693,0.)
%\rput(0,.3){$\alpha$}
%\rput(3,-.3){$x_0$}
%\rput(1.4462603202968598,-.3){$x_1$}
%\rput(0.4167223713682693,-.3){$x_2$}
%%\end{scriptsize}
%\end{pspicture*}

\paragraph*{Partie B}

Soit la fonction $f$ définie sur $\mathbb R$ par :\quad
$f(x) = x^3 - x^2 - 20x - 12$.

\medskip

Considérons la suite définie par la méthode de Newton en partant de $x_0=2$

\subparagraph{}\textit{Écrire un programme en Python qui :
\begin{itemize}
\item définit la fonction $f$ ainsi que sa fonction dérivée que vous noterez $fp$,
\item calcule et affiche les valeurs successives de l'algorithme de Newton.
\end{itemize}}


\begin{center}
\includegraphics[width=.5\textwidth]{images/fig2}
\end{center}
%\begin{center}
%\psset{xunit=1.0cm,yunit=.1cm,
%algebraic=true,dimen=middle,dotstyle=*,dotsize=3pt 0,linewidth=0.3pt,arrowsize=3pt 2,arrowinset=0.25}
%\begin{pspicture*}(-5.,-55.)(5.,25.)
%%\multips(0,-50)(0,10.0){9}{\psline[linestyle=dashed,linecap=1,dash=1.5pt 1.5pt,linewidth=0.4pt,linecolor=lightgray]{c-c}(-5.,0)(5.,0)}
%%\multips(-4,0)(2.0,0){6}{\psline[linestyle=dashed,linecap=1,dash=1.5pt 1.5pt,linewidth=0.4pt,linecolor=lightgray]{c-c}(0,-55.)(0,25.)}
%\psaxes[labelFontSize=\scriptstyle,xAxis=true,yAxis=true,Dx=2.,Dy=10.,ticksize=-2pt 0,subticks=2]{->}(0,0)(-5.,-55.)(5.,25.)
%\psplot[plotpoints=200,linewidth=.8pt]{-5.0}{5.0}{x^(3.0)-x^(2.0)-20.0*x-12.0}
%%\psset{linestyle=dotted}
%\psplot{-3.}{4.}{(8.-4.*x)/1.}
%\psplot{-3.5}{3.}{(-24.-12.*x)/1.}
%\psdots(2,-48)(-2,16)
%\end{pspicture*}
%\end{center}


\subsection*{Exercice : Avec des listes}
\setcounter{subparagraph}{0}

Lors d'une expérience on mesure un phénomène numérique au cours du temps et on dresse deux listes (de même longueur) :
\begin{itemize}
\item \texttt{V} : la liste des mesures supposées monotones, 
\item \texttt{T} : la liste des temps (en seconde, dans l'ordre croissant) correspondant à chaque mesure.
\end{itemize}
Exemple : \texttt{T=[0, 2, \ldots]} et \texttt{V=[0.5, 0.55, \ldots]} signifie que 0.5 a été mesuré à 0s, puis la valeur suivante  (0.55) a été prise à 2s etc.


\subparagraph{}\textit{Écrire une fonction \texttt{solution(valeur,V,T)}, qui renvoie un
encadrement du temps pour lequel les mesures encadrent elles-même la valeur.}

Indication :
\begin{itemize}
\item privilégiez la dichotomie,
\item déterminez les indices \texttt{i} et \texttt{j} de cet encadrement,
\item observez que le test d'arrêt de recherche de ces entiers ne dépend pas d'une précision.
\end{itemize}

\subparagraph{}\textit{Tester ce programme avec les listes : \texttt{T=[0, 2, 3, 5, 6, 8, 10]} et \texttt{V=[0.5, 0.55, 0.7, 0.9, 1, 1.5, 1.6]} avec 0.6  pour la valeur.}

\subparagraph{}\textit{Modifier votre fonction \texttt{solution} afin de gérer (par un message d'erreur) le cas où il n'y a pas d'encadrement possible car \texttt{valeur} serait incompatible.}

\end{document}

