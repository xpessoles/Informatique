\documentclass[10pt,oneside]{article}
\input{style/coursHeadings}
\usepackage{algorithm}
\usepackage{algorithmic}


% Python sources
\usepackage{listings}
\usepackage{textcomp}
\usepackage{setspace}
%\usepackage{palatino}

%\usepackage{color}
\definecolor{Bleu}{rgb}{0.1,0.1,1.0}
\definecolor{Noir}{rgb}{0,0,0}
\definecolor{Grau}{rgb}{0.5,0.5,0.5}
\definecolor{DunkelGrau}{rgb}{0.15,0.15,0.15}
\definecolor{Hellbraun}{rgb}{0.5,0.25,0.0}
\definecolor{Magenta}{rgb}{1.0,0.0,1.0}
\definecolor{Gris}{gray}{0.5}
\definecolor{Vert}{rgb}{0,0.5,0}
\definecolor{SourceHintergrund}{rgb}{1,1.0,0.95}

%
\renewcommand{\lstlistlistingname}{Listings}
\renewcommand{\lstlistingname}{Listing}

\lstnewenvironment{python}[1][]{
\lstset{
language=python,
basicstyle=\ttfamily\footnotesize\setstretch{1}, 	
stringstyle=\color{red}, 
showstringspaces=false, 
alsoletter={1234567890},
otherkeywords={\ , \}, \{},
keywordstyle=\color{blue},
emph={access,and,break,class,continue,def,del,elif ,else,
except,exec,finally,for,from,global,if,import,in,i s,
lambda,not,or,pass,print,raise,return,try,while},
emphstyle=\color{black}\bfseries,
emph={[2]True, False, None, self},
emphstyle=[2]\color{green},
emph={[3]from, import, as},
emphstyle=[3]\color{blue},
upquote=true,
morecomment=[s]{"""}{"""},
commentstyle=\color{Hellbraun}\slshape, 
%emph={[4]1, 2, 3, 4, 5, 6, 7, 8, 9, 0},
emphstyle=[4]\color{blue},
literate=*{:}{{\textcolor{blue}:}}{1}
{=}{{\textcolor{blue}=}}{1}
{-}{{\textcolor{blue}-}}{1}
{+}{{\textcolor{blue}+}}{1}
{*}{{\textcolor{blue}*}}{1}
{!}{{\textcolor{blue}!}}{1}
{(}{{\textcolor{blue}(}}{1}
{)}{{\textcolor{blue})}}{1}
{[}{{\textcolor{blue}[}}{1}
{]}{{\textcolor{blue}]}}{1}
{<}{{\textcolor{blue}<}}{1}
{>}{{\textcolor{blue}>}}{1},
%framexleftmargin=1mm, framextopmargin=1mm, frame=shadowbox, rulesepcolor=\color{blue},#1
backgroundcolor=\color{SourceHintergrund}, 
framexleftmargin=1mm, framexrightmargin=1mm, framextopmargin=1mm, frame=single, framerule=1pt, rulecolor=\color{black},#1
}}{}

\usepackage{array,pstricks,pstricks-add,pst-all,pst-plot}


%Si le boolen xp est vrai : compilation pour xabi
%Sinon compilation Damien
\newboolean{xp}
\setboolean{xp}{true}

\newboolean{prof}
\setboolean{prof}{false}

\def\xxtitre{\ifthenelse{\boolean{xp}}{
CI 3 : Ingénierie Numérique \& Simulation
}}

\def\xxsoustitre{\ifthenelse{\boolean{xp}}{
TP -- Méthodes d'intégration numérique}{
}}


\def\xxauteur{\ifthenelse{\boolean{xp}}{
\noindent \textsl{Éric Olivi} \\
\textsl{Xavier Pessoles}
}{
}}


\def\xxpied{\ifthenelse{\boolean{xp}}{
CI 3 : Ingénierie Numérique \& Simulation -- TP \\
Intégration numérique -- \ifthenelse{\boolean{prof}}{P}{E}%
}{
}}

\usepackage[%
    pdftitle={Ingénierie Numérique et Simulation},
    pdfauthor={Xavier Pessoles},
    colorlinks=true,
    linkcolor=blue,
    citecolor=magenta]{hyperref}



\usepackage{pifont}
\sloppy
\hyphenpenalty 10000


\begin{document}


\usepackage[%
    pdftitle={\xxtitre},
    pdfauthor={\xxauteur},
    colorlinks=true,
    linkcolor=blue,
    citecolor=magenta]{hyperref}

\usepackage{pifont}


% \makeatletter \let\ps@plain\ps@empty \makeatother
%% DEBUT DU DOCUMENT
%% =================
\sloppy
\hyphenpenalty 10000

\newcommand{\Pointilles}[1][3]{%
\multido{}{#1}{\makebox[\linewidth]{\dotfill}\\[\parskip]
}}


\colorlet{shadecolor}{orange!15}

\newtheorem{theorem}{Theorem}


\begin{document}


\newboolean{prof}
\setboolean{prof}{true}
%------------- En tetes et Pieds de Pages ------------


\pagestyle{fancy}
\renewcommand{\headrulewidth}{0pt}
%\renewcommand{\headrulewidth}{0.2pt} %pour mettre le trait en haut

\fancyhead{}
\fancyhead[L]{%
%\footnotesize{\textit{\textsf{Lycée François Premier}}}%
\noindent\noindent\begin{minipage}[c]{2.6cm}
\includegraphics[width=2cm]{png/logo_ptsi.png}%
\end{minipage}
}

\fancyhead[C]{\rule{12cm}{.5pt}}  %pour mettre le petit trait en haut


\fancyhead[R]{%
\noindent\begin{minipage}[c]{3cm}
\begin{flushright}
\footnotesize{\textit{\textsf{Informatique}}}%
\end{flushright}
\end{minipage}
}

\renewcommand{\footrulewidth}{0.2pt}

\fancyfoot[C]{\footnotesize{\bfseries \thepage}}
\fancyfoot[L]{%
\begin{minipage}[c]{.2\linewidth}
%\noindent\footnotesize{{Damien Iceta}}
\noindent\footnotesize{\textsc{Xavier Pessoles}\\\textsc{Damien Iceta}}
\end{minipage}
%\begin{minipage}[c]{.15\linewidth}
%\includegraphics[width=2cm]{png/logoCC.png}
%\end{minipage}
}

\ifthenelse{\boolean{prof}}{%
\fancyfoot[R]{\footnotesize{\xxpiedd}}}



\begin{center}
 \Large\textsc{\xxtitre}
\end{center}

\begin{center}
 \large\textsc{\xxsoustitre}
\end{center}





\subsection*{Test-introductif}

\subparagraph*{}
\textit{Déterminer l'affichage final.}

\begin{py}
\begin{python}
L=[10,11,12,13,14,15,16,17,18,19]
n=len(L)
LL=[]
k=0
while k<n-1 :
    LL.append(L[k])
    k=k+2
k=1

while k<n :
    LL.append(L[k])
    k=k+2

print(LL)
\end{python}
\end{py}

\subsection*{Exercice 1}

Le but est d'obtenir un encadrement de
\quad $\displaystyle I=4\int_0^1\frac{\text{d}x}{1+x^2}$

\begin{minipage}[c]{.6\linewidth}
\subparagraph{}
\textit{Compléter cet algorithme et le coder en python afin d'obtenir une valeur approchée de $I$ par la méthode des rectangles à gauche en utilisant les champs suivants : \fbox{x+h}
\fbox{(b-a)/n}
\fbox{h}
\fbox{somme+f(x)}
\fbox{f(a)}.}

\subparagraph{}
\textit{Modifier cet algorithme pour que la méthode soit celle des rectangles à droite.}

\subparagraph{}
\textit{Modifier cet algorithme afin d'afficher les résultats des deux méthodes.}

\subparagraph{}
\textit{Augmenter le nombre de subdivisions.}

\subparagraph{}
\textit{Justifier que la méthode des rectangles à droite donne un minorant de $I$ et que la méthode des rectangles à gauche donne un majorant. }
\end{minipage} \hfill
\begin{minipage}[c]{.35\linewidth}
\begin{py}
\begin{python}
a=0
b=1
n=100
h=
x=a
somme=
for k in range(1,n) :
    x=
    somme=
print(somme* 
\end{python}
\end{py}
\end{minipage}


%\subsection*{Exercice 2}
%\setcounter{subparagraph}{0}
%\subparagraph{}
%\textit{Écrire 4 fonctions nommées \textsf{integraleRectangleGauche}, \textsf{integraleRectangleDroite},
%\textsf{integraleRectangleMilieu}, \textsf{integrationTrapeze} prenant comme arguments une fonction \textsf{f}, deux réels  \textsf{a} et \textsf{b} tels que \textsf{a<b} et un entier \textsf{n} représentant le nombre d'échantillons.}
% 
%\subparagraph{}
%\textit{Écrire une fonction nommée \textsf{integrale} prenant comme arguments une chaîne de caractère correspondant au type d'intégration souhaitée, une fonction, les réels $a$ et $b$ ainsi que le nombre d'échantillon.}
 
 
\newpage


\subsection*{Exercice 2}
\setcounter{subparagraph}{0} 
Pour tout $n\in\mathbb{N}$ soit l'intégrale : 
$$I_n=\int_0^1\frac{x}{1+x^n} \text{d}x$$

\subparagraph{}
\textit{Écrire une fonction d'argument $n$ qui renvoie une valeur approchée de $I_n$ par une des méthodes des rectangles (avec une subdivision de 100 intervalles).}

\subparagraph{}
\textit{Afficher quelques  valeurs de cette suite afin de conjecturer sa monotonie et son comportement asymptotique.}

\subsection*{Exercice 3}
\setcounter{subparagraph}{0}
Lors d'une expérience on mesure un phénomène numérique au cours du temps et on dresse deux listes (de même longueur) :
\begin{itemize}
\item \texttt{V} : la liste des mesures, 
\item \texttt{T} : la liste des temps (en seconde, dans l'ordre croissant) correspondant à chaque mesure.
\end{itemize}
Exemple : \texttt{T=[0, 2, \ldots]} et \texttt{V=[1.5, 0.8, \ldots]} signifie que 1.5 a été mesuré à 0s, puis la valeur suivante  (0.8) a été prise à 2s etc.

\subparagraph{}
\textit{Écrire un programme, qui à partir de ces deux listes, renvoie une valeur approchée de l'intégrale du phénomène au cours du temps par la méthode des rectangles.}

\subparagraph{}
\textit{Tester ce programme avec les listes : \texttt{T=[0, 2, 3, 5, 6, 8]} et \texttt{V=[1.5, 0.8, 0.7, 0.7, 0.9, 1.8]}.}


\subsection*{Exercice 3}
\setcounter{subparagraph}{0}


\begin{minipage}{4cm}
%\psset{unit=3cm}
%\begin{pspicture*}(-0.2,-0.2)(1.1,1.1)
%\psset{algebraic=true,dimen=middle,dotstyle=o,dotsize=3pt 0,linewidth=0.8pt,arrowsize=3pt 2,arrowinset=0.25}
%\multips(0,0)(0,0.2){7}{\psline[linestyle=dashed,linecap=1,dash=1.5pt 1.5pt,linewidth=0.4pt,linecolor=lightgray]{c-c}(-0.1,0)(1.1,0)}
%\multips(0,0)(0.2,0){7}{\psline[linestyle=dashed,linecap=1,dash=1.5pt 1.5pt,linewidth=0.4pt,linecolor=lightgray]{c-c}(0,-0.1)(0,1.1)}
%\psaxes[labelFontSize=\scriptstyle,xAxis=true,yAxis=true,Dx=1,Dy=1,ticksize=-2pt 0,subticks=2]{->}(0,0)(-0.1,-0.1)(1.1,1.1)
%\psplot[plotpoints=200]{-0.1}{1.1}{x^(2.0)}
%\end{pspicture*}
\includegraphics[width=\textwidth]{images/courbe}
\end{minipage}
\hfill
\begin{minipage}{12cm}
On démontre que la longueur $L$ de la courbe $y=x^2$ pour $x\in[0;1]$ dans un repère orthonormal est donnée par :
$$L=\int_0^1\sqrt{1+4x^2}\text{d}x$$

\subparagraph*{}\textit{Calculer $L$ à $10^{-3}$ près.}

\medskip

On pourra commencer par justifier que si $\varphi(x)=\sqrt{1+4x^2}$ pour tout $x\in[0;1]$ alors $\varphi'(x)\leq\frac{4}{\sqrt{5}}<1.8$ .
\end{minipage}\hfill\hbox{}


\end{document}

