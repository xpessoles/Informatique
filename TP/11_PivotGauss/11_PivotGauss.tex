\documentclass[10pt,oneside]{article}
\input{style/coursHeadings}
\usepackage{algorithm}
\usepackage{algorithmic}


% Python sources
\usepackage{listings}
\usepackage{textcomp}
\usepackage{setspace}
%\usepackage{palatino}

%\usepackage{color}
\definecolor{Bleu}{rgb}{0.1,0.1,1.0}
\definecolor{Noir}{rgb}{0,0,0}
\definecolor{Grau}{rgb}{0.5,0.5,0.5}
\definecolor{DunkelGrau}{rgb}{0.15,0.15,0.15}
\definecolor{Hellbraun}{rgb}{0.5,0.25,0.0}
\definecolor{Magenta}{rgb}{1.0,0.0,1.0}
\definecolor{Gris}{gray}{0.5}
\definecolor{Vert}{rgb}{0,0.5,0}
\definecolor{SourceHintergrund}{rgb}{1,1.0,0.95}

%
\renewcommand{\lstlistlistingname}{Listings}
\renewcommand{\lstlistingname}{Listing}

\lstnewenvironment{python}[1][]{
\lstset{
language=python,
basicstyle=\ttfamily\footnotesize\setstretch{1}, 	
stringstyle=\color{red}, 
showstringspaces=false, 
alsoletter={1234567890},
otherkeywords={\ , \}, \{},
keywordstyle=\color{blue},
emph={access,and,break,class,continue,def,del,elif ,else,
except,exec,finally,for,from,global,if,import,in,i s,
lambda,not,or,pass,print,raise,return,try,while},
emphstyle=\color{black}\bfseries,
emph={[2]True, False, None, self},
emphstyle=[2]\color{green},
emph={[3]from, import, as},
emphstyle=[3]\color{blue},
upquote=true,
morecomment=[s]{"""}{"""},
commentstyle=\color{Hellbraun}\slshape, 
%emph={[4]1, 2, 3, 4, 5, 6, 7, 8, 9, 0},
emphstyle=[4]\color{blue},
literate=*{:}{{\textcolor{blue}:}}{1}
{=}{{\textcolor{blue}=}}{1}
{-}{{\textcolor{blue}-}}{1}
{+}{{\textcolor{blue}+}}{1}
{*}{{\textcolor{blue}*}}{1}
{!}{{\textcolor{blue}!}}{1}
{(}{{\textcolor{blue}(}}{1}
{)}{{\textcolor{blue})}}{1}
{[}{{\textcolor{blue}[}}{1}
{]}{{\textcolor{blue}]}}{1}
{<}{{\textcolor{blue}<}}{1}
{>}{{\textcolor{blue}>}}{1},
%framexleftmargin=1mm, framextopmargin=1mm, frame=shadowbox, rulesepcolor=\color{blue},#1
backgroundcolor=\color{SourceHintergrund}, 
framexleftmargin=1mm, framexrightmargin=1mm, framextopmargin=1mm, frame=single, framerule=1pt, rulecolor=\color{black},#1
}}{}

%Si le boolen xp est vrai : compilation pour xabi
%Sinon compilation Damien
\newboolean{xp}
\setboolean{xp}{true}

\newboolean{prof}
\setboolean{prof}{false}

\def\xxtitre{\ifthenelse{\boolean{xp}}{
CI 3 : Ingénierie Numérique \& Simulation
}}

\def\xxsoustitre{\ifthenelse{\boolean{xp}}{
TP -- Résolution de l'équation $f(x)=0$.}{
}}


\def\xxauteur{\ifthenelse{\boolean{xp}}{
\noindent \textsl{Eric Olivi} \\
\textsl{Xavier Pessoles}
}{
}}


\def\xxpied{\ifthenelse{\boolean{xp}}{
CI 3 : Ingénierie Numérique \& Simulation -- TP \\
Pivot de Gauss -- \ifthenelse{\boolean{prof}}{P}{E}%
}{
}}

\usepackage[%
    pdftitle={Ingénierie Numérique et Simulation},
    pdfauthor={Xavier Pessoles},
    colorlinks=true,
    linkcolor=blue,
    citecolor=magenta]{hyperref}



\usepackage{pifont}
\sloppy
\hyphenpenalty 10000


\begin{document}


\usepackage[%
    pdftitle={\xxtitre},
    pdfauthor={\xxauteur},
    colorlinks=true,
    linkcolor=blue,
    citecolor=magenta]{hyperref}

\usepackage{pifont}


% \makeatletter \let\ps@plain\ps@empty \makeatother
%% DEBUT DU DOCUMENT
%% =================
\sloppy
\hyphenpenalty 10000

\newcommand{\Pointilles}[1][3]{%
\multido{}{#1}{\makebox[\linewidth]{\dotfill}\\[\parskip]
}}


\colorlet{shadecolor}{orange!15}

\newtheorem{theorem}{Theorem}


\begin{document}


\newboolean{prof}
\setboolean{prof}{true}
%------------- En tetes et Pieds de Pages ------------


\pagestyle{fancy}
\renewcommand{\headrulewidth}{0pt}
%\renewcommand{\headrulewidth}{0.2pt} %pour mettre le trait en haut

\fancyhead{}
\fancyhead[L]{%
%\footnotesize{\textit{\textsf{Lycée François Premier}}}%
\noindent\noindent\begin{minipage}[c]{2.6cm}
\includegraphics[width=2cm]{png/logo_ptsi.png}%
\end{minipage}
}

\fancyhead[C]{\rule{12cm}{.5pt}}  %pour mettre le petit trait en haut


\fancyhead[R]{%
\noindent\begin{minipage}[c]{3cm}
\begin{flushright}
\footnotesize{\textit{\textsf{Informatique}}}%
\end{flushright}
\end{minipage}
}

\renewcommand{\footrulewidth}{0.2pt}

\fancyfoot[C]{\footnotesize{\bfseries \thepage}}
\fancyfoot[L]{%
\begin{minipage}[c]{.2\linewidth}
%\noindent\footnotesize{{Damien Iceta}}
\noindent\footnotesize{\textsc{Xavier Pessoles}\\\textsc{Damien Iceta}}
\end{minipage}
%\begin{minipage}[c]{.15\linewidth}
%\includegraphics[width=2cm]{png/logoCC.png}
%\end{minipage}
}

\ifthenelse{\boolean{prof}}{%
\fancyfoot[R]{\footnotesize{\xxpiedd}}}



\begin{center}
 \Large\textsc{\xxtitre}
\end{center}

\begin{center}
 \large\textsc{\xxsoustitre}
\end{center}


Une matrice $M\in\mathcal{M}_{n,p}(\mathbb{R})$, avec $n$ lignes et $p$ colonnes, est implémentée sous forme d'une liste de longueur $n$, composée de listes, chacune contenant $p$ nombres.



\medskip

Rappel sur la sélection des listes:

\hfil\begin{tabular}{ll}
si : & \texttt{M=[[a,b],[e,f],[u,v]]}\\
alors : &  \texttt{M} est à 3 lignes et 2 colonnes,\\
&  \texttt{M[2][0]} désigne l'élément \texttt{u}
\end{tabular}


\subsection*{Méthode de Gauss.}

Remarque : si \texttt{M} est une telle liste alors :

\hfil \texttt{nLign=len(M)} \quad et \quad \texttt{nCol=len(M[0])}

sont, respectivement, le nombre de lignes et le nombre de colonnes de \texttt{M}.

\subsubsection*{Combinaison de deux lignes.}


\subparagraph{}
\textit{Écrire une procédure \quad\texttt{transv(M,i,lambda,j)}\quad qui remplace  la ligne n°\texttt{i}, L$_{\texttt{i}}$, par :}

\hfil L$_{\texttt{i}}$ $\longleftarrow$\quad L$_{\texttt{i}}$ + \texttt{lambda}$\times$L$_{\texttt{j}}$

\medskip

\subparagraph{}
\textit{Écrire une procédure \quad\texttt{echang(M,i,j)}\quad qui échange les lignes n°\texttt{i} et n°\texttt{j} dans \texttt{M}.}

\subsubsection*{Le pivot}



\setlength{\columnseprule}{.2pt}

Soit la procédure :
\begin{minipage}[c]{.47\linewidth}
\begin{verbatim}
def megalytero(M,i) :
    anotato=abs(M[i][i])
    j=i
    for k in range(i+1,nLign) :
        if abs(M[k][i])>anotato :
            anotato=abs(M[k][i])
            j=k
    return j
\end{verbatim}
\end{minipage}\hfill
\begin{minipage}[c]{.47\linewidth}
 Soit $M = \begin{pmatrix}
4&0&1&0&0&1&2\\0&-1&5&1&4&0&2\\0&0&2&1&1&1&9\\0&0&1&3&8&8&2\\0&0&0&6&5&4&3\\0&0&8&1&2&8&1\\0&0&3&0&3&3&2 \end{pmatrix}$
\end{minipage}

\subparagraph{}
\textit{Quelle valeur renvoi  \texttt{megalytero(M,2)} ?}



\subsubsection*{Conclusion}


\subparagraph{}
\textit{Compléter la procédure suivante en utilisant les notations et les procédures précédentes afin de modifier \texttt{M} par l'algorithme de Gauss :}
%
\begin{verbatim}
def Pivot(M):
    for i in ............... :
        i0=megalytero(M,i)
        ................
        if M[i][i] ..... : 
            for z in range( ... , ... ):
                transv(... , ... , .......)
    return
\end{verbatim}

\subsection*{Un exemple}




Soit \texttt{M=[[0,1,2,3,4,5],[6,7,8,9,0,1],[2,3,4,5,6,7],[8,9,0,1,2,3],[4,5,6,7,8,9],[0,1,2,3,4,5]]}.

\subparagraph{}
\textit{Écrire une série d'instructions qui, pour tout \texttt{k} entre 0 et \texttt{nCol}, affichent :}

\hfil L$_\texttt{k,0}$\qquad L$_\texttt{k,1}$\qquad\ldots\qquad L$_\texttt{k,nCol}$

\textit{où deux espaces séparent chaque terme de cette ligne.}

\subparagraph{}
\textit{Compléter la question précédente afin d'afficher la matrice \texttt{M}.}

\subparagraph{}
\textit{Afficher la matrice obtenue par l'algorithme de Gauss sur \texttt{M}.}

\subparagraph{}
\textit{En déduire un majorant du rang de \texttt{M}.}


\end{document}

\subsection*{Déterminant.}

Écrire une fonction \texttt{determinant(M)} qui renvoie le déterminant d'une matrice carrée M.

Il faudra vérifier que la matrice \texttt{M} est bien carrée.

\subsection*{Inversion.}

Écrire une fonction \texttt{inversion(M)} qui renvoie l'inverse d'une matrice carrée \texttt{M}.

\end{document}