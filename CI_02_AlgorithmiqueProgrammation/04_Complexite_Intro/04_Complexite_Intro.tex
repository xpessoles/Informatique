\documentclass[10pt,oneside]{article}
\input{style/coursHeadings}
\usepackage{algorithm}
\usepackage{algorithmic}


% Python sources
\usepackage{listings}
\usepackage{textcomp}
\usepackage{setspace}
%\usepackage{palatino}

%\usepackage{color}
\definecolor{Bleu}{rgb}{0.1,0.1,1.0}
\definecolor{Noir}{rgb}{0,0,0}
\definecolor{Grau}{rgb}{0.5,0.5,0.5}
\definecolor{DunkelGrau}{rgb}{0.15,0.15,0.15}
\definecolor{Hellbraun}{rgb}{0.5,0.25,0.0}
\definecolor{Magenta}{rgb}{1.0,0.0,1.0}
\definecolor{Gris}{gray}{0.5}
\definecolor{Vert}{rgb}{0,0.5,0}
\definecolor{SourceHintergrund}{rgb}{1,1.0,0.95}

%
\renewcommand{\lstlistlistingname}{Listings}
\renewcommand{\lstlistingname}{Listing}

\lstnewenvironment{python}[1][]{
\lstset{
language=python,
basicstyle=\ttfamily\footnotesize\setstretch{1}, 	
stringstyle=\color{red}, 
showstringspaces=false, 
alsoletter={1234567890},
otherkeywords={\ , \}, \{},
keywordstyle=\color{blue},
emph={access,and,break,class,continue,def,del,elif ,else,
except,exec,finally,for,from,global,if,import,in,i s,
lambda,not,or,pass,print,raise,return,try,while},
emphstyle=\color{black}\bfseries,
emph={[2]True, False, None, self},
emphstyle=[2]\color{green},
emph={[3]from, import, as},
emphstyle=[3]\color{blue},
upquote=true,
morecomment=[s]{"""}{"""},
commentstyle=\color{Hellbraun}\slshape, 
%emph={[4]1, 2, 3, 4, 5, 6, 7, 8, 9, 0},
emphstyle=[4]\color{blue},
literate=*{:}{{\textcolor{blue}:}}{1}
{=}{{\textcolor{blue}=}}{1}
{-}{{\textcolor{blue}-}}{1}
{+}{{\textcolor{blue}+}}{1}
{*}{{\textcolor{blue}*}}{1}
{!}{{\textcolor{blue}!}}{1}
{(}{{\textcolor{blue}(}}{1}
{)}{{\textcolor{blue})}}{1}
{[}{{\textcolor{blue}[}}{1}
{]}{{\textcolor{blue}]}}{1}
{<}{{\textcolor{blue}<}}{1}
{>}{{\textcolor{blue}>}}{1},
%framexleftmargin=1mm, framextopmargin=1mm, frame=shadowbox, rulesepcolor=\color{blue},#1
backgroundcolor=\color{SourceHintergrund}, 
framexleftmargin=1mm, framexrightmargin=1mm, framextopmargin=1mm, frame=single, framerule=1pt, rulecolor=\color{black},#1
}}{}


%Si le boolen xp est vrai : compilation pour xabi
%Sinon compilation Damien
\newboolean{xp}
\setboolean{xp}{true}

%\newboolean{prof}
%\setboolean{prof}{true}

\def\xxtitre{\ifthenelse{\boolean{xp}}{
CI 2 : Algorithmique \& Programmation}{
Chapitre  -- }}

\def\xxsoustitre{\ifthenelse{\boolean{xp}}{
Chapitre 4 -- Introduction à la complexité}{
Partie  -- }}

\def\xxauteur{\ifthenelse{\boolean{xp}}{
Xavier \textsc{Pessoles} \\ Damien \textsc{Iceta}}{
Damien \textsc{Iceta} \\ Xavier \textsc{Pessoles}}}

\def\xxpied{\ifthenelse{\boolean{xp}}{
Cours -- CI 2 : Algorithmique \& Programmation\\
Ch. 3 : Introduction à la complexité}{
\xxtitre}}

\def\xxcathegorie{\ifthenelse{\boolean{xp}}{
2013 -- 2014 \\
Xavier \textsc{Pessoles}}{
Informatique - Cours}}

\ifthenelse{\boolean{xp}}{\usepackage[%
    pdftitle={Représentation des nombres},
    pdfauthor={Xavier Pessoles},
    colorlinks=true,
    linkcolor=blue,
    citecolor=magenta]{hyperref}

\usepackage{pifont}
%\usepackage{lastpage}

% \makeatletter \let\ps@plain\ps@empty \makeatother
%% DEBUT DU DOCUMENT
%% =================
\sloppy
\hyphenpenalty 10000


\colorlet{shadecolor}{orange!15}

\newtheorem{theorem}{Theorem}


\begin{document}


%\newboolean{prof}
%\setboolean{prof}{true}
% \makeatletter \let\ps@plain\ps@empty \makeatother
%% DEBUT DU DOCUMENT
%% =================




%------------- En tetes et Pieds de Pages ------------


\pagestyle{fancy}
\ifthenelse{\boolean{xp}}{%
\renewcommand{\headrulewidth}{0pt}}{%
\renewcommand{\headrulewidth}{0.2pt}} %pour mettre le trait en haut
%\renewcommand{\headrulewidth}{0.2pt}

\fancyhead{}
\fancyhead[L]{%
\noindent\begin{minipage}[c]{2.6cm}%
\includegraphics[width=2cm]{png/logo_ptsi.png}%
\end{minipage}}


\fancyhead[C]{\rule{12cm}{.5pt}}



\fancyhead[R]{%
\noindent\begin{minipage}[c]{3cm}
\begin{flushright}
\footnotesize{\textit{\textsf{Informatique}}}%
\end{flushright}
\end{minipage}
}



\fancyhead[C]{\rule{12cm}{.5pt}}

\renewcommand{\footrulewidth}{0.2pt}

\fancyfoot[C]{\footnotesize{\bfseries \thepage}}
\fancyfoot[L]{%
\begin{minipage}[c]{.2\linewidth}
\noindent\footnotesize{{\xxauteur}}
\end{minipage}
\ifthenelse{\boolean{xp}}{}{%
\begin{minipage}[c]{.15\linewidth}
\includegraphics[width=2cm]{png/logoCC.png}
\end{minipage}}
}

\ifthenelse{\boolean{prof}}{%
\fancyfoot[R]{\footnotesize{\xxpied}}}

\begin{center}
 \huge\textsc{\xxtitre}
\end{center}

\begin{center}
 \LARGE\textsc{\xxsoustitre}
\end{center}

\vspace{.5cm}
}{\input{style/enteteDI}}


%---------------------------------------------------------------------------



\begin{savoir}
\textsc{Savoirs :}
\begin{itemize}
\item s'interroger sur l'efficacité algorithmique temporelle.
\end{itemize}
\end{savoir}

\section{Premier exemple}
On introduit les algorithmes de tri suivant : 
\begin{py}
\begin{python}
#Tri par sélection
def tri_selection(tab):
    for i in range(0,len(tab)):
        indice = i
        for j in range(i+1,len(tab)):
            if tab[j]<tab[indice]:
               indice = j
        tab[i],tab[indice]=tab[indice],tab[i]
    return tab
\end{python}
\end{py}

\begin{py}
\begin{python}    
#Tri par insertion
def tri_insertion(tab):
    for i in range(1,len(tab)):
        a=tab[i] 
        j=i-1    
        while j>=0 and tab[j]>a:
            tab[j+1]=tab[j]
            j=j-1
        tab[j+1]=a
    return tab
\end{python}
\end{py}

\begin{py}
\begin{python}
def shellSort(array):
     "Shell sort using Shell's (original) gap sequence: n/2, n/4, ..., 1."
     "http://en.wikibooks.org/wiki/Algorithm_Implementation/Sorting/Shell_sort#Python"
     gap = len(array) // 2
     # loop over the gaps
     while gap > 0:
         # do the insertion sort
         for i in range(gap, len(array)):
             val = array[i]
             j = i
             while j >= gap and array[j - gap] > val:
                 array[j] = array[j - gap]
                 j -= gap
             array[j] = val
         gap //= 2
\end{python}
\end{py}

La figure ci-dessous montre le temps en secondes pour trier des tableaux de 1 à 1000 éléments en utilisant les méthodes de tri suivant : 
\begin{itemize}
\item tri par sélection;
\item tri par insertion; 
\item tri shell;
\item méthode de tri utilisée par Python.
\end{itemize}

Le premier graphe montre le temps de calcul et le second une estimation du nombre d'opérations. 

\begin{minipage}[c]{.48\linewidth}
\begin{center}
\includegraphics[width=.95\textwidth]{images/figure_1}
\end{center}
\end{minipage}\hfill
\begin{minipage}[c]{.48\linewidth}
\begin{center}
\includegraphics[width=.95\textwidth]{images/figure_2}
\end{center}
\end{minipage}

\section{Deuxième exemple}
On prend maintenant l'exemple de la recherche d'un élément dans une liste : 

\begin{py}
\begin{python}
def recherche(x,tab):
    for i in range(len(tab)):
        if tab[i]==x:
            return True
    return False
\end{python}
\end{py}


\begin{py}
\begin{python}
def recherche_dichotomique(x, a):
    g, d = 0, len(a)-1
    while g <= d:
        m = (g + d) // 2
        if a[m] == x:
            return True
        if a[m] < x:
            g = m+1
        else:
            d = m-1
    return None
\end{python}
\end{py}


\begin{minipage}[c]{.48\linewidth}
\begin{center}
\includegraphics[width=.95\textwidth]{images/figure_3}
\end{center}
\end{minipage}\hfill
\begin{minipage}[c]{.48\linewidth}
\begin{center}
\includegraphics[width=.95\textwidth]{images/figure_4}
\end{center}
\end{minipage}


%\section{Troisième exemple}


%\begin{thebibliography}{2}
%\bibitem{denis}{François Denis \url{http://pageperso.lif.univ-mrs.fr/~francois.denis/algoL2/chap1.pdf}}
%\bibitem{soyeur}{Alain Soyeur \url{http://asoyeur.free.fr/}}
%\bibitem{wack}{Wack et Al., \textit{L'informatique pour tous en classes préparatoires aux grandes écoles}, Editions Eyrolles.}
%\bibitem{yb}{Yves Benjamin, Vérification de la validité d'un programme.}
%\end{thebibliography}
\end{document}
