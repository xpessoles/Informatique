\documentclass[10pt,oneside]{article}
\input{style/coursHeadings}
\usepackage{algorithm}
\usepackage{algorithmic}


% Python sources
\usepackage{listings}
\usepackage{textcomp}
\usepackage{setspace}
%\usepackage{palatino}

%\usepackage{color}
\definecolor{Bleu}{rgb}{0.1,0.1,1.0}
\definecolor{Noir}{rgb}{0,0,0}
\definecolor{Grau}{rgb}{0.5,0.5,0.5}
\definecolor{DunkelGrau}{rgb}{0.15,0.15,0.15}
\definecolor{Hellbraun}{rgb}{0.5,0.25,0.0}
\definecolor{Magenta}{rgb}{1.0,0.0,1.0}
\definecolor{Gris}{gray}{0.5}
\definecolor{Vert}{rgb}{0,0.5,0}
\definecolor{SourceHintergrund}{rgb}{1,1.0,0.95}

%
\renewcommand{\lstlistlistingname}{Listings}
\renewcommand{\lstlistingname}{Listing}

\lstnewenvironment{python}[1][]{
\lstset{
language=python,
basicstyle=\ttfamily\footnotesize\setstretch{1}, 	
stringstyle=\color{red}, 
showstringspaces=false, 
alsoletter={1234567890},
otherkeywords={\ , \}, \{},
keywordstyle=\color{blue},
emph={access,and,break,class,continue,def,del,elif ,else,
except,exec,finally,for,from,global,if,import,in,i s,
lambda,not,or,pass,print,raise,return,try,while},
emphstyle=\color{black}\bfseries,
emph={[2]True, False, None, self},
emphstyle=[2]\color{green},
emph={[3]from, import, as},
emphstyle=[3]\color{blue},
upquote=true,
morecomment=[s]{"""}{"""},
commentstyle=\color{Hellbraun}\slshape, 
%emph={[4]1, 2, 3, 4, 5, 6, 7, 8, 9, 0},
emphstyle=[4]\color{blue},
literate=*{:}{{\textcolor{blue}:}}{1}
{=}{{\textcolor{blue}=}}{1}
{-}{{\textcolor{blue}-}}{1}
{+}{{\textcolor{blue}+}}{1}
{*}{{\textcolor{blue}*}}{1}
{!}{{\textcolor{blue}!}}{1}
{(}{{\textcolor{blue}(}}{1}
{)}{{\textcolor{blue})}}{1}
{[}{{\textcolor{blue}[}}{1}
{]}{{\textcolor{blue}]}}{1}
{<}{{\textcolor{blue}<}}{1}
{>}{{\textcolor{blue}>}}{1},
%framexleftmargin=1mm, framextopmargin=1mm, frame=shadowbox, rulesepcolor=\color{blue},#1
backgroundcolor=\color{SourceHintergrund}, 
framexleftmargin=1mm, framexrightmargin=1mm, framextopmargin=1mm, frame=single, framerule=1pt, rulecolor=\color{black},#1
}}{}


%Si le boolen xp est vrai : compilation pour xabi
%Sinon compilation Damien
\newboolean{xp}
\setboolean{xp}{true}

%\newboolean{prof}
%\setboolean{prof}{true}

\def\xxtitre{\ifthenelse{\boolean{xp}}{
CI 2 : Algorithmique \& Programmation}{
Chapitre  -- }}

\def\xxsoustitre{\ifthenelse{\boolean{xp}}{
Chapitre 3 -- Variant et invariant de boucle}{
Partie  -- }}

\def\xxauteur{\ifthenelse{\boolean{xp}}{
Xavier \textsc{Pessoles} \\ Damien \textsc{Iceta}}{
Damien \textsc{Iceta} \\ Xavier \textsc{Pessoles}}}

\def\xxpied{\ifthenelse{\boolean{xp}}{
Cours -- CI 2 : Algorithmique \& Programmation\\
Ch. 3 : Invariance de boucle}{
\xxtitre}}

\def\xxcathegorie{\ifthenelse{\boolean{xp}}{
2013 -- 2014 \\
Xavier \textsc{Pessoles}}{
Informatique - Cours}}

\ifthenelse{\boolean{xp}}{\usepackage[%
    pdftitle={Représentation des nombres},
    pdfauthor={Xavier Pessoles},
    colorlinks=true,
    linkcolor=blue,
    citecolor=magenta]{hyperref}

\usepackage{pifont}
%\usepackage{lastpage}

% \makeatletter \let\ps@plain\ps@empty \makeatother
%% DEBUT DU DOCUMENT
%% =================
\sloppy
\hyphenpenalty 10000


\colorlet{shadecolor}{orange!15}

\newtheorem{theorem}{Theorem}


\begin{document}


%\newboolean{prof}
%\setboolean{prof}{true}
% \makeatletter \let\ps@plain\ps@empty \makeatother
%% DEBUT DU DOCUMENT
%% =================




%------------- En tetes et Pieds de Pages ------------


\pagestyle{fancy}
\ifthenelse{\boolean{xp}}{%
\renewcommand{\headrulewidth}{0pt}}{%
\renewcommand{\headrulewidth}{0.2pt}} %pour mettre le trait en haut
%\renewcommand{\headrulewidth}{0.2pt}

\fancyhead{}
\fancyhead[L]{%
\noindent\begin{minipage}[c]{2.6cm}%
\includegraphics[width=2cm]{png/logo_ptsi.png}%
\end{minipage}}


\fancyhead[C]{\rule{12cm}{.5pt}}



\fancyhead[R]{%
\noindent\begin{minipage}[c]{3cm}
\begin{flushright}
\footnotesize{\textit{\textsf{Informatique}}}%
\end{flushright}
\end{minipage}
}



\fancyhead[C]{\rule{12cm}{.5pt}}

\renewcommand{\footrulewidth}{0.2pt}

\fancyfoot[C]{\footnotesize{\bfseries \thepage}}
\fancyfoot[L]{%
\begin{minipage}[c]{.2\linewidth}
\noindent\footnotesize{{\xxauteur}}
\end{minipage}
\ifthenelse{\boolean{xp}}{}{%
\begin{minipage}[c]{.15\linewidth}
\includegraphics[width=2cm]{png/logoCC.png}
\end{minipage}}
}

\ifthenelse{\boolean{prof}}{%
\fancyfoot[R]{\footnotesize{\xxpied}}}

\begin{center}
 \huge\textsc{\xxtitre}
\end{center}

\begin{center}
 \LARGE\textsc{\xxsoustitre}
\end{center}

\vspace{.5cm}
}{\input{style/enteteDI}}


%---------------------------------------------------------------------------



\begin{savoir}
\textsc{Savoirs :}
\begin{itemize}
\item justifier qu'une itération (ou boucle) produit l'effet attendu au moyen d'un invariant;
\item démontrer qu'une boucle se termine effectivement.
%\item s'interroger sur l'efficacité algorithmique temporelle.
\end{itemize}
\end{savoir}

L'algorithme suivant permet de réaliser une division euclidienne : 
\begin{pseudo}
\begin{algorithm}[H]
\KwData{$a,b \in\mathbb{N}^*$}
reste $\gets$ a\\
quotient $\gets$ 0\\
\Tq{reste $\geq$ b}{
reste $\gets$ reste $-$ b\\
quotient $\gets$ quotient $+$ 1\\
}
\end{algorithm}
\end{pseudo}
 
\begin{prob}
\begin{itemize}
\item Comment s'assurer que l'algorithme en question réalise bien une division euclidienne ?
\item Comment s'assurer que la boucle \textsf{while} se termine ?
\end{itemize}
\end{prob}
\setlength{\parskip}{0ex plus 0.2ex minus 0ex}
 \renewcommand{\contentsname}{}
 \renewcommand{\baselinestretch}{1}

\tableofcontents

 \renewcommand{\baselinestretch}{1.2}
\setlength{\parskip}{2ex plus 0.5ex minus 0.2ex}

% \vspace{1cm}
\textit{Ce document évolue. Merci de signaler toutes erreurs ou coquilles.}

\section{Variant}
\subsection{Définition}
En utilisant l'exemple précédent de la division euclidienne, pour sortir de la boucle, il est nécessaire que le reste devienne, au cours de l'algorithme, inférieur à la valeur de $b$. Dans ce cas, on constate que $b$ étant un entier naturel, la valeur du reste va diminuer à chaque tour de boucle. Ainsi, on peut intuiter que cet algorithme se terminera. 

\begin{defi}
\textbf{Variant de boucle}

Soit une condition booléenne permettant de sortir d'une boucle constituée d'une comparaison entre une variable et une constante de types entiers positifs. La variable est un variant de boucle si :
\begin{itemize}
\item elle reste positive tout au long de l'algorithme;
\item elle décroit (ou croit) strictement à chaque itération de la boucle. 
\end{itemize}

Ainsi, après un nombre fini d'itérations, on est sûr que la boucle se terminera. 

\end{defi}

\begin{rem}
Un variant de boucle \textbf{permet} de s'assurer qu'une boucle se terminera.

Un variant de boucle \textbf{ne permet pas} de s'assurer qu'un algorithme fournit la réponse attendue.
\end{rem}

\subsection{Exemple -- L'algorithme d'Euclide.}


Cet algorithme permet de calculer le PGCD de deux nombres entiers. Il se base sur le fait que si $a$ et $b$ sont deux entiers naturels non nuls, $pgcd(a,b)=pgcd(b, a \text{mod} b)$. 

\begin{minipage}[c]{.47\linewidth}
\begin{pseudo}
\begin{algorithm}[H]
\KwData{$a,b \in\mathbb{N}^*$, $a<b$}
$x\gets a$\\
$y\gets b$\\
\Tq{$y\neq 0$}{
$r\gets$ reste de la division euclidienne de $x$ par $y$\\
$x\gets y$\\
$y\gets r$
}

\end{algorithm}
\end{pseudo}
\end{minipage}
\begin{minipage}[c]{.47\linewidth}\hfill
\begin{exemple}
Calculons le PGCD de 240 et 64 :
\begin{eqnarray*}
240 = 3\cdot 64 + 48 \\
64 = 1\cdot 48 + 16 \\
48 = 3\cdot 16 + 0 \\
\end{eqnarray*}
\end{exemple}
\end{minipage}

Dans cet exemple, la condition pour sortir de la boucle est que $y$ devienne nul. $y$ est un variant de boucle si $y$ est un entier positif qui qui décroit strictement à chaque itération de boucle. Dans ces conditions, on est sûr que la boucle se terminera. 

Le reste de la division euclidienne de $x$ par $y$, $r$, est un entier positif strictement inférieur à $y$. De plus, à chaque itération, $y$ prend la valeur de $r$. $y$ décroit donc à chaque itération. 

En conséquence, $y$ est un variant de boucle.


\section{Invariant de boucle}
\subsection{Problématique}
Lors de la réalisation d'un algorithme, il est nécessaire
\begin{itemize} 
\item de vérifier que celui-ci permet bien de répondre au problème initial;
\item de s'assurer que celui-ci se termine sans quoi le résultat du problème ne sera jamais délivré à l'utilisateur.
\end{itemize}

On s'intéresse ici uniquement aux structures itératives. (Les structures récursives seront traitées en seconde année.)

\begin{defi}
\textbf{Invariant de boucle \cite{denis}}

Un invariant de boucle est une propriété ou une formule logique, 
\begin{itemize}
\item qui est vérifiée après la phase d'initialisation;
\item qui reste vraie après l'exécution d'une itération;
\item et qui, conjointement à la condition d'arrêt, permet de montrer que le résultat attendu et bien le résultat calculé. 
\end{itemize}
\end{defi}


\begin{methode}
\cite{soyeur}
\begin{enumerate}
\item Définir les préconditions (état des variables avant d’entrer dans la boucle).
\item Définir un invariant de boucle.
\item Prouver l’invariant (correspond à $\mathcal{P}(n) \Longleftarrow \mathcal{P}(n + 1)$).
\item Montrer la terminaison du programme.
\item Condition de sortie de boucle + invariant de boucle $\Longleftarrow$ postcondition.
\end{enumerate}
\end{methode}

\subsection{Exemple division euclidienne}
Utilisons la méthode définie précédemment. 

\textbf{1. }Initialement, l'utilisateur saisit les entiers $a$ et $b$. Le reste prend la valeur de $a$ et me quotient prend la valeur 0. 

\textbf{2. }On postule que la quantité $b\cdot\text{quotient}+\text{reste}= a $ est un invariant de boucle.

\textbf{3. } Démontrons par récurrence le postulat précédent.
\begin{itemize}
\item Avant d'entrer dans la boucle, on a bien $b\cdot\text{quotient}_0+\text{reste}_0= b\cdot 0 + a = a $. Le postulat est donc vrai.
\item Admettons le postulat vrai au rang $n$ : $b\cdot\text{quotient}_n+\text{reste}_n = a$.
\item Montrons que le postulat est vrai au rang $n+1$ :
\begin{eqnarray*}
b\cdot\text{quotient}_{n+1}+\text{reste}_{n+1} =  
b\left(\text{quotient}_{n}+1\right)+\text{reste}_{n}-b=  
b\text{quotient}_{n} +\text{reste}_{n}  +b-b = a
\end{eqnarray*}
La relation est donc vraie au rang $n+1$. 
\end{itemize}

\textbf{4. } On a montré la terminaison du programme grâce au variant de boucle. 

\textbf{5. } En sortie de programme, l'invariant de boucle reste vérifié. 


\subsection{Exemple -- L'algorithme d'Euclide.}


Cet algorithme permet de calculer le PGCD de deux nombres entiers. Il se base sur le fait que si $a$ et $b$ sont deux entiers naturels non nuls, $pgcd(a,b)=pgcd(b, a \text{mod} b)$. 


\begin{pseudo}
\begin{algorithm}[H]
\KwData{$a,b \in\mathbb{N}^*$, $a<b$}
$x\gets a$\\
$y\gets b$\\
\Tq{$y\neq 0$}{
$r\gets$ reste de la division euclidienne de $x$ par $y$\\
$x\gets y$\\
$y\gets r$
}

\end{algorithm}
\end{pseudo}

\begin{exemple}
Calculons le PGCD de 240 et 64 :
\begin{eqnarray*}
240 = 3\cdot 64 + 48 \\
64 = 1\cdot 48 + 16 \\
48 = 3\cdot 16 + 0 \\
\end{eqnarray*}

$pgcd(240,64)$ est le dernier reste non nul à savoir 16.

\end{exemple}

\begin{demo}
Soient $a$ et $b$, tels que $(a;b)\in \mathbb{N}^*$ tels que $a > b$. Soient $(q;r)\in \mathbb{N}$ tels que $a=b\cdot q+r$. Avec $r$ le reste de la division euclidienne de $a$ par $b$ et $q$ le quotient.

\textbf{Montrons que $PGCD(a,b)=PGCD(b,r)$.}

Soit $d\in\mathbb{N}$ un diviseur commun de $a$ et $b$. Dans ce cas, $d$ divise $a-b\cdot q$; donc $d$ divise $r$.

En conclusion, si $d$ divise $a$ et $b$ alors $PGCD(a,b)=PGCD(b,r)$.

Réciproquement, soit $d\in\mathbb{N}$ un diviseur commun de $b$ et $r$. Dans ce cas, $d$ divise $b\cdot q+r$; donc $d$ divise $a$.

En conclusion, si $d$ divise $b$ et $r$ alors $PGCD(a,b)=PGCD(b,r)$.

On a donc $PGCD(a,b)=PGCD(b,r)$.

\end{demo}


\begin{demo}
\textbf{Preuve par invariant de boucle de l'algorithme d'Euclide}

\begin{enumerate}
\item Initialement, $a$ et $b$ sont des données saisies par l'utilisateur. On a $x_0 = a$ et $y_0 = b$. 
\item On postule que l'invariant de boucle est donné par $x=y\cdot q + r \Longleftrightarrow r = x - y\cdot q$.
\item 
\begin{itemize}
\item Lors de la première itération, $r_1$ est tel que $r_1 = x_1 - y_1\cdot q_1$.
\item Lors de la nième itération, $r_n$ est tel que $r_n = x_n - y_n\cdot q_n$.
\item \`A l'itération suivante, $x_{n+1}=y_n$ et $y_{n+1}=r_n$, on peut donc trouver $r_{n+1}$ tel que $r_{n+1} = x_{n+1} - y_{n+1}\cdot q_{n+1}$.
\end{itemize}
\item On a montré précédemment que l'itération se terminait. 
\item En sortant de la boucle, l'invariant reste vrai.
\end{enumerate}

%Montrons que la propriété  $x=y\cdot q+r$ est un invariant de boucle :
%\begin{itemize}
%\item au premier incrément, $r$ étant le reste de la division euclidienne de $x$ par $y$, on a donc bien  $x=y\cdot q+r$;
%\item à l'incrément suivant, $x'\gets y$ et $y' \gets r$. On peut donc trouver $r'$ tel que $x'=y'\cdot q'+r'$. La propriété reste donc vraie;
%\item par définition de la division euclidienne, $r<y$; donc à chaque incrément la quantité $y$ diminue. La boucle pourra donc s'arrêter.
%\end{itemize}
\end{demo}

%\subsection{Crible d'Ératosthène}
%Cet algorithme permet de déterminer les nombre premiers d'une liste de $n$ nombre entiers.
%
%\begin{pseudo}
%\begin{algorithm}[H]
%\KwData{$n \in\mathbb{N}^*$}
%$tab$ contient la liste des $n$ entiers\\
%\For{$i=1$ \KwTo $n$}{
%\For{$j=n$ \KwTo $i$}{
%\If{$tab[j]$ modulo $tab[i] =0$}{
%Supprimer $tab[j]$
%}}
%}
%
%\end{algorithm}
%\end{pseudo}




\begin{thebibliography}{2}
\bibitem{denis}{François Denis \url{http://pageperso.lif.univ-mrs.fr/~francois.denis/algoL2/chap1.pdf}}
\bibitem{soyeur}{Alain Soyeur \url{http://asoyeur.free.fr/}}
\bibitem{wack}{Wack et Al., \textit{L'informatique pour tous en classes préparatoires aux grandes écoles}, Editions Eyrolles.}
\bibitem{Yves Benjamin, Vérification de la validité d'un programme.}
\end{thebibliography}
\end{document}
