\documentclass[10pt]{article}
\input{style/coursHeadings}
\usepackage{algorithm}
\usepackage{algorithmic}


% Python sources
\usepackage{listings}
\usepackage{textcomp}
\usepackage{setspace}
%\usepackage{palatino}

%\usepackage{color}
\definecolor{Bleu}{rgb}{0.1,0.1,1.0}
\definecolor{Noir}{rgb}{0,0,0}
\definecolor{Grau}{rgb}{0.5,0.5,0.5}
\definecolor{DunkelGrau}{rgb}{0.15,0.15,0.15}
\definecolor{Hellbraun}{rgb}{0.5,0.25,0.0}
\definecolor{Magenta}{rgb}{1.0,0.0,1.0}
\definecolor{Gris}{gray}{0.5}
\definecolor{Vert}{rgb}{0,0.5,0}
\definecolor{SourceHintergrund}{rgb}{1,1.0,0.95}

%
\renewcommand{\lstlistlistingname}{Listings}
\renewcommand{\lstlistingname}{Listing}

\lstnewenvironment{python}[1][]{
\lstset{
language=python,
basicstyle=\ttfamily\footnotesize\setstretch{1}, 	
stringstyle=\color{red}, 
showstringspaces=false, 
alsoletter={1234567890},
otherkeywords={\ , \}, \{},
keywordstyle=\color{blue},
emph={access,and,break,class,continue,def,del,elif ,else,
except,exec,finally,for,from,global,if,import,in,i s,
lambda,not,or,pass,print,raise,return,try,while},
emphstyle=\color{black}\bfseries,
emph={[2]True, False, None, self},
emphstyle=[2]\color{green},
emph={[3]from, import, as},
emphstyle=[3]\color{blue},
upquote=true,
morecomment=[s]{"""}{"""},
commentstyle=\color{Hellbraun}\slshape, 
%emph={[4]1, 2, 3, 4, 5, 6, 7, 8, 9, 0},
emphstyle=[4]\color{blue},
literate=*{:}{{\textcolor{blue}:}}{1}
{=}{{\textcolor{blue}=}}{1}
{-}{{\textcolor{blue}-}}{1}
{+}{{\textcolor{blue}+}}{1}
{*}{{\textcolor{blue}*}}{1}
{!}{{\textcolor{blue}!}}{1}
{(}{{\textcolor{blue}(}}{1}
{)}{{\textcolor{blue})}}{1}
{[}{{\textcolor{blue}[}}{1}
{]}{{\textcolor{blue}]}}{1}
{<}{{\textcolor{blue}<}}{1}
{>}{{\textcolor{blue}>}}{1},
%framexleftmargin=1mm, framextopmargin=1mm, frame=shadowbox, rulesepcolor=\color{blue},#1
backgroundcolor=\color{SourceHintergrund}, 
framexleftmargin=1mm, framexrightmargin=1mm, framextopmargin=1mm, frame=single, framerule=1pt, rulecolor=\color{black},#1
}}{}
%%%%%%%%%%%%
% Définition des vecteurs 
%%%%%%%%%%%%
\newcommand{\vect}[1]{\overrightarrow{#1}}
\newcommand{\axe}[2]{\left(#1,\vect{#2}\right)}
\newcommand{\couple}[2]{\left(#1,\vect{#2}\right)}
\newcommand{\angl}[2]{\left(\vect{#1},\vect{#2}\right)}

\newcommand{\rep}[1]{\mathcal{R}_{#1}}
\newcommand{\quadruplet}[4]{\left(#1;#2,#3,#4 \right)}
\newcommand{\repere}[4]{\left(#1;\vect{#2},\vect{#3},\vect{#4} \right)}
\newcommand{\base}[3]{\left(\vect{#1},\vect{#2},\vect{#3} \right)}


\newcommand{\vx}[1]{\vect{x_{#1}}}
\newcommand{\vy}[1]{\vect{y_{#1}}}
\newcommand{\vz}[1]{\vect{z_{#1}}}

\newcommand{\norm}[1]{\ensuremath{\left\Vert {#1}\right\Vert}}
\newcommand{\Ker}{\mathop{\mathrm{Ker}}\nolimits}

% d droit pour le calcul différentiel
\newcommand{\dd}{\text{d}}

\newcommand{\inertie}[2]{I_{#1}\left( #2\right)}
\newcommand{\matinertie}[7]{
\begin{pmatrix}
#1 & #6 & #5 \\
#6 & #2 & #4 \\
#5 & #4 & #3 \\
\end{pmatrix}_{#7}}
%%%%%%%%%%%%
% Définition des torseurs 
%%%%%%%%%%%%

\newcommand{\ec}[2]{%
\mathcal{E}_c\left(#1/#2\right)}

\newcommand{\pext}[3]{%
\mathcal{P}\left(#1\rightarrow#2/#3\right)}

\newcommand{\pint}[3]{%
\mathcal{P}\left(#1 \stackrel{\text{#3}}{\leftrightarrow} #2\right)}


 \newcommand{\torseur}[1]{%
\left\{{#1}\right\}
}

\newcommand{\torseurcin}[3]{%
\left\{\mathcal{#1} \left(#2/#3 \right) \right\}
}

\newcommand{\torseurci}[2]{%
\left\{\sigma \left(#1/#2 \right) \right\}
}
\newcommand{\torseurdyn}[2]{%
\left\{\mathcal{D} \left(#1/#2 \right) \right\}
}


\newcommand{\torseurstat}[3]{%
\left\{\mathcal{#1} \left(#2\rightarrow #3 \right) \right\}
}


 \newcommand{\torseurc}[8]{%
%\left\{#1 \right\}=
\left\{
{#1}
\right\}
 = 
\left\{%
\begin{array}{cc}%
{#2} & {#5}\\%
{#3} & {#6}\\%
{#4} & {#7}\\%
\end{array}%
\right\}_{#8}%
}

 \newcommand{\torseurcol}[7]{
\left\{%
\begin{array}{cc}%
{#1} & {#4}\\%
{#2} & {#5}\\%
{#3} & {#6}\\%
\end{array}%
\right\}_{#7}%
}

 \newcommand{\torseurl}[3]{%
%\left\{\mathcal{#1}\right\}_{#2}=%
\left\{%
\begin{array}{l}%
{#1} \\%
{#2} %
\end{array}%
\right\}_{#3}%
}

% Vecteur vitesse
 \newcommand{\vectv}[3]{%
\vect{V\left( {#1} \in {#2}/{#3}\right)}
}

% Vecteur force
\newcommand{\vectf}[2]{%
\vect{R\left( {#1} \rightarrow {#2}\right)}
}

% Vecteur moment stat
\newcommand{\vectm}[3]{%
\vect{\mathcal{M}\left( {#1}, {#2} \rightarrow {#3}\right)}
}




% Vecteur résultante cin
\newcommand{\vectrc}[2]{%
\vect{R_c \left( {#1}/ {#2}\right)}
}
% Vecteur moment cin
\newcommand{\vectmc}[3]{%
\vect{\sigma \left( {#1}, {#2} /{#3}\right)}
}


% Vecteur résultante dyn
\newcommand{\vectrd}[2]{%
\vect{R_d \left( {#1}/ {#2}\right)}
}
% Vecteur moment dyn
\newcommand{\vectmd}[3]{%
\vect{\delta \left( {#1}, {#2} /{#3}\right)}
}

% Vecteur accélération
 \newcommand{\vectg}[3]{%
\vect{\Gamma \left( {#1} \in {#2}/{#3}\right)}
}

% Vecteur omega
 \newcommand{\vecto}[2]{%
\vect{\Omega\left( {#1}/{#2}\right)}
}
% }$$\left\{\mathcal{#1} \right\}_{#2} =%
% \left\{%
% \begin{array}{c}%
%  #3 \\%
%  #4 %
% \end{array}%
% \right\}_{#5}}

\newcommand{\N}{\mathbb{N}}
\newcommand{\Z}{\mathbb{Z}}
\newcommand{\R}{\mathbb{R}}
\newcommand{\C}{\mathbb{C}}
\newcommand{\K}{\mathbb{K}}

\newcommand{\cA}{\mathscr{A}}
\newcommand{\cM}{\mathscr{M}}
\newcommand{\cL}{\mathscr{L}}
\newcommand{\cS}{\mathscr{S}}

\newcommand{\python}{\texttt{Python}}

\newcommand{\z}[1]{\Z_{#1}}
\newcommand{\ztimes}[1]{\Z_{#1}^{\times}}
\newcommand{\ii}[1]{[\![#1[\![}
\newcommand{\iif}[1]{[\![#1]\!]}
\newcommand{\llbr}{\ensuremath{\llbracket}}
\newcommand{\rrbr}{\ensuremath{\rrbracket}}
%\newcommand{\p}[1]{\left(#1\right)}
\newcommand{\ens}[1]{\left\{ #1 \right\}}
\newcommand{\croch}[1]{\left[ #1 \right]}
%\newcommand{\of}[1]{\lstinline{#1}}
% \newcommand{\py}[2]{%
%   \begin{tabular}{|l}
%     \lstinline+>>>+\textbf{\of{#1}}\\
%     \of{#2}
%   \end{tabular}\par{}
% }
\newcommand{\floor}[1]{\left\lfloor#1\right\rfloor}
\newcommand{\ceil}[1]{\left\lceil#1\right\rceil}
\newcommand{\abs}[1]{\left|#1\right|}


% Binaire, octal, hexa
\newcommand{\hex}[1]{\underline{\text{\texttt{#1}}}_{16}}
\newcommand{\oct}[1]{\underline{\text{\texttt{#1}}}_{8}}
\newcommand{\bin}[1]{\underline{\text{\texttt{#1}}}_{2}}
\DeclareMathOperator{\mmod}{\texttt{\%}}


% Fonctions et systèmes
\newcommand{\fct}[5][t]{%
  \begin{array}[#1]{rcl}
    #2 & \rightarrow & #3\\
    #4 & \mapsto     & #5\\
  \end{array}
}
\newcommand{\fonction}[5]{#1 : \left\{\begin{array}{rcl} #2& \longrightarrow &#3 \\ #4 &\longmapsto & #5\end{array}\right.}
\newenvironment{systeme}{\left\{ \begin{array}{rcl}}{\end{array}\right.}

% Matrices
\newcommand{\mat}[1]{
  \begin{pmatrix}
    #1
  \end{pmatrix}
}
\newcommand{\inv}{\ensuremath{^{-1}}}
\newcommand{\bpm}{\begin{pmatrix}}
\newcommand{\epm}{\end{pmatrix}}


% bases de données
\newcommand{\relat}[1]{\textsc{#1}}
\newcommand{\attr}[1]{\emph{#1}}
\newcommand{\prim}[1]{\uline{#1}}
\newcommand{\foreign}[1]{\#\textsl{#1}}


% Bases de données

\newcommand{\att}{\ensuremath{\mathbf{att}}}
\newcommand{\dom}{\ensuremath{\mathbf{dom}}}
\newcommand{\sort}{\ensuremath{\mathbf{sort}}}
\newcommand{\relname}{\ensuremath{\mathbf{relname}}}
\newcommand{\var}{\ensuremath{\mathbf{var}}}
\newcommand{\FILM}{\ensuremath{\mathtt{FILM}}}
\newcommand{\JOUE}{\ensuremath{\mathtt{JOUE}}}
\newcommand{\PERSONNE}{\ensuremath{\mathtt{PERSONNE}}}
\newcommand{\PERSONNAGE}{\ensuremath{\mathtt{PERSONNAGE}}}

\newcommand{\ttid}{\ensuremath{\mathtt{id}}}
\newcommand{\tttitre}{\ensuremath{\mathtt{titre}}}
\newcommand{\ttdate}{\ensuremath{\mathtt{date}}}
\newcommand{\ttidr}{\ensuremath{\mathtt{idrealisateur}}}
\newcommand{\ttdatenais}{\ensuremath{\mathtt{datenaissance}}}
\newcommand{\ttnom}{\ensuremath{\mathtt{nom}}}
\newcommand{\ttprenom}{\ensuremath{\mathtt{prenom}}}
\newcommand{\ttidacteur}{\ensuremath{\mathtt{idacteur}}}
\newcommand{\ttidfilm}{\ensuremath{\mathtt{idfilm}}}
\newcommand{\ttidpersonnage}{\ensuremath{\mathtt{idpersonnage}}}

\newcommand{\fv}{\mathrm{libre}}
\newcommand{\sem}[1]{[\![ #1 ]\!]}

\input{style/macros_Titres}
\input{style/macros_Frames}

%Si le boolen xp est vrai : compilation pour xabi
%Sinon compilation Damien
\newif\ifprof
%\proftrue
\proffalse

\newif\ifxp
\xptrue
%\xpfalse

\newif\iftd
\tdtrue
%\tdfalse

\usepackage[%
    pdftitle={Introduction à l'algorithmique - Applications},
    pdfauthor={Xavier Pessoles},
    colorlinks=true,
    linkcolor=blue,
    citecolor=magenta]{hyperref}

\def\discipline{Informatique}
\def\xxtitre{%
\ifxp
CI 2 : Algorithmique \& Programmation
\else
\fi
}

\def\xxsoustitre{%
\ifxp
Chapitre 2 -- Introduction à l'algorithmique \\
Exercices d'application 1
\else
\fi}

\def\xxauteur{%
\ifxp
Patrick \textsc{Beynet}\\
Xavier \textsc{Pessoles}
\else
\fi}

\def\xxpied{%
\ifxp
CI 2 : Algorithmique \& Programmation\\
Ch. 2 : Introduction à l'algorithmique -- Applications 01
\else
\fi}


%---------------------------------------------------------------------------


\begin{document}
\ifxp
\usepackage[%
    pdftitle={Représentation des nombres},
    pdfauthor={Xavier Pessoles},
    colorlinks=true,
    linkcolor=blue,
    citecolor=magenta]{hyperref}

\usepackage{pifont}
%\usepackage{lastpage}

% \makeatletter \let\ps@plain\ps@empty \makeatother
%% DEBUT DU DOCUMENT
%% =================
\sloppy
\hyphenpenalty 10000


\colorlet{shadecolor}{orange!15}

\newtheorem{theorem}{Theorem}


\begin{document}


%\newboolean{prof}
%\setboolean{prof}{true}
% \makeatletter \let\ps@plain\ps@empty \makeatother
%% DEBUT DU DOCUMENT
%% =================




%------------- En tetes et Pieds de Pages ------------


\pagestyle{fancy}
\ifthenelse{\boolean{xp}}{%
\renewcommand{\headrulewidth}{0pt}}{%
\renewcommand{\headrulewidth}{0.2pt}} %pour mettre le trait en haut
%\renewcommand{\headrulewidth}{0.2pt}

\fancyhead{}
\fancyhead[L]{%
\noindent\begin{minipage}[c]{2.6cm}%
\includegraphics[width=2cm]{png/logo_ptsi.png}%
\end{minipage}}


\fancyhead[C]{\rule{12cm}{.5pt}}



\fancyhead[R]{%
\noindent\begin{minipage}[c]{3cm}
\begin{flushright}
\footnotesize{\textit{\textsf{Informatique}}}%
\end{flushright}
\end{minipage}
}



\fancyhead[C]{\rule{12cm}{.5pt}}

\renewcommand{\footrulewidth}{0.2pt}

\fancyfoot[C]{\footnotesize{\bfseries \thepage}}
\fancyfoot[L]{%
\begin{minipage}[c]{.2\linewidth}
\noindent\footnotesize{{\xxauteur}}
\end{minipage}
\ifthenelse{\boolean{xp}}{}{%
\begin{minipage}[c]{.15\linewidth}
\includegraphics[width=2cm]{png/logoCC.png}
\end{minipage}}
}

\ifthenelse{\boolean{prof}}{%
\fancyfoot[R]{\footnotesize{\xxpied}}}

\begin{center}
 \huge\textsc{\xxtitre}
\end{center}

\begin{center}
 \LARGE\textsc{\xxsoustitre}
\end{center}

\vspace{.5cm}

\else
\input{style/enteteDI}
\fi

\begin{comp}
\begin{itemize}
\item \textit{Alg -- C1 :} comprendre un algorithme et expliquer ce qu’il fait;
\item \textit{Alg -- C2 :} modifier un algorithme existant pour obtenir un résultat différent;
\item \textit{Alg -- C4 :} expliquer le fonctionnement d’un algorithme;
\item \textit{Alg -- C10 :} concevoir l’en-tête (ou la spécification) d’une fonction, puis la fonction elle-même;
\item \textit{Alg -- C11 :} traduire un algorithme dans un langage de programmation;
\item \textit{Alg -- C13 :} rechercher une information au sein d’une documentation en ligne, analyser des exemples fournis dans cette documentation;
\item \textit{Alg -- C14 :} documenter une fonction, un programme plus complexe.
\end{itemize}
\end{comp}

\subsection*{Exercice 1}
\setcounter{subparagraph}{0}
%\subparagraph{}\textit{En utilisant des exemples, expliquer le fonctionnement de l'algorithme 1.}

\subparagraph{}\textit{Traduire cet algorithme en Python en implémentant le fonction \textsf{is\_even} renvoyant \textsf{True} si un entier est pair, \textsf{False} sinon. Vous n'oublierez pas de documenter la fonction.}


\ifprof
\begin{py}
\begin{python}
# Exercice 1
def is_even(n):
    """
    Fonction permettant de savoir si un nombre est pair ou impair
    Entrées : 
     * n(int) : nombre entier
    Sorties : 
     * un booleén valant True si le nombre est pair, False sinon
    """
    return n%2==0
\end{python}
\end{py}
\else
\begin{pseudo}
\begin{center}
\begin{tabular}{p{.9\textwidth}}
\hline
\textbf{Algorithme 1 :} Pair ou impair ? \\
\hline
\textbf{Données :} $n$ : un entier \\
\textbf{Résultat :} $r$ : un booléen vrai si $n$ est pair, faux si $n$ est impair. \\
\\
\textbf{Si} $n$ \textbf{mod} 2 == 0 \textbf{alors} \\
\hspace{.4cm} $r$ $\leftarrow$ Vrai \\
\textbf{Sinon} \\
\hspace{.4cm} $r$ $\leftarrow$ Faux \\
\textbf{Fin Si} \\
\hline
\end{tabular}
\end{center}
\end{pseudo}
\fi

\subsection*{Exercice 2}
\setcounter{subparagraph}{0}
\subparagraph{}\textit{Réaliser l'algorithme en utilisant une boucle \textsf{Pour} permettant de calculer la somme des $n$ premiers entiers. Vous utiliserez la syntaxe Python ou Pseudo code.}

\subparagraph{}\textit{Réaliser l'algorithme en utilisant une boucle \textsf{Tant que} permettant de calculer la somme des $n$ premiers entiers. Vous utiliserez la syntaxe Python.}


\ifprof
\begin{pseudo}
\begin{center}
\begin{tabular}{p{.9\textwidth}}
\hline
\textbf{Algorithme 2 :} Somme des $n$ premiers entiers \\
\hline
\textbf{Données :} $n$ : un entier \\
\textbf{Résultat :} $S$ : le résultat de la somme des $n$ premiers entiers. \\
\\
$S \leftarrow 0$ \\
\textbf{Pour} $i=1$ \textbf{à} $n$ \textbf{faire} \\
\hspace{.4cm} $S$ $\leftarrow$ $S+i$ \\
\textbf{Fin Pour} \\
\hline
\end{tabular}
\end{center}
\end{pseudo}
\begin{py}
\begin{minipage}[c]{.47\linewidth}
\begin{python}
def somme_entiers_for(n):
    """
    Fonction permettant de calculer
     la somme des n premiers entiers
    Entrées : 
     * n(int) : nombre entier
    Sortie : 
     * S (int) : résultat
    """
    S=0
    for i in range(n+1):
       S = S+i
    return S
\end{python}
\end{minipage} \hfill
\begin{minipage}[c]{.47\linewidth}
\begin{python}    
def somme_entiers_while(n):
    """
    Fonction permettant de calculer 
     la somme des n premiers entiers
    Entrées : 
     * n(int) : nombre entier
    Sortie : 
     * S (int) : résultat
    """
    S=0
    i=n
    while i!=0: 
       S = S+i
       i=i-1
    return S
\end{python}
\end{minipage}
\end{py}
\else
\fi

\subsection*{Exercice 3 -- Calcul de $2^n$}
\setcounter{subparagraph}{0}
\subparagraph{}\textit{Implémenter la fonction \textsf{P2\_explicite(n)} en utilisant la méthode $n$ disponible dans la bibliothèque de fonction \textsf{math}.}

\ifprof
\begin{pseudo}
Évaluer le nombre 2 à la puissance $n$ de manière explicite avec $n\in \mathbb{N}$. On définit de manière explicite la suite $u_n =2^n$.
\begin{center}
\begin{tabular}{p{.9\textwidth}}
\hline
\textbf{Algorithme 3 :} Puissance de 2, méthode explicite\\
\hline
\textbf{Données :} $n$ : un nombre entier \\
\textbf{Résultat :} un entier égal à la nième puissance de 2 \\
\\
\textbf{P2\_explicite}(n) \\
\hspace{.4cm} \textbf{Retourner} $2\wedge n$ \\
\hline
\end{tabular}
\end{center}
\end{pseudo}
\else
\begin{py}
\begin{python}
>>> help(pow)
Help on built-in function pow in module builtins:

pow(...)
    pow(x, y[, z]) -> number
    
    With two arguments, equivalent to x**y.  With three arguments,
    equivalent to (x**y) % z, but may be more efficient (e.g. for ints).
\end{python}
\end{py}
\fi


\subparagraph{}\textit{Implémenter la fonction \textsf{P2\_iterative(n)} en utilisant une 
boucle \textsf{Tant que}.}

\ifprof
\begin{pseudo}
\begin{center}
\begin{tabular}{p{.9\textwidth}}
\hline
\textbf{Algorithme 3 :} Puissance de 2, méthode itérative\\
\hline
\textbf{Données :} $n$ : un nombre entier \\
\textbf{Résultat :} un entier égal à la nième puissance de 2 \\
\\
\textbf{P2\_itérative}(n) \\
\hspace{.4cm} $x\leftarrow 1$ \\
\hspace{.4cm} \textbf{Tant que } $n>0$ \textbf{faire} : \\
\hspace{.8cm} $x\leftarrow 2 *x$ \\
\hspace{.8cm} $n\leftarrow n-1$ \\
\hspace{.4cm} \textbf{Fin Tant que}\\
\hspace{.4cm} \textbf{Retourner} $x$ \\
\hline
\end{tabular}
\end{center}
\end{pseudo}

\begin{py}
\begin{minipage}[c]{.47\linewidth}
\begin{python}
def P2_explicite(n):
    """
    Calcul de 2^n
    Entrée :
     * n (int) : un nombre entier
    Sortie :
     * x (int) : nieme puissance de 2
    """
    return math.pow(2,n)
\end{python}
\end{minipage} \hfill
\begin{minipage}[c]{.47\linewidth}
\begin{python}
def P2_iterative(n):
    """
    Calcul de 2^n
    Entrée :
     * n (int) : un nombre entier
    Sortie :
     * x (int) : nieme puissance de 2
    """
    x = 1
    while n>0 :
        x=2*x
        n=n-1
    return x
\end{python}
\end{minipage}
\end{py}

\else
\fi

\subsection*{Exercice 4}
\setcounter{subparagraph}{0}
L'algorithme suivant permet de calculer le nième terme de la suite de Syracuse. 
\begin{pseudo}
\begin{center}
\begin{tabular}{p{.9\textwidth}}
\hline
\textbf{Algorithme 4 :} Suite de Syracuse\\
\hline
%\textbf{Données :} $n$ : un nombre entier \\
%\textbf{Résultat :} $syr$  entier résultat du nième terme de la suite de Syracuse.  \\
%\\
\textbf{Syracuse}(n) \\
\hspace{.4cm} $syr\leftarrow n$ \\
\hspace{.4cm} \textbf{Tant que } $syr \neq 1$ \textbf{faire} : \\
\hspace{.8cm} \textbf{Si } $syr$ \textbf{mod} 2 == 0 \textbf{alors} : \\
\hspace{1.2cm} $syr\leftarrow syr/2$ \\
\hspace{.8cm} \textbf{Sinon}\\
\hspace{1.2cm} $syr\leftarrow 3 *syr+1$ \\
\hspace{.8cm} \textbf{Fin si}\\
\hspace{.4cm} \textbf{Fin Tant que}\\
\hspace{.4cm} \textbf{Retourner} $syr$ \\
\hline
\end{tabular}
\end{center}
\end{pseudo}

\subparagraph{}\textit{Calculer \textsf{Syracuse(10)} et \textsf{Syracuse(12)} et observer l'évolution de la variable \textsf{syr}. En déduire la conjecture de Syracuse. }

\subparagraph{}\textit{Donner les spécifications de la fonction.}

\subparagraph{}\textit{Donner l'énoncé mathématique de la suite de Syracuse.}

\subparagraph{} \textit{On appelle \textbf{temps de vol} le plus petite indice $n$ tel que $u_n = 1$. Modifier l'algorithme pour le calculer.}

\subparagraph{} \textit{On appelle \textbf{altitude} la valeur maximale de la suite. Modifier l'algorithme pour la calculer.}


\subsection*{Exercice 5}
On donne l'algorithme suivant. 

\begin{pseudo}
\begin{center}
\begin{tabular}{p{.9\textwidth}}
\hline
\textbf{Algorithme 5 :} Insertion d'un élément dans une liste de nombres triés par ordre croissant\\
\hline
\textbf{Données :} 
\begin{itemize}
\item $T$ : une liste de nombres triés par ordre croissant $T[1..n]$;
\item $x$ : un nombre.
\end{itemize} \\

\textbf{Résultat :} $T$ : une liste de nombre triés par ordre croissant  $T[1..n+1]$ \\
\\
\textbf{Insertion\_element}(T,x) \\
\hspace{.4cm} $i \leftarrow n$\\
\hspace{.4cm} \textbf{Tant que } $T[i]>x$ \textbf{ou} $i>0$ \textbf{faire} : \\
\hspace{.8cm} $T[i+1] \leftarrow T[i]$ \\
\hspace{.8cm} $i\leftarrow i-1$ \\
\hspace{.4cm} \textbf{Fin Tant que}\\
\hspace{.4cm} $T[i+1]\leftarrow x$ \\
\hline
\end{tabular}
\end{center}
\end{pseudo}

\subparagraph*{}\textit{Expliquer le processus permettant d'insérer un élément dans un tableau. Vous pourrez sur les exemples suivants :
\begin{itemize}
\item $T=[1,2,4,5]$, \textsf{Insertion\_element(T,3)};
\item $T=[1,2,4,5]$, \textsf{Insertion\_element(T,6)};
\end{itemize}}


\end{document}

Conjecture de Syracuse : la suite de Syracuse d’un nombre entier $N$ est définie par
récurrence de la manière suivante :

$$
\forall n \in \mathbb{N} : 
\left\{
\begin{array}{l}
\text{si } n=0, u_0 = N \\
\text{sinon, si  }u_n \text{ est pair, } u_{n+1} = \dfrac{u_n}{2} \\
\text{sinon, si  }u_n \text{ est impair, } u_{n+1} = 3u_n+1 \\
\end{array}
\right.
$$






\begin{pseudo}
On souhaite calculer le nombre réel $x^n$ où $n$ est un entier positif.
\begin{center}
\begin{tabular}{p{.9\textwidth}}
\hline
\textbf{Algorithme 8 :} Exponentiation <<naïve>>\\
\hline
\textbf{Données :} 
\begin{itemize}
\item $n$ : un entier positif;
\item $x$ : un nombre réel.
\end{itemize} 
\textbf{Résultat :} $r$ : un nombre tel que $r=x^n$ \\
\\
\textbf{Exponentiation}(x,n) \\
\hspace{.4cm} \textbf{Si } $n==0$ \textbf{alors} \\
\hspace{.8cm} \textbf{Retourner } 1 \\
\hspace{.4cm} \textbf{Sinon } \\
\hspace{.8cm} $r\leftarrow x$ \\
\hspace{.8cm} \textbf{Pour } $i$ \textbf{de} 2 \textbf{à} $n$ \textbf{faire} : \\
\hspace{1.2cm} $r \leftarrow r*x$\\
\hspace{.8cm} \textbf{Fin Pour}\\
\hspace{.8cm} \textbf{Retourner r}\\
\hspace{.4cm} \textbf{Fin Si} \\
\hline
\end{tabular}
\end{center}
\end{pseudo}



\begin{pseudo}
On souhaite calculer le nombre réel $x^n$ où $n$ est un entier positif.
\begin{center}
\begin{tabular}{p{.9\textwidth}}
\hline
\textbf{Algorithme 8 :} Exponentiation rapide itérative\\
\hline
\textbf{Données :} 
\begin{itemize}
\item $n$ : un entier positif;
\item $x$ : un nombre réel.
\end{itemize} 
\textbf{Résultat :} $r$ : un nombre tel que $r=x^n$ \\
\\
\textbf{Exponentiation\_rapide\_iterative}(x,n) \\
\hspace{.4cm} \textbf{Si } $n==0$ \textbf{alors} \\
\hspace{.8cm} \textbf{Retourner } 1 \\
\hspace{.4cm} \textbf{Sinon } \\
\hspace{.8cm} $r\leftarrow 1$ \\
\hspace{.8cm} \textbf{Tant que } $n>0$ \textbf{faire} : \\
\hspace{1.2cm} \textbf{Si } $n$ \textbf{mod} 2==1 \textbf{alors} : \\
\hspace{1.6cm} $r\leftarrow r*x$ \\
\hspace{1.2cm} \textbf{Fin Si } \\
\hspace{1.2cm} $x\leftarrow x*x$ \\
\hspace{1.2cm} $n\leftarrow n$\textbf{div} 2 \\
\hspace{0.8cm} \textbf{Fin Tant que } \\
\hspace{0.8cm} \textbf{Retourner }$r$ \\
\hspace{0.4cm} \textbf{Fin Si} \\
\hline
\end{tabular}
\end{center}
\end{pseudo}



%\begin{pseudo}
%\begin{center}
%\begin{tabular}{p{.9\textwidth}}
%\hline
%\textbf{Algorithme 3 :} Devinette \\
%\hline
%\textbf{Données :} $n$ : un entier compris entre $1$ et $10$, généré aléatoirement \\
%\\
%$i \leftarrow 0$ \\
%$k \leftarrow 0$ \\
%\textbf{Tant que} $i\neq n$ \textbf{faire} \\
%\hspace{.4cm} $i\leftarrow $\textbf{Entrer}(``\textit{Donner un nombre compris entre 1 et 10 :}'') \\
%\hspace{.4cm} $k\leftarrow k+1$\\
%\textbf{Fin Tant que} \\
%\textbf{Afficher}(``\textit{Gagné en },$k$,\textit{coups !}'') \\
%\hline
%\end{tabular}
%\end{center}
%\end{pseudo}


\end{document}



