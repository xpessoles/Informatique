\documentclass[10pt,fleqn]{article} % Default font size and left-justified equations
\usepackage[%
    pdftitle={Informatique : capteur rayonnement},
    pdfauthor={Xavier Pessoles}]{hyperref}

%%%%%%%%%%%%%%%%%%%%%%%%%%%%%%%%%%%%%%%%%
% Original author:
% Mathias Legrand (legrand.mathias@gmail.com) with modifications by:
% Vel (vel@latextemplates.com)
% License:
% CC BY-NC-SA 3.0 (http://creativecommons.org/licenses/by-nc-sa/3.0/)
%%%%%%%%%%%%%%%%%%%%%%%%%%%%%%%%%%%%%%%%%



%----------------------------------------------------------------------------------------
%	MAIN TABLE OF CONTENTS
%----------------------------------------------------------------------------------------


% Part text styling
\titlecontents{part}[0cm]
{\addvspace{20pt}\centering\large\bfseries}
{}
{}
{}

% Chapter text styling
\titlecontents{chapter}[1.25cm] % Indentation
{\addvspace{12pt}\large\sffamily\bfseries} % Spacing and font options for chapters
{\color{bleuxp!60}\contentslabel[\Large\thecontentslabel]{1.25cm}\color{bleuxp}} % Chapter number
{\color{bleuxp}}  
{\color{bleuxp!60}\normalsize\;\titlerule*[.5pc]{.}\;\thecontentspage} % Page number

% Section text styling
\titlecontents{section}[1.25cm] % Indentation
{\addvspace{3pt}\sffamily\bfseries} % Spacing and font options for sections
{\color{bleuxp!60}\contentslabel[\thecontentslabel]{1.25cm} \color{bleuxp}} % Section number
{\color{bleuxp}}
{\hfill\color{bleuxp!60}\thecontentspage} % Page number
[]

% Subsection text styling
\titlecontents{subsection}[1.25cm] % Indentation
{\addvspace{1pt}\sffamily\small} % Spacing and font options for subsections
{\contentslabel[\thecontentslabel]{1.25cm}} % Subsection number
{}
{\ \titlerule*[.5pc]{.}\;\thecontentspage} % Page number
[]


% Subsection text styling
\titlecontents{subsubsection}[1.25cm] % Indentation
{\addvspace{1pt}\sffamily\small} % Spacing and font options for subsections
{\contentslabel[\thecontentslabel]{1.25cm}} % Subsection number
{}
{\ \titlerule*[.5pc]{.}\;\thecontentspage} % Page number
[]

% List of figures
\titlecontents{figure}[0em]
{\addvspace{-5pt}\sffamily}
{\thecontentslabel\hspace*{1em}}
{}
{\ \titlerule*[.5pc]{.}\;\thecontentspage}
[]

% List of tables
\titlecontents{table}[0em]
{\addvspace{-5pt}\sffamily}
{\thecontentslabel\hspace*{1em}}
{}
{\ \titlerule*[.5pc]{.}\;\thecontentspage}
[]

%----------------------------------------------------------------------------------------
%	MINI TABLE OF CONTENTS IN PART HEADS
%----------------------------------------------------------------------------------------

% Chapter text styling
\titlecontents{lchapter}[0em] % Indenting
{\addvspace{15pt}\large\sffamily\bfseries} % Spacing and font options for chapters
{\color{bleuxp}\contentslabel[\Large\thecontentslabel]{1.25cm}\color{bleuxp}} % Chapter number
{}  
{\color{bleuxp}\normalsize\sffamily\bfseries\;\titlerule*[.5pc]{.}\;\thecontentspage} % Page number

% Section text styling
\titlecontents{lsection}[0em] % Indenting
{\sffamily\small} % Spacing and font options for sections
{\contentslabel[\thecontentslabel]{1.25cm}} % Section number
{}
{}

% Subsection text styling
\titlecontents{lsubsection}[.5em] % Indentation
{\normalfont\footnotesize\sffamily} % Font settings
{}
{}
{}

%----------------------------------------------------------------------------------------
%	PAGE HEADERS
%----------------------------------------------------------------------------------------




\pagestyle{fancy}
 \renewcommand{\headrulewidth}{0pt}
 \fancyhead{}
 
 % ENTETES de page
 \fancyhead[L]{%
 \begin{tikzpicture}[overlay]
\node(logo) at (1,0)
    {\includegraphics[width=2cm]{logo_lycee.png}};
\end{tikzpicture}
 %\noindent\begin{minipage}[c]{2.6cm}%
 %\includegraphics[width=2cm]{logo_lycee.png}%
 %\end{minipage}
}

\fancyhead[C]{\rule{8cm}{.5pt}}

 \fancyhead[R]{%
 \noindent\begin{minipage}[c]{3cm}
 \begin{flushright}
 \footnotesize{\textit{\textsf{\xxtete}}}%
 \end{flushright}
 \end{minipage}
}

 \fancyfoot{}
 % PIEDS de page
\fancyfoot[C]{\rule{12cm}{.5pt}}
\renewcommand{\footrulewidth}{0.2pt}
\fancyfoot[C]{\footnotesize{\bfseries \thepage}}
\fancyfoot[L]{ 
\begin{minipage}[c]{.4\linewidth}
\noindent\footnotesize{{\xxauteur}}
\end{minipage}}

\fancyfoot[R]{\footnotesize{\xxpied}
\ifthenelse{\isodd{\value{page}}}{
\begin{tikzpicture}[overlay]
\node[shape=rectangle, 
      rounded corners = .25 cm,
	  draw= bleuxp,
	  line width=2pt, 
	  fill = bleuxp!10,
	  minimum width  = 2.5cm,
	  minimum height = 3cm,] at (\xxposongletx,\xxposonglety) {};
\node at (\xxposonglettext,\xxposonglety) {\rotatebox{90}{\textbf{\large\color{bleuxp}{\xxonglet}}}};
%{};
\end{tikzpicture}}{}
}



%
%
%
% Removes the header from odd empty pages at the end of chapters
\makeatletter
%\renewcommand{\cleardoublepage}{
%\clearpage\ifodd\c@page\else
%\hbox{}
%\vspace*{\fill}
%\thispagestyle{empty}
%\newpage
%\fi}

%\fancypagestyle{plain}{%
%\fancyhf{} % vide l’en-tête et le pied~de~page.
%%\fancyfoot[C]{\bfseries \thepage} % numéro de la page en cours en gras
%% et centré en pied~de~page.
%\fancyfoot[R]{\footnotesize{\xxpied}}
%\fancyfoot[C]{\rule{12cm}{.5pt}}
%\renewcommand{\footrulewidth}{0.2pt}
%\fancyfoot[C]{\footnotesize{\bfseries \thepage}}
%\fancyfoot[L]{ 
%\begin{minipage}[c]{.4\linewidth}
%\noindent\footnotesize{{\xxauteur}}
%\end{minipage}}}

\fancypagestyle{plain}{%
\fancyhf{} % vide l’en-tête et le pied~de~page.
\fancyfoot[C]{\rule{12cm}{.5pt}}
\renewcommand{\footrulewidth}{0.2pt}
\fancyfoot[C]{\footnotesize{\bfseries \thepage}}
\fancyfoot[L]{ 
\begin{minipage}[c]{.4\linewidth}
\noindent\footnotesize{{\xxauteur}}
\end{minipage}}
\fancyfoot[R]{\footnotesize{\xxpied}}
}







%----------------------------------------------------------------------------------------
%	SECTION NUMBERING IN THE MARGIN
%----------------------------------------------------------------------------------------
\setcounter{secnumdepth}{3}
\setcounter{tocdepth}{2}



\makeatletter
\renewcommand{\@seccntformat}[1]{\llap{\textcolor{bleuxp}{\csname the#1\endcsname}\hspace{1em}}}                    
\renewcommand{\section}{\@startsection{section}{1}{\z@}
{-4ex \@plus -1ex \@minus -.4ex}
{1ex \@plus.2ex }
{\normalfont\large\sffamily\bfseries}}
\renewcommand{\subsection}{\@startsection {subsection}{2}{\z@}
{-3ex \@plus -0.1ex \@minus -.4ex}
{0.5ex \@plus.2ex }
{\normalfont\sffamily\bfseries}}
\renewcommand{\subsubsection}{\@startsection {subsubsection}{3}{\z@}
{-2ex \@plus -0.1ex \@minus -.2ex}
{.2ex \@plus.2ex }
{\normalfont\small\sffamily\bfseries}}                        
\renewcommand\paragraph{\@startsection{paragraph}{4}{\z@}
{-2ex \@plus-.2ex \@minus .2ex}
{.1ex}
{\normalfont\small\sffamily\bfseries}}

%----------------------------------------------------------------------------------------
%	PART HEADINGS
%----------------------------------------------------------------------------------------


%----------------------------------------------------------------------------------------
%	CHAPTER HEADINGS
%----------------------------------------------------------------------------------------

% \newcommand{\thechapterimage}{}%
% \newcommand{\chapterimage}[1]{\renewcommand{\thechapterimage}{#1}}%
% \def\@makechapterhead#1{%
% {\parindent \z@ \raggedright \normalfont
% \ifnum \c@secnumdepth >\m@ne
% \if@mainmatter
% \begin{tikzpicture}[remember picture,overlay]
% \node at (current page.north west)
% {\begin{tikzpicture}[remember picture,overlay]
% \node[anchor=north west,inner sep=0pt] at (0,0) {\includegraphics[width=\paperwidth]{\thechapterimage}};
% \draw[anchor=west] (\Gm@lmargin,-9cm) node [line width=2pt,rounded corners=15pt,draw=bleuxp,fill=white,fill opacity=0.5,inner sep=15pt]{\strut\makebox[22cm]{}};
% \draw[anchor=west] (\Gm@lmargin+.3cm,-9cm) node {\huge\sffamily\bfseries\color{black}\thechapter. #1\strut};
% \end{tikzpicture}};
% \end{tikzpicture}
% \else
% \begin{tikzpicture}[remember picture,overlay]
% \node at (current page.north west)
% {\begin{tikzpicture}[remember picture,overlay]
% \node[anchor=north west,inner sep=0pt] at (0,0) {\includegraphics[width=\paperwidth]{\thechapterimage}};
% \draw[anchor=west] (\Gm@lmargin,-9cm) node [line width=2pt,rounded corners=15pt,draw=bleuxp,fill=white,fill opacity=0.5,inner sep=15pt]{\strut\makebox[22cm]{}};
% \draw[anchor=west] (\Gm@lmargin+.3cm,-9cm) node {\huge\sffamily\bfseries\color{black}#1\strut};
% \end{tikzpicture}};
% \end{tikzpicture}
% \fi\fi\par\vspace*{270\p@}}}

%-------------------------------------------

\def\@makeschapterhead#1{%
\begin{tikzpicture}[remember picture,overlay]
\node at (current page.north west)
{\begin{tikzpicture}[remember picture,overlay]
\node[anchor=north west,inner sep=0pt] at (0,0) {\includegraphics[width=\paperwidth]{\thechapterimage}};
\draw[anchor=west] (\Gm@lmargin,-9cm) node [line width=2pt,rounded corners=15pt,draw=bleuxp,fill=white,fill opacity=0.5,inner sep=15pt]{\strut\makebox[22cm]{}};
\draw[anchor=west] (\Gm@lmargin+.3cm,-9cm) node {\huge\sffamily\bfseries\color{black}#1\strut};
\end{tikzpicture}};
\end{tikzpicture}
\par\vspace*{270\p@}}
\makeatother



%----------------------------------------------------------------------------------------
%	
%----------------------------------------------------------------------------------------

\newcommand{\thechapterimage}{}%
\newcommand{\chapterimage}[1]{\renewcommand{\thechapterimage}{#1}}%
\def\@makechapterhead#1{%
{\parindent \z@ \raggedright \normalfont
\begin{tikzpicture}[remember picture,overlay]
\node at (current page.north west)
{\begin{tikzpicture}[remember picture,overlay]
\node[anchor=north west,inner sep=0pt] at (0,0) {\includegraphics[width=\paperwidth]{\thechapterimage}};
%\draw[anchor=west] (\Gm@lmargin,-9cm) node [line width=2pt,rounded corners=15pt,draw=bleuxp,fill=white,fill opacity=0.5,inner sep=15pt]{\strut\makebox[22cm]{}};
%\draw[anchor=west] (\Gm@lmargin+.3cm,-9cm) node {\huge\sffamily\bfseries\color{black}\thechapter. #1\strut};
\end{tikzpicture}};
\end{tikzpicture}
\par\vspace*{270\p@}
}}


%% Questions et exercices
\newcounter{numques}%Création d'un compteur qui s'appelle numques
\setcounter{numques}{0}%initialisation du compteur
\newcommand{\question}[1]{%Création d'une macro ayant un paramètre
\addtocounter{numques}{1}%chaque fois que cette macro est appelée, elle ajoute 1 au compteur numexos
\textbf{Question\, \textcolor{bleuxp}{\thenumques}\,}\,\textit{#1}}

\newcounter{numexo}%Création d'un compteur qui s'appelle numques
\setcounter{numexo}{0}%initialisation du compteur
\newcommand{\exer}[1]{%Création d'une macro ayant un paramètre
\refstepcounter{numexo} % incrément compteur et label
%\addtocounter{numexo}{1}%chaque fois que cette macro est appelée, elle ajoute 1 au compteur numexo
\noindent\textsf{\textbf{Exercice\, \textcolor{bleuxp}{\thenumexo}\, -- \, #1}}}



% \makeatletter             
% \renewcommand{\subparagraph}{\@startsection{exo}{5}{\z@}%
                                    % {-2ex \@plus-.2ex \@minus .2ex}%
                                    % {0ex}%               
% {\normalfont\bfseries Question \hspace{.7cm} }}
% \makeatother
% \renewcommand{\thesubparagraph}{\arabic{subparagraph}} 
% \makeatletter


%%%%%%%%%%%%
% Définition des vecteurs 
%%%%%%%%%%%%
\newcommand{\vect}[1]{\overrightarrow{#1}}
\newcommand{\axe}[2]{\left(#1,\vect{#2}\right)}
\newcommand{\couple}[2]{\left(#1,\vect{#2}\right)}
\newcommand{\angl}[2]{\left(\vect{#1},\vect{#2}\right)}

\newcommand{\rep}[1]{\mathcal{R}_{#1}}
\newcommand{\quadruplet}[4]{\left(#1;#2,#3,#4 \right)}
\newcommand{\repere}[4]{\left(#1;\vect{#2},\vect{#3},\vect{#4} \right)}
\newcommand{\base}[3]{\left(\vect{#1},\vect{#2},\vect{#3} \right)}


\newcommand{\vx}[1]{\vect{x_{#1}}}
\newcommand{\vy}[1]{\vect{y_{#1}}}
\newcommand{\vz}[1]{\vect{z_{#1}}}

\newcommand{\norm}[1]{\ensuremath{\left\Vert {#1}\right\Vert}}
\newcommand{\Ker}{\mathop{\mathrm{Ker}}\nolimits}

% d droit pour le calcul différentiel
\newcommand{\dd}{\text{d}}

\newcommand{\inertie}[2]{I_{#1}\left( #2\right)}
\newcommand{\matinertie}[7]{
\begin{pmatrix}
#1 & #6 & #5 \\
#6 & #2 & #4 \\
#5 & #4 & #3 \\
\end{pmatrix}_{#7}}
%%%%%%%%%%%%
% Définition des torseurs 
%%%%%%%%%%%%

\newcommand{\ec}[2]{%
\mathcal{E}_c\left(#1/#2\right)}

\newcommand{\pext}[3]{%
\mathcal{P}\left(#1\rightarrow#2/#3\right)}

\newcommand{\pint}[3]{%
\mathcal{P}\left(#1 \stackrel{\text{#3}}{\leftrightarrow} #2\right)}


 \newcommand{\torseur}[1]{%
\left\{{#1}\right\}
}

\newcommand{\torseurcin}[3]{%
\left\{\mathcal{#1} \left(#2/#3 \right) \right\}
}

\newcommand{\torseurci}[2]{%
\left\{\sigma \left(#1/#2 \right) \right\}
}
\newcommand{\torseurdyn}[2]{%
\left\{\mathcal{D} \left(#1/#2 \right) \right\}
}


\newcommand{\torseurstat}[3]{%
\left\{\mathcal{#1} \left(#2\rightarrow #3 \right) \right\}
}


 \newcommand{\torseurc}[8]{%
%\left\{#1 \right\}=
\left\{
{#1}
\right\}
 = 
\left\{%
\begin{array}{cc}%
{#2} & {#5}\\%
{#3} & {#6}\\%
{#4} & {#7}\\%
\end{array}%
\right\}_{#8}%
}

 \newcommand{\torseurcol}[7]{
\left\{%
\begin{array}{cc}%
{#1} & {#4}\\%
{#2} & {#5}\\%
{#3} & {#6}\\%
\end{array}%
\right\}_{#7}%
}

 \newcommand{\torseurl}[3]{%
%\left\{\mathcal{#1}\right\}_{#2}=%
\left\{%
\begin{array}{l}%
{#1} \\%
{#2} %
\end{array}%
\right\}_{#3}%
}

% Vecteur vitesse
 \newcommand{\vectv}[3]{%
\vect{V\left( {#1} \in {#2}/{#3}\right)}
}

% Vecteur force
\newcommand{\vectf}[2]{%
\vect{R\left( {#1} \rightarrow {#2}\right)}
}

% Vecteur moment stat
\newcommand{\vectm}[3]{%
\vect{\mathcal{M}\left( {#1}, {#2} \rightarrow {#3}\right)}
}




% Vecteur résultante cin
\newcommand{\vectrc}[2]{%
\vect{R_c \left( {#1}/ {#2}\right)}
}
% Vecteur moment cin
\newcommand{\vectmc}[3]{%
\vect{\sigma \left( {#1}, {#2} /{#3}\right)}
}


% Vecteur résultante dyn
\newcommand{\vectrd}[2]{%
\vect{R_d \left( {#1}/ {#2}\right)}
}
% Vecteur moment dyn
\newcommand{\vectmd}[3]{%
\vect{\delta \left( {#1}, {#2} /{#3}\right)}
}

% Vecteur accélération
 \newcommand{\vectg}[3]{%
\vect{\Gamma \left( {#1} \in {#2}/{#3}\right)}
}

% Vecteur omega
 \newcommand{\vecto}[2]{%
\vect{\Omega\left( {#1}/{#2}\right)}
}
% }$$\left\{\mathcal{#1} \right\}_{#2} =%
% \left\{%
% \begin{array}{c}%
%  #3 \\%
%  #4 %
% \end{array}%
% \right\}_{#5}}

\newcommand{\N}{\mathbb{N}}
\newcommand{\Z}{\mathbb{Z}}
\newcommand{\R}{\mathbb{R}}
\newcommand{\C}{\mathbb{C}}
\newcommand{\K}{\mathbb{K}}

\newcommand{\cA}{\mathscr{A}}
\newcommand{\cM}{\mathscr{M}}
\newcommand{\cL}{\mathscr{L}}
\newcommand{\cS}{\mathscr{S}}

\newcommand{\python}{\texttt{Python}}

\newcommand{\z}[1]{\Z_{#1}}
\newcommand{\ztimes}[1]{\Z_{#1}^{\times}}
\newcommand{\ii}[1]{[\![#1[\![}
\newcommand{\iif}[1]{[\![#1]\!]}
\newcommand{\llbr}{\ensuremath{\llbracket}}
\newcommand{\rrbr}{\ensuremath{\rrbracket}}
%\newcommand{\p}[1]{\left(#1\right)}
\newcommand{\ens}[1]{\left\{ #1 \right\}}
\newcommand{\croch}[1]{\left[ #1 \right]}
%\newcommand{\of}[1]{\lstinline{#1}}
% \newcommand{\py}[2]{%
%   \begin{tabular}{|l}
%     \lstinline+>>>+\textbf{\of{#1}}\\
%     \of{#2}
%   \end{tabular}\par{}
% }
\newcommand{\floor}[1]{\left\lfloor#1\right\rfloor}
\newcommand{\ceil}[1]{\left\lceil#1\right\rceil}
\newcommand{\abs}[1]{\left|#1\right|}


% Binaire, octal, hexa
\newcommand{\hex}[1]{\underline{\text{\texttt{#1}}}_{16}}
\newcommand{\oct}[1]{\underline{\text{\texttt{#1}}}_{8}}
\newcommand{\bin}[1]{\underline{\text{\texttt{#1}}}_{2}}
\DeclareMathOperator{\mmod}{\texttt{\%}}


% Fonctions et systèmes
\newcommand{\fct}[5][t]{%
  \begin{array}[#1]{rcl}
    #2 & \rightarrow & #3\\
    #4 & \mapsto     & #5\\
  \end{array}
}
\newcommand{\fonction}[5]{#1 : \left\{\begin{array}{rcl} #2& \longrightarrow &#3 \\ #4 &\longmapsto & #5\end{array}\right.}
\newenvironment{systeme}{\left\{ \begin{array}{rcl}}{\end{array}\right.}

% Matrices
\newcommand{\mat}[1]{
  \begin{pmatrix}
    #1
  \end{pmatrix}
}
\newcommand{\inv}{\ensuremath{^{-1}}}
\newcommand{\bpm}{\begin{pmatrix}}
\newcommand{\epm}{\end{pmatrix}}


% bases de données
\newcommand{\relat}[1]{\textsc{#1}}
\newcommand{\attr}[1]{\emph{#1}}
\newcommand{\prim}[1]{\uline{#1}}
\newcommand{\foreign}[1]{\#\textsl{#1}}


% Bases de données

\newcommand{\att}{\ensuremath{\mathbf{att}}}
\newcommand{\dom}{\ensuremath{\mathbf{dom}}}
\newcommand{\sort}{\ensuremath{\mathbf{sort}}}
\newcommand{\relname}{\ensuremath{\mathbf{relname}}}
\newcommand{\var}{\ensuremath{\mathbf{var}}}
\newcommand{\FILM}{\ensuremath{\mathtt{FILM}}}
\newcommand{\JOUE}{\ensuremath{\mathtt{JOUE}}}
\newcommand{\PERSONNE}{\ensuremath{\mathtt{PERSONNE}}}
\newcommand{\PERSONNAGE}{\ensuremath{\mathtt{PERSONNAGE}}}

\newcommand{\ttid}{\ensuremath{\mathtt{id}}}
\newcommand{\tttitre}{\ensuremath{\mathtt{titre}}}
\newcommand{\ttdate}{\ensuremath{\mathtt{date}}}
\newcommand{\ttidr}{\ensuremath{\mathtt{idrealisateur}}}
\newcommand{\ttdatenais}{\ensuremath{\mathtt{datenaissance}}}
\newcommand{\ttnom}{\ensuremath{\mathtt{nom}}}
\newcommand{\ttprenom}{\ensuremath{\mathtt{prenom}}}
\newcommand{\ttidacteur}{\ensuremath{\mathtt{idacteur}}}
\newcommand{\ttidfilm}{\ensuremath{\mathtt{idfilm}}}
\newcommand{\ttidpersonnage}{\ensuremath{\mathtt{idpersonnage}}}

\newcommand{\fv}{\mathrm{libre}}
\newcommand{\sem}[1]{[\![ #1 ]\!]}


\fichetrue
%\fichefalse

\proftrue
%\proffalse

%\tdtrue
\tdfalse

%\courstrue
\coursfalse

% -------------------------------------
% Déclaration des titres
% -------------------------------------

\def\discipline{Informatique \ifprof \\ Corrigé \else \fi}
\def\xxtete{Informatique}

\def\classe{PT -- PT $\star$}
\def\xxnumpartie{DS 2}
\def\xxpartie{Devoir Surveillé 2}

\def\xxnumchapitre{Mesures du rayonnement $\gamma$ $\;$ }
\def\xxchapitre{\subparagraph{$\;$ \\ }}

\def\xxtitreexo{Mesures du rayonnement $\gamma$}
\def\xxsourceexo{\hspace{.2cm} }

\def\xxposongletx{2}
\def\xxposonglettext{1.45}
\def\xxposonglety{20}
\def\xxonglet{\textsf{DM 01}}

\def\xxactivite{}
\def\xxauteur{Patrick Beynet \\ \textsl{La Martinière Monplaisir}}

\def\xxcompetences{%
\texttt{%
\subparagraph{Savoirs et compétences :}\\
\noindent \subparagraph{Résoudre :} à partir des modèles retenus :
\begin{itemize}[label=\ding{112},font=\color{ocre}] 
\item choisir une méthode de résolution analytique, graphique, numérique;
\item mettre en \oe{}uvre une méthode de résolution.
\end{itemize}
\begin{itemize}[label=\ding{112},font=\color{ocre}] 
\item \subparagraph{Rés -- C1.1 :} Loi entrée sortie géométrique et cinématique -- Fermeture géométrique.
\end{itemize}
%
%\noindent \subparagraph{Mod2 -- C4.1 :} Représentation par schéma bloc.
}}

\def\xxfigures{
%\includegraphics[width=.8\textwidth]{images/prot_01}
}%figues de la page de garde

\def\xxpied{%
DS 2 -- Mesures du rayonnement $\gamma$}


\setcounter{secnumdepth}{5}
%---------------------------------------------------------------------------


\begin{document}

%\chapterimage{png/Fond_Cin}
\pagestyle{empty}


%%%%%%%% PAGE DE GARDE COURS
\ifcours
% ==== BANDEAU DES TITRES ==== 
\begin{tikzpicture}[remember picture,overlay]
\node at (current page.north west)
{\begin{tikzpicture}[remember picture,overlay]
\node[anchor=north west,inner sep=0pt] at (0,0) {\includegraphics[width=\paperwidth]{\thechapterimage}};
\draw[anchor=west] (-2cm,-8cm) node [line width=2pt,rounded corners=15pt,draw=ocre,fill=white,fill opacity=0.6,inner sep=40pt]{\strut\makebox[22cm]{}};
\draw[anchor=west] (1cm,-8cm) node {\huge\sffamily\bfseries\color{black} %
\begin{minipage}{1cm}
\rotatebox{90}{\LARGE\sffamily\textsc{\color{ocre}\textbf{\xxnumpartie}}}
\end{minipage} \hfill
\begin{minipage}[c]{14cm}
\begin{titrepartie}
\begin{flushright}
\renewcommand{\baselinestretch}{1.1} 
\Large\sffamily\textsc{\textbf{\xxpartie}}
\renewcommand{\baselinestretch}{1} 
\end{flushright}
\end{titrepartie}
\end{minipage} \hfill
\begin{minipage}[c]{3.5cm}
{\large\sffamily\textsc{\textbf{\color{ocre} \discipline}}}
\end{minipage} 
 };
\end{tikzpicture}};
\end{tikzpicture}
% ==== FIN BANDEAU DES TITRES ==== 


% ==== ONGLET 
\begin{tikzpicture}[overlay]
\node[shape=rectangle, 
      rounded corners = .25 cm,
	  draw= ocre,
	  line width=2pt, 
	  fill = ocre!10,
	  minimum width  = 2.5cm,
	  minimum height = 3cm,] at (18.3cm,0) {};
\node at (17.7cm,0) {\rotatebox{90}{\textbf{\Large\color{ocre}{\classe}}}};
%{};
\end{tikzpicture}
% ==== FIN ONGLET 


\vspace{3.5cm}

\begin{tikzpicture}[remember picture,overlay]
\draw[anchor=west] (-2cm,-6cm) node {\huge\sffamily\bfseries\color{black} %
\begin{minipage}{2cm}
\begin{center}
\LARGE\sffamily\textsc{\color{ocre}\textbf{\xxactivite}}
\end{center}
\end{minipage} \hfill
\begin{minipage}[c]{15cm}
\begin{titrechapitre}
\renewcommand{\baselinestretch}{1.1} 
\Large\sffamily\textsc{\textbf{\xxnumchapitre}}

\Large\sffamily\textsc{\textbf{\xxchapitre}}
\vspace{.5cm}

\renewcommand{\baselinestretch}{1} 
\normalsize\normalfont
\xxcompetences
\end{titrechapitre}
\end{minipage}  };
\end{tikzpicture}
\vfill

\begin{flushright}
\begin{minipage}[c]{.3\linewidth}
\begin{center}
\xxfigures
\end{center}
\end{minipage}\hfill
\begin{minipage}[c]{.6\linewidth}
\startcontents
%\printcontents{}{1}{}
\printcontents{}{1}{}
\end{minipage}
\end{flushright}

\begin{tikzpicture}[remember picture,overlay]
\draw[anchor=west] (4.5cm,-.7cm) node {
\begin{minipage}[c]{.2\linewidth}
\begin{flushright}
\includegraphics[width=2cm]{logoCC}
\end{flushright}
\end{minipage}
\begin{minipage}[c]{.2\linewidth}
\textsl{\xxauteur} \\
\textsl{\classe}
\end{minipage}
 };
\end{tikzpicture}

\newpage
\pagestyle{fancy}

%\newpage
%\pagestyle{fancy}

\else
\fi
%% FIN PAGE DE GARDE DES COURS

%%%%%%%% PAGE DE GARDE TD
\iftd
%\begin{tikzpicture}[remember picture,overlay]
%\node at (current page.north west)
%{\begin{tikzpicture}[remember picture,overlay]
%\draw[anchor=west] (-2cm,-3.25cm) node [line width=2pt,rounded corners=15pt,draw=ocre,fill=white,fill opacity=0.6,inner sep=40pt]{\strut\makebox[22cm]{}};
%\draw[anchor=west] (1cm,-3.25cm) node {\huge\sffamily\bfseries\color{black} %
%\begin{minipage}{1cm}
%\rotatebox{90}{\LARGE\sffamily\textsc{\color{ocre}\textbf{\xxnumpartie}}}
%\end{minipage} \hfill
%\begin{minipage}[c]{13.5cm}
%\begin{titrepartie}
%\begin{flushright}
%\renewcommand{\baselinestretch}{1.1} 
%\Large\sffamily\textsc{\textbf{\xxpartie}}
%\renewcommand{\baselinestretch}{1} 
%\end{flushright}
%\end{titrepartie}
%\end{minipage} \hfill
%\begin{minipage}[c]{3.5cm}
%{\large\sffamily\textsc{\textbf{\color{ocre} \discipline}}}
%\end{minipage} 
% };
%\end{tikzpicture}};
%\end{tikzpicture}

%%%%%%%%%% PAGE DE GARDE TD %%%%%%%%%%%%%%%
%\begin{tikzpicture}[overlay]
%\node[shape=rectangle, 
%      rounded corners = .25 cm,
%	  draw= ocre,
%	  line width=2pt, 
%	  fill = ocre!10,
%	  minimum width  = 2.5cm,
%	  minimum height = 2.5cm,] at (18.5cm,0) {};
%\node at (17.7cm,0) {\rotatebox{90}{\textbf{\Large\color{ocre}{\classe}}}};
%%{};
%\end{tikzpicture}

% PARTIE ET CHAPITRE
%\begin{tikzpicture}[remember picture,overlay]
%\draw[anchor=west] (-1cm,-2.1cm) node {\large\sffamily\bfseries\color{black} %
%\begin{minipage}[c]{15cm}
%\begin{flushleft}
%\xxnumchapitre \\
%\xxchapitre
%\end{flushleft}
%\end{minipage}  };
%\end{tikzpicture}

% BANDEAU EXO
\iflivret % SI LIVRET
\begin{tikzpicture}[remember picture,overlay]
\draw[anchor=west] (-2cm,-3.3cm) node {\huge\sffamily\bfseries\color{black} %
\begin{minipage}{5cm}
\begin{center}
\LARGE\sffamily\color{ocre}\textbf{\textsc{\xxactivite}}

\begin{center}
\xxfigures
\end{center}

\end{center}
\end{minipage} \hfill
\begin{minipage}[c]{12cm}
\begin{titrechapitre}
\renewcommand{\baselinestretch}{1.1} 
\large\sffamily\textbf{\textsc{\xxtitreexo}}

\small\sffamily{\textbf{\textit{\color{black!70}\xxsourceexo}}}
\vspace{.5cm}

\renewcommand{\baselinestretch}{1} 
\normalsize\normalfont
\xxcompetences
\end{titrechapitre}
\end{minipage}};
\end{tikzpicture}
\else % ELSE NOT LIVRET
\begin{tikzpicture}[remember picture,overlay]
\draw[anchor=west] (-2cm,-4.5cm) node {\huge\sffamily\bfseries\color{black} %
\begin{minipage}{5cm}
\begin{center}
\LARGE\sffamily\color{ocre}\textbf{\textsc{\xxactivite}}

\begin{center}
\xxfigures
\end{center}

\end{center}
\end{minipage} \hfill
\begin{minipage}[c]{12cm}
\begin{titrechapitre}
\renewcommand{\baselinestretch}{1.1} 
\large\sffamily\textbf{\textsc{\xxtitreexo}}

\small\sffamily{\textbf{\textit{\color{black!70}\xxsourceexo}}}
\vspace{.5cm}

\renewcommand{\baselinestretch}{1} 
\normalsize\normalfont
\xxcompetences
\end{titrechapitre}
\end{minipage}};
\end{tikzpicture}

\fi

\else   % FIN IF TD
\fi


%%%%%%%% PAGE DE GARDE FICHE
\iffiche
\begin{tikzpicture}[remember picture,overlay]
\node at (current page.north west)
{\begin{tikzpicture}[remember picture,overlay]
\draw[anchor=west] (-2cm,-2.25cm) node [line width=2pt,rounded corners=15pt,draw=ocre,fill=white,fill opacity=0.6,inner sep=40pt]{\strut\makebox[22cm]{}};
\draw[anchor=west] (1cm,-2.25cm) node {\huge\sffamily\bfseries\color{black} %
\begin{minipage}{1cm}
\rotatebox{90}{\LARGE\sffamily\textsc{\color{ocre}\textbf{\xxnumpartie}}}
\end{minipage} \hfill
\begin{minipage}[c]{14cm}
\begin{titrepartie}
\begin{flushright}
\renewcommand{\baselinestretch}{1.1} 
\large\sffamily\textsc{\textbf{\xxpartie} \\} 

\vspace{.2cm}

\normalsize\sffamily\textsc{\textbf{\xxnumchapitre -- \xxchapitre}}
\renewcommand{\baselinestretch}{1} 
\end{flushright}
\end{titrepartie}
\end{minipage} \hfill
\begin{minipage}[c]{3.5cm}
{\large\sffamily\textsc{\textbf{\color{ocre} \discipline}}}
\end{minipage} 
 };
\end{tikzpicture}};
\end{tikzpicture}

\iflivret
\begin{tikzpicture}[overlay]
\node[shape=rectangle, 
      rounded corners = .25 cm,
	  draw= ocre,
	  line width=2pt, 
	  fill = ocre!10,
	  minimum width  = 2.5cm,
	  minimum height = 2.5cm,] at (18.5cm,1.1cm) {};
\node at (17.9cm,1.1cm) {\rotatebox{90}{\textsf{\textbf{\large\color{ocre}{\classe}}}}};
%{};
\end{tikzpicture}
\else
\begin{tikzpicture}[overlay]
\node[shape=rectangle, 
      rounded corners = .25 cm,
	  draw= ocre,
	  line width=2pt, 
	  fill = ocre!10,
	  minimum width  = 2.5cm,
%	  minimum height = 2.5cm,] at (18.5cm,1.1cm) {};
	  minimum height = 2.5cm,] at (18.6cm,1cm) {};
\node at (18cm,1cm) {\rotatebox{90}{\textsf{\textbf{\large\color{ocre}{\classe}}}}};
%{};
\end{tikzpicture}

\fi

\else
\fi



\vspace{2cm}
\pagestyle{fancy}
\thispagestyle{plain}


Un capteur mesure pendant plusieurs années le rayonnement gamma émis par un lointain pulsar.
Pour chaque rayon gamma reçu, repéré par son rang $i (0 \leq i < n)$, on mesure son énergie $a_i$ et l’instant $t_i$ de sa détection.
L’unité d’énergie est le kilo électron-volt (keV), l’unité de temps est le dixième de seconde (1/10 s), l’origine des temps correspond au début de la campagne de mesures.
On supposera $n > 0$.
Pour tout $i$, la quantité $a_i$ est un entier naturel ($a_i \in \mathbb{N}$).
Les valeurs $a_i$ sont rangées dans un tableau $a$ de $n$ éléments de type entier. Ces mesures n’ont pas lieu à intervalles réguliers.
On note les dates $t_i$ (exprimées en 1/10 s, $t_i \in \mathbb{N}$) dans un autre tableau $t$ de $n$ éléments de type entier.
Pour tout $i$ et $j$ tels que $0 \le i < j < n$, on a donc $t_i < t_j$ .

\begin{center}

\begin{tabular}{p{.5cm}|c|c|c|c|c|c|c|c|c|c|c|c|c|c|c|c|c|c|c|c|}
\cline{2-21}
$a$&&&&&&&&&&&&&&&&&&&&\\
\cline{2-21}
\end{tabular}

\vspace{.1cm}

\begin{tabular}{p{.5cm}|c|c|c|c|c|c|c|c|c|c|c|c|c|c|c|c|c|c|c|c|}
\cline{2-21}
$t$&&&&&&&&&&&&&&&&&&&&\\
\cline{2-21}
\end{tabular}
\end{center}


\setcounter{exo}{0}
\subparagraph{} Écrire une fonction \texttt{compte(x,a)} qui retourne le nombre de fois où la valeur $x$ apparaît dans le tableau $a$.


\subparagraph{} En déduire une fonction \texttt{occurrences(a)} qui retourne un tableau $r$ tel que pour tout $i (0 \le i < n)$, l’élément $r_i$ est le nombre de fois où $a_i$ apparaît dans $a$.

\section{Partie 1}

\textbf{DONNER L'ALGO}

On cherche à calculer les périodes $T$ de temps pendant lesquelles le rayonnement reste d’énergie constante.

$T = t_j - t_i$ avec $a_i = a_k = a_j$ pour tout $k$ tel que $i \le k \le j$

\subparagraph{} Écrire une fonction \texttt{maxconstant(a,t)} qui retourne la période la plus grande période $T$ pendant laquelle le rayonnement reste constant.

\section{Partie 2}

Soit $r$ le tableau calculé à la question 2.

\subparagraph{} Écrire une fonction \texttt{maxoccurrences(a,occ)} qui retourne, en temps linéaire en $n$, les indices $i_1$ et $i_2$ de deux rayonnements $m_1$ et $m_2$ qui apparaissent le plus grand nombre de fois dans le tableau de mesures $a$.
(Si le tableau $a$ contient des valeurs toutes identiques, on posera $i_2=m_2 = -1$).

On veut maintenant réorganiser les tableaux de mesures $a$ et de dates $t$ pour mettre en tête toutes les mesures donnant $m_1$, puis celles valant $m_2$, puis toutes les autres.
La réorganisation des tableaux $a$ et $t$ demandée est donc telle que :

$0\le k<b \Rightarrow a_k=m_1$

$b\le k<r \Rightarrow a_k=m_2$

$r\le k<n \Rightarrow a_k \notin {m_1,m_2}$

Après réorganisation, le tableau $t$ vérifie toujours que $t_i$ est la date à laquelle s’est produit le rayonnement $a_i (0 \le i < n)$.



\subparagraph{} Écrire une fonction \texttt{trier(a,t,m1,m2)} qui réordonne, les tableaux $a$ et $t$ pour regrouper en tête les deux mesures les plus fréquentes, comme indiqué précédemment.

Indication : on parcourra les tableaux $a$ et $t$ (dans le sens des indices croissants) en maintenant une décomposition de la forme suivante :

\begin{figure}[h]
\begin{center}
\includegraphics[scale=0.15]{images/Image1.jpg} 
\end{center}
\end{figure}

avec $a_i$ valant respectivement $m_1$, $m_2$ et une valeur non prise dans {$m_1, m_2$} dans les zones 1, 2 et 4.

Au début les tableaux $a$ et $t$ sont de la forme :




\begin{figure}[h]
\begin{center}
\includegraphics[scale=0.21]{images/Image2.jpg} 
\end{center}
\end{figure}

À la fin les mêmes tableaux $a$ et $t$ sont de la forme :



\begin{figure}[h]
\begin{center}
\includegraphics[scale=0.21]{images/Image3.jpg} 
\end{center}
\end{figure}

\subparagraph{} La fonction précédente garde-t-elle la croissance des dates à l’intérieur de chaque zone, c’est-à-dire que $i < j$ implique $t_i < t_j$ pour $i$ et $j$ dans une même zone ? Justifier votre réponse.


\end{document}
